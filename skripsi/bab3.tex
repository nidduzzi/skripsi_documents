% !TEX root = ./skripsi.tex
\chapter{Research Methodology}\label{sec:research_methodology}

% Secara umum, metode penelitian yang digunakan pada pekerjaan penelitian disampaikan pada bab ini. Judul bab tidak harus seperti yang dituliskan. Dalam kata lain bisa diubah sesuai kebutuhan.

\section{Research Design}
\noindent This study is carried out under the framework of a research design. In this section, this design will be laid out and explained. The design includes the process from the beginning to the end of the overall study. The research design is visually depicted in \lccref{fig:research_design_diagram}.

\begin{figure}[h]
  \centering
  \tikzfig{figures/research_design_diagram}
  \caption{Research design diagram}\label{fig:research_design_diagram}
\end{figure}

The study is carried out in 9 main phases. Each phase has a specific aim to accomplish such that the subsequent phases are able to proceed. Each phase is explained as follows:

\begin{enumerate}
  \item Problem Specification

        The first phase of the study is dedicated to specifying and identifying the problem to be studied and will define the solution this study sets out to discover.

  \item Literature Review



  \item Design of Computational Model



  \item Implementation of Computational Model



  \item Planning of Experimental Scenarios



  \item Data Generation and Retrieval



  \item Data Preprocessing



  \item Experiments



  \item Analysis and Discussion


\end{enumerate}

\section{Sub Bab \texorpdfstring{$\alpha$}{α}}
\noindent Misal kita mau masukin tabel, kita bisa juga.
\begin{table}[H]
  \caption{Tabel Sederhana Pertama}
  \label{co:tabel1}
  \begin{center}
    \begin{tabular}{ |c|c|c| }
      \hline
      $G$     & $\text{dim }G$      & $\text{dim }F$ \\
      \hline
      $SU(N)$ & $N^2 -1$            & $N$            \\
      $SO(N)$ & $\frac{1}{2}N(N-1)$ & $N$            \\
      $Sp(N)$ & $N(2N+1)$           & $2N$           \\
      $E_6$   & $78$                & $27$           \\
      $E_7$   & $133$               & $56$           \\
      $E_8$   & $248$               & $248$          \\
      $F_4$   & $52$                & $6$            \\
      $G_2$   & $14$                & $7$            \\
      \hline
    \end{tabular}
  \end{center}
\end{table}

\section{Sub Bab \texorpdfstring{$\beta$}{ꞵ}}
\noindent Atau bisa juga seperti berikut.
\begin{table}[H]
  \caption{Tabel Sederhana Kedua}
  \label{co:tabel2}
  \begin{center}
    \begin{tabularx}{0.8\textwidth} {
      |>{\raggedright\arraybackslash}X
      |>{\raggedright\arraybackslash}X
      |>{\raggedright\arraybackslash}X
      |}
      \hline
      $G$     & $\text{dim }G$      & $\text{dim }F$ \\
      \hline
      $SU(N)$ & $N^2 -1$            & $N$            \\
      $SO(N)$ & $\frac{1}{2}N(N-1)$ & $N$            \\
      $Sp(N)$ & $N(2N+1)$           & $2N$           \\
      $E_6$   & $78$                & $27$           \\
      $E_7$   & $133$               & $56$           \\
      $E_8$   & $248$               & $248$          \\
      $F_4$   & $52$                & $6$            \\
      $G_2$   & $14$                & $7$            \\
      \hline
    \end{tabularx}
  \end{center}
\end{table}
