\chapter{METODOLOGI PENELITIAN}
\label{sec:metodologi_penelitian}

Secara umum, metode penelitian yang digunakan pada pekerjaan penelitian disampaikan pada bab ini. Judul bab tidak harus seperti yang dituliskan. Dalam kata lain bisa diubah sesuai kebutuhan.

\section{Model Komputasi}
\noindent The DeepONet architecture contains an element that is difficult to optimize.
% \begin{mini}
%   {W_b}{y=(W_b\cdot f+b_b)^\top\cdot(W_t\cdot v+b_t)+\xi
% \end{mini}

\begin{mini!}|s|
{\substack{\vb*{W_b},\vb*{W_t},\vb*{\xi}}}{\frac{1}{2}\Vert \vb*{W_b}\Vert_F^2 + \frac{1}{2}\Vert \vb*{W_t}\Vert_F^2 + \frac{c}{2}\sum_{i=1}^{N}\xi_i^2} {}{}
\addConstraint{y_i}{=(\vb*{W_b} \vb*{f_i})^\top(\vb*{W_t} \vb*{v_i})+\xi_i,\quad}{i=1,\ldots,N}
\end{mini!}
Converting the optimization problem above to the lagrange formulation
\begin{equation}
  \begin{aligned}
    L(\vb*{W_b},\vb*{W_t},\vb*{\xi};\vb*{\alpha})= {} & \frac{1}{2}\Vert \vb*{W_b}\Vert_F^2 + \frac{1}{2}\Vert \vb*{W_t}\Vert_F^2 +\frac{c}{2}\sum_{i=1}^{N}\xi_i^2 \\ &+ \sum_{i=1}^{N}\alpha_i(y_i-\xi_i-(\vb*{W_b} \vb*{f_i})^\top(\vb*{W_t} \vb*{v_i}))
  \end{aligned}
\end{equation}
Taking partial derivatives with respect to 
\begin{equation}
  \begin{aligned}
    \pdv{L(W_b,b_b,W_t,b_t,\xi;\alpha)}{W_b}                               & = 0                                                                                \\
    \frac{2}{2}W_b + 0 + 0 - \sum_{i=1}^{l}\alpha_i((W_t v_i+b_t)f_i^\top) & = 0                                                                                \\
    W_b - \sum_{i=1}^{l}\alpha_i((W_t v_i+b_t)f_i^\top)                    & = 0                                                                                \\
    W_b                                                                    & = \sum_{i=1}^{l}\alpha_i(W_t v_i f_i^\top+b_t f_i^\top)                            \\
    W_b                                                                    & = \sum_{i=1}^{l}(\alpha_i W_t v_i f_i^\top)+ \sum_{i=1}^{l}(\alpha_i b_t f_i^\top) \\
    W_b                                                                    & = W_t \sum_{i=1}^{l}(\alpha_i v_i f_i^\top)+ b_t \sum_{i=1}^{l}(\alpha_i f_i^\top) \\
    W_b                                                                    & = W_t \sum_{i=1}^{l}(\alpha_i v_i f_i^\top)+ b_t \sum_{i=1}^{l}(\alpha_i f_i^\top) \\
  \end{aligned}
\end{equation}
\begin{equation}
  \begin{aligned}
    \pdv{L(W_b,b_b,W_t,b_t,\xi;\alpha)}{W_t}                                        & = 0                                                                                                                                                                                    \\
    0 + \frac{2}{2}W_t + 0 - \sum_{i=1}^{l}\alpha_i((W_b f_i + b_b)v_i^\top)        & = 0                                                                                                                                                                                    \\
    W_t - \sum_{i=1}^{l}\alpha_i((W_b f_i + b_b)v_i^\top)                           & = 0                                                                                                                                                                                    \\
    W_t                                                                             & = \sum_{i=1}^{l}\alpha_i(W_b f_i v_i^\top + b_b v_i^\top)                                                                                                                              \\
    W_t                                                                             & = \sum_{i=1}^{l}\alpha_i((\sum_{j=1}^{l}\alpha_j(W_t v_j f_j^\top + b_t f_j^\top)) f_i v_i^\top+b_b v_i^\top)                                                                          \\
    W_t                                                                             & = \sum_{i=1}^{l}(\sum_{j=1}^{l}(\alpha_i\alpha_jW_t v_j f_j^\top f_i v_i^\top + \alpha_i\alpha_j b_t f_j^\top f_i v_i^\top) + \alpha_i b_b v_i^\top)                                   \\
    W_t                                                                             & = \sum_{i=1}^{l}(\sum_{j=1}^{l}(\alpha_i\alpha_jW_t v_j f_j^\top f_i v_i^\top) + \sum_{j=1}^{l}(\alpha_i\alpha_j b_t f_j^\top f_i v_i^\top) + \alpha_i b_b v_i^\top)                   \\
    W_t - \sum_{i=1}^{l}\sum_{j=1}^{l}\alpha_i\alpha_jW_t v_j f_j^\top f_i v_i^\top & = \sum_{i=1}^{l}(\sum_{j=1}^{l}(\alpha_i\alpha_j b_t f_j^\top f_i v_i^\top) + \alpha_i b_b v_i^\top)                                                                                   \\
    W_t(I - \sum_{i=1}^{l}\sum_{j=1}^{l}\alpha_i\alpha_j v_j f_j^\top f_i v_i^\top) & = \sum_{i=1}^{l}(\sum_{j=1}^{l}(\alpha_i\alpha_j b_t f_j^\top f_i v_i^\top) + \alpha_i b_b v_i^\top)                                                                                   \\
    W_t                                                                             & = \sum_{i=1}^{l}(\sum_{j=1}^{l}(\alpha_i\alpha_j b_t f_j^\top f_i v_i^\top) + \alpha_i b_b v_i^\top) (I - \sum_{i=1}^{l}\sum_{j=1}^{l}\alpha_i\alpha_j v_j f_j^\top f_i v_i^\top)^{-1} \\
    W_t                                                                             & = \sum_{i=1}^{l}(\alpha_i(\sum_{j=1}^{l}(\alpha_j b_t f_j^\top f_i) + b_b)v_i^\top ) (I - \sum_{i=1}^{l}\sum_{j=1}^{l}\alpha_i\alpha_j v_j f_j^\top f_i v_i^\top)^{-1}                 \\
  \end{aligned}
\end{equation}
\begin{equation}
  \begin{aligned}
    \pdv{L(W_b,b_b,W_t,b_t,\xi;\alpha)}{b_b}        & = 0 \\
    0 + 0 + 0 - \sum_{i=1}^{l}\alpha_i(W_t v_i+b_t) & = 0 \\
    \sum_{i=1}^{l}\alpha_i(W_t v_i+b_t)             & = 0
  \end{aligned}
\end{equation}
\begin{equation}
  \begin{aligned}
    \pdv{L(W_b,b_b,W_t,b_t,\xi;\alpha)}{b_t}        & = 0 \\
    0 + 0 + 0 - \sum_{i=1}^{l}\alpha_i(W_b f_i+b_b) & = 0 \\
    \sum_{i=1}^{l}\alpha_i(W_b f_i+b_b)             & = 0
  \end{aligned}
\end{equation}
\begin{equation}
  \begin{aligned}
    \pdv{L(W_b,b_b,W_t,b_t,\xi;\alpha)}{\xi}                         & = 0        \\
    0 + 0 + \frac{2c}{2}\sum_{i=1}^{l}\xi_i - \sum_{i=1}^{l}\alpha_i & = 0        \\
    \sum_{i=1}^{l}c\xi_i - \alpha_i                                  & = 0        \\
    c\xi_i                                                           & = \alpha_i \\
  \end{aligned}
\end{equation}
\begin{equation}
  \begin{aligned}
    \pdv{L(W_b,b_b,W_t,b_t,\xi;\alpha)}{\alpha}                         & = 0 \\
    0 + 0 + 0 + \sum_{i=1}^{l}y_i-\xi_i-(W_b f_i+b_b)^\top(W_t v_i+b_t) & = 0 \\
    \sum_{i=1}^{l}y_i-\xi_i-(W_b f_i+b_b)^\top(W_t v_i+b_t)             & = 0
  \end{aligned}
\end{equation}




\section{Sub Bab \texorpdfstring{$\alpha$}{α}}
\noindent Misal kita mau masukin tabel, kita bisa juga.
\begin{table}[H]
  \caption{Tabel Sederhana Pertama}
  \label{co:tabel1}
  \begin{center}
    \begin{tabular}{ |c|c|c| }
      \hline
      $G$     & $\text{dim }G$      & $\text{dim }F$ \\
      \hline
      $SU(N)$ & $N^2 -1$            & $N$            \\
      $SO(N)$ & $\frac{1}{2}N(N-1)$ & $N$            \\
      $Sp(N)$ & $N(2N+1)$           & $2N$           \\
      $E_6$   & $78$                & $27$           \\
      $E_7$   & $133$               & $56$           \\
      $E_8$   & $248$               & $248$          \\
      $F_4$   & $52$                & $6$            \\
      $G_2$   & $14$                & $7$            \\
      \hline
    \end{tabular}
  \end{center}
\end{table}

\section{Sub Bab \texorpdfstring{$\beta$}{ꞵ}}
\noindent Atau bisa juga seperti berikut.
\begin{table}[H]
  \caption{Tabel Sederhana Kedua}
  \label{co:tabel2}
  \begin{center}
    \begin{tabularx}{0.8\textwidth} {
      |>{\raggedright\arraybackslash}X
      |>{\raggedright\arraybackslash}X
      |>{\raggedright\arraybackslash}X
      |}
      \hline
      $G$     & $\text{dim }G$      & $\text{dim }F$ \\
      \hline
      $SU(N)$ & $N^2 -1$            & $N$            \\
      $SO(N)$ & $\frac{1}{2}N(N-1)$ & $N$            \\
      $Sp(N)$ & $N(2N+1)$           & $2N$           \\
      $E_6$   & $78$                & $27$           \\
      $E_7$   & $133$               & $56$           \\
      $E_8$   & $248$               & $248$          \\
      $F_4$   & $52$                & $6$            \\
      $G_2$   & $14$                & $7$            \\
      \hline
    \end{tabularx}
  \end{center}
\end{table}
