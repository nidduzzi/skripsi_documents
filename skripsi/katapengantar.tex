\clearpage
\phantomsection
\addcontentsline{toc}{chapter}{KATA PENGANTAR}
\begin{center}
 \textbf{\large KATA PENGANTAR}\\[3em]
\end{center}
%-----------------------------------------

\noindent Kata pengantar berperan sebagai gerbang masuk bagi pembaca dan mendapat sajian ringkas tentang hal-hal terkait paparan pada buku Tugas Akhir (TA). Sajian ini sejatinya merupakan pengenalan umum bagi pembaca tentang isi tulisan. Hal ini berbeda dengan abstrak yang mendeskripsikan pekerjaan dan hasil penelitian secara lebih teknis.

Kata pengantar merupakan wadah penulis untuk mengenalkan dan mempromosikan pekerjaan dan hasil penelitian dengan bahasa yang sederhana, sehingga pembaca tertarik untuk menelusuri lebih jauh dengan mencermati seluruh paparan pada buku TA. Ini salah satu tujuan kata pengantar. Contoh paragraf yang mengantar pembaca pada isi Buku TA: \textit{Template} \LaTeX\ diberikan berikut ini.

Menuliskan pekerjaan dan hasil penelitian TA dalam suatu laporan buku TA memerlukan panduan standar. Panduan ini dibuat dalam beberapa dokumen, yang salah satunya adalah Buku TA: \textit{Template} \LaTeX. Suatu template adalah cetakan yang siap dituang oleh curahan buah pikiran yang keluar dari pengalaman dalam melakukan pekerjaan penelitian dan hasil-hasilnya. Mencermati cetakan yang memberikan sejumlah contoh dapat memperlancar penulisan laporan tersebut menjadi suatu produk, yaitu buku TA.

Tujuan lain dari Kata Pengantar adalah memberi tempat untuk menyampaikan rasa syukur dan terima kasih kepada banyak pihak, misalnya keluarga, staf akademik, staf tenaga kependidikan, teman, individu atau komunitas pemberi dukungan dan inspirasi, dan institusi pendukung pendanaan seperti pemberi beasiswa atau dana penelitian, atau pendukung akses fasilitas.

\textit{Pengorbanan, kegigihan, dedikasi, dan penuh tanggung jawab dari para pahlawan pekerja medis dalam perawatan pasien terpapar Covid-19 telah memberi inspirasi melalui nilai-nilai kejuangan tanpa pamrih. Inspirasi inilah yang membangkitkan spirit pamungkas pada penyelesaian Buku TA: Template} \LaTeX\ \textit{ini. Suatu inspirasi selalu bekerja dan mengena secara tidak langsung. Banyak berterima kasih atas inspirasi yang memantik spirit ini.}

\newpage
Tidak ada sub bab/bagian pada Kata Pengantar, namun daftar rincian diperkenankan. Pada bagian indentitas akhir, seperti berikut ini, dituliskan nama mahasiswa dan NIM, bukan \textit{penulis}, dan tidak perlu ditandatangani. Berikut adalah contoh penulisan rincian yang berisi ucapan terima kasih:
\begin{itemize}
 \item \pembimbingsatu dan \pembimbingdua selaku dosen pembimbing tugas akhir.
 \item Jyesta, sebagai teman bimbingan yang selalu bersedia untuk diajak berdiskusi selama penelitian.
 \item Rekan-rekan Spectranova, ............
\end{itemize}

\begin{flushright}
Bandung, \today\\[0.25cm]
\penulis \\
\nim
\end{flushright}
