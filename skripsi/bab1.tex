\chapter{PENDAHULUAN}
\section{Latar Belakang}
\label{sec:latarbelakang} %pelabelan ini opsional, biar bisa di klik kalo mengarahkan ke section tertentu.

\noindent Bagian ini mendeskripsikan gambaran umum, konteks, dan posisi penelitian TA dalam konstelasi perkembangan pengetahuan yang telah dicapai. Penjelasan yang dituliskan menjadi penting karena dengan landasan yang kuat, maka pekerjaan penelitian dapat terarah dilakukan. Hal ini lebih spesifik dan tegas disampaikan pada sub-sub bab berikutnya.

Beberapa pustaka utama yang berperan dominan dapat disampaikan di sini untuk memberi gambaran tentang letak penelitian TA dalam konstelasi keilmuan yang dicapai. Hasil-hasil dari pustaka terbaru dapat menopang Latar Belakang ini menjadi lebih kuat.

Sangat wajar apabila isi sub bab setelah Latar Belakang ini mengalami penyesuaian saat sejumlah hasil penelitian sudah diperoleh dan dianalisis. Pada dasarnya, hal ini dimungkinkan apabila ada penyesuaian kecil, karena fokus penelitian sejatinya sudah jelas sedari awal, namun hasil-hasil yang diperoleh dapat memperbaharui beberapa butir isi sub bab. Oleh karena itu, finalisasi isi Pendahuluan ini biasanya dilakukan menjelang akhir pembuatan laporan penelitian yang dituangkan dalam buku TA.


\section{Rumusan dan Batasan Masalah}
\noindent Bagian ini menjadi salah satu bagian penting dalam Pendahuluan. Setelah paparan Latar Belakang, maka masalah yang diangkat pada pekerjaan penelitian perlu dirumuskan dengan baik. Perumusan ini sebaiknya dibahasakan tidak dalam bentuk kalimat pertanyaan, melainkan kalimat aktif, dan dapat memuat lebih dari satu rumusan.

Sejalan dengan ini, setiap masalah yang diangkat selalu memiliki batas. Ada batasan, asumsi, atau kriteria yang menjadi pembatas atas masalah yang diangkat dalam penelitian TA, sehingga arah penelitian dapat fokus. Batasan ini perlu dituliskan secara tegas, dan dapat saja memuat lebih dari satu.


\section{Tujuan}
\label{sec:tujuan}
Bagian ini secara tegas menuliskan tujuan pekerjaan penelitian TA, yang dapat memuat lebih dari satu. Pemilihan kata kerja pada Tujuan ini sangat penting karena menggambarkan arah fokus dari jalinan upaya yang dilakukan.


\section{Metodologi}
Di sini disampaikan metodologi yang diterapkan pada pekerjaan penelitian TA. Beberpa di antaranya adalah pengamatan dan akuisisi data, eksperimen numerik, studi pustaka, teoretik atau analitik, dan semi analitik dengan komplemen numerik.



\section{Sistematika Penulisan}
\noindent Bagian ini adalah penutup Bab I yang menyampaikan  secara ringkas isi setiap  bab. Karena pembaca sudah sampai akhir Bab I, yang  berarti  sudah  mengetahui isinya, maka tidak perlu ditulis lagi rincian Bab I. Sebaiknya langsung dituliskan secara ringkas isi rincian bab-bab selanjutnya, misalnya, \textit{Setelah Pendahuluan pada Bab I ini, Bab II akan mengulas tentang ...}

Apabila diperlukan, dapat dituliskan konvensi khusus yang digunakan pada penulisan naskah buku TA ini, misalnya tanda titik menggantikan tanda desimal karena alasan kemudahan dan kejelasan dalam formulasi matematika.