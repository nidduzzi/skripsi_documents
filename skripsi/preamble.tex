\usepackage{geometry}
\geometry{verbose,tmargin=3cm,bmargin=3cm,lmargin=4cm,rmargin=3cm}
\usepackage[pdftex,bookmarks=true,unicode=true,pdfusetitle,bookmarksnumbered=true,bookmarksopen=true,breaklinks=true,pdfborder={0 0 1},backref=none,colorlinks=false]{hyperref}

%%%% identitas (kalo file pdf nya di klik kanan, info..)
\hypersetup{pdfauthor={\penulis},
  pdfsubject={SKRIPSI},
  pdftitle={\judul},
  pdfkeywords={\keywords}}

% \usepackage[htt]{hyphenat}
\usepackage[none]{hyphenat}
\sloppy
%\raggedright		%fix justify
%\hyphenpenalty=10000
%\hbadness=10000 


\usepackage{listing}
\usepackage{graphicx}      %menghandle gambar
\usepackage[inkscapelatex=false]{svg} 	%handle svg with \includesvg
\usepackage{tabularx}
\usepackage{float}
\usepackage[hang,nooneline,scriptsize,md]{subfigure}
\usepackage{epsfig}
\usepackage[font=small,labelfont=bf]{caption}
\usepackage[titletoc]{appendix}

\usepackage[utf8]{inputenc}
\usepackage[backend=biber, style=apa]{biblatex} %TODO change style to apa or other style, backref=true
\addbibresource{references.bib}
%\usepackage{cite}          % buat cite pake BibTex
%\bibliographystyle{apalike}
% \usepackage[bahasa]{babel} %plugin bahasa 
\usepackage[american]{babel} %plugin english  
\usepackage{csquotes}
\usepackage{xpatch}

\usepackage[T1]{fontenc}   % biar bisa kombinasi font
\usepackage{mathptmx}
\usepackage{amssymb}
\usepackage{amsmath}
\usepackage{mathrsfs}
\usepackage{dsfont}
\usepackage{listings}
\usepackage{relsize}
\usepackage{physics}
\usepackage{optidef}  % for writing optimization problems easier
\usepackage{algpseudocode}
\usepackage{algorithm}
\renewcommand{\algorithmicrequire}{\textbf{Input:}}
\renewcommand{\algorithmicensure}{\textbf{Output:}}
\usepackage{booktabs}
\usepackage{cleveref}
%lowercase non abbreviation referece
\newcommand{\lccref}[1]{\lcnamecref{#1}~\labelcref{#1}}
\makeatletter
\def\lcfirstnamecrefs#1,#2\@nil{\lcnamecrefs{#1}}
\newcommand{\lcfirstnamecref}[1]{\lcfirstnamecrefs #1,\@nil}
\makeatother
\newcommand{\lccrefs}[1]{\lcfirstnamecref{#1}~\labelcref{#1}}

%%% pengaturan global
\graphicspath{{figures/}}   % letak direktori penyimpanan gambar

%format judul
\usepackage{titlesec}
\titleformat{\chapter}[display]
{\bfseries\Large}
{%
  % \vskip-5em
  \filcenter
  \LARGE\MakeUppercase \chaptertitlename\
  \LARGE\thechapter}
{0mm}
{\filcenter}
\titleformat*{\section}{\bfseries\large}
\titleformat*{\subsection}{\bfseries}
\titlespacing*{\chapter}{0pt}{-1.5\baselineskip}{1em}

%menambahkan titik titik di daftar isi, gambar, tabel
\usepackage[subfigure]{tocloft}
\renewcommand{\cftpartleader}{\dotfill}
\renewcommand{\cftpartafterpnum}{\cftparfillskip}
\renewcommand{\cftchapleader}{\dotfill}
\renewcommand{\cftsecleader}{\dotfill}
\renewcommand{\cftsubsecleader}{\dotfill}

%typeset
\linespread{1.5}
\setlength{\emergencystretch}{3em}

\usepackage[onehalfspacing]{setspace}      %buat kombinasi spacing
\usepackage{indentfirst}
%\setlength{\parindent}{0.6cm}
\usepackage{parskip} 	%safely change the space inserted between paragraphs

\pagestyle{plain}
\setlength{\parindent}{20pt}

%tambah "bab" x di daftar isi 
\renewcommand*\cftchappresnum{\MakeUppercase{bab}~}
\renewcommand\chaptername{BAB}
\settowidth{\cftchapnumwidth}{\cftchappresnum}
\renewcommand{\cftchapaftersnumb}{\quad}
\addtocontents{toc}{
  %\renewcommand\protect*\protect\cftchappresnum{\MakeUppercase{\chaptername}~}}
  \protect\renewcommand*\protect\cftchappresnum{\MakeUppercase{\chaptername}~}}

%Penomoran Bab berupa bilangan romawi
\renewcommand{\thechapter}{\Roman{chapter}}%
\renewcommand{\thesection}{\arabic{chapter}.\arabic{section}}%
\renewcommand{\thesubsection}{\arabic{chapter}.\arabic{section}.\arabic{subsection}}%
\renewcommand{\thesubsubsection}{\arabic{chapter}.\arabic{section}.\arabic{subsection}.\arabic{subsubsection}}%

%Penomoran gambar & table
\renewcommand{\thefigure}{\arabic{chapter}.\arabic{figure}}
\renewcommand{\thetable}{\arabic{chapter}.\arabic{table}}

%Penomoran equations
\renewcommand{\theequation}{\arabic{chapter}.\arabic{equation}}

%tambah "gambar" dan "table" di daftar gambar dan daftar tabel
\usepackage{chngcntr}
%\counterwithout{figure}{chapter}
\renewcommand{\cftfigpresnum}{\bfseries Gambar\ }
\renewcommand{\cfttabpresnum}{\bfseries Tabel\ }
\newlength{\mylenf}
\settowidth{\mylenf}{\cftfigpresnum}
\setlength{\cftfignumwidth}{\dimexpr\mylenf+2em}
\settowidth{\mylenf}{\cfttabpresnum}
\setlength{\cfttabnumwidth}{\dimexpr\mylenf+2em}
\makeatletter

%bikin listoftable tanpa judul, gara2 aneh
%\counterwithout{table}{chapter}
\newcommand\daftartabel{%
  \phantomsection
  \@starttoc{lof}%
  \bigskip
  \@starttoc{lot}}
\makeatother



%ilangin judul bibli 
\makeatletter
\renewenvironment{thebibliography}[1]{%
  \section*{\refname}%
  \@mkboth{\MakeUppercase\refname}{\MakeUppercase\refname}%
  \list{\@biblabel{\@arabic\c@enumiv}}%
  {\settowidth\labelwidth{\@biblabel{#1}}%
    \leftmargin\labelwidth
    \advance\leftmargin\labelsep
    \@openbib@code
    \usecounter{enumiv}%
    \let\p@enumiv\@empty
    \renewcommand\theenumiv{\@arabic\c@enumiv}}%
  \sloppy
  \clubpenalty4000
  \@clubpenalty \clubpenalty
  \widowpenalty4000%
  \sfcode`\.\@m}
{\def\@noitemerr
  {\@latex@warning{Empty `thebibliography' environment}}%
  \endlist}
\makeatother


%lampiran
\makeatletter
\newcommand\appendix@chapter[1]{%
  \refstepcounter{chapter}%
  \orig@chapter*{LAMPIRAN \@Alph\c@chapter \\  #1}\vspace{1.5em}%
  \addcontentsline{toc}{chapter}{LAMPIRAN \@Alph\c@chapter: #1}%
}
\let\orig@chapter\chapter
\g@addto@macro\appendix{\let\chapter\appendix@chapter}
\makeatother



