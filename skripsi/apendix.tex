% !TEX root = ./skripsi.tex
\appendix

\chapter{EXAMPLE COMPUTATION OF LSSVR}
% Example computation
In this example we will be using the function \(2x^2+4\). The values of this function can be seen in \lccref{table:lssvr_example_function_values}.

\begin{table}[H]
  \centering
  \begin{tabular}{@{}lll@{}}
    \toprule
    No & x          & y          \\ \midrule
    1  & 0.0        & 4.0        \\
    2  & 0.33\ldots & 4.22\ldots \\
    3  & 0.66\ldots & 4.88\ldots \\
    4  & 1.0        & 6.0        \\
    \bottomrule
  \end{tabular}
  \caption{Example data of function \(2x^2+4\)}\label{table:lssvr_example_function_values}
\end{table}
For \(\Omega_{i,j}=K(x_i,x_j)=\exp\left(-\frac{\norm*{x_i-x_j}^2}{2\sigma^2}\right)\)

For example, with \(\sigma=1\), \(x_i=0.0\), \& \(x_j = 0.33\dots \)
\begin{equation}
  \begin{split}
    K(0.0,0.33) & = \exp\left(-\frac{\norm*{0.0-0.33}^2}{2{(1)}^2}\right) \\
                & = \exp\left(-\frac{0.33^2}{2}\right)                  \\
                & = 0.9460
  \end{split}
\end{equation}

\begin{equation}
  \Omega\gets
  \begin{bmatrix}
    1.0000 & 0.9460 & 0.8007 & 0.6065 \\
    0.9460 & 1.0000 & 0.9460 & 0.8007 \\
    0.8007 & 0.9460 & 1.0000 & 0.9460 \\
    0.6065 & 0.8007 & 0.9460 & 1.0000 \\
  \end{bmatrix}
\end{equation}

\begin{equation}
  \vb{I}\frac{1}{C}\to\vb{I}\frac{1}{5}\to
  \begin{bmatrix}
    0.2000 & 0.0000 & 0.0000 & 0.0000 \\
    0.0000 & 0.2000 & 0.0000 & 0.0000 \\
    0.0000 & 0.0000 & 0.2000 & 0.0000 \\
    0.0000 & 0.0000 & 0.0000 & 0.2000
  \end{bmatrix}
\end{equation}

\begin{equation}
  \Omega+\vb{I}\frac{1}{5}\to H\to
  \begin{bmatrix}
    1.2000 & 0.9460 & 0.8007 & 0.6065 \\
    0.9460 & 1.2000 & 0.9460 & 0.8007 \\
    0.8007 & 0.9460 & 1.2000 & 0.9460 \\
    0.6065 & 0.8007 & 0.9460 & 1.2000 \\
  \end{bmatrix}
\end{equation}

\begin{equation}
  A\to
  \begin{bmatrix}
    0.0000 & 1.0000 & 1.0000 & 1.0000 & 1.0000 \\
    1.0000 & 1.2000 & 0.9460 & 0.8007 & 0.6065 \\
    1.0000 & 0.9460 & 1.2000 & 0.9460 & 0.8007 \\
    1.0000 & 0.8007 & 0.9460 & 1.2000 & 0.9460 \\
    1.0000 & 0.6065 & 0.8007 & 0.9460 & 1.2000 \\
  \end{bmatrix}
\end{equation}

\begin{equation}
  B\to
  \begin{bmatrix}
    0.0000 \\
    4.0000 \\
    4.2222 \\
    4.8889 \\
    6.0000 \\
  \end{bmatrix}
\end{equation}

\begin{equation}
  A^{\dag}\to
  \begin{bmatrix}
    -0.8994 & 0.4348  & 0.0652  & 0.0652  & 0.4348  \\
    0.4348  & 2.0686  & -1.6490 & -0.5292 & 0.1096  \\
    0.0652  & -1.6490 & 3.3774  & -1.1992 & -0.5292 \\
    0.0652  & -0.5292 & -1.1992 & 3.3774  & -1.6490 \\
    0.4348  & 0.1096  & -0.5292 & -1.6490 & 2.0686  \\
  \end{bmatrix}
\end{equation}

\begin{equation}
  A^{\dag}B\to S\to
  \begin{bmatrix}
    4.9421  \\
    -0.6177 \\
    -1.3737 \\
    -0.5625 \\
    2.5538  \\
  \end{bmatrix}
\end{equation}

\begin{equation}
  b\to 4.9421
\end{equation}

\begin{equation}
  \alpha \to
  \begin{bmatrix}
    -0.6177 \\
    -1.3737 \\
    -0.5625 \\
    2.5538  \\
  \end{bmatrix}
\end{equation}

Prediction
\begin{equation}
  U\to
  \begin{bmatrix}
    0.3 \\
    0.2 \\
    0.5
  \end{bmatrix}
\end{equation}

\begin{equation}
  \Omega\to
  \begin{bmatrix}
    0.9560 & 0.9994 & 0.9350 & 0.7827 \\
    0.9802 & 0.9912 & 0.8968 & 0.7261 \\
    0.8825 & 0.9862 & 0.9862 & 0.8825 \\
  \end{bmatrix}
\end{equation}

\begin{equation}
  \Omega\alpha\to
  \begin{bmatrix}
    -0.4904 \\
    -0.6170 \\
    -0.2008 \\
  \end{bmatrix}
\end{equation}

\begin{equation}
  \Omega\alpha+b\vb{1}_{m}\to v\to
  \begin{bmatrix}
    4.4516 \\
    4.3251 \\
    4.7413 \\
  \end{bmatrix}
\end{equation}
Where \(\vb{1}_{m}\) is a vector of 1s with the length of \(U\).

\chapter{BURGERS' EQUATION COMPARISON}\label{sec:burgers_comparison}
Comparison of SpectralSVR against the Method of Lines with finite differences. The numerical methods used for comparison are lsoda and Implicit Adams-Bashforth-Moulton methods.
\begin{figure}[H]
  \centering
  \begin{adjustwidth}{-0.05\linewidth}{-0.05\linewidth}
    \begin{subfigure}{0.49\linewidth}
      \begin{adjustbox}{width=\linewidth}
        \input{figures/comparisons/burgers_rollout_model_pred_0.0.pgf}
      \end{adjustbox}
      \caption{SpectralSVR prediction.}\label{fig:comp_model_pred_0.0}
    \end{subfigure}
    \begin{subfigure}{0.49\linewidth}
      \begin{adjustbox}{width=\linewidth}
        \input{figures/comparisons/burgers_rollout_model_diff_0.0.pgf}
      \end{adjustbox}
      \caption{SpectralSVR difference with target.}\label{fig:comp_model_diff_0.0}
    \end{subfigure}
    \begin{subfigure}{0.49\linewidth}
      \begin{adjustbox}{width=\linewidth}
        \input{figures/comparisons/burgers_rollout_spo_pred_0.0.pgf}
      \end{adjustbox}
      \caption{SciPy \& NumPy prediction.}\label{fig:comp_spo_pred_0.0}
    \end{subfigure}
    \begin{subfigure}{0.49\linewidth}
      \begin{adjustbox}{width=\linewidth}
        \input{figures/comparisons/burgers_rollout_spo_diff_0.0.pgf}
      \end{adjustbox}
      \caption{SciPy \& NumPy difference with target.}\label{fig:comp_spo_diff_0.0}
    \end{subfigure}
    \begin{subfigure}{0.49\linewidth}
      \begin{adjustbox}{width=\linewidth}
        \input{figures/comparisons/burgers_rollout_tdo_pred_0.0.pgf}
      \end{adjustbox}
      \caption{torchdiffeq \& PyTorch prediction.}\label{fig:comp_tdo_pred_0.0}
    \end{subfigure}
    \begin{subfigure}{0.49\linewidth}
      \begin{adjustbox}{width=\linewidth}
        \input{figures/comparisons/burgers_rollout_tdo_diff_0.0.pgf}
      \end{adjustbox}
      \caption{torchdiffeq \& PyTorch difference with target.}\label{fig:comp_tdo_diff_0.0}
    \end{subfigure}
  \end{adjustwidth}
  \caption{Comparison of our model (SpectralSVR) and numerical methods for the Forced Burgers equation with viscosity \(\nu=0.0\).}\label{fig:comparison_burgers_0.0}
\end{figure}

\begin{figure}[H]
  \centering
  \begin{adjustwidth}{-0.05\linewidth}{-0.05\linewidth}
    \begin{subfigure}{0.49\linewidth}
      \begin{adjustbox}{width=\linewidth}
        \input{figures/comparisons/burgers_rollout_model_pred_0.01.pgf}
      \end{adjustbox}
      \caption{SpectralSVR prediction.}\label{fig:comp_model_pred_0.01}
    \end{subfigure}
    \begin{subfigure}{0.49\linewidth}
      \begin{adjustbox}{width=\linewidth}
        \input{figures/comparisons/burgers_rollout_model_diff_0.01.pgf}
      \end{adjustbox}
      \caption{SpectralSVR difference with target.}\label{fig:comp_model_diff_0.01}
    \end{subfigure}
    \begin{subfigure}{0.49\linewidth}
      \begin{adjustbox}{width=\linewidth}
        \input{figures/comparisons/burgers_rollout_spo_pred_0.01.pgf}
      \end{adjustbox}
      \caption{SciPy \& NumPy prediction.}\label{fig:comp_spo_pred_0.01}
    \end{subfigure}
    \begin{subfigure}{0.49\linewidth}
      \begin{adjustbox}{width=\linewidth}
        \input{figures/comparisons/burgers_rollout_spo_diff_0.01.pgf}
      \end{adjustbox}
      \caption{SciPy \& NumPy difference with target.}\label{fig:comp_spo_diff_0.01}
    \end{subfigure}
    \begin{subfigure}{0.49\linewidth}
      \begin{adjustbox}{width=\linewidth}
        \input{figures/comparisons/burgers_rollout_tdo_pred_0.01.pgf}
      \end{adjustbox}
      \caption{torchdiffeq \& PyTorch prediction.}\label{fig:comp_tdo_pred_0.01}
    \end{subfigure}
    \begin{subfigure}{0.49\linewidth}
      \begin{adjustbox}{width=\linewidth}
        \input{figures/comparisons/burgers_rollout_tdo_diff_0.01.pgf}
      \end{adjustbox}
      \caption{torchdiffeq \& PyTorch difference with target.}\label{fig:comp_tdo_diff_0.01}
    \end{subfigure}
  \end{adjustwidth}
  \caption{Comparison of our model (SpectralSVR) and numerical methods for the Forced Burgers equation with viscosity \(\nu=0.01\).}\label{fig:comparison_burgers_0.01}
\end{figure}
\begin{table}
  \caption{Metrics for predicting the exact solution for SpectralSVR (Ours) method}\label{table:comparison_exact_metrics_model}
  \centering
  \begin{tabular}{lcccc}
    \toprule
    \(\nu \) & MSE  & RMSE & MAE  & sMAPE \\
    \midrule
    0.0      & 0.01 & 0.07 & 0.06 & 1.99  \\
    0.01     & 0.00 & 0.07 & 0.06 & 1.61  \\
    0.1      & 0.00 & 0.02 & 0.02 & 1.47  \\
    \bottomrule
  \end{tabular}
\end{table}
\begin{table}
  \caption{Metrics for predicting the exact solution for the lsoda method}\label{table:comparison_exact_metrics_lsoda}
  \centering
  \begin{tabular}{lcccc}
    \toprule
    \(\nu \) & MSE  & RMSE & MAE  & sMAPE \\
    \midrule
    0.0      & 0.00 & 0.00 & 0.00 & 0.00  \\
    0.01     & 0.00 & 0.00 & 0.00 & 0.09  \\
    0.1      & 0.00 & 0.03 & 0.02 & 1.56  \\
    \bottomrule
  \end{tabular}
\end{table}
\begin{table}
  \caption{Metrics for predicting the exact solution for the IABM method}\label{table:comparison_exact_metrics_iabm}
  \centering
  \begin{tabular}{lcccc}
    \toprule
    \(\nu \) & MSE  & RMSE & MAE  & sMAPE \\
    \midrule
    0.0      & 0.00 & 0.00 & 0.00 & 0.00  \\
    0.01     & nan  & nan  & nan  & nan   \\
    0.1      & nan  & nan  & nan  & nan   \\
    \bottomrule
  \end{tabular}
\end{table}
% \section{PROGRAM SATU}
% \section{PROGRAM DUA}


% \chapter{GAMBAR-GAMBAR}

