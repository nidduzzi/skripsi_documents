% !TEX root = ./skripsi.tex
\appendix

\chapter{EXAMPLE COMPUTATION OF LSSVR}
% Example computation
In this example we will be using the function \(2x^2+4\). The values of this function can be seen in \lccref{table:1}.

\begin{table}[H]
  \centering
  \begin{tabular}{@{}lll@{}}
    \toprule
    No & x          & y          \\ \midrule
    1  & 0.0        & 4.0        \\
    2  & 0.33\ldots & 4.22\ldots \\
    3  & 0.66\ldots & 4.88\ldots \\
    4  & 1.0        & 6.0        \\
    \bottomrule
  \end{tabular}
  \caption{Example data of function \(2x^2+4\)}\label{table:lssvr_example_function_values}
\end{table}
For \(\Omega_{i,j}=K(x_i,x_j)=exp \left(-\frac{\norm*{x_i-x_j}^2}{2\sigma^2}\right)\)

For example, with \(\sigma=1\), \(x_i=0.0\), \& \(x_j = 0.33\dots\)
\begin{equation}
  \begin{split}
    K(0.0,0.33) & = \exp\left(-\frac{\norm*{0.0-0.33}^2}{2(1)^2}\right) \\
                & = \exp\left(-\frac{0.33^2}{2}\right)                  \\
                & = 0.9460
  \end{split}
\end{equation}

\begin{equation}
  \Omega\gets
  \begin{bmatrix}
    1.0000 & 0.9460 & 0.8007 & 0.6065 \\
    0.9460 & 1.0000 & 0.9460 & 0.8007 \\
    0.8007 & 0.9460 & 1.0000 & 0.9460 \\
    0.6065 & 0.8007 & 0.9460 & 1.0000 \\
  \end{bmatrix}
\end{equation}

\begin{equation}
  \vb{I}\frac{1}{C}\to\vb{I}\frac{1}{5}\to
  \begin{bmatrix}
    0.2000 & 0.0000 & 0.0000 & 0.0000 \\
    0.0000 & 0.2000 & 0.0000 & 0.0000 \\
    0.0000 & 0.0000 & 0.2000 & 0.0000 \\
    0.0000 & 0.0000 & 0.0000 & 0.2000
  \end{bmatrix}
\end{equation}

\begin{equation}
  \Omega+\vb{I}\frac{1}{5}\to H\to
  \begin{bmatrix}
    1.2000 & 0.9460 & 0.8007 & 0.6065 \\
    0.9460 & 1.2000 & 0.9460 & 0.8007 \\
    0.8007 & 0.9460 & 1.2000 & 0.9460 \\
    0.6065 & 0.8007 & 0.9460 & 1.2000 \\
  \end{bmatrix}
\end{equation}

\begin{equation}
  A\to
  \begin{bmatrix}
    0.0000 & 1.0000 & 1.0000 & 1.0000 & 1.0000 \\
    1.0000 & 1.2000 & 0.9460 & 0.8007 & 0.6065 \\
    1.0000 & 0.9460 & 1.2000 & 0.9460 & 0.8007 \\
    1.0000 & 0.8007 & 0.9460 & 1.2000 & 0.9460 \\
    1.0000 & 0.6065 & 0.8007 & 0.9460 & 1.2000 \\
  \end{bmatrix}
\end{equation}

\begin{equation}
  B\to
  \begin{bmatrix}
    0.0000 \\
    4.0000 \\
    4.2222 \\
    4.8889 \\
    6.0000 \\
  \end{bmatrix}
\end{equation}

\begin{equation}
  A^{\dag}\to
  \begin{bmatrix}
    -0.8994 & 0.4348  & 0.0652  & 0.0652  & 0.4348  \\
    0.4348  & 2.0686  & -1.6490 & -0.5292 & 0.1096  \\
    0.0652  & -1.6490 & 3.3774  & -1.1992 & -0.5292 \\
    0.0652  & -0.5292 & -1.1992 & 3.3774  & -1.6490 \\
    0.4348  & 0.1096  & -0.5292 & -1.6490 & 2.0686  \\
  \end{bmatrix}
\end{equation}

\begin{equation}
  A^{\dag}B\to S\to
  \begin{bmatrix}
    4.9421  \\
    -0.6177 \\
    -1.3737 \\
    -0.5625 \\
    2.5538  \\
  \end{bmatrix}
\end{equation}

\begin{equation}
  b\to 4.9421
\end{equation}

\begin{equation}
  \alpha \to
  \begin{bmatrix}
    -0.6177 \\
    -1.3737 \\
    -0.5625 \\
    2.5538  \\
  \end{bmatrix}
\end{equation}

Prediction
\begin{equation}
  U\to
  \begin{bmatrix}
    0.3 \\
    0.2 \\
    0.5
  \end{bmatrix}
\end{equation}

\begin{equation}
  \Omega\to
  \begin{bmatrix}
    0.9560 & 0.9994 & 0.9350 & 0.7827 \\
    0.9802 & 0.9912 & 0.8968 & 0.7261 \\
    0.8825 & 0.9862 & 0.9862 & 0.8825 \\
  \end{bmatrix}
\end{equation}

\begin{equation}
  \Omega\alpha\to
  \begin{bmatrix}
    -0.4904 \\
    -0.6170 \\
    -0.2008 \\
  \end{bmatrix}
\end{equation}

\begin{equation}
  \Omega\alpha+b\vb{1}_{m}\to v\to
  \begin{bmatrix}
    4.4516 \\
    4.3251 \\
    4.7413 \\
  \end{bmatrix}
\end{equation}
Where \(\vb{1}_{m}\) is a vector of 1s with the length of \(U\).

% \section{PROGRAM SATU}
% \section{PROGRAM DUA}


% \chapter{GAMBAR-GAMBAR}

