% !TEX root = ./skripsi.tex
\chapter{CONCLUSIONS AND FUTURE WORK}\label{sec:conclusions_and_future_work}

\section{Conclusion}
\noindent Based on the results of the research into \MakeLowercase{\judul}, the following conclusions can be drawn:
\begin{enumerate}
    \item The proposed SpectralSVR model is able to learn the relationships defined by PDEs in the spectral domain as an operator regression model. This is done with a computational model that with four phases which are data preparation, data transformation into the spectral domain and appropriate scale, model training of the LSSVR, and finally model evaluation by transforming the spectral results into the physical domain.
    \item The proposed model's capabilities are verified using the derivative equation, the Burgers' equation, and the 2-meter temperature data from the ERA5 dataset. For more concrete verification, the model was tasked with and successfully solved problems based on exact analytical solutions to the derivative equation and the Burgers' equation. This revealed that for some configurations the model is able to successfully generalize to exact solutions. The model is able to work with some level of noise, with performance deteriorating with high levels of noise. Stiffness and nonlinearity can also be handled to an extent. And when solving IVPs, the model's solution converge and stays stable. The model's performance does suffer when the solution is a type of function that was not part of the training set.
    \item Interpretation of the proposed model can be done to a certain extent using two tools which are the correlation image and p-matrix developed by \textcite{ustunVisualisationInterpretationSupport2007}. The p-matrix specifically can be used to reveal how the input coefficients contribute to the output coefficients. This allows the user to see the learned relationship similar to the spectral method with its equations of the solution in terms of the coefficients of parameters.
\end{enumerate}
% \noindent Bab ini merupakan pamungkas berupa rincian rangkuman yang merupakan simpulan dari analisis yang telah dilakukan. Simpulan ini menyajikan sejumlah hal penting yang disampaikan secara ringkas, padat, dan utuh, yang menjawab tujuan penelitian yang dituliskan pada Bab Pendahuluan. Sangat mungkin   ada beberapa konsekuensi dan implikasi yang ditimbulkan dari simpulan yang dihasilkan, yang sepatutnya menjadi perhatian pada penelitian berikutnya. Judul bab dapat disesuaikan, namun umumnya ada \textit{Simpulan} yang memang mendominasi isi bab ini.

\section{Future Work}
During the work on this study, some limitations were imposed, and many new questions also went unanswered. Here, we would like to list some potential avenues for future work:
\begin{itemize}
    \item The synthetic data used in scenarios 1 and 2 were generated using MMS\@. While mathematically correct, this data is not realistic. Incorporating more realistic data from traditional numerical methods or other sources could enhance the model's ability to predict more realistic solutions. In addition, a more diverse set of training data would allow the model to better avoid errors like the prediction of the exact solution for the inviscid Burgers' equation.
    \item For the 2-meter temperature data, we only measured how well the model performed in predicting the temperature for six hours in the future. Other works may be interested in measuring how well auto-regression of weather using this model compares to the target data.
    \item Comparison of inverse prediction results with traditional methods, such as the Method of Regularization or iterative methods \autocite{groetschInverseProblemsMathematical1993,vogelComputationalMethodsInverse2002}.
    \item Measure and compare the proposed model against other machine learning-based methods.
    \item Different problems require other kinds of basis functions to be able to represent them well. For example, global weather prediction would benefit from using spherical harmonics \autocite{bonevSphericalFourierNeural2023}. Other problems such as seismic data would be better represented using wavelet bases \autocite{chakrabortyFrequencytimeDecompositionSeismic1995}.
    \item The use better metrics for measuring performance, such as the \enquote{standard error of regression} as it may be inadequate for nonlinear cases \autocite{spiessEvaluationR2Inadequate2010}.
    \item Our study focused on what the model learned and whether it is capable of doing operator regression. Further study into optimizing the model would be useful for problems like overcoming noise or stiffness.
    \item Further insight into how the model performs, specifically in conditions that may induce the double penalty phenomenon would need to be undertaken for optimizing the model. This would include analyzing the error by wave number. Based on the information of error by wave number, the model can be tuned accordingly. For example, using stronger regularization (a higher regularization parameter) for higher wave numbers.
    \item The current work is limited to smooth and dense functions. However, many data sources do not provide data with this property. Because of this, expanding the model's capability would require support for sparse data by employing data assimilation techniques, such as the Lomb-Scargle periodogram \autocite{vanderplasUnderstandingLombScargle2018}, and 4D-Var data assimilation \autocite{puNumericalWeatherPrediction2018,parkDataAssimilationAtmospheric2013}.
    \item Deeper understanding of the relationship learned by the model would need separation of terms in how the input coefficients contribute to the output coefficients. This would require better visualization and interpretation tool than what is provided by the p-matrix. Another way to extend understanding is research into the relationship between the contribution values of the p-matrix and the equations from the spectral method.
\end{itemize}
% \noindent Sejumlah ide yang muncul ketika melaksanakan penelitian TA dapat menjadi bahan atau topik untuk pekerjaan selanjutnya. Hal ini dapat berupa perbaikan atau ragam lain dari apa yang telah dilakukan sepanjang penelitian. Sub bab ini menjadi sumber informasi penting bagi, utamanya mahasiswa, yang akan melakukan penelitian lanjutan.