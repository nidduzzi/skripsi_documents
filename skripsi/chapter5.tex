% !TEX root = ./skripsi.tex
\chapter{SIMPULAN DAN SARAN}

\section{Simpulan}
\noindent Bab ini merupakan pamungkas berupa rincian rangkuman yang merupakan simpulan dari analisis yang telah dilakukan. Simpulan ini menyajikan sejumlah hal penting yang disampaikan secara ringkas, padat, dan utuh, yang menjawab tujuan penelitian yang dituliskan pada Bab Pendahuluan. Sangat mungkin   ada beberapa konsekuensi dan implikasi yang ditimbulkan dari simpulan yang dihasilkan, yang sepatutnya menjadi perhatian pada penelitian berikutnya. Judul bab dapat disesuaikan, namun umumnya ada \textit{Simpulan} yang memang mendominasi isi bab ini.

\section{Saran}
\noindent Sejumlah ide yang muncul ketika melaksanakan penelitian TA dapat menjadi bahan atau topik untuk pekerjaan selanjutnya. Hal ini dapat berupa perbaikan atau ragam lain dari apa yang telah dilakukan sepanjang penelitian. Sub bab ini menjadi sumber informasi penting bagi, utamanya mahasiswa, yang akan melakukan penelitian lanjutan.