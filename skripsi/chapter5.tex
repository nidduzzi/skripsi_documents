% !TEX root = ./skripsi.tex
\chapter{CONCLUSIONS AND FUTURE WORK}

\section{Conclusion}
\noindent Based on the results of the research into \MakeLowercase{\judul}, several conclusions can be drawn.
\begin{enumerate}
    \item The proposed SpectralSVR model is able to learn the relationships defined by PDEs. This is done with a computational model that with four phases which are data preparation, data transformation, model training, and finally model evaluation.
    \item Interpretation of the proposed model can be done to a certain extent using two tools which are the correlation image and p-matrix developed by \textcite{ustunVisualisationInterpretationSupport2007}.\@\item The proposed model is also capable in dealing with noise and partial data as shown in scenario 1 and 3.
    \item The model's capabilities are also verified using exact solution. This revealed that for some configurations the model is able to successfully generalize to exact solutions.
\end{enumerate}
% \noindent Bab ini merupakan pamungkas berupa rincian rangkuman yang merupakan simpulan dari analisis yang telah dilakukan. Simpulan ini menyajikan sejumlah hal penting yang disampaikan secara ringkas, padat, dan utuh, yang menjawab tujuan penelitian yang dituliskan pada Bab Pendahuluan. Sangat mungkin   ada beberapa konsekuensi dan implikasi yang ditimbulkan dari simpulan yang dihasilkan, yang sepatutnya menjadi perhatian pada penelitian berikutnya. Judul bab dapat disesuaikan, namun umumnya ada \textit{Simpulan} yang memang mendominasi isi bab ini.

\section{Future Work}
During the work on this study, some limitations were imposed, and many new questions also went unanswered. Here, we would like to list some potential avenues for future work:
\begin{itemize}
    % TODO: reword these points
    \item The synthetic data used in scenarios 1 and 2 were generated using MMS\@. While mathematically correct, this data is not realistic. Incorporating more realistic data from traditional numerical methods or other sources could enhance the model's ability to predict more realistic solutions.
    \item Test how the auto-regression of weather compares to the target data.
    \item Compare inverse prediction results with traditional derivatives, such as the spectral method or FDM\@.
    \item Compare the proposed method with other methods in comparable situations.
    \item Measure and compare the efficiency (time) of computing solutions against traditional and machine learning-based methods.
    \item Add more basis functions, such as spherical harmonics or wavelet bases.
    \item Incorporate more regression models, specifically more support vector regression models, to continue utilizing the correlation image and p-matrix.
    \item Use better metrics for measuring performance, such as the \enquote{standard error of regression}.
    \item Analyze the performance with different hyperparameters or data sizes and present the results in charts.
    \item Perform an analysis of the error by wave number.
    \item Solve the problem of double penalties that cause the model to favor smooth solutions \autocite{brownForecastsSpatialFields2011,NIPS2017_44a2e080}. Try using stronger regularization (a higher regularization parameter) for higher wave numbers.
    \item Add support for sparse data by employing data assimilation techniques, such as the Lomb-Scargle periodogram \autocite{vanderplasUnderstandingLombScargle2018}, and 4D-Var data assimilation \autocite{puNumericalWeatherPrediction2018,parkDataAssimilationAtmospheric2013}.
    \item Study the contribution patterns for different terms in a PDE using the p-matrix or other methods.
\end{itemize}
% \noindent Sejumlah ide yang muncul ketika melaksanakan penelitian TA dapat menjadi bahan atau topik untuk pekerjaan selanjutnya. Hal ini dapat berupa perbaikan atau ragam lain dari apa yang telah dilakukan sepanjang penelitian. Sub bab ini menjadi sumber informasi penting bagi, utamanya mahasiswa, yang akan melakukan penelitian lanjutan.