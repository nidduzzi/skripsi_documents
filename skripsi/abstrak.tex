% !TEX root = ./skripsi.tex
\clearpage
\phantomsection{}
\addcontentsline{toc}{chapter}{ABSTRAK}
\begin{center}
\textbf{\large  ABSTRAK}\\[0.5cm]
\textbf{\large Aproksimasi Solusi \textit{Govering Equations} dengan \emph{Fourier Transform} dan \textit{Support Vector Machine}}\\[0.5cm]
\textbf{Oleh}\\
\textbf{\penulis}\\
\textbf{NIM:\@\nim}\\[2em]
\end{center}

\noindent Model numerik sebuah sistem merupakan bagian penting dari ilmu pengetahuan dan \textit{engineering}. Penggunaan \textit{Machine Learning} (ML) dalam ruang ini untuk pembelajaran operator menyediakan alternatif sebagai \textit{data-driven surrogates}.\ \textit{Fourier Transform} menyediakan komponen kunci untuk mempelajari hubungan antara suatu fungsi dan turunannya. Berdasarkan \textit{Spectral Neural Operators (SNO)}, kami mengusulkan kerangka kerja berbasis \textit{Support Vector Machine (SVM)} untuk mempelajari persamaan dasar yang mengatur suatu sistem berdasarkan data. Kami mempelajari kelayakan dan interpretabilitas kerangka kerja yang diusulkan pada persamaan turunan dan persamaan Burgers. Model ini mampu belajar dari data acak yang secara matematis benar dan sebagian mampu melakukan generalisasi ke solusi eksak dari persamaan Burgers. Model yang dipelajari diinterpretasikan dan diverifikasi untuk mempelajari kontribusi yang benar dari koefisien fungsi masukan terhadap koefisien fungsi keluaran. Model yang diajukan menyelesaikan solusi \textit{Burgers' equation} hingga 33 kali lebih cepat daripada metode numerik tradisional lsoda.
% \noindent Abstrak merupakan penjelasan singkat dan padat tentang pekerjaan dan hasil penelitian TA, yang dituliskan secara teknis. Abstrak memiliki karakter tegas dan komprehensif, dan hanya dapat dituliskan setelah pekerjaan penelitian telah mencapai tahap tertentu, dan karenanya ada hasil penelitian yang dapat dilaporkan. Abstrak ditulis menjelang akhir penyelesaian penulisan buku TA.

% Secara umum, abstrak memuat beberapa komponen penting, yaitu: konteks atau cakupan pekerjaan penelitian, tujuan penelitian, metodologi yang digunakan selama penelitian, hasil-hasil penting yang dapat ditambahkan dengan implikasinya, dan simpulan dari penelitian. Dengan demikian, suatu abstrak tidak dapat dituliskan apabila penelitian belum mencapai hasil tertentu, apalagi kalau penelitiannya pun belum dilakukan.

% Panjang abstrak sebaiknya dicukupkan dalam satu halaman, termasuk kata kunci. Tiga kata kunci dipandang cukup, yang masing-masingnya memuat paduan kata utama, yang dapat merepresentasikan isi Abstrak. Halaman Abstrak tidak memuat informasi judul dan penulis, sehingga tidak secara langsung dapat digunakan sebagai lembaran Abstrak Sidang TA yang disediakan untuk hadirin, yang memerlukan tambahan (sekurangnya) dua informasi tersebut.


\noindent Kata kunci: \katakunci{}
