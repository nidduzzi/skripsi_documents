% !TEX root = ./technical_doc.tex
\chapter{REQUIREMENTS}
The requirements of the software can be divided into two main categories of functional and non-functional requirements. The functional requirements of the software are:\nopagebreak
\begin{itemize}
    \item The ability to transform between discrete function values and the coefficients of their Fourier series approximation.
    \item The ability to visualize functions in their coefficient representation
    \item The ability to represent differential equations using Fourier series and the coefficient relation between parameters and solution.
    \item Generation of random functions.
    \item Generate parameters and solutions for the derivative equation and Burgers' equation.
    \item A way to represent complex numbers as a pair of real numbers.
    \item An LSSVR model for regressing the coefficient relationship of parameter and solutions.
    \item A way to normalize inputs to the LSSVR.
    \item A way to measure the accuracy and efficiency of the model.
    \item A way to interpret the coefficient relationship learned by the model.
    \item The ability to use the tools in a Jupyter Notebooks or Python scripts.
\end{itemize}

\noindent The non-functional requirements for the software are:\nopagebreak
\begin{itemize}
    \item Working memory footprint that fits within 16 GB\@.
    \item Efficient and fast performance.
    \item Reliability and the consistency.
    \item Extendable to other problems.
\end{itemize}
\filbreak
\noindent The information requirements of the software are:
\begin{itemize}
    \item Reference of derivative and Burgers' equations.
    \item Mathematical formulation of LSSVR.
    \item Mathematical formulation of the Discrete Fourier Transform (DFT) and properties of the basis and its coefficients.
\end{itemize}

\noindent The external interface requirements are:\nopagebreak
\begin{itemize}
    \item Data compatibility with Zarr or other data formats.
\end{itemize}