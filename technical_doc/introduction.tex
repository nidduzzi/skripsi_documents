% !TEX root = ./technical_doc.tex
\chapter{INTRODUCTION}

\section{Purpose}

The purpose of this technical document is to provide a comprehensive, interpretable guide for all stakeholders to understand the library. The requirements that were gathered will first be presented. Then there will be an explanation of the architecture and designs that will go towards fulfilling the requirements and constraints. This includes the different modules and their roles in performing operator learning in the Fourier domain using LSSVR\@. The designs will also discuss the way the modules will interface with external stimuli. From the designs, the implementation will be tested on its adherence to restrictions and the designs as a whole.

\section{Scope}
This document will cover the requirements, design and analysis, implementation, and testing of the library. The library is a software product developed with the purpose of providing functionality relevant to our proposed SpectralSVR operator learning approach.

\section{Outline}
This document will be organized as follows:
\begin{enumerate}
    \item Introduction: This chapter provides context to the purpose of this document. This chapter also describes the scope and structure of this document.
    \item Requirements: This chapter explains the software requirements, including functional requirements, non-functional requirements, information requirements, user characteristics, and external interface requirements.
    \item Use Case: This chapter explains the model used based on the results of the software requirements analysis. The model created consists of a rich picture diagram that visualize the use case for the software product.
    \item Design: This chapter explains the software design using Sequence Diagram and module descriptions.
    \item Implementation: This chapter explains the implementation of the software in a specific programming language.
    \item Testing: This chapter contains the process of testing the software using specific test cases. The current results of said test cases are also presented.
\end{enumerate}
