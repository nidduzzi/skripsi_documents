% !TEX root = ./technical_doc.tex
\chapter{INTRODUCTION}

\section{Purpose}

This document is an example of a software manual created using \LaTeX\ and Overleaf. It provides a nice clean format for producing this type of technical document. 

Simply start writing your document and use the Recompile button to view the updated PDF preview. Examples of commonly used commands and features can be found in the sections below, to help you get started.

To change the logo shown on the front page, simply upload your own logo image file (via the file-tree menu) to replace the default logo.pdf file. 

Once you're familiar with the editor, you can find various project settings in the Overleaf menu, accessed via the button in the very top left of the editor. To view tutorials, user guides, and further documentation, please visit our \url{https://www.overleaf.com/learn}. If you haven't used \LaTeX\ before, our \url{https://www.overleaf.com/learn/latex/Learn_LaTeX_in_30_minutes} tutorial is an excellent place to start.

For large documents, it can be useful to split the document into multiple source files.
Double-clicking on text in Overleaf's PDF preview will take you directly to the corresponding code in the editor.
