% !TEX root = ./article.tex
\documentclass[preprint,12pt,times,authoryear]{elsarticle}
\usepackage{geometry}
\geometry{verbose,tmargin=3cm,bmargin=3cm,lmargin=4cm,rmargin=3cm}
\usepackage[pdftex,bookmarks=true,unicode=true,pdfusetitle,bookmarksnumbered=true,bookmarksopen=true,breaklinks=true,pdfborder={0 0 1},backref=true,colorlinks=false]{hyperref}

%%%% identitas (kalo file pdf nya di klik kanan, info..)
\hypersetup{pdfauthor={\penulis},
  pdfsubject={SKRIPSI},
  pdftitle={\judul},
  pdfkeywords={\keywords}}

\usepackage[htt]{hyphenat}
\usepackage{listing}
\usepackage{graphicx}      %menghandle gambar
\usepackage{tabularx}
\usepackage{float}
\usepackage[hang,nooneline,scriptsize,md]{subfigure}
\usepackage{epsfig}
\usepackage[font=small,labelfont=bf]{caption}
\usepackage[onehalfspacing]{setspace}      %buat kombinasi spacing
\usepackage{indentfirst}
%\setlength{\parindent}{0.6cm}
\usepackage{parskip}
\usepackage[titletoc]{appendix}
\usepackage{cite}          % buat cite pake BibTex
\bibliographystyle{apalike}
% \usepackage[bahasa]{babel} %plugin bahasa 
\usepackage[american]{babel} %plugin english  

\usepackage[utf8]{inputenc}
\usepackage[T1]{fontenc}   % biar bisa kombinasi font
\usepackage{mathptmx}
\usepackage{amssymb}
\usepackage{amsmath}
\usepackage{mathrsfs}
\usepackage{dsfont}
\usepackage{listings}
\usepackage{relsize}
\usepackage{physics}
\usepackage{optidef}  % for writing optimization problems easier

%%% pengaturan global
\graphicspath{{gambar/}}   % letak direktori penyimpanan gambar
%\usepackage[authoryear,round]{natbib}

%\citestyle{nature}


%\usepackage[fixlanguage]{babelbib}
%\selectbiblanguage{bahasa}

\pagestyle{plain}
\setlength{\parindent}{20pt}

%format judul
\usepackage{titlesec}
\titleformat{\chapter}[display]
{\bfseries\Large}
{%
  % \vskip-5em
  \filcenter
  \LARGE\MakeUppercase \chaptertitlename\
  \LARGE\thechapter}
{0mm}
{\filcenter}
\titleformat*{\section}{\bfseries\large}
\titleformat*{\subsection}{\bfseries}
\titlespacing*{\chapter}{0pt}{-1.5\baselineskip}{1em}

%menambahkan titik titik di daftar isi, gambar, tabel
\usepackage[subfigure]{tocloft}
\renewcommand{\cftpartleader}{\dotfill}
\renewcommand{\cftpartafterpnum}{\cftparfillskip}
\renewcommand{\cftchapleader}{\dotfill}
\renewcommand{\cftsecleader}{\dotfill}
\renewcommand{\cftsubsecleader}{\dotfill}

%tambah "bab" x di daftar isi 
\renewcommand*\cftchappresnum{\MakeUppercase{bab}~}
\renewcommand\chaptername{BAB}
\settowidth{\cftchapnumwidth}{\cftchappresnum}
\renewcommand{\cftchapaftersnumb}{\quad}
\addtocontents{toc}{
  %\renewcommand\protect*\protect\cftchappresnum{\MakeUppercase{\chaptername}~}}
  \protect\renewcommand*\protect\cftchappresnum{\MakeUppercase{\chaptername}~}}

%Penomoran Bab berupa bilangan romawi
\renewcommand{\thechapter}{\Roman{chapter}}%
\renewcommand{\thesection}{\arabic{chapter}.\arabic{section}}%
\renewcommand{\thesubsection}{\arabic{chapter}.\arabic{section}.\arabic{subsection}}%
\renewcommand{\thesubsubsection}{\arabic{chapter}.\arabic{section}.\arabic{subsection}.\arabic{subsubsection}}%

%Penomoran gambar & table
\renewcommand{\thefigure}{\arabic{chapter}.\arabic{figure}}
\renewcommand{\thetable}{\arabic{chapter}.\arabic{table}}

%Penomoran equations
\renewcommand{\theequation}{\arabic{chapter}.\arabic{equation}}

%tambah "gambar" dan "table" di daftar gambar dan daftar tabel
\usepackage{chngcntr}
%\counterwithout{figure}{chapter}
\renewcommand{\cftfigpresnum}{\bfseries Gambar\ }
\renewcommand{\cfttabpresnum}{\bfseries Tabel\ }
\newlength{\mylenf}
\settowidth{\mylenf}{\cftfigpresnum}
\setlength{\cftfignumwidth}{\dimexpr\mylenf+2em}
\settowidth{\mylenf}{\cfttabpresnum}
\setlength{\cfttabnumwidth}{\dimexpr\mylenf+2em}
\makeatletter

%bikin listoftable tanpa judul, gara2 aneh
%\counterwithout{table}{chapter}
\newcommand\daftartabel{%
  \phantomsection
  \@starttoc{lof}%
  \bigskip
  \@starttoc{lot}}
\makeatother



%ilangin judul bibli 
\makeatletter
\renewenvironment{thebibliography}[1]{%
  \section*{\refname}%
  \@mkboth{\MakeUppercase\refname}{\MakeUppercase\refname}%
  \list{\@biblabel{\@arabic\c@enumiv}}%
  {\settowidth\labelwidth{\@biblabel{#1}}%
    \leftmargin\labelwidth
    \advance\leftmargin\labelsep
    \@openbib@code
    \usecounter{enumiv}%
    \let\p@enumiv\@empty
    \renewcommand\theenumiv{\@arabic\c@enumiv}}%
  \sloppy
  \clubpenalty4000
  \@clubpenalty \clubpenalty
  \widowpenalty4000%
  \sfcode`\.\@m}
{\def\@noitemerr
  {\@latex@warning{Empty `thebibliography' environment}}%
  \endlist}
\makeatother


%lampiran
\makeatletter
\newcommand\appendix@chapter[1]{%
  \refstepcounter{chapter}%
  \orig@chapter*{LAMPIRAN \@Alph\c@chapter \\  #1}\vspace{1.5em}%
  \addcontentsline{toc}{chapter}{LAMPIRAN \@Alph\c@chapter: #1}%
}
\let\orig@chapter\chapter
\g@addto@macro\appendix{\let\chapter\appendix@chapter}
\makeatother




\begin{document}
\begin{frontmatter}
  \title{\judul}
  % \author{
  %   Lala Septem Riza\\
  %   \texttt{lala.s.riza@upi.edu}
  %   \and
  %   Ahmad Izzuddin\\
  %   \texttt{ahmadizzuddin@upi.edu}
  \author[upi]{Lala Septem Riza}
  \ead{lala.s.riza@upi.edu}
  \author[upi]{Ahmad Izzuddin}
  \ead{ahmadizzuddin@upi.edu}

  \affiliation[upi]{organization={Department of Computer Science Education Universitas Pendidikan Indonesia},%Department and Organization
    addressline={Jl. Dr.\ Setiabudi No.229},
    postcode={40154},
    city={Bandung},
    country={Indonesia}}
  \begin{abstract}
    Numerical models of systems are a crucial part of science and engineering. The use of machine learning in this space for operator learning provides an alternative as data-driven surrogates. The Fourier Transform provides a key component for learning the relationship between a function and its derivatives. Building on Spectral Neural Operators (SNO), we propose a Support Vector Machine (SVM) based framework to learn the underlying governing equations of a system based on data. We study the viability and interpretability of the proposed framework on the derivative equation and the Burgers' equation. The model is able to learn from mathematically correct random data and is able to partially generalize to an exact solution of the Burgers' equation. The learned model is interpreted and verified to have learned the correct contributions of the input function coefficients to the output function coefficients.
  \end{abstract}
  \begin{keyword}
    %% keywords here, in the form: keyword \sep keyword
    Operator Learning \sep{} LSSVR \sep{} PDE
    % keyword one \sep keyword two
    %% PACS codes here, in the form: \PACS code \sep code
    % \PACS 0000 \sep 1111
    %% MSC codes here, in the form: \MSC code \sep code
    %% or \MSC[2008] code \sep code (2000 is the default)
    \MSC{} 35{-}04 \sep{} 68T99
  \end{keyword}

\end{frontmatter}
\section{Introduction}
Governing equations have been a critical part of the technological revolution. They help us understand better the systems that exist in our world \citep{braunDifferentialEquationsTheir1993}. Typically, governing equations are differential equations. For systems with many variables, they are described by Partial Differential Equations (PDE). For example, the spread of heat on a cooking surface is described by the heat equation, with the temperature as a function of time and space. Differential equations are used because often, the behavior systems are easier to describe in the way they change.

Typically, when modeling a system with PDEs, there are different problem types based on the conditions that are imposed on the solution. The main problem types are where the solutions are constrained by its boundaries. For equations with a temporal component, an initial value can be imposed as a restriction on solution's value at that initial point in time. This problem is called the Initial Value Problem (IVP). The Boundary Value Problem (BVP) on the other hand, defines the restrictions on the solution at the spatial boundary of the system being simulated, such as values or derivatives the solution must have at the boundary. The Initial-Boundary Value Problem (IBVP) combines both conditions imposed on the solution. These problems are considered part of the larger group of forward problems. This is because, the objective is to determine the system's response in terms of the solution towards causal factors, namely parameters and imposed conditions \citep{groetschInverseProblemsMathematical1993, vogelComputationalMethodsInverse2002}. The inverse problem, on the other hand, is the reverse process where parameters or conditions such as the initial value is the objective. Simply put, in inverse problems, the effects are used to compute the cause that led to it.

While some PDEs have analytical solutions, they are restrictive in ways such as the initial condition, or other parameters \citep{selvaduraiPartialDifferentialEquations2000,koprivaImplementingSpectralMethods2009, olverIntroductionPartialDifferential2014, schiesserNumericalMethodLines2012, wazwazPartialDifferentialEquations2010}. Because of this, modeling a system with differential equations usually involve the use of numerical methods. With their long history, there are now several well established general methods such as Finite Difference Method (FDM), Finite Element Method (FEM), and the Spectral Method. With mesh based methods such as FDM, evaluating the solutions at higher resolutions means either recomputing the solution with a higher resolution which increases the computational cost, or interpolation which can depend on the problem. For multiscale problems like Numerical Weather Prediction (NWP), the complex interactions are very difficult to model due to the vast differences in scale \citep{frankCharacterizingDynamicsMultiscale2024}. Also, the user needs to know the equations involved before modeling the system. Which may not be available in some fields like ecology \citep{holmesPartialDifferentialEquations1994,turchinDoesPopulationEcology2001} and epidemiology \citep{brauerMathematicalModelsEpidemiology2019}.

These challenges have spurred interest in modeling systems with the use of Machine Learning (ML) as surrogates. There are two types of modeling with ML models. The first kind is akin to function regression, where the model predicts solution values from coordinates and parameters. An example, \citet{raissiPhysicsinformedNeuralNetworks2019} proposed Physics-Informed Neural Networks (PINN). The specific implementation they used is a Feedforward Neural Network (FNN) as an approximation for the solution.  To enforce the PDE, the loss is computed with PDE residuals and other constraints. The partial derivatives for the residual are computed with ease using the automatic differentiation with respect to the input coordinates. Other works utilize convolutional neural networks (CNN) to compute the solution from input functions such as forcing terms or initial conditions. This approach generally means discretizing the functions on a grid and using these as training data. One study by \citet{wangPhysicsinformedDeepLearning2020} predicts turbulent flow using spatial and temporal decomposition and a specialized U-Net, an architecture based on CNNs, to predict the velocity field from the decomposition of the previous velocity field. Part of the loss function is a regularization term for zero divergence in the velocity field to enforce incompressible fluid flow. This term was calculated using finite differences since auto differentiation cannot be used without the coordinates in the model inputs. While the use of CNNs mean that discretization is implied, solutions of different initial conditions or parameter functions can be computed by inference and no retraining is required. This property is especially useful for many-query problems such as computing gradients for inverse problems.

% TODO: talk about the relationship of derivatives in the fourier domain with an example equation
This is where operator learning models that predicts basis function coefficients come in. Basis functions are collections of functions that share common properties and can be used to represent other functions in a linear combination. For example, the sine and cosines in the real Fourier series. This is the concept used by \citet{fanaskovSpectralNeuralOperators2023} for their proposed model, termed Spectral Neural Operator (SNO). The model Neural Network (NN) is trained on features of input function coefficients which are computed using Fourier or Chebyshev transforms and labels of output function coefficients using the same transforms. This separates the concern of discretization from the problem of learning the relationships enforced by some system between functions.\ \citet{du2024neural} extends the concept of mapping coefficients by proposing residuals in the spectral domain and leveraging Parseval's Identity to compute the spectral analog to the loss term in PINNs. This allows for self supervised learning in the spectral domain. The same benefits incorporating physics into PINNs also apply here without the pain points introduced by discretized model inputs and outputs. Another approach by \citet{luLearningNonlinearOperators2021}, essentially uses learned basis functions in the form of NNs. This has the benefit of the basis functions being custom-made to fit the problem and equations being modeled \citep{meurisMachinelearningbasedSpectralMethods2023}. These methods show promise and provide solutions to parameters via the relatively fast process of inference.

The use of NNs does come with its own downsides. First, the loss functions used in their optimization represents a landscape with many local minima. This problem becomes worse with the addition of PDE residual terms that cause disparities in the gradients of each individual loss term and \citep{rathoreChallengesTrainingPINNs2024,NEURIPS2021_df438e52,basirCriticalInvestigationFailure2022}. This has motivated the use of ML algorithms with a convex loss landscape for modeling. Support Vector Machines (SVM) are one such family of algorithms \citep{vapnikNatureStatisticalLearning2000}. The appeal of SVMs is the fact that the model is formulated as a quadratic programming problem. This means there are strong guarantees for convergence, generalization, and complexity. Another formulation called Least-Squares Support Vector Machines (LSSVM) reformulates the problem as a linear system \citep{suykensLeastSquaresSupport2005}. This leads to an easier problem that can be computed faster by well established algorithms like the many implementations of least squares solvers. Another advantage of the linear formulation is that this can be easily parallelized to exploit hardware like graphics processing units more widely known as GPUs in contrast to the commonly used Sequential Minimal Optimization (SMO) used for SVMs with quadratic objective functions.

An early work using SVMs to solve PDEs by \citet{youxiwuSVMSolvingForward2005} introduced a method for solving the forward problem of Electro-Impedance Tomography. This work solved for the mathematical model of EIT which is given by Maxwell's equations by modeling the trial function as using an \(\epsilon \)-SVR model. Another approach much more similar to PINNs was presented by \citet{mehrkanoonLearningSolutionsPartial2015}. The residual and initial/boundary conditions are imposed as equality constraints on an LSSVM objective function. A different study by \citet{leakeAnalyticallyEmbeddingDifferential2019}, the incorporation of physics into the model is done slightly differently by utilizing the theory of functional connections to directly embed constraints into the solution. This means that the proposed method would satisfy the boundary condition exactly. These approaches, however, only learn a function regression problem constrained by the PDE\@. Meaning they are also not practical for many-query problems.

In this paper, we propose an operator learning model based on basis functions and LSSVM for regression, named SpectralSVR\@. Here we will focus on demonstrating the model's capability, starting with a proof of concept with the simple derivative equation. And then, the nonlinear PDE solving capabilities will be shown by approximating solutions to the Burgers' equation.% Finally, we will also use the ERA5 dataset, specifically the could optimized version provided by WeatherBench 2 \citep{raspWeatherBench2Benchmark2023}. %TODO: determine if the inverse problem need to be included

\section{Methods}
This section describes the proposed method and the case studies we used.
\subsection{SpectralSVR}
% Model design
The design of the proposed model is built upon performing regression in the Fourier domain. This means that any function values that will be used by the model, should be transformed into their coefficients. And, any predictions from the model, which are coefficients, should be transformed back into function values. The training process is shown in \lccref{fig:spectralsvr_training}. The process starts with determining the feature and label functions. Typically, the feature functions would include the forcing term and other parameter functions such as the solution at the current time step. The label functions typically is the solution function or the solution function at the next time step. The Fourier coefficients are then computed using the DFT\@. The coefficients are then flattened for each sample. A scaling function is fitted on the flattened feature coefficients and the feature coefficients are then scaled using the fitted scaling function. Finally, Using the scaled feature and label coefficients, the training function computes the parameters of the model.

\begin{figure}[H]
  \centering
  \tikzfig{figures/training_spectralsvr}
  \caption{SpectralSVR training diagram}\label{fig:spectralsvr_training}
\end{figure}

The prediction process is shown in \lccref{fig:spectralsvr_prediction}. The process starts by transforming the feature functions into their coefficients with the DFT\@. These coefficients are then scaled with the scaling function fitted during training. The scaled feature coefficients are then used with the learned parameters to produce the predicted coefficients using the LSSVR prediction function. Finally, the inverse DFT is computed on the predicted coefficients to obtain the predicted function value.
\begin{figure}[H]
  \centering
  \tikzfig{figures/prediction_spectralsvr}
  \caption{SpectralSVR prediction diagram}\label{fig:spectralsvr_prediction}
\end{figure}

To interpret the relationship learned by the LSSVR between the feature and label coefficients, we use two tools called the correlation image and p-matrix introduced by \citet{ustunVisualisationInterpretationSupport2007}.

\subsection{Case Studies}
% case studies
% antiderivative
The model's ability will be verified and demonstrated using two equations. The data will be generated using the method of manufactured solutions \citep{roacheCodeVerificationMethod2002,salariCodeVerificationMethod2000,vedovotoApplicationMethodManufactured2011}. First, the model will be tasked to learn the basic relationship defined by the simple derivative \lccref{eq:fourier_series_derivative}. The solution functions are generated as Fourier polynomials with random coefficients sampled from a normal distribution. The generated solution function takes the form shown in \lccref{eq:fourier_speed}. The coefficients of the derivative function in \lccref{eq:fourier_acceleration} is computed using \lccref{eq:derivative_coeff}.

\begin{align}
  \dv{u\left( t \right)}{t}                                        & = a\left(t\right) \label{eq:fourier_series_derivative}                                                            \\
  u\left( t \right)                                                & = \sum_{k} \hat{u}_k e^{2\pi ikt/T} \label{eq:fourier_speed}                                                      \\
  a\left( t \right)                                                & = \sum_{k} \hat{a}_k e^{2\pi ikt/T} \label{eq:fourier_acceleration}                                               \\
  \sum_{k} \hat{u}_k\times \left( 2\pi ik/T \right) e^{2\pi ikt/T} & = \sum_{k} \hat{a}_k e^{2\pi ikt/T} \label{eq:example_spectral_method_fourier_substituted}                        \\
  \hat{a}_k                                                        & = \hat{u}_k\left(2\pi ik/T \right)                                                    \label{eq:derivative_coeff}
\end{align}

In total, we generate \num{5000} unique functions with \num{50} complex coefficients each. We also add noise to the function values at three levels of the average function standard deviation. The entire set has three versions with different levels of noise which are 5\%, 10\% and 50\%.

% burgers
The second problem the model will be presented with is solving the Burgers' equation by stepping through time. Because of the nonlinearity of this equation, it is often used to test the capability of numerical solvers \citep{woodExactSolutionBurgers2006,wazwazPartialDifferentialEquations2010,koprivaImplementingSpectralMethods2009}. The formulation of the Burgers' equation we use is the forced viscous Burgers' equation in 1-dimension as shown in \lccref{eq:forced_viscous_burgers} where the viscosity is \(\nu{}\).
\begin{align}
  \pdv{u}{t}+u\pdv{u}{x}-\nu\pdv[2]{u}{x} & = f \label{eq:forced_viscous_burgers}
\end{align}

Data generation is done similarly to the derivative equation. With the solution as in \lccref{eq:fourier_field} and the forcing term \lccref{eq:fourier_force}, the equation for the forcing term is shown in \lccref{eq:forced_viscous_burgers_coeff}. The nonlinear term is approximated, for efficiency, in the physical domain and the coefficients subsequently computed as \(\hat{uu}_k\).
\begin{align}
  u\left(x, t \right) & = \sum_{k_{x}} \sum_{k_{t}} \hat{u}_k e^{2\pi i(k_{x}x/L+k_{t}t/T)} \label{eq:fourier_field}                                         \\
  f\left(x, t \right) & = \sum_{k_{x}}\sum_{k_{t}} \hat{f}_k e^{2\pi i(k_{x}x/L+k_{t}t/T)} \label{eq:fourier_force}                                          \\
  \hat{f}_k           & = (2\pi i k_{t}/T) \hat{u}_k + (2\pi i k_{x}/L)\hat{uu}_k - \nu{(2\pi i k_{x}/L)}^2\hat{u}_k \label{eq:forced_viscous_burgers_coeff}
\end{align}

The functions generated for the Burgers' equation will use three different values of \(\nu{}\) to in order to be able to analyze how the model behaves with varying levels of stiffness and nonlinearity. The values for viscosity are \num{0.0} for the inviscid equation, \num{0.01}, and \num{0.1} for a more stiff PDE\@. For each viscosity level, we generate \num{500} unique solution functions and compute the corresponding forcing term using \lccref{eq:forced_viscous_burgers_coeff}. Each function has 8 modes in space and 8 in time.

% WeatherBench 2
% The final dataset we use is the ERA5 dataset. The cloud optimized version from WeatherBench 2 we use will have a grid size of 64 in longitude by 32 in latitude. The model will attempt to partially model the complex weather data using only the 2-meter temperature variable. For the reanalysis dataset, we limit the time range to the years 1995{-}2013 for training and 2020{-}2022 for testing. The specific time ranges for training are chosen to balance the diversity within the data and capacity of the machine the processing will be done on. We also use the climatology provided by WeatherBench 2 which is the averages for hours 0, 6, 12, and 18 for each day of the year from 1990 until 2019.

\section{Results and Discussion}
In this section, we will present the results of the experiments on each equation and the ERA5 dataset. The metrics we will show in this section are evaluated between the function values evaluated from the predicted coefficients and the function values of the test label coefficients.

\subsection{Scenario 1: Derivative Equation}
For the derivative equation, metrics between the noisy targets and predictions are shown in \lccref{table:scenario_1_function_metrics}. The metrics shows that the model is able to generalize from the training data and learn the simple linear antiderivative operator. Focusing on the R\textsuperscript{2} scores, we can see that the model is off by 0.04 to the perfect score of 1.0 for the lowest noise level. While increasing noise levels does degrade the performance, this is partly due to the target values also having been perturbed.

For the function values of the 50\% noise dataset, the values range on average from \num{-2.5} to \num{2.49}. This was approximated since there is no direct way to compute the amplitude of the function from just the coefficients. Here we used the inverse transform and extracted the maximum and minimum values of each function from the discrete values. Since the RMSE in \lccref{table:scenario_1_function_metrics} for the 50\% noise dataset is \num{0.63}, the error ratio comes to \num{0.126}. One side note about \lccref{table:scenario_1_function_metrics} is that the kernel scaling hyperparameter is the same for all noise levels because of the scaling which is 10.
\begin{table}[H]
  \caption{Performance metrics of function value from evaluated coefficient prediction in scenario 1 by noise level.}\label{table:scenario_1_function_metrics}
  \centering
  \begin{tabular}{lrrrrrrrr}
    \toprule
    Noise level & MSE  & RMSE & MAE  & R\textsuperscript{2} & sMAPE \\
    \midrule
    5\%         & 0.03 & 0.18 & 0.15 & 0.97                 & 0.39  \\
    10\%        & 0.08 & 0.29 & 0.23 & 0.92                 & 0.55  \\
    50\%        & 0.62 & 0.79 & 0.63 & 0.49                 & 1.05  \\
    \bottomrule
  \end{tabular}
\end{table}

The metrics between the predictions and the noise-free version of the target function values can be seen in \lccref{table:scenario_1_clean_function_metrics}. The metrics show that the model predictions are slightly closer to the unperturbed functions compared to the perturbed versions. This can be seen as the influence of the independent measurement noise in the perturbed targets being removed from the testing metrics. The metrics show that the model performs slightly better with more pronounced differences for the higher noise levels.
\begin{table}[H]
  \caption{Performance metrics of function value from evaluated coefficient prediction compared to unperturbed targets in scenario 1 by noise level.}\label{table:scenario_1_clean_function_metrics}
  \centering
  \begin{tabular}{lrrrrrrr}
    \toprule
    Noise level & MSE  & RMSE & MAE  & R\textsuperscript{2} & sMAPE \\
    \midrule
    5\%         & 0.03 & 0.18 & 0.14 & 0.97                 & 0.38  \\
    10\%        & 0.07 & 0.27 & 0.21 & 0.93                 & 0.53  \\
    50\%        & 0.37 & 0.61 & 0.49 & 0.62                 & 0.94  \\
    \bottomrule
  \end{tabular}
\end{table}

The next part of this scenario is the results from predicting the exact solution of \(f(t)=2\pi\cos(2\pi t)\) which is \(u(t)=\sin(2\pi t)\). The function values will be computed, and the values put through the preprocessing steps as all other function values. The results are presented in \lccrefs{table:scenario_1_exact_function_metrics}. For 5\% and 10\% noise levels, the model shows that it has learned the relation. However, the 50\% noise level, the model has been unable to predict the results well enough. The marked difference in error across all metrics between the 50\% and the lower noise levels is more apparent for this specific exact problem compared to the test sets. A clear picture of this can be seen when we compare the sMAPE metric. For the test set, the predictions on average results in a value of 1, however, for this exact problem the sMAPE value is 0.23 to 1.65.
\begin{table}[H]
  \caption{Performance metrics of evaluated function values of coefficient prediction of exact antiderivative in scenario 1 by noise level.}\label{table:scenario_1_exact_function_metrics}
  \centering
  \begin{tabular}{lrrrr}
    \toprule
    Noise level & MSE  & RMSE & MAE  & sMAPE \\
    \midrule
    5\%         & 0.00 & 0.06 & 0.05 & 0.23  \\
    10\%        & 0.03 & 0.18 & 0.15 & 0.36  \\
    50\%        & 0.41 & 0.64 & 0.58 & 1.65  \\
    \bottomrule
  \end{tabular}
\end{table}

To test the generalization capabilities of the model, we use the sine function with a period of one shown in \lccref{eq:sine_function}. The derivative is simply \lccref{eq:sine_derivative}. The functions are discretized and noise is added. Finally, the values go through the same transformations to become feature and target coefficients.
\begin{align}
  f(t) & = 2\pi \cos\left(2\pi t\right)\label{eq:sine_derivative} \\
  u(t) & = \sin\left(2\pi t\right)\label{eq:sine_function}
\end{align}

The predictions of the exact equation is shown in \lccref{fig:antiderivative_exact}. Visually, we can see that as the noise level increases, the model predictions become worse. Another observation is how the higher the noise level, the more the predicted functions become closer to zero. This is explained by the double penalty phenomenon \citep{lledoScaledependentVerificationPrecipitation2023}. The loss function used for LSSVR penalizes the model for predicting a non-zero value that turns out to be wrong compared to predicting a zero value. This results in a model with predictions that are close to the mean of the training data.
\begin{figure}[H]
  \centering
  \begin{subfigure}{\linewidth}
    \begin{adjustbox}{width=\linewidth}
      \begingroup%
\makeatletter%
\begin{pgfpicture}%
\pgfpathrectangle{\pgfpointorigin}{\pgfqpoint{8.000000in}{2.000000in}}%
\pgfusepath{use as bounding box, clip}%
\begin{pgfscope}%
\pgfsetbuttcap%
\pgfsetmiterjoin%
\pgfsetlinewidth{0.000000pt}%
\definecolor{currentstroke}{rgb}{0.000000,0.000000,0.000000}%
\pgfsetstrokecolor{currentstroke}%
\pgfsetstrokeopacity{0.000000}%
\pgfsetdash{}{0pt}%
\pgfpathmoveto{\pgfqpoint{0.000000in}{0.000000in}}%
\pgfpathlineto{\pgfqpoint{8.000000in}{0.000000in}}%
\pgfpathlineto{\pgfqpoint{8.000000in}{2.000000in}}%
\pgfpathlineto{\pgfqpoint{0.000000in}{2.000000in}}%
\pgfpathlineto{\pgfqpoint{0.000000in}{0.000000in}}%
\pgfpathclose%
\pgfusepath{}%
\end{pgfscope}%
\begin{pgfscope}%
\pgfsetbuttcap%
\pgfsetmiterjoin%
\pgfsetlinewidth{0.000000pt}%
\definecolor{currentstroke}{rgb}{0.000000,0.000000,0.000000}%
\pgfsetstrokecolor{currentstroke}%
\pgfsetstrokeopacity{0.000000}%
\pgfsetdash{}{0pt}%
\pgfpathmoveto{\pgfqpoint{0.697450in}{0.517039in}}%
\pgfpathlineto{\pgfqpoint{7.958330in}{0.517039in}}%
\pgfpathlineto{\pgfqpoint{7.958330in}{1.958330in}}%
\pgfpathlineto{\pgfqpoint{0.697450in}{1.958330in}}%
\pgfpathlineto{\pgfqpoint{0.697450in}{0.517039in}}%
\pgfpathclose%
\pgfusepath{}%
\end{pgfscope}%
\begin{pgfscope}%
\pgfsetbuttcap%
\pgfsetroundjoin%
\definecolor{currentfill}{rgb}{0.000000,0.000000,0.000000}%
\pgfsetfillcolor{currentfill}%
\pgfsetlinewidth{0.803000pt}%
\definecolor{currentstroke}{rgb}{0.000000,0.000000,0.000000}%
\pgfsetstrokecolor{currentstroke}%
\pgfsetdash{}{0pt}%
\pgfsys@defobject{currentmarker}{\pgfqpoint{0.000000in}{-0.048611in}}{\pgfqpoint{0.000000in}{0.000000in}}{%
\pgfpathmoveto{\pgfqpoint{0.000000in}{0.000000in}}%
\pgfpathlineto{\pgfqpoint{0.000000in}{-0.048611in}}%
\pgfusepath{stroke,fill}%
}%
\begin{pgfscope}%
\pgfsys@transformshift{1.027490in}{0.517039in}%
\pgfsys@useobject{currentmarker}{}%
\end{pgfscope}%
\end{pgfscope}%
\begin{pgfscope}%
\definecolor{textcolor}{rgb}{0.000000,0.000000,0.000000}%
\pgfsetstrokecolor{textcolor}%
\pgfsetfillcolor{textcolor}%
\pgftext[x=1.027490in,y=0.419816in,,top]{\color{textcolor}{\rmfamily\fontsize{12.000000}{14.400000}\selectfont\catcode`\^=\active\def^{\ifmmode\sp\else\^{}\fi}\catcode`\%=\active\def%{\%}0.00}}%
\end{pgfscope}%
\begin{pgfscope}%
\pgfsetbuttcap%
\pgfsetroundjoin%
\definecolor{currentfill}{rgb}{0.000000,0.000000,0.000000}%
\pgfsetfillcolor{currentfill}%
\pgfsetlinewidth{0.803000pt}%
\definecolor{currentstroke}{rgb}{0.000000,0.000000,0.000000}%
\pgfsetstrokecolor{currentstroke}%
\pgfsetdash{}{0pt}%
\pgfsys@defobject{currentmarker}{\pgfqpoint{0.000000in}{-0.048611in}}{\pgfqpoint{0.000000in}{0.000000in}}{%
\pgfpathmoveto{\pgfqpoint{0.000000in}{0.000000in}}%
\pgfpathlineto{\pgfqpoint{0.000000in}{-0.048611in}}%
\pgfusepath{stroke,fill}%
}%
\begin{pgfscope}%
\pgfsys@transformshift{1.852590in}{0.517039in}%
\pgfsys@useobject{currentmarker}{}%
\end{pgfscope}%
\end{pgfscope}%
\begin{pgfscope}%
\definecolor{textcolor}{rgb}{0.000000,0.000000,0.000000}%
\pgfsetstrokecolor{textcolor}%
\pgfsetfillcolor{textcolor}%
\pgftext[x=1.852590in,y=0.419816in,,top]{\color{textcolor}{\rmfamily\fontsize{12.000000}{14.400000}\selectfont\catcode`\^=\active\def^{\ifmmode\sp\else\^{}\fi}\catcode`\%=\active\def%{\%}0.25}}%
\end{pgfscope}%
\begin{pgfscope}%
\pgfsetbuttcap%
\pgfsetroundjoin%
\definecolor{currentfill}{rgb}{0.000000,0.000000,0.000000}%
\pgfsetfillcolor{currentfill}%
\pgfsetlinewidth{0.803000pt}%
\definecolor{currentstroke}{rgb}{0.000000,0.000000,0.000000}%
\pgfsetstrokecolor{currentstroke}%
\pgfsetdash{}{0pt}%
\pgfsys@defobject{currentmarker}{\pgfqpoint{0.000000in}{-0.048611in}}{\pgfqpoint{0.000000in}{0.000000in}}{%
\pgfpathmoveto{\pgfqpoint{0.000000in}{0.000000in}}%
\pgfpathlineto{\pgfqpoint{0.000000in}{-0.048611in}}%
\pgfusepath{stroke,fill}%
}%
\begin{pgfscope}%
\pgfsys@transformshift{2.677690in}{0.517039in}%
\pgfsys@useobject{currentmarker}{}%
\end{pgfscope}%
\end{pgfscope}%
\begin{pgfscope}%
\definecolor{textcolor}{rgb}{0.000000,0.000000,0.000000}%
\pgfsetstrokecolor{textcolor}%
\pgfsetfillcolor{textcolor}%
\pgftext[x=2.677690in,y=0.419816in,,top]{\color{textcolor}{\rmfamily\fontsize{12.000000}{14.400000}\selectfont\catcode`\^=\active\def^{\ifmmode\sp\else\^{}\fi}\catcode`\%=\active\def%{\%}0.50}}%
\end{pgfscope}%
\begin{pgfscope}%
\pgfsetbuttcap%
\pgfsetroundjoin%
\definecolor{currentfill}{rgb}{0.000000,0.000000,0.000000}%
\pgfsetfillcolor{currentfill}%
\pgfsetlinewidth{0.803000pt}%
\definecolor{currentstroke}{rgb}{0.000000,0.000000,0.000000}%
\pgfsetstrokecolor{currentstroke}%
\pgfsetdash{}{0pt}%
\pgfsys@defobject{currentmarker}{\pgfqpoint{0.000000in}{-0.048611in}}{\pgfqpoint{0.000000in}{0.000000in}}{%
\pgfpathmoveto{\pgfqpoint{0.000000in}{0.000000in}}%
\pgfpathlineto{\pgfqpoint{0.000000in}{-0.048611in}}%
\pgfusepath{stroke,fill}%
}%
\begin{pgfscope}%
\pgfsys@transformshift{3.502790in}{0.517039in}%
\pgfsys@useobject{currentmarker}{}%
\end{pgfscope}%
\end{pgfscope}%
\begin{pgfscope}%
\definecolor{textcolor}{rgb}{0.000000,0.000000,0.000000}%
\pgfsetstrokecolor{textcolor}%
\pgfsetfillcolor{textcolor}%
\pgftext[x=3.502790in,y=0.419816in,,top]{\color{textcolor}{\rmfamily\fontsize{12.000000}{14.400000}\selectfont\catcode`\^=\active\def^{\ifmmode\sp\else\^{}\fi}\catcode`\%=\active\def%{\%}0.75}}%
\end{pgfscope}%
\begin{pgfscope}%
\pgfsetbuttcap%
\pgfsetroundjoin%
\definecolor{currentfill}{rgb}{0.000000,0.000000,0.000000}%
\pgfsetfillcolor{currentfill}%
\pgfsetlinewidth{0.803000pt}%
\definecolor{currentstroke}{rgb}{0.000000,0.000000,0.000000}%
\pgfsetstrokecolor{currentstroke}%
\pgfsetdash{}{0pt}%
\pgfsys@defobject{currentmarker}{\pgfqpoint{0.000000in}{-0.048611in}}{\pgfqpoint{0.000000in}{0.000000in}}{%
\pgfpathmoveto{\pgfqpoint{0.000000in}{0.000000in}}%
\pgfpathlineto{\pgfqpoint{0.000000in}{-0.048611in}}%
\pgfusepath{stroke,fill}%
}%
\begin{pgfscope}%
\pgfsys@transformshift{4.327890in}{0.517039in}%
\pgfsys@useobject{currentmarker}{}%
\end{pgfscope}%
\end{pgfscope}%
\begin{pgfscope}%
\definecolor{textcolor}{rgb}{0.000000,0.000000,0.000000}%
\pgfsetstrokecolor{textcolor}%
\pgfsetfillcolor{textcolor}%
\pgftext[x=4.327890in,y=0.419816in,,top]{\color{textcolor}{\rmfamily\fontsize{12.000000}{14.400000}\selectfont\catcode`\^=\active\def^{\ifmmode\sp\else\^{}\fi}\catcode`\%=\active\def%{\%}1.00}}%
\end{pgfscope}%
\begin{pgfscope}%
\pgfsetbuttcap%
\pgfsetroundjoin%
\definecolor{currentfill}{rgb}{0.000000,0.000000,0.000000}%
\pgfsetfillcolor{currentfill}%
\pgfsetlinewidth{0.803000pt}%
\definecolor{currentstroke}{rgb}{0.000000,0.000000,0.000000}%
\pgfsetstrokecolor{currentstroke}%
\pgfsetdash{}{0pt}%
\pgfsys@defobject{currentmarker}{\pgfqpoint{0.000000in}{-0.048611in}}{\pgfqpoint{0.000000in}{0.000000in}}{%
\pgfpathmoveto{\pgfqpoint{0.000000in}{0.000000in}}%
\pgfpathlineto{\pgfqpoint{0.000000in}{-0.048611in}}%
\pgfusepath{stroke,fill}%
}%
\begin{pgfscope}%
\pgfsys@transformshift{5.152990in}{0.517039in}%
\pgfsys@useobject{currentmarker}{}%
\end{pgfscope}%
\end{pgfscope}%
\begin{pgfscope}%
\definecolor{textcolor}{rgb}{0.000000,0.000000,0.000000}%
\pgfsetstrokecolor{textcolor}%
\pgfsetfillcolor{textcolor}%
\pgftext[x=5.152990in,y=0.419816in,,top]{\color{textcolor}{\rmfamily\fontsize{12.000000}{14.400000}\selectfont\catcode`\^=\active\def^{\ifmmode\sp\else\^{}\fi}\catcode`\%=\active\def%{\%}1.25}}%
\end{pgfscope}%
\begin{pgfscope}%
\pgfsetbuttcap%
\pgfsetroundjoin%
\definecolor{currentfill}{rgb}{0.000000,0.000000,0.000000}%
\pgfsetfillcolor{currentfill}%
\pgfsetlinewidth{0.803000pt}%
\definecolor{currentstroke}{rgb}{0.000000,0.000000,0.000000}%
\pgfsetstrokecolor{currentstroke}%
\pgfsetdash{}{0pt}%
\pgfsys@defobject{currentmarker}{\pgfqpoint{0.000000in}{-0.048611in}}{\pgfqpoint{0.000000in}{0.000000in}}{%
\pgfpathmoveto{\pgfqpoint{0.000000in}{0.000000in}}%
\pgfpathlineto{\pgfqpoint{0.000000in}{-0.048611in}}%
\pgfusepath{stroke,fill}%
}%
\begin{pgfscope}%
\pgfsys@transformshift{5.978090in}{0.517039in}%
\pgfsys@useobject{currentmarker}{}%
\end{pgfscope}%
\end{pgfscope}%
\begin{pgfscope}%
\definecolor{textcolor}{rgb}{0.000000,0.000000,0.000000}%
\pgfsetstrokecolor{textcolor}%
\pgfsetfillcolor{textcolor}%
\pgftext[x=5.978090in,y=0.419816in,,top]{\color{textcolor}{\rmfamily\fontsize{12.000000}{14.400000}\selectfont\catcode`\^=\active\def^{\ifmmode\sp\else\^{}\fi}\catcode`\%=\active\def%{\%}1.50}}%
\end{pgfscope}%
\begin{pgfscope}%
\pgfsetbuttcap%
\pgfsetroundjoin%
\definecolor{currentfill}{rgb}{0.000000,0.000000,0.000000}%
\pgfsetfillcolor{currentfill}%
\pgfsetlinewidth{0.803000pt}%
\definecolor{currentstroke}{rgb}{0.000000,0.000000,0.000000}%
\pgfsetstrokecolor{currentstroke}%
\pgfsetdash{}{0pt}%
\pgfsys@defobject{currentmarker}{\pgfqpoint{0.000000in}{-0.048611in}}{\pgfqpoint{0.000000in}{0.000000in}}{%
\pgfpathmoveto{\pgfqpoint{0.000000in}{0.000000in}}%
\pgfpathlineto{\pgfqpoint{0.000000in}{-0.048611in}}%
\pgfusepath{stroke,fill}%
}%
\begin{pgfscope}%
\pgfsys@transformshift{6.803190in}{0.517039in}%
\pgfsys@useobject{currentmarker}{}%
\end{pgfscope}%
\end{pgfscope}%
\begin{pgfscope}%
\definecolor{textcolor}{rgb}{0.000000,0.000000,0.000000}%
\pgfsetstrokecolor{textcolor}%
\pgfsetfillcolor{textcolor}%
\pgftext[x=6.803190in,y=0.419816in,,top]{\color{textcolor}{\rmfamily\fontsize{12.000000}{14.400000}\selectfont\catcode`\^=\active\def^{\ifmmode\sp\else\^{}\fi}\catcode`\%=\active\def%{\%}1.75}}%
\end{pgfscope}%
\begin{pgfscope}%
\pgfsetbuttcap%
\pgfsetroundjoin%
\definecolor{currentfill}{rgb}{0.000000,0.000000,0.000000}%
\pgfsetfillcolor{currentfill}%
\pgfsetlinewidth{0.803000pt}%
\definecolor{currentstroke}{rgb}{0.000000,0.000000,0.000000}%
\pgfsetstrokecolor{currentstroke}%
\pgfsetdash{}{0pt}%
\pgfsys@defobject{currentmarker}{\pgfqpoint{0.000000in}{-0.048611in}}{\pgfqpoint{0.000000in}{0.000000in}}{%
\pgfpathmoveto{\pgfqpoint{0.000000in}{0.000000in}}%
\pgfpathlineto{\pgfqpoint{0.000000in}{-0.048611in}}%
\pgfusepath{stroke,fill}%
}%
\begin{pgfscope}%
\pgfsys@transformshift{7.628290in}{0.517039in}%
\pgfsys@useobject{currentmarker}{}%
\end{pgfscope}%
\end{pgfscope}%
\begin{pgfscope}%
\definecolor{textcolor}{rgb}{0.000000,0.000000,0.000000}%
\pgfsetstrokecolor{textcolor}%
\pgfsetfillcolor{textcolor}%
\pgftext[x=7.628290in,y=0.419816in,,top]{\color{textcolor}{\rmfamily\fontsize{12.000000}{14.400000}\selectfont\catcode`\^=\active\def^{\ifmmode\sp\else\^{}\fi}\catcode`\%=\active\def%{\%}2.00}}%
\end{pgfscope}%
\begin{pgfscope}%
\definecolor{textcolor}{rgb}{0.000000,0.000000,0.000000}%
\pgfsetstrokecolor{textcolor}%
\pgfsetfillcolor{textcolor}%
\pgftext[x=4.327890in,y=0.202965in,,top]{\color{textcolor}{\rmfamily\fontsize{12.000000}{14.400000}\selectfont\catcode`\^=\active\def^{\ifmmode\sp\else\^{}\fi}\catcode`\%=\active\def%{\%}Time (hours)}}%
\end{pgfscope}%
\begin{pgfscope}%
\pgfsetbuttcap%
\pgfsetroundjoin%
\definecolor{currentfill}{rgb}{0.000000,0.000000,0.000000}%
\pgfsetfillcolor{currentfill}%
\pgfsetlinewidth{0.803000pt}%
\definecolor{currentstroke}{rgb}{0.000000,0.000000,0.000000}%
\pgfsetstrokecolor{currentstroke}%
\pgfsetdash{}{0pt}%
\pgfsys@defobject{currentmarker}{\pgfqpoint{-0.048611in}{0.000000in}}{\pgfqpoint{-0.000000in}{0.000000in}}{%
\pgfpathmoveto{\pgfqpoint{-0.000000in}{0.000000in}}%
\pgfpathlineto{\pgfqpoint{-0.048611in}{0.000000in}}%
\pgfusepath{stroke,fill}%
}%
\begin{pgfscope}%
\pgfsys@transformshift{0.697450in}{0.550824in}%
\pgfsys@useobject{currentmarker}{}%
\end{pgfscope}%
\end{pgfscope}%
\begin{pgfscope}%
\definecolor{textcolor}{rgb}{0.000000,0.000000,0.000000}%
\pgfsetstrokecolor{textcolor}%
\pgfsetfillcolor{textcolor}%
\pgftext[x=0.258521in, y=0.487510in, left, base]{\color{textcolor}{\rmfamily\fontsize{12.000000}{14.400000}\selectfont\catcode`\^=\active\def^{\ifmmode\sp\else\^{}\fi}\catcode`\%=\active\def%{\%}\ensuremath{-}10}}%
\end{pgfscope}%
\begin{pgfscope}%
\pgfsetbuttcap%
\pgfsetroundjoin%
\definecolor{currentfill}{rgb}{0.000000,0.000000,0.000000}%
\pgfsetfillcolor{currentfill}%
\pgfsetlinewidth{0.803000pt}%
\definecolor{currentstroke}{rgb}{0.000000,0.000000,0.000000}%
\pgfsetstrokecolor{currentstroke}%
\pgfsetdash{}{0pt}%
\pgfsys@defobject{currentmarker}{\pgfqpoint{-0.048611in}{0.000000in}}{\pgfqpoint{-0.000000in}{0.000000in}}{%
\pgfpathmoveto{\pgfqpoint{-0.000000in}{0.000000in}}%
\pgfpathlineto{\pgfqpoint{-0.048611in}{0.000000in}}%
\pgfusepath{stroke,fill}%
}%
\begin{pgfscope}%
\pgfsys@transformshift{0.697450in}{1.172991in}%
\pgfsys@useobject{currentmarker}{}%
\end{pgfscope}%
\end{pgfscope}%
\begin{pgfscope}%
\definecolor{textcolor}{rgb}{0.000000,0.000000,0.000000}%
\pgfsetstrokecolor{textcolor}%
\pgfsetfillcolor{textcolor}%
\pgftext[x=0.494189in, y=1.109677in, left, base]{\color{textcolor}{\rmfamily\fontsize{12.000000}{14.400000}\selectfont\catcode`\^=\active\def^{\ifmmode\sp\else\^{}\fi}\catcode`\%=\active\def%{\%}0}}%
\end{pgfscope}%
\begin{pgfscope}%
\pgfsetbuttcap%
\pgfsetroundjoin%
\definecolor{currentfill}{rgb}{0.000000,0.000000,0.000000}%
\pgfsetfillcolor{currentfill}%
\pgfsetlinewidth{0.803000pt}%
\definecolor{currentstroke}{rgb}{0.000000,0.000000,0.000000}%
\pgfsetstrokecolor{currentstroke}%
\pgfsetdash{}{0pt}%
\pgfsys@defobject{currentmarker}{\pgfqpoint{-0.048611in}{0.000000in}}{\pgfqpoint{-0.000000in}{0.000000in}}{%
\pgfpathmoveto{\pgfqpoint{-0.000000in}{0.000000in}}%
\pgfpathlineto{\pgfqpoint{-0.048611in}{0.000000in}}%
\pgfusepath{stroke,fill}%
}%
\begin{pgfscope}%
\pgfsys@transformshift{0.697450in}{1.795158in}%
\pgfsys@useobject{currentmarker}{}%
\end{pgfscope}%
\end{pgfscope}%
\begin{pgfscope}%
\definecolor{textcolor}{rgb}{0.000000,0.000000,0.000000}%
\pgfsetstrokecolor{textcolor}%
\pgfsetfillcolor{textcolor}%
\pgftext[x=0.388151in, y=1.731844in, left, base]{\color{textcolor}{\rmfamily\fontsize{12.000000}{14.400000}\selectfont\catcode`\^=\active\def^{\ifmmode\sp\else\^{}\fi}\catcode`\%=\active\def%{\%}10}}%
\end{pgfscope}%
\begin{pgfscope}%
\definecolor{textcolor}{rgb}{0.000000,0.000000,0.000000}%
\pgfsetstrokecolor{textcolor}%
\pgfsetfillcolor{textcolor}%
\pgftext[x=0.202965in,y=1.237684in,,bottom,rotate=90.000000]{\color{textcolor}{\rmfamily\fontsize{12.000000}{14.400000}\selectfont\catcode`\^=\active\def^{\ifmmode\sp\else\^{}\fi}\catcode`\%=\active\def%{\%}Acceleration}}%
\end{pgfscope}%
\begin{pgfscope}%
\pgfpathrectangle{\pgfqpoint{0.697450in}{0.517039in}}{\pgfqpoint{7.260880in}{1.441291in}}%
\pgfusepath{clip}%
\pgfsetrectcap%
\pgfsetroundjoin%
\pgfsetlinewidth{1.505625pt}%
\definecolor{currentstroke}{rgb}{0.121569,0.466667,0.705882}%
\pgfsetstrokecolor{currentstroke}%
\pgfsetdash{}{0pt}%
\pgfpathmoveto{\pgfqpoint{1.027490in}{1.550725in}}%
\pgfpathlineto{\pgfqpoint{1.060660in}{1.547590in}}%
\pgfpathlineto{\pgfqpoint{1.127000in}{1.545483in}}%
\pgfpathlineto{\pgfqpoint{1.193339in}{1.543095in}}%
\pgfpathlineto{\pgfqpoint{1.226509in}{1.539398in}}%
\pgfpathlineto{\pgfqpoint{1.259679in}{1.532978in}}%
\pgfpathlineto{\pgfqpoint{1.292849in}{1.522990in}}%
\pgfpathlineto{\pgfqpoint{1.326019in}{1.508634in}}%
\pgfpathlineto{\pgfqpoint{1.359188in}{1.489604in}}%
\pgfpathlineto{\pgfqpoint{1.392358in}{1.466692in}}%
\pgfpathlineto{\pgfqpoint{1.425528in}{1.442072in}}%
\pgfpathlineto{\pgfqpoint{1.458698in}{1.418920in}}%
\pgfpathlineto{\pgfqpoint{1.491868in}{1.400279in}}%
\pgfpathlineto{\pgfqpoint{1.525038in}{1.387613in}}%
\pgfpathlineto{\pgfqpoint{1.591377in}{1.372808in}}%
\pgfpathlineto{\pgfqpoint{1.624547in}{1.361610in}}%
\pgfpathlineto{\pgfqpoint{1.657717in}{1.341668in}}%
\pgfpathlineto{\pgfqpoint{1.690887in}{1.311320in}}%
\pgfpathlineto{\pgfqpoint{1.724057in}{1.272708in}}%
\pgfpathlineto{\pgfqpoint{1.757227in}{1.231128in}}%
\pgfpathlineto{\pgfqpoint{1.790397in}{1.192943in}}%
\pgfpathlineto{\pgfqpoint{1.823566in}{1.162951in}}%
\pgfpathlineto{\pgfqpoint{1.856736in}{1.142422in}}%
\pgfpathlineto{\pgfqpoint{1.889906in}{1.128716in}}%
\pgfpathlineto{\pgfqpoint{1.923076in}{1.116638in}}%
\pgfpathlineto{\pgfqpoint{1.956246in}{1.100871in}}%
\pgfpathlineto{\pgfqpoint{1.989416in}{1.078317in}}%
\pgfpathlineto{\pgfqpoint{2.022585in}{1.049262in}}%
\pgfpathlineto{\pgfqpoint{2.088925in}{0.985557in}}%
\pgfpathlineto{\pgfqpoint{2.122095in}{0.958637in}}%
\pgfpathlineto{\pgfqpoint{2.155265in}{0.937248in}}%
\pgfpathlineto{\pgfqpoint{2.188435in}{0.920210in}}%
\pgfpathlineto{\pgfqpoint{2.287944in}{0.875930in}}%
\pgfpathlineto{\pgfqpoint{2.321114in}{0.862781in}}%
\pgfpathlineto{\pgfqpoint{2.354284in}{0.852933in}}%
\pgfpathlineto{\pgfqpoint{2.387454in}{0.847202in}}%
\pgfpathlineto{\pgfqpoint{2.453793in}{0.841514in}}%
\pgfpathlineto{\pgfqpoint{2.486963in}{0.834955in}}%
\pgfpathlineto{\pgfqpoint{2.520133in}{0.822433in}}%
\pgfpathlineto{\pgfqpoint{2.553303in}{0.804329in}}%
\pgfpathlineto{\pgfqpoint{2.586473in}{0.783929in}}%
\pgfpathlineto{\pgfqpoint{2.619643in}{0.766287in}}%
\pgfpathlineto{\pgfqpoint{2.652813in}{0.756219in}}%
\pgfpathlineto{\pgfqpoint{2.685982in}{0.756337in}}%
\pgfpathlineto{\pgfqpoint{2.719152in}{0.766014in}}%
\pgfpathlineto{\pgfqpoint{2.818662in}{0.812610in}}%
\pgfpathlineto{\pgfqpoint{2.851832in}{0.821730in}}%
\pgfpathlineto{\pgfqpoint{2.918171in}{0.831478in}}%
\pgfpathlineto{\pgfqpoint{2.951341in}{0.838935in}}%
\pgfpathlineto{\pgfqpoint{2.984511in}{0.852011in}}%
\pgfpathlineto{\pgfqpoint{3.017681in}{0.871184in}}%
\pgfpathlineto{\pgfqpoint{3.117191in}{0.941188in}}%
\pgfpathlineto{\pgfqpoint{3.150360in}{0.958978in}}%
\pgfpathlineto{\pgfqpoint{3.183530in}{0.972640in}}%
\pgfpathlineto{\pgfqpoint{3.216700in}{0.984479in}}%
\pgfpathlineto{\pgfqpoint{3.249870in}{0.998007in}}%
\pgfpathlineto{\pgfqpoint{3.283040in}{1.016533in}}%
\pgfpathlineto{\pgfqpoint{3.316210in}{1.041792in}}%
\pgfpathlineto{\pgfqpoint{3.349379in}{1.073156in}}%
\pgfpathlineto{\pgfqpoint{3.415719in}{1.141569in}}%
\pgfpathlineto{\pgfqpoint{3.448889in}{1.170923in}}%
\pgfpathlineto{\pgfqpoint{3.482059in}{1.194071in}}%
\pgfpathlineto{\pgfqpoint{3.515229in}{1.211917in}}%
\pgfpathlineto{\pgfqpoint{3.548399in}{1.227677in}}%
\pgfpathlineto{\pgfqpoint{3.581568in}{1.245516in}}%
\pgfpathlineto{\pgfqpoint{3.614738in}{1.268700in}}%
\pgfpathlineto{\pgfqpoint{3.647908in}{1.298065in}}%
\pgfpathlineto{\pgfqpoint{3.714248in}{1.364903in}}%
\pgfpathlineto{\pgfqpoint{3.747418in}{1.393655in}}%
\pgfpathlineto{\pgfqpoint{3.780587in}{1.414911in}}%
\pgfpathlineto{\pgfqpoint{3.813757in}{1.428679in}}%
\pgfpathlineto{\pgfqpoint{3.880097in}{1.446484in}}%
\pgfpathlineto{\pgfqpoint{3.913267in}{1.458472in}}%
\pgfpathlineto{\pgfqpoint{3.946437in}{1.475166in}}%
\pgfpathlineto{\pgfqpoint{4.012776in}{1.515002in}}%
\pgfpathlineto{\pgfqpoint{4.045946in}{1.531060in}}%
\pgfpathlineto{\pgfqpoint{4.079116in}{1.541253in}}%
\pgfpathlineto{\pgfqpoint{4.112286in}{1.546074in}}%
\pgfpathlineto{\pgfqpoint{4.178626in}{1.551256in}}%
\pgfpathlineto{\pgfqpoint{4.211795in}{1.557627in}}%
\pgfpathlineto{\pgfqpoint{4.244965in}{1.567629in}}%
\pgfpathlineto{\pgfqpoint{4.278135in}{1.579203in}}%
\pgfpathlineto{\pgfqpoint{4.311305in}{1.588958in}}%
\pgfpathlineto{\pgfqpoint{4.344475in}{1.593711in}}%
\pgfpathlineto{\pgfqpoint{4.377645in}{1.591750in}}%
\pgfpathlineto{\pgfqpoint{4.410815in}{1.583262in}}%
\pgfpathlineto{\pgfqpoint{4.443985in}{1.569813in}}%
\pgfpathlineto{\pgfqpoint{4.477154in}{1.553325in}}%
\pgfpathlineto{\pgfqpoint{4.543494in}{1.515903in}}%
\pgfpathlineto{\pgfqpoint{4.609834in}{1.474484in}}%
\pgfpathlineto{\pgfqpoint{4.676173in}{1.433735in}}%
\pgfpathlineto{\pgfqpoint{4.709343in}{1.416180in}}%
\pgfpathlineto{\pgfqpoint{4.775683in}{1.384833in}}%
\pgfpathlineto{\pgfqpoint{4.808853in}{1.366564in}}%
\pgfpathlineto{\pgfqpoint{4.842023in}{1.343331in}}%
\pgfpathlineto{\pgfqpoint{4.875193in}{1.314473in}}%
\pgfpathlineto{\pgfqpoint{4.941532in}{1.248398in}}%
\pgfpathlineto{\pgfqpoint{4.974702in}{1.218569in}}%
\pgfpathlineto{\pgfqpoint{5.007872in}{1.194610in}}%
\pgfpathlineto{\pgfqpoint{5.041042in}{1.176317in}}%
\pgfpathlineto{\pgfqpoint{5.074212in}{1.160872in}}%
\pgfpathlineto{\pgfqpoint{5.107381in}{1.144152in}}%
\pgfpathlineto{\pgfqpoint{5.140551in}{1.122707in}}%
\pgfpathlineto{\pgfqpoint{5.173721in}{1.095411in}}%
\pgfpathlineto{\pgfqpoint{5.240061in}{1.032112in}}%
\pgfpathlineto{\pgfqpoint{5.273231in}{1.003646in}}%
\pgfpathlineto{\pgfqpoint{5.306401in}{0.980763in}}%
\pgfpathlineto{\pgfqpoint{5.339570in}{0.963050in}}%
\pgfpathlineto{\pgfqpoint{5.405910in}{0.932091in}}%
\pgfpathlineto{\pgfqpoint{5.439080in}{0.913416in}}%
\pgfpathlineto{\pgfqpoint{5.505419in}{0.870117in}}%
\pgfpathlineto{\pgfqpoint{5.538589in}{0.851104in}}%
\pgfpathlineto{\pgfqpoint{5.571759in}{0.837576in}}%
\pgfpathlineto{\pgfqpoint{5.604929in}{0.830147in}}%
\pgfpathlineto{\pgfqpoint{5.638099in}{0.827121in}}%
\pgfpathlineto{\pgfqpoint{5.671269in}{0.825342in}}%
\pgfpathlineto{\pgfqpoint{5.704439in}{0.821718in}}%
\pgfpathlineto{\pgfqpoint{5.737609in}{0.814662in}}%
\pgfpathlineto{\pgfqpoint{5.803948in}{0.794348in}}%
\pgfpathlineto{\pgfqpoint{5.837118in}{0.786391in}}%
\pgfpathlineto{\pgfqpoint{5.870288in}{0.783025in}}%
\pgfpathlineto{\pgfqpoint{5.903458in}{0.784691in}}%
\pgfpathlineto{\pgfqpoint{5.936628in}{0.790071in}}%
\pgfpathlineto{\pgfqpoint{6.002967in}{0.802847in}}%
\pgfpathlineto{\pgfqpoint{6.036137in}{0.807045in}}%
\pgfpathlineto{\pgfqpoint{6.102477in}{0.813201in}}%
\pgfpathlineto{\pgfqpoint{6.135647in}{0.818911in}}%
\pgfpathlineto{\pgfqpoint{6.168817in}{0.828564in}}%
\pgfpathlineto{\pgfqpoint{6.201986in}{0.842494in}}%
\pgfpathlineto{\pgfqpoint{6.235156in}{0.859809in}}%
\pgfpathlineto{\pgfqpoint{6.334666in}{0.916220in}}%
\pgfpathlineto{\pgfqpoint{6.401006in}{0.951420in}}%
\pgfpathlineto{\pgfqpoint{6.434176in}{0.970880in}}%
\pgfpathlineto{\pgfqpoint{6.467346in}{0.993208in}}%
\pgfpathlineto{\pgfqpoint{6.500515in}{1.018651in}}%
\pgfpathlineto{\pgfqpoint{6.600025in}{1.101865in}}%
\pgfpathlineto{\pgfqpoint{6.633195in}{1.126766in}}%
\pgfpathlineto{\pgfqpoint{6.699534in}{1.171046in}}%
\pgfpathlineto{\pgfqpoint{6.732704in}{1.193388in}}%
\pgfpathlineto{\pgfqpoint{6.765874in}{1.217934in}}%
\pgfpathlineto{\pgfqpoint{6.799044in}{1.245065in}}%
\pgfpathlineto{\pgfqpoint{6.865384in}{1.302068in}}%
\pgfpathlineto{\pgfqpoint{6.898553in}{1.327874in}}%
\pgfpathlineto{\pgfqpoint{6.931723in}{1.350099in}}%
\pgfpathlineto{\pgfqpoint{7.031233in}{1.406324in}}%
\pgfpathlineto{\pgfqpoint{7.064402in}{1.428328in}}%
\pgfpathlineto{\pgfqpoint{7.163912in}{1.501505in}}%
\pgfpathlineto{\pgfqpoint{7.197082in}{1.518972in}}%
\pgfpathlineto{\pgfqpoint{7.230252in}{1.529481in}}%
\pgfpathlineto{\pgfqpoint{7.263422in}{1.533780in}}%
\pgfpathlineto{\pgfqpoint{7.329761in}{1.534891in}}%
\pgfpathlineto{\pgfqpoint{7.362931in}{1.537533in}}%
\pgfpathlineto{\pgfqpoint{7.396101in}{1.543163in}}%
\pgfpathlineto{\pgfqpoint{7.462441in}{1.557631in}}%
\pgfpathlineto{\pgfqpoint{7.495611in}{1.561955in}}%
\pgfpathlineto{\pgfqpoint{7.528780in}{1.562513in}}%
\pgfpathlineto{\pgfqpoint{7.561950in}{1.559737in}}%
\pgfpathlineto{\pgfqpoint{7.628290in}{1.550725in}}%
\pgfpathlineto{\pgfqpoint{7.628290in}{1.550725in}}%
\pgfusepath{stroke}%
\end{pgfscope}%
\begin{pgfscope}%
\pgfpathrectangle{\pgfqpoint{0.697450in}{0.517039in}}{\pgfqpoint{7.260880in}{1.441291in}}%
\pgfusepath{clip}%
\pgfsetrectcap%
\pgfsetroundjoin%
\pgfsetlinewidth{1.505625pt}%
\definecolor{currentstroke}{rgb}{1.000000,0.498039,0.054902}%
\pgfsetstrokecolor{currentstroke}%
\pgfsetdash{}{0pt}%
\pgfpathmoveto{\pgfqpoint{1.027490in}{1.540298in}}%
\pgfpathlineto{\pgfqpoint{1.060660in}{1.545068in}}%
\pgfpathlineto{\pgfqpoint{1.093830in}{1.553801in}}%
\pgfpathlineto{\pgfqpoint{1.127000in}{1.559922in}}%
\pgfpathlineto{\pgfqpoint{1.160169in}{1.557282in}}%
\pgfpathlineto{\pgfqpoint{1.193339in}{1.543180in}}%
\pgfpathlineto{\pgfqpoint{1.226509in}{1.519733in}}%
\pgfpathlineto{\pgfqpoint{1.259679in}{1.492866in}}%
\pgfpathlineto{\pgfqpoint{1.292849in}{1.469422in}}%
\pgfpathlineto{\pgfqpoint{1.326019in}{1.453861in}}%
\pgfpathlineto{\pgfqpoint{1.359188in}{1.446229in}}%
\pgfpathlineto{\pgfqpoint{1.392358in}{1.442387in}}%
\pgfpathlineto{\pgfqpoint{1.425528in}{1.436286in}}%
\pgfpathlineto{\pgfqpoint{1.458698in}{1.423060in}}%
\pgfpathlineto{\pgfqpoint{1.491868in}{1.401307in}}%
\pgfpathlineto{\pgfqpoint{1.591377in}{1.318169in}}%
\pgfpathlineto{\pgfqpoint{1.624547in}{1.297260in}}%
\pgfpathlineto{\pgfqpoint{1.657717in}{1.280676in}}%
\pgfpathlineto{\pgfqpoint{1.690887in}{1.265632in}}%
\pgfpathlineto{\pgfqpoint{1.724057in}{1.249351in}}%
\pgfpathlineto{\pgfqpoint{1.757227in}{1.230688in}}%
\pgfpathlineto{\pgfqpoint{1.823566in}{1.190832in}}%
\pgfpathlineto{\pgfqpoint{1.856736in}{1.173029in}}%
\pgfpathlineto{\pgfqpoint{1.923076in}{1.139896in}}%
\pgfpathlineto{\pgfqpoint{1.956246in}{1.119555in}}%
\pgfpathlineto{\pgfqpoint{1.989416in}{1.093967in}}%
\pgfpathlineto{\pgfqpoint{2.022585in}{1.063632in}}%
\pgfpathlineto{\pgfqpoint{2.055755in}{1.031359in}}%
\pgfpathlineto{\pgfqpoint{2.088925in}{1.000995in}}%
\pgfpathlineto{\pgfqpoint{2.122095in}{0.975569in}}%
\pgfpathlineto{\pgfqpoint{2.155265in}{0.955820in}}%
\pgfpathlineto{\pgfqpoint{2.221605in}{0.924427in}}%
\pgfpathlineto{\pgfqpoint{2.254774in}{0.906198in}}%
\pgfpathlineto{\pgfqpoint{2.287944in}{0.883834in}}%
\pgfpathlineto{\pgfqpoint{2.354284in}{0.832855in}}%
\pgfpathlineto{\pgfqpoint{2.387454in}{0.810681in}}%
\pgfpathlineto{\pgfqpoint{2.420624in}{0.794417in}}%
\pgfpathlineto{\pgfqpoint{2.453793in}{0.784789in}}%
\pgfpathlineto{\pgfqpoint{2.486963in}{0.780713in}}%
\pgfpathlineto{\pgfqpoint{2.553303in}{0.780396in}}%
\pgfpathlineto{\pgfqpoint{2.586473in}{0.780216in}}%
\pgfpathlineto{\pgfqpoint{2.652813in}{0.776702in}}%
\pgfpathlineto{\pgfqpoint{2.685982in}{0.774963in}}%
\pgfpathlineto{\pgfqpoint{2.719152in}{0.775342in}}%
\pgfpathlineto{\pgfqpoint{2.752322in}{0.779668in}}%
\pgfpathlineto{\pgfqpoint{2.785492in}{0.789369in}}%
\pgfpathlineto{\pgfqpoint{2.818662in}{0.804853in}}%
\pgfpathlineto{\pgfqpoint{2.851832in}{0.825043in}}%
\pgfpathlineto{\pgfqpoint{2.918171in}{0.868479in}}%
\pgfpathlineto{\pgfqpoint{2.951341in}{0.885477in}}%
\pgfpathlineto{\pgfqpoint{2.984511in}{0.897253in}}%
\pgfpathlineto{\pgfqpoint{3.050851in}{0.912445in}}%
\pgfpathlineto{\pgfqpoint{3.084021in}{0.922839in}}%
\pgfpathlineto{\pgfqpoint{3.117191in}{0.938420in}}%
\pgfpathlineto{\pgfqpoint{3.183530in}{0.979082in}}%
\pgfpathlineto{\pgfqpoint{3.216700in}{0.996009in}}%
\pgfpathlineto{\pgfqpoint{3.249870in}{1.006249in}}%
\pgfpathlineto{\pgfqpoint{3.316210in}{1.014670in}}%
\pgfpathlineto{\pgfqpoint{3.349379in}{1.025696in}}%
\pgfpathlineto{\pgfqpoint{3.382549in}{1.050540in}}%
\pgfpathlineto{\pgfqpoint{3.415719in}{1.091584in}}%
\pgfpathlineto{\pgfqpoint{3.448889in}{1.145124in}}%
\pgfpathlineto{\pgfqpoint{3.482059in}{1.202281in}}%
\pgfpathlineto{\pgfqpoint{3.515229in}{1.252339in}}%
\pgfpathlineto{\pgfqpoint{3.548399in}{1.287040in}}%
\pgfpathlineto{\pgfqpoint{3.581568in}{1.303906in}}%
\pgfpathlineto{\pgfqpoint{3.614738in}{1.307064in}}%
\pgfpathlineto{\pgfqpoint{3.647908in}{1.305225in}}%
\pgfpathlineto{\pgfqpoint{3.681078in}{1.307828in}}%
\pgfpathlineto{\pgfqpoint{3.714248in}{1.321154in}}%
\pgfpathlineto{\pgfqpoint{3.747418in}{1.346153in}}%
\pgfpathlineto{\pgfqpoint{3.813757in}{1.412827in}}%
\pgfpathlineto{\pgfqpoint{3.846927in}{1.442594in}}%
\pgfpathlineto{\pgfqpoint{3.880097in}{1.465904in}}%
\pgfpathlineto{\pgfqpoint{3.979607in}{1.521355in}}%
\pgfpathlineto{\pgfqpoint{4.012776in}{1.544895in}}%
\pgfpathlineto{\pgfqpoint{4.045946in}{1.570162in}}%
\pgfpathlineto{\pgfqpoint{4.079116in}{1.592684in}}%
\pgfpathlineto{\pgfqpoint{4.112286in}{1.607476in}}%
\pgfpathlineto{\pgfqpoint{4.145456in}{1.611152in}}%
\pgfpathlineto{\pgfqpoint{4.178626in}{1.603292in}}%
\pgfpathlineto{\pgfqpoint{4.211795in}{1.586571in}}%
\pgfpathlineto{\pgfqpoint{4.278135in}{1.546085in}}%
\pgfpathlineto{\pgfqpoint{4.311305in}{1.531669in}}%
\pgfpathlineto{\pgfqpoint{4.344475in}{1.524437in}}%
\pgfpathlineto{\pgfqpoint{4.377645in}{1.523845in}}%
\pgfpathlineto{\pgfqpoint{4.443985in}{1.531273in}}%
\pgfpathlineto{\pgfqpoint{4.477154in}{1.532310in}}%
\pgfpathlineto{\pgfqpoint{4.510324in}{1.528381in}}%
\pgfpathlineto{\pgfqpoint{4.543494in}{1.519385in}}%
\pgfpathlineto{\pgfqpoint{4.676173in}{1.470960in}}%
\pgfpathlineto{\pgfqpoint{4.709343in}{1.458956in}}%
\pgfpathlineto{\pgfqpoint{4.742513in}{1.441920in}}%
\pgfpathlineto{\pgfqpoint{4.775683in}{1.416289in}}%
\pgfpathlineto{\pgfqpoint{4.808853in}{1.380968in}}%
\pgfpathlineto{\pgfqpoint{4.875193in}{1.295311in}}%
\pgfpathlineto{\pgfqpoint{4.908363in}{1.258124in}}%
\pgfpathlineto{\pgfqpoint{4.941532in}{1.232120in}}%
\pgfpathlineto{\pgfqpoint{4.974702in}{1.217643in}}%
\pgfpathlineto{\pgfqpoint{5.007872in}{1.209704in}}%
\pgfpathlineto{\pgfqpoint{5.041042in}{1.199905in}}%
\pgfpathlineto{\pgfqpoint{5.074212in}{1.180222in}}%
\pgfpathlineto{\pgfqpoint{5.107381in}{1.146927in}}%
\pgfpathlineto{\pgfqpoint{5.140551in}{1.102699in}}%
\pgfpathlineto{\pgfqpoint{5.173721in}{1.055752in}}%
\pgfpathlineto{\pgfqpoint{5.206891in}{1.016221in}}%
\pgfpathlineto{\pgfqpoint{5.240061in}{0.991458in}}%
\pgfpathlineto{\pgfqpoint{5.273231in}{0.982486in}}%
\pgfpathlineto{\pgfqpoint{5.339570in}{0.983913in}}%
\pgfpathlineto{\pgfqpoint{5.372740in}{0.974269in}}%
\pgfpathlineto{\pgfqpoint{5.405910in}{0.949455in}}%
\pgfpathlineto{\pgfqpoint{5.439080in}{0.911419in}}%
\pgfpathlineto{\pgfqpoint{5.472250in}{0.867868in}}%
\pgfpathlineto{\pgfqpoint{5.505419in}{0.828500in}}%
\pgfpathlineto{\pgfqpoint{5.538589in}{0.800612in}}%
\pgfpathlineto{\pgfqpoint{5.571759in}{0.786204in}}%
\pgfpathlineto{\pgfqpoint{5.604929in}{0.781864in}}%
\pgfpathlineto{\pgfqpoint{5.638099in}{0.781238in}}%
\pgfpathlineto{\pgfqpoint{5.671269in}{0.778597in}}%
\pgfpathlineto{\pgfqpoint{5.704439in}{0.771594in}}%
\pgfpathlineto{\pgfqpoint{5.737609in}{0.761913in}}%
\pgfpathlineto{\pgfqpoint{5.770779in}{0.753701in}}%
\pgfpathlineto{\pgfqpoint{5.803948in}{0.750889in}}%
\pgfpathlineto{\pgfqpoint{5.837118in}{0.754955in}}%
\pgfpathlineto{\pgfqpoint{5.870288in}{0.764308in}}%
\pgfpathlineto{\pgfqpoint{5.903458in}{0.775420in}}%
\pgfpathlineto{\pgfqpoint{5.936628in}{0.784875in}}%
\pgfpathlineto{\pgfqpoint{5.969798in}{0.791066in}}%
\pgfpathlineto{\pgfqpoint{6.069307in}{0.801560in}}%
\pgfpathlineto{\pgfqpoint{6.102477in}{0.807391in}}%
\pgfpathlineto{\pgfqpoint{6.168817in}{0.822636in}}%
\pgfpathlineto{\pgfqpoint{6.201986in}{0.832236in}}%
\pgfpathlineto{\pgfqpoint{6.235156in}{0.845486in}}%
\pgfpathlineto{\pgfqpoint{6.268326in}{0.864998in}}%
\pgfpathlineto{\pgfqpoint{6.301496in}{0.892081in}}%
\pgfpathlineto{\pgfqpoint{6.367836in}{0.959672in}}%
\pgfpathlineto{\pgfqpoint{6.401006in}{0.989420in}}%
\pgfpathlineto{\pgfqpoint{6.434176in}{1.009546in}}%
\pgfpathlineto{\pgfqpoint{6.467346in}{1.019003in}}%
\pgfpathlineto{\pgfqpoint{6.533685in}{1.024539in}}%
\pgfpathlineto{\pgfqpoint{6.566855in}{1.035664in}}%
\pgfpathlineto{\pgfqpoint{6.600025in}{1.059681in}}%
\pgfpathlineto{\pgfqpoint{6.633195in}{1.096178in}}%
\pgfpathlineto{\pgfqpoint{6.699534in}{1.182600in}}%
\pgfpathlineto{\pgfqpoint{6.732704in}{1.217532in}}%
\pgfpathlineto{\pgfqpoint{6.765874in}{1.241145in}}%
\pgfpathlineto{\pgfqpoint{6.799044in}{1.254571in}}%
\pgfpathlineto{\pgfqpoint{6.865384in}{1.270349in}}%
\pgfpathlineto{\pgfqpoint{6.898553in}{1.282790in}}%
\pgfpathlineto{\pgfqpoint{6.931723in}{1.301599in}}%
\pgfpathlineto{\pgfqpoint{6.964893in}{1.326248in}}%
\pgfpathlineto{\pgfqpoint{6.998063in}{1.354916in}}%
\pgfpathlineto{\pgfqpoint{7.097572in}{1.447710in}}%
\pgfpathlineto{\pgfqpoint{7.130742in}{1.476343in}}%
\pgfpathlineto{\pgfqpoint{7.163912in}{1.501126in}}%
\pgfpathlineto{\pgfqpoint{7.197082in}{1.520425in}}%
\pgfpathlineto{\pgfqpoint{7.230252in}{1.533652in}}%
\pgfpathlineto{\pgfqpoint{7.263422in}{1.541922in}}%
\pgfpathlineto{\pgfqpoint{7.329761in}{1.554318in}}%
\pgfpathlineto{\pgfqpoint{7.362931in}{1.562947in}}%
\pgfpathlineto{\pgfqpoint{7.396101in}{1.572897in}}%
\pgfpathlineto{\pgfqpoint{7.429271in}{1.581111in}}%
\pgfpathlineto{\pgfqpoint{7.462441in}{1.583879in}}%
\pgfpathlineto{\pgfqpoint{7.495611in}{1.579048in}}%
\pgfpathlineto{\pgfqpoint{7.528780in}{1.567647in}}%
\pgfpathlineto{\pgfqpoint{7.561950in}{1.553891in}}%
\pgfpathlineto{\pgfqpoint{7.595120in}{1.543381in}}%
\pgfpathlineto{\pgfqpoint{7.628290in}{1.540298in}}%
\pgfpathlineto{\pgfqpoint{7.628290in}{1.540298in}}%
\pgfusepath{stroke}%
\end{pgfscope}%
\begin{pgfscope}%
\pgfpathrectangle{\pgfqpoint{0.697450in}{0.517039in}}{\pgfqpoint{7.260880in}{1.441291in}}%
\pgfusepath{clip}%
\pgfsetrectcap%
\pgfsetroundjoin%
\pgfsetlinewidth{1.505625pt}%
\definecolor{currentstroke}{rgb}{0.172549,0.627451,0.172549}%
\pgfsetstrokecolor{currentstroke}%
\pgfsetdash{}{0pt}%
\pgfpathmoveto{\pgfqpoint{1.027490in}{1.675638in}}%
\pgfpathlineto{\pgfqpoint{1.060660in}{1.691418in}}%
\pgfpathlineto{\pgfqpoint{1.093830in}{1.616616in}}%
\pgfpathlineto{\pgfqpoint{1.127000in}{1.477873in}}%
\pgfpathlineto{\pgfqpoint{1.160169in}{1.326677in}}%
\pgfpathlineto{\pgfqpoint{1.193339in}{1.217262in}}%
\pgfpathlineto{\pgfqpoint{1.226509in}{1.185334in}}%
\pgfpathlineto{\pgfqpoint{1.259679in}{1.236385in}}%
\pgfpathlineto{\pgfqpoint{1.292849in}{1.347233in}}%
\pgfpathlineto{\pgfqpoint{1.326019in}{1.478290in}}%
\pgfpathlineto{\pgfqpoint{1.359188in}{1.589909in}}%
\pgfpathlineto{\pgfqpoint{1.392358in}{1.655575in}}%
\pgfpathlineto{\pgfqpoint{1.425528in}{1.667477in}}%
\pgfpathlineto{\pgfqpoint{1.458698in}{1.634119in}}%
\pgfpathlineto{\pgfqpoint{1.491868in}{1.573060in}}%
\pgfpathlineto{\pgfqpoint{1.525038in}{1.502992in}}%
\pgfpathlineto{\pgfqpoint{1.558208in}{1.438314in}}%
\pgfpathlineto{\pgfqpoint{1.591377in}{1.387126in}}%
\pgfpathlineto{\pgfqpoint{1.624547in}{1.351759in}}%
\pgfpathlineto{\pgfqpoint{1.657717in}{1.330267in}}%
\pgfpathlineto{\pgfqpoint{1.690887in}{1.317907in}}%
\pgfpathlineto{\pgfqpoint{1.724057in}{1.308478in}}%
\pgfpathlineto{\pgfqpoint{1.757227in}{1.295900in}}%
\pgfpathlineto{\pgfqpoint{1.790397in}{1.276061in}}%
\pgfpathlineto{\pgfqpoint{1.823566in}{1.248267in}}%
\pgfpathlineto{\pgfqpoint{1.923076in}{1.150213in}}%
\pgfpathlineto{\pgfqpoint{1.956246in}{1.119983in}}%
\pgfpathlineto{\pgfqpoint{1.989416in}{1.085518in}}%
\pgfpathlineto{\pgfqpoint{2.022585in}{1.039873in}}%
\pgfpathlineto{\pgfqpoint{2.055755in}{0.979396in}}%
\pgfpathlineto{\pgfqpoint{2.122095in}{0.837699in}}%
\pgfpathlineto{\pgfqpoint{2.155265in}{0.785413in}}%
\pgfpathlineto{\pgfqpoint{2.188435in}{0.765235in}}%
\pgfpathlineto{\pgfqpoint{2.221605in}{0.781519in}}%
\pgfpathlineto{\pgfqpoint{2.254774in}{0.825321in}}%
\pgfpathlineto{\pgfqpoint{2.287944in}{0.876763in}}%
\pgfpathlineto{\pgfqpoint{2.321114in}{0.912556in}}%
\pgfpathlineto{\pgfqpoint{2.354284in}{0.915403in}}%
\pgfpathlineto{\pgfqpoint{2.387454in}{0.880977in}}%
\pgfpathlineto{\pgfqpoint{2.420624in}{0.819412in}}%
\pgfpathlineto{\pgfqpoint{2.453793in}{0.750907in}}%
\pgfpathlineto{\pgfqpoint{2.486963in}{0.697761in}}%
\pgfpathlineto{\pgfqpoint{2.520133in}{0.676543in}}%
\pgfpathlineto{\pgfqpoint{2.553303in}{0.693439in}}%
\pgfpathlineto{\pgfqpoint{2.586473in}{0.743917in}}%
\pgfpathlineto{\pgfqpoint{2.619643in}{0.815657in}}%
\pgfpathlineto{\pgfqpoint{2.652813in}{0.892653in}}%
\pgfpathlineto{\pgfqpoint{2.685982in}{0.958770in}}%
\pgfpathlineto{\pgfqpoint{2.719152in}{1.000262in}}%
\pgfpathlineto{\pgfqpoint{2.752322in}{1.007824in}}%
\pgfpathlineto{\pgfqpoint{2.785492in}{0.978772in}}%
\pgfpathlineto{\pgfqpoint{2.818662in}{0.919005in}}%
\pgfpathlineto{\pgfqpoint{2.851832in}{0.843265in}}%
\pgfpathlineto{\pgfqpoint{2.885002in}{0.772128in}}%
\pgfpathlineto{\pgfqpoint{2.918171in}{0.725488in}}%
\pgfpathlineto{\pgfqpoint{2.951341in}{0.714549in}}%
\pgfpathlineto{\pgfqpoint{2.984511in}{0.736162in}}%
\pgfpathlineto{\pgfqpoint{3.017681in}{0.773132in}}%
\pgfpathlineto{\pgfqpoint{3.050851in}{0.801734in}}%
\pgfpathlineto{\pgfqpoint{3.084021in}{0.803743in}}%
\pgfpathlineto{\pgfqpoint{3.117191in}{0.777300in}}%
\pgfpathlineto{\pgfqpoint{3.150360in}{0.740507in}}%
\pgfpathlineto{\pgfqpoint{3.183530in}{0.724728in}}%
\pgfpathlineto{\pgfqpoint{3.216700in}{0.759748in}}%
\pgfpathlineto{\pgfqpoint{3.249870in}{0.857628in}}%
\pgfpathlineto{\pgfqpoint{3.283040in}{1.003596in}}%
\pgfpathlineto{\pgfqpoint{3.316210in}{1.159417in}}%
\pgfpathlineto{\pgfqpoint{3.349379in}{1.278632in}}%
\pgfpathlineto{\pgfqpoint{3.382549in}{1.326870in}}%
\pgfpathlineto{\pgfqpoint{3.415719in}{1.297423in}}%
\pgfpathlineto{\pgfqpoint{3.448889in}{1.214279in}}%
\pgfpathlineto{\pgfqpoint{3.482059in}{1.120975in}}%
\pgfpathlineto{\pgfqpoint{3.515229in}{1.060753in}}%
\pgfpathlineto{\pgfqpoint{3.548399in}{1.057955in}}%
\pgfpathlineto{\pgfqpoint{3.581568in}{1.109730in}}%
\pgfpathlineto{\pgfqpoint{3.647908in}{1.273203in}}%
\pgfpathlineto{\pgfqpoint{3.681078in}{1.334840in}}%
\pgfpathlineto{\pgfqpoint{3.714248in}{1.376408in}}%
\pgfpathlineto{\pgfqpoint{3.747418in}{1.413807in}}%
\pgfpathlineto{\pgfqpoint{3.780587in}{1.466079in}}%
\pgfpathlineto{\pgfqpoint{3.813757in}{1.541322in}}%
\pgfpathlineto{\pgfqpoint{3.846927in}{1.630036in}}%
\pgfpathlineto{\pgfqpoint{3.880097in}{1.709975in}}%
\pgfpathlineto{\pgfqpoint{3.913267in}{1.759577in}}%
\pgfpathlineto{\pgfqpoint{3.946437in}{1.771752in}}%
\pgfpathlineto{\pgfqpoint{3.979607in}{1.759331in}}%
\pgfpathlineto{\pgfqpoint{4.012776in}{1.748250in}}%
\pgfpathlineto{\pgfqpoint{4.045946in}{1.761871in}}%
\pgfpathlineto{\pgfqpoint{4.079116in}{1.805466in}}%
\pgfpathlineto{\pgfqpoint{4.112286in}{1.860456in}}%
\pgfpathlineto{\pgfqpoint{4.145456in}{1.892817in}}%
\pgfpathlineto{\pgfqpoint{4.178626in}{1.872034in}}%
\pgfpathlineto{\pgfqpoint{4.211795in}{1.790436in}}%
\pgfpathlineto{\pgfqpoint{4.244965in}{1.671734in}}%
\pgfpathlineto{\pgfqpoint{4.278135in}{1.562784in}}%
\pgfpathlineto{\pgfqpoint{4.311305in}{1.511624in}}%
\pgfpathlineto{\pgfqpoint{4.344475in}{1.542575in}}%
\pgfpathlineto{\pgfqpoint{4.377645in}{1.641559in}}%
\pgfpathlineto{\pgfqpoint{4.410815in}{1.760012in}}%
\pgfpathlineto{\pgfqpoint{4.443985in}{1.836356in}}%
\pgfpathlineto{\pgfqpoint{4.477154in}{1.824791in}}%
\pgfpathlineto{\pgfqpoint{4.510324in}{1.717109in}}%
\pgfpathlineto{\pgfqpoint{4.576664in}{1.372000in}}%
\pgfpathlineto{\pgfqpoint{4.609834in}{1.249953in}}%
\pgfpathlineto{\pgfqpoint{4.643003in}{1.209303in}}%
\pgfpathlineto{\pgfqpoint{4.676173in}{1.240268in}}%
\pgfpathlineto{\pgfqpoint{4.709343in}{1.302863in}}%
\pgfpathlineto{\pgfqpoint{4.742513in}{1.349482in}}%
\pgfpathlineto{\pgfqpoint{4.775683in}{1.349491in}}%
\pgfpathlineto{\pgfqpoint{4.808853in}{1.303348in}}%
\pgfpathlineto{\pgfqpoint{4.842023in}{1.239717in}}%
\pgfpathlineto{\pgfqpoint{4.875193in}{1.198161in}}%
\pgfpathlineto{\pgfqpoint{4.908363in}{1.207141in}}%
\pgfpathlineto{\pgfqpoint{4.941532in}{1.268838in}}%
\pgfpathlineto{\pgfqpoint{4.974702in}{1.358026in}}%
\pgfpathlineto{\pgfqpoint{5.007872in}{1.434529in}}%
\pgfpathlineto{\pgfqpoint{5.041042in}{1.461976in}}%
\pgfpathlineto{\pgfqpoint{5.074212in}{1.423098in}}%
\pgfpathlineto{\pgfqpoint{5.107381in}{1.324620in}}%
\pgfpathlineto{\pgfqpoint{5.140551in}{1.190877in}}%
\pgfpathlineto{\pgfqpoint{5.173721in}{1.051157in}}%
\pgfpathlineto{\pgfqpoint{5.206891in}{0.928202in}}%
\pgfpathlineto{\pgfqpoint{5.240061in}{0.833341in}}%
\pgfpathlineto{\pgfqpoint{5.273231in}{0.769078in}}%
\pgfpathlineto{\pgfqpoint{5.306401in}{0.735372in}}%
\pgfpathlineto{\pgfqpoint{5.339570in}{0.734215in}}%
\pgfpathlineto{\pgfqpoint{5.372740in}{0.768952in}}%
\pgfpathlineto{\pgfqpoint{5.405910in}{0.838955in}}%
\pgfpathlineto{\pgfqpoint{5.472250in}{1.031462in}}%
\pgfpathlineto{\pgfqpoint{5.505419in}{1.104351in}}%
\pgfpathlineto{\pgfqpoint{5.538589in}{1.129155in}}%
\pgfpathlineto{\pgfqpoint{5.571759in}{1.097543in}}%
\pgfpathlineto{\pgfqpoint{5.604929in}{1.020680in}}%
\pgfpathlineto{\pgfqpoint{5.638099in}{0.925068in}}%
\pgfpathlineto{\pgfqpoint{5.671269in}{0.841343in}}%
\pgfpathlineto{\pgfqpoint{5.704439in}{0.791489in}}%
\pgfpathlineto{\pgfqpoint{5.737609in}{0.780785in}}%
\pgfpathlineto{\pgfqpoint{5.770779in}{0.798230in}}%
\pgfpathlineto{\pgfqpoint{5.803948in}{0.824672in}}%
\pgfpathlineto{\pgfqpoint{5.837118in}{0.843855in}}%
\pgfpathlineto{\pgfqpoint{5.870288in}{0.850305in}}%
\pgfpathlineto{\pgfqpoint{5.903458in}{0.850059in}}%
\pgfpathlineto{\pgfqpoint{5.936628in}{0.854316in}}%
\pgfpathlineto{\pgfqpoint{5.969798in}{0.870118in}}%
\pgfpathlineto{\pgfqpoint{6.002967in}{0.893663in}}%
\pgfpathlineto{\pgfqpoint{6.036137in}{0.910336in}}%
\pgfpathlineto{\pgfqpoint{6.069307in}{0.901752in}}%
\pgfpathlineto{\pgfqpoint{6.102477in}{0.856243in}}%
\pgfpathlineto{\pgfqpoint{6.135647in}{0.777218in}}%
\pgfpathlineto{\pgfqpoint{6.168817in}{0.684903in}}%
\pgfpathlineto{\pgfqpoint{6.201986in}{0.610099in}}%
\pgfpathlineto{\pgfqpoint{6.235156in}{0.582552in}}%
\pgfpathlineto{\pgfqpoint{6.268326in}{0.619030in}}%
\pgfpathlineto{\pgfqpoint{6.301496in}{0.716200in}}%
\pgfpathlineto{\pgfqpoint{6.367836in}{0.989929in}}%
\pgfpathlineto{\pgfqpoint{6.401006in}{1.098295in}}%
\pgfpathlineto{\pgfqpoint{6.434176in}{1.154185in}}%
\pgfpathlineto{\pgfqpoint{6.467346in}{1.153751in}}%
\pgfpathlineto{\pgfqpoint{6.500515in}{1.111419in}}%
\pgfpathlineto{\pgfqpoint{6.533685in}{1.053607in}}%
\pgfpathlineto{\pgfqpoint{6.566855in}{1.008753in}}%
\pgfpathlineto{\pgfqpoint{6.600025in}{0.997240in}}%
\pgfpathlineto{\pgfqpoint{6.633195in}{1.024705in}}%
\pgfpathlineto{\pgfqpoint{6.666364in}{1.081035in}}%
\pgfpathlineto{\pgfqpoint{6.699534in}{1.145379in}}%
\pgfpathlineto{\pgfqpoint{6.732704in}{1.195375in}}%
\pgfpathlineto{\pgfqpoint{6.765874in}{1.217008in}}%
\pgfpathlineto{\pgfqpoint{6.799044in}{1.210976in}}%
\pgfpathlineto{\pgfqpoint{6.832214in}{1.192569in}}%
\pgfpathlineto{\pgfqpoint{6.865384in}{1.184590in}}%
\pgfpathlineto{\pgfqpoint{6.898553in}{1.206093in}}%
\pgfpathlineto{\pgfqpoint{6.931723in}{1.262034in}}%
\pgfpathlineto{\pgfqpoint{6.964893in}{1.339110in}}%
\pgfpathlineto{\pgfqpoint{6.998063in}{1.410626in}}%
\pgfpathlineto{\pgfqpoint{7.031233in}{1.449018in}}%
\pgfpathlineto{\pgfqpoint{7.064402in}{1.440469in}}%
\pgfpathlineto{\pgfqpoint{7.097572in}{1.394343in}}%
\pgfpathlineto{\pgfqpoint{7.130742in}{1.341839in}}%
\pgfpathlineto{\pgfqpoint{7.163912in}{1.323177in}}%
\pgfpathlineto{\pgfqpoint{7.197082in}{1.368418in}}%
\pgfpathlineto{\pgfqpoint{7.230252in}{1.480852in}}%
\pgfpathlineto{\pgfqpoint{7.263422in}{1.631547in}}%
\pgfpathlineto{\pgfqpoint{7.296592in}{1.768925in}}%
\pgfpathlineto{\pgfqpoint{7.329761in}{1.840075in}}%
\pgfpathlineto{\pgfqpoint{7.362931in}{1.814588in}}%
\pgfpathlineto{\pgfqpoint{7.396101in}{1.699705in}}%
\pgfpathlineto{\pgfqpoint{7.429271in}{1.539211in}}%
\pgfpathlineto{\pgfqpoint{7.462441in}{1.395779in}}%
\pgfpathlineto{\pgfqpoint{7.495611in}{1.324375in}}%
\pgfpathlineto{\pgfqpoint{7.528780in}{1.348612in}}%
\pgfpathlineto{\pgfqpoint{7.561950in}{1.450822in}}%
\pgfpathlineto{\pgfqpoint{7.595120in}{1.580274in}}%
\pgfpathlineto{\pgfqpoint{7.628290in}{1.675638in}}%
\pgfpathlineto{\pgfqpoint{7.628290in}{1.675638in}}%
\pgfusepath{stroke}%
\end{pgfscope}%
\begin{pgfscope}%
\pgfsetrectcap%
\pgfsetmiterjoin%
\pgfsetlinewidth{0.803000pt}%
\definecolor{currentstroke}{rgb}{0.000000,0.000000,0.000000}%
\pgfsetstrokecolor{currentstroke}%
\pgfsetdash{}{0pt}%
\pgfpathmoveto{\pgfqpoint{0.697450in}{0.517039in}}%
\pgfpathlineto{\pgfqpoint{0.697450in}{1.958330in}}%
\pgfusepath{stroke}%
\end{pgfscope}%
\begin{pgfscope}%
\pgfsetrectcap%
\pgfsetmiterjoin%
\pgfsetlinewidth{0.803000pt}%
\definecolor{currentstroke}{rgb}{0.000000,0.000000,0.000000}%
\pgfsetstrokecolor{currentstroke}%
\pgfsetdash{}{0pt}%
\pgfpathmoveto{\pgfqpoint{7.958330in}{0.517039in}}%
\pgfpathlineto{\pgfqpoint{7.958330in}{1.958330in}}%
\pgfusepath{stroke}%
\end{pgfscope}%
\begin{pgfscope}%
\pgfsetrectcap%
\pgfsetmiterjoin%
\pgfsetlinewidth{0.803000pt}%
\definecolor{currentstroke}{rgb}{0.000000,0.000000,0.000000}%
\pgfsetstrokecolor{currentstroke}%
\pgfsetdash{}{0pt}%
\pgfpathmoveto{\pgfqpoint{0.697450in}{0.517039in}}%
\pgfpathlineto{\pgfqpoint{7.958330in}{0.517039in}}%
\pgfusepath{stroke}%
\end{pgfscope}%
\begin{pgfscope}%
\pgfsetrectcap%
\pgfsetmiterjoin%
\pgfsetlinewidth{0.803000pt}%
\definecolor{currentstroke}{rgb}{0.000000,0.000000,0.000000}%
\pgfsetstrokecolor{currentstroke}%
\pgfsetdash{}{0pt}%
\pgfpathmoveto{\pgfqpoint{0.697450in}{1.958330in}}%
\pgfpathlineto{\pgfqpoint{7.958330in}{1.958330in}}%
\pgfusepath{stroke}%
\end{pgfscope}%
\begin{pgfscope}%
\pgfsetbuttcap%
\pgfsetmiterjoin%
\definecolor{currentfill}{rgb}{1.000000,1.000000,1.000000}%
\pgfsetfillcolor{currentfill}%
\pgfsetfillopacity{0.800000}%
\pgfsetlinewidth{1.003750pt}%
\definecolor{currentstroke}{rgb}{0.800000,0.800000,0.800000}%
\pgfsetstrokecolor{currentstroke}%
\pgfsetstrokeopacity{0.800000}%
\pgfsetdash{}{0pt}%
\pgfpathmoveto{\pgfqpoint{6.625713in}{1.091110in}}%
\pgfpathlineto{\pgfqpoint{7.841663in}{1.091110in}}%
\pgfpathquadraticcurveto{\pgfqpoint{7.874997in}{1.091110in}}{\pgfqpoint{7.874997in}{1.124444in}}%
\pgfpathlineto{\pgfqpoint{7.874997in}{1.841663in}}%
\pgfpathquadraticcurveto{\pgfqpoint{7.874997in}{1.874997in}}{\pgfqpoint{7.841663in}{1.874997in}}%
\pgfpathlineto{\pgfqpoint{6.625713in}{1.874997in}}%
\pgfpathquadraticcurveto{\pgfqpoint{6.592379in}{1.874997in}}{\pgfqpoint{6.592379in}{1.841663in}}%
\pgfpathlineto{\pgfqpoint{6.592379in}{1.124444in}}%
\pgfpathquadraticcurveto{\pgfqpoint{6.592379in}{1.091110in}}{\pgfqpoint{6.625713in}{1.091110in}}%
\pgfpathlineto{\pgfqpoint{6.625713in}{1.091110in}}%
\pgfpathclose%
\pgfusepath{stroke,fill}%
\end{pgfscope}%
\begin{pgfscope}%
\pgfsetrectcap%
\pgfsetroundjoin%
\pgfsetlinewidth{1.505625pt}%
\definecolor{currentstroke}{rgb}{0.121569,0.466667,0.705882}%
\pgfsetstrokecolor{currentstroke}%
\pgfsetdash{}{0pt}%
\pgfpathmoveto{\pgfqpoint{6.659046in}{1.740036in}}%
\pgfpathlineto{\pgfqpoint{6.825713in}{1.740036in}}%
\pgfpathlineto{\pgfqpoint{6.992379in}{1.740036in}}%
\pgfusepath{stroke}%
\end{pgfscope}%
\begin{pgfscope}%
\definecolor{textcolor}{rgb}{0.000000,0.000000,0.000000}%
\pgfsetstrokecolor{textcolor}%
\pgfsetfillcolor{textcolor}%
\pgftext[x=7.125713in,y=1.681702in,left,base]{\color{textcolor}{\rmfamily\fontsize{12.000000}{14.400000}\selectfont\catcode`\^=\active\def^{\ifmmode\sp\else\^{}\fi}\catcode`\%=\active\def%{\%}5% std}}%
\end{pgfscope}%
\begin{pgfscope}%
\pgfsetrectcap%
\pgfsetroundjoin%
\pgfsetlinewidth{1.505625pt}%
\definecolor{currentstroke}{rgb}{1.000000,0.498039,0.054902}%
\pgfsetstrokecolor{currentstroke}%
\pgfsetdash{}{0pt}%
\pgfpathmoveto{\pgfqpoint{6.659046in}{1.495407in}}%
\pgfpathlineto{\pgfqpoint{6.825713in}{1.495407in}}%
\pgfpathlineto{\pgfqpoint{6.992379in}{1.495407in}}%
\pgfusepath{stroke}%
\end{pgfscope}%
\begin{pgfscope}%
\definecolor{textcolor}{rgb}{0.000000,0.000000,0.000000}%
\pgfsetstrokecolor{textcolor}%
\pgfsetfillcolor{textcolor}%
\pgftext[x=7.125713in,y=1.437074in,left,base]{\color{textcolor}{\rmfamily\fontsize{12.000000}{14.400000}\selectfont\catcode`\^=\active\def^{\ifmmode\sp\else\^{}\fi}\catcode`\%=\active\def%{\%}10% std}}%
\end{pgfscope}%
\begin{pgfscope}%
\pgfsetrectcap%
\pgfsetroundjoin%
\pgfsetlinewidth{1.505625pt}%
\definecolor{currentstroke}{rgb}{0.172549,0.627451,0.172549}%
\pgfsetstrokecolor{currentstroke}%
\pgfsetdash{}{0pt}%
\pgfpathmoveto{\pgfqpoint{6.659046in}{1.250778in}}%
\pgfpathlineto{\pgfqpoint{6.825713in}{1.250778in}}%
\pgfpathlineto{\pgfqpoint{6.992379in}{1.250778in}}%
\pgfusepath{stroke}%
\end{pgfscope}%
\begin{pgfscope}%
\definecolor{textcolor}{rgb}{0.000000,0.000000,0.000000}%
\pgfsetstrokecolor{textcolor}%
\pgfsetfillcolor{textcolor}%
\pgftext[x=7.125713in,y=1.192445in,left,base]{\color{textcolor}{\rmfamily\fontsize{12.000000}{14.400000}\selectfont\catcode`\^=\active\def^{\ifmmode\sp\else\^{}\fi}\catcode`\%=\active\def%{\%}50% std}}%
\end{pgfscope}%
\end{pgfpicture}%
\makeatother%
\endgroup%

    \end{adjustbox}
    \caption{}\label{fig:antiderivative_exact_input}
  \end{subfigure}
  \begin{subfigure}{\linewidth}
    \begin{adjustbox}{width=\linewidth}
      \begingroup%
\makeatletter%
\begin{pgfpicture}%
\pgfpathrectangle{\pgfpointorigin}{\pgfqpoint{8.000000in}{2.000000in}}%
\pgfusepath{use as bounding box, clip}%
\begin{pgfscope}%
\pgfsetbuttcap%
\pgfsetmiterjoin%
\pgfsetlinewidth{0.000000pt}%
\definecolor{currentstroke}{rgb}{0.000000,0.000000,0.000000}%
\pgfsetstrokecolor{currentstroke}%
\pgfsetstrokeopacity{0.000000}%
\pgfsetdash{}{0pt}%
\pgfpathmoveto{\pgfqpoint{0.000000in}{0.000000in}}%
\pgfpathlineto{\pgfqpoint{8.000000in}{0.000000in}}%
\pgfpathlineto{\pgfqpoint{8.000000in}{2.000000in}}%
\pgfpathlineto{\pgfqpoint{0.000000in}{2.000000in}}%
\pgfpathlineto{\pgfqpoint{0.000000in}{0.000000in}}%
\pgfpathclose%
\pgfusepath{}%
\end{pgfscope}%
\begin{pgfscope}%
\pgfsetbuttcap%
\pgfsetmiterjoin%
\pgfsetlinewidth{0.000000pt}%
\definecolor{currentstroke}{rgb}{0.000000,0.000000,0.000000}%
\pgfsetstrokecolor{currentstroke}%
\pgfsetstrokeopacity{0.000000}%
\pgfsetdash{}{0pt}%
\pgfpathmoveto{\pgfqpoint{0.593772in}{0.517039in}}%
\pgfpathlineto{\pgfqpoint{7.958330in}{0.517039in}}%
\pgfpathlineto{\pgfqpoint{7.958330in}{1.958330in}}%
\pgfpathlineto{\pgfqpoint{0.593772in}{1.958330in}}%
\pgfpathlineto{\pgfqpoint{0.593772in}{0.517039in}}%
\pgfpathclose%
\pgfusepath{}%
\end{pgfscope}%
\begin{pgfscope}%
\pgfsetbuttcap%
\pgfsetroundjoin%
\definecolor{currentfill}{rgb}{0.000000,0.000000,0.000000}%
\pgfsetfillcolor{currentfill}%
\pgfsetlinewidth{0.803000pt}%
\definecolor{currentstroke}{rgb}{0.000000,0.000000,0.000000}%
\pgfsetstrokecolor{currentstroke}%
\pgfsetdash{}{0pt}%
\pgfsys@defobject{currentmarker}{\pgfqpoint{0.000000in}{-0.048611in}}{\pgfqpoint{0.000000in}{0.000000in}}{%
\pgfpathmoveto{\pgfqpoint{0.000000in}{0.000000in}}%
\pgfpathlineto{\pgfqpoint{0.000000in}{-0.048611in}}%
\pgfusepath{stroke,fill}%
}%
\begin{pgfscope}%
\pgfsys@transformshift{0.928524in}{0.517039in}%
\pgfsys@useobject{currentmarker}{}%
\end{pgfscope}%
\end{pgfscope}%
\begin{pgfscope}%
\definecolor{textcolor}{rgb}{0.000000,0.000000,0.000000}%
\pgfsetstrokecolor{textcolor}%
\pgfsetfillcolor{textcolor}%
\pgftext[x=0.928524in,y=0.419816in,,top]{\color{textcolor}{\rmfamily\fontsize{12.000000}{14.400000}\selectfont\catcode`\^=\active\def^{\ifmmode\sp\else\^{}\fi}\catcode`\%=\active\def%{\%}0.00}}%
\end{pgfscope}%
\begin{pgfscope}%
\pgfsetbuttcap%
\pgfsetroundjoin%
\definecolor{currentfill}{rgb}{0.000000,0.000000,0.000000}%
\pgfsetfillcolor{currentfill}%
\pgfsetlinewidth{0.803000pt}%
\definecolor{currentstroke}{rgb}{0.000000,0.000000,0.000000}%
\pgfsetstrokecolor{currentstroke}%
\pgfsetdash{}{0pt}%
\pgfsys@defobject{currentmarker}{\pgfqpoint{0.000000in}{-0.048611in}}{\pgfqpoint{0.000000in}{0.000000in}}{%
\pgfpathmoveto{\pgfqpoint{0.000000in}{0.000000in}}%
\pgfpathlineto{\pgfqpoint{0.000000in}{-0.048611in}}%
\pgfusepath{stroke,fill}%
}%
\begin{pgfscope}%
\pgfsys@transformshift{1.765406in}{0.517039in}%
\pgfsys@useobject{currentmarker}{}%
\end{pgfscope}%
\end{pgfscope}%
\begin{pgfscope}%
\definecolor{textcolor}{rgb}{0.000000,0.000000,0.000000}%
\pgfsetstrokecolor{textcolor}%
\pgfsetfillcolor{textcolor}%
\pgftext[x=1.765406in,y=0.419816in,,top]{\color{textcolor}{\rmfamily\fontsize{12.000000}{14.400000}\selectfont\catcode`\^=\active\def^{\ifmmode\sp\else\^{}\fi}\catcode`\%=\active\def%{\%}0.25}}%
\end{pgfscope}%
\begin{pgfscope}%
\pgfsetbuttcap%
\pgfsetroundjoin%
\definecolor{currentfill}{rgb}{0.000000,0.000000,0.000000}%
\pgfsetfillcolor{currentfill}%
\pgfsetlinewidth{0.803000pt}%
\definecolor{currentstroke}{rgb}{0.000000,0.000000,0.000000}%
\pgfsetstrokecolor{currentstroke}%
\pgfsetdash{}{0pt}%
\pgfsys@defobject{currentmarker}{\pgfqpoint{0.000000in}{-0.048611in}}{\pgfqpoint{0.000000in}{0.000000in}}{%
\pgfpathmoveto{\pgfqpoint{0.000000in}{0.000000in}}%
\pgfpathlineto{\pgfqpoint{0.000000in}{-0.048611in}}%
\pgfusepath{stroke,fill}%
}%
\begin{pgfscope}%
\pgfsys@transformshift{2.602287in}{0.517039in}%
\pgfsys@useobject{currentmarker}{}%
\end{pgfscope}%
\end{pgfscope}%
\begin{pgfscope}%
\definecolor{textcolor}{rgb}{0.000000,0.000000,0.000000}%
\pgfsetstrokecolor{textcolor}%
\pgfsetfillcolor{textcolor}%
\pgftext[x=2.602287in,y=0.419816in,,top]{\color{textcolor}{\rmfamily\fontsize{12.000000}{14.400000}\selectfont\catcode`\^=\active\def^{\ifmmode\sp\else\^{}\fi}\catcode`\%=\active\def%{\%}0.50}}%
\end{pgfscope}%
\begin{pgfscope}%
\pgfsetbuttcap%
\pgfsetroundjoin%
\definecolor{currentfill}{rgb}{0.000000,0.000000,0.000000}%
\pgfsetfillcolor{currentfill}%
\pgfsetlinewidth{0.803000pt}%
\definecolor{currentstroke}{rgb}{0.000000,0.000000,0.000000}%
\pgfsetstrokecolor{currentstroke}%
\pgfsetdash{}{0pt}%
\pgfsys@defobject{currentmarker}{\pgfqpoint{0.000000in}{-0.048611in}}{\pgfqpoint{0.000000in}{0.000000in}}{%
\pgfpathmoveto{\pgfqpoint{0.000000in}{0.000000in}}%
\pgfpathlineto{\pgfqpoint{0.000000in}{-0.048611in}}%
\pgfusepath{stroke,fill}%
}%
\begin{pgfscope}%
\pgfsys@transformshift{3.439169in}{0.517039in}%
\pgfsys@useobject{currentmarker}{}%
\end{pgfscope}%
\end{pgfscope}%
\begin{pgfscope}%
\definecolor{textcolor}{rgb}{0.000000,0.000000,0.000000}%
\pgfsetstrokecolor{textcolor}%
\pgfsetfillcolor{textcolor}%
\pgftext[x=3.439169in,y=0.419816in,,top]{\color{textcolor}{\rmfamily\fontsize{12.000000}{14.400000}\selectfont\catcode`\^=\active\def^{\ifmmode\sp\else\^{}\fi}\catcode`\%=\active\def%{\%}0.75}}%
\end{pgfscope}%
\begin{pgfscope}%
\pgfsetbuttcap%
\pgfsetroundjoin%
\definecolor{currentfill}{rgb}{0.000000,0.000000,0.000000}%
\pgfsetfillcolor{currentfill}%
\pgfsetlinewidth{0.803000pt}%
\definecolor{currentstroke}{rgb}{0.000000,0.000000,0.000000}%
\pgfsetstrokecolor{currentstroke}%
\pgfsetdash{}{0pt}%
\pgfsys@defobject{currentmarker}{\pgfqpoint{0.000000in}{-0.048611in}}{\pgfqpoint{0.000000in}{0.000000in}}{%
\pgfpathmoveto{\pgfqpoint{0.000000in}{0.000000in}}%
\pgfpathlineto{\pgfqpoint{0.000000in}{-0.048611in}}%
\pgfusepath{stroke,fill}%
}%
\begin{pgfscope}%
\pgfsys@transformshift{4.276051in}{0.517039in}%
\pgfsys@useobject{currentmarker}{}%
\end{pgfscope}%
\end{pgfscope}%
\begin{pgfscope}%
\definecolor{textcolor}{rgb}{0.000000,0.000000,0.000000}%
\pgfsetstrokecolor{textcolor}%
\pgfsetfillcolor{textcolor}%
\pgftext[x=4.276051in,y=0.419816in,,top]{\color{textcolor}{\rmfamily\fontsize{12.000000}{14.400000}\selectfont\catcode`\^=\active\def^{\ifmmode\sp\else\^{}\fi}\catcode`\%=\active\def%{\%}1.00}}%
\end{pgfscope}%
\begin{pgfscope}%
\pgfsetbuttcap%
\pgfsetroundjoin%
\definecolor{currentfill}{rgb}{0.000000,0.000000,0.000000}%
\pgfsetfillcolor{currentfill}%
\pgfsetlinewidth{0.803000pt}%
\definecolor{currentstroke}{rgb}{0.000000,0.000000,0.000000}%
\pgfsetstrokecolor{currentstroke}%
\pgfsetdash{}{0pt}%
\pgfsys@defobject{currentmarker}{\pgfqpoint{0.000000in}{-0.048611in}}{\pgfqpoint{0.000000in}{0.000000in}}{%
\pgfpathmoveto{\pgfqpoint{0.000000in}{0.000000in}}%
\pgfpathlineto{\pgfqpoint{0.000000in}{-0.048611in}}%
\pgfusepath{stroke,fill}%
}%
\begin{pgfscope}%
\pgfsys@transformshift{5.112932in}{0.517039in}%
\pgfsys@useobject{currentmarker}{}%
\end{pgfscope}%
\end{pgfscope}%
\begin{pgfscope}%
\definecolor{textcolor}{rgb}{0.000000,0.000000,0.000000}%
\pgfsetstrokecolor{textcolor}%
\pgfsetfillcolor{textcolor}%
\pgftext[x=5.112932in,y=0.419816in,,top]{\color{textcolor}{\rmfamily\fontsize{12.000000}{14.400000}\selectfont\catcode`\^=\active\def^{\ifmmode\sp\else\^{}\fi}\catcode`\%=\active\def%{\%}1.25}}%
\end{pgfscope}%
\begin{pgfscope}%
\pgfsetbuttcap%
\pgfsetroundjoin%
\definecolor{currentfill}{rgb}{0.000000,0.000000,0.000000}%
\pgfsetfillcolor{currentfill}%
\pgfsetlinewidth{0.803000pt}%
\definecolor{currentstroke}{rgb}{0.000000,0.000000,0.000000}%
\pgfsetstrokecolor{currentstroke}%
\pgfsetdash{}{0pt}%
\pgfsys@defobject{currentmarker}{\pgfqpoint{0.000000in}{-0.048611in}}{\pgfqpoint{0.000000in}{0.000000in}}{%
\pgfpathmoveto{\pgfqpoint{0.000000in}{0.000000in}}%
\pgfpathlineto{\pgfqpoint{0.000000in}{-0.048611in}}%
\pgfusepath{stroke,fill}%
}%
\begin{pgfscope}%
\pgfsys@transformshift{5.949814in}{0.517039in}%
\pgfsys@useobject{currentmarker}{}%
\end{pgfscope}%
\end{pgfscope}%
\begin{pgfscope}%
\definecolor{textcolor}{rgb}{0.000000,0.000000,0.000000}%
\pgfsetstrokecolor{textcolor}%
\pgfsetfillcolor{textcolor}%
\pgftext[x=5.949814in,y=0.419816in,,top]{\color{textcolor}{\rmfamily\fontsize{12.000000}{14.400000}\selectfont\catcode`\^=\active\def^{\ifmmode\sp\else\^{}\fi}\catcode`\%=\active\def%{\%}1.50}}%
\end{pgfscope}%
\begin{pgfscope}%
\pgfsetbuttcap%
\pgfsetroundjoin%
\definecolor{currentfill}{rgb}{0.000000,0.000000,0.000000}%
\pgfsetfillcolor{currentfill}%
\pgfsetlinewidth{0.803000pt}%
\definecolor{currentstroke}{rgb}{0.000000,0.000000,0.000000}%
\pgfsetstrokecolor{currentstroke}%
\pgfsetdash{}{0pt}%
\pgfsys@defobject{currentmarker}{\pgfqpoint{0.000000in}{-0.048611in}}{\pgfqpoint{0.000000in}{0.000000in}}{%
\pgfpathmoveto{\pgfqpoint{0.000000in}{0.000000in}}%
\pgfpathlineto{\pgfqpoint{0.000000in}{-0.048611in}}%
\pgfusepath{stroke,fill}%
}%
\begin{pgfscope}%
\pgfsys@transformshift{6.786696in}{0.517039in}%
\pgfsys@useobject{currentmarker}{}%
\end{pgfscope}%
\end{pgfscope}%
\begin{pgfscope}%
\definecolor{textcolor}{rgb}{0.000000,0.000000,0.000000}%
\pgfsetstrokecolor{textcolor}%
\pgfsetfillcolor{textcolor}%
\pgftext[x=6.786696in,y=0.419816in,,top]{\color{textcolor}{\rmfamily\fontsize{12.000000}{14.400000}\selectfont\catcode`\^=\active\def^{\ifmmode\sp\else\^{}\fi}\catcode`\%=\active\def%{\%}1.75}}%
\end{pgfscope}%
\begin{pgfscope}%
\pgfsetbuttcap%
\pgfsetroundjoin%
\definecolor{currentfill}{rgb}{0.000000,0.000000,0.000000}%
\pgfsetfillcolor{currentfill}%
\pgfsetlinewidth{0.803000pt}%
\definecolor{currentstroke}{rgb}{0.000000,0.000000,0.000000}%
\pgfsetstrokecolor{currentstroke}%
\pgfsetdash{}{0pt}%
\pgfsys@defobject{currentmarker}{\pgfqpoint{0.000000in}{-0.048611in}}{\pgfqpoint{0.000000in}{0.000000in}}{%
\pgfpathmoveto{\pgfqpoint{0.000000in}{0.000000in}}%
\pgfpathlineto{\pgfqpoint{0.000000in}{-0.048611in}}%
\pgfusepath{stroke,fill}%
}%
\begin{pgfscope}%
\pgfsys@transformshift{7.623577in}{0.517039in}%
\pgfsys@useobject{currentmarker}{}%
\end{pgfscope}%
\end{pgfscope}%
\begin{pgfscope}%
\definecolor{textcolor}{rgb}{0.000000,0.000000,0.000000}%
\pgfsetstrokecolor{textcolor}%
\pgfsetfillcolor{textcolor}%
\pgftext[x=7.623577in,y=0.419816in,,top]{\color{textcolor}{\rmfamily\fontsize{12.000000}{14.400000}\selectfont\catcode`\^=\active\def^{\ifmmode\sp\else\^{}\fi}\catcode`\%=\active\def%{\%}2.00}}%
\end{pgfscope}%
\begin{pgfscope}%
\definecolor{textcolor}{rgb}{0.000000,0.000000,0.000000}%
\pgfsetstrokecolor{textcolor}%
\pgfsetfillcolor{textcolor}%
\pgftext[x=4.276051in,y=0.202965in,,top]{\color{textcolor}{\rmfamily\fontsize{12.000000}{14.400000}\selectfont\catcode`\^=\active\def^{\ifmmode\sp\else\^{}\fi}\catcode`\%=\active\def%{\%}Time (hours)}}%
\end{pgfscope}%
\begin{pgfscope}%
\pgfsetbuttcap%
\pgfsetroundjoin%
\definecolor{currentfill}{rgb}{0.000000,0.000000,0.000000}%
\pgfsetfillcolor{currentfill}%
\pgfsetlinewidth{0.803000pt}%
\definecolor{currentstroke}{rgb}{0.000000,0.000000,0.000000}%
\pgfsetstrokecolor{currentstroke}%
\pgfsetdash{}{0pt}%
\pgfsys@defobject{currentmarker}{\pgfqpoint{-0.048611in}{0.000000in}}{\pgfqpoint{-0.000000in}{0.000000in}}{%
\pgfpathmoveto{\pgfqpoint{-0.000000in}{0.000000in}}%
\pgfpathlineto{\pgfqpoint{-0.048611in}{0.000000in}}%
\pgfusepath{stroke,fill}%
}%
\begin{pgfscope}%
\pgfsys@transformshift{0.593772in}{0.603813in}%
\pgfsys@useobject{currentmarker}{}%
\end{pgfscope}%
\end{pgfscope}%
\begin{pgfscope}%
\definecolor{textcolor}{rgb}{0.000000,0.000000,0.000000}%
\pgfsetstrokecolor{textcolor}%
\pgfsetfillcolor{textcolor}%
\pgftext[x=0.260881in, y=0.540499in, left, base]{\color{textcolor}{\rmfamily\fontsize{12.000000}{14.400000}\selectfont\catcode`\^=\active\def^{\ifmmode\sp\else\^{}\fi}\catcode`\%=\active\def%{\%}\ensuremath{-}1}}%
\end{pgfscope}%
\begin{pgfscope}%
\pgfsetbuttcap%
\pgfsetroundjoin%
\definecolor{currentfill}{rgb}{0.000000,0.000000,0.000000}%
\pgfsetfillcolor{currentfill}%
\pgfsetlinewidth{0.803000pt}%
\definecolor{currentstroke}{rgb}{0.000000,0.000000,0.000000}%
\pgfsetstrokecolor{currentstroke}%
\pgfsetdash{}{0pt}%
\pgfsys@defobject{currentmarker}{\pgfqpoint{-0.048611in}{0.000000in}}{\pgfqpoint{-0.000000in}{0.000000in}}{%
\pgfpathmoveto{\pgfqpoint{-0.000000in}{0.000000in}}%
\pgfpathlineto{\pgfqpoint{-0.048611in}{0.000000in}}%
\pgfusepath{stroke,fill}%
}%
\begin{pgfscope}%
\pgfsys@transformshift{0.593772in}{1.235446in}%
\pgfsys@useobject{currentmarker}{}%
\end{pgfscope}%
\end{pgfscope}%
\begin{pgfscope}%
\definecolor{textcolor}{rgb}{0.000000,0.000000,0.000000}%
\pgfsetstrokecolor{textcolor}%
\pgfsetfillcolor{textcolor}%
\pgftext[x=0.390511in, y=1.172132in, left, base]{\color{textcolor}{\rmfamily\fontsize{12.000000}{14.400000}\selectfont\catcode`\^=\active\def^{\ifmmode\sp\else\^{}\fi}\catcode`\%=\active\def%{\%}0}}%
\end{pgfscope}%
\begin{pgfscope}%
\pgfsetbuttcap%
\pgfsetroundjoin%
\definecolor{currentfill}{rgb}{0.000000,0.000000,0.000000}%
\pgfsetfillcolor{currentfill}%
\pgfsetlinewidth{0.803000pt}%
\definecolor{currentstroke}{rgb}{0.000000,0.000000,0.000000}%
\pgfsetstrokecolor{currentstroke}%
\pgfsetdash{}{0pt}%
\pgfsys@defobject{currentmarker}{\pgfqpoint{-0.048611in}{0.000000in}}{\pgfqpoint{-0.000000in}{0.000000in}}{%
\pgfpathmoveto{\pgfqpoint{-0.000000in}{0.000000in}}%
\pgfpathlineto{\pgfqpoint{-0.048611in}{0.000000in}}%
\pgfusepath{stroke,fill}%
}%
\begin{pgfscope}%
\pgfsys@transformshift{0.593772in}{1.867080in}%
\pgfsys@useobject{currentmarker}{}%
\end{pgfscope}%
\end{pgfscope}%
\begin{pgfscope}%
\definecolor{textcolor}{rgb}{0.000000,0.000000,0.000000}%
\pgfsetstrokecolor{textcolor}%
\pgfsetfillcolor{textcolor}%
\pgftext[x=0.390511in, y=1.803766in, left, base]{\color{textcolor}{\rmfamily\fontsize{12.000000}{14.400000}\selectfont\catcode`\^=\active\def^{\ifmmode\sp\else\^{}\fi}\catcode`\%=\active\def%{\%}1}}%
\end{pgfscope}%
\begin{pgfscope}%
\definecolor{textcolor}{rgb}{0.000000,0.000000,0.000000}%
\pgfsetstrokecolor{textcolor}%
\pgfsetfillcolor{textcolor}%
\pgftext[x=0.205325in,y=1.237684in,,bottom,rotate=90.000000]{\color{textcolor}{\rmfamily\fontsize{12.000000}{14.400000}\selectfont\catcode`\^=\active\def^{\ifmmode\sp\else\^{}\fi}\catcode`\%=\active\def%{\%}Velocity}}%
\end{pgfscope}%
\begin{pgfscope}%
\pgfpathrectangle{\pgfqpoint{0.593772in}{0.517039in}}{\pgfqpoint{7.364558in}{1.441291in}}%
\pgfusepath{clip}%
\pgfsetrectcap%
\pgfsetroundjoin%
\pgfsetlinewidth{1.505625pt}%
\definecolor{currentstroke}{rgb}{0.121569,0.466667,0.705882}%
\pgfsetstrokecolor{currentstroke}%
\pgfsetdash{}{0pt}%
\pgfpathmoveto{\pgfqpoint{0.928524in}{1.296620in}}%
\pgfpathlineto{\pgfqpoint{0.962168in}{1.328422in}}%
\pgfpathlineto{\pgfqpoint{0.995811in}{1.363146in}}%
\pgfpathlineto{\pgfqpoint{1.029455in}{1.401294in}}%
\pgfpathlineto{\pgfqpoint{1.164029in}{1.563328in}}%
\pgfpathlineto{\pgfqpoint{1.197672in}{1.598715in}}%
\pgfpathlineto{\pgfqpoint{1.231316in}{1.630898in}}%
\pgfpathlineto{\pgfqpoint{1.264959in}{1.660236in}}%
\pgfpathlineto{\pgfqpoint{1.298603in}{1.687161in}}%
\pgfpathlineto{\pgfqpoint{1.332246in}{1.711972in}}%
\pgfpathlineto{\pgfqpoint{1.365889in}{1.734849in}}%
\pgfpathlineto{\pgfqpoint{1.433176in}{1.775900in}}%
\pgfpathlineto{\pgfqpoint{1.534107in}{1.833030in}}%
\pgfpathlineto{\pgfqpoint{1.567750in}{1.851305in}}%
\pgfpathlineto{\pgfqpoint{1.601394in}{1.867695in}}%
\pgfpathlineto{\pgfqpoint{1.635037in}{1.880711in}}%
\pgfpathlineto{\pgfqpoint{1.668681in}{1.889189in}}%
\pgfpathlineto{\pgfqpoint{1.702324in}{1.892817in}}%
\pgfpathlineto{\pgfqpoint{1.735968in}{1.892308in}}%
\pgfpathlineto{\pgfqpoint{1.769611in}{1.889107in}}%
\pgfpathlineto{\pgfqpoint{1.836898in}{1.879958in}}%
\pgfpathlineto{\pgfqpoint{1.870542in}{1.874552in}}%
\pgfpathlineto{\pgfqpoint{1.904185in}{1.867203in}}%
\pgfpathlineto{\pgfqpoint{1.937829in}{1.856210in}}%
\pgfpathlineto{\pgfqpoint{1.971472in}{1.840295in}}%
\pgfpathlineto{\pgfqpoint{2.005116in}{1.819241in}}%
\pgfpathlineto{\pgfqpoint{2.038759in}{1.794023in}}%
\pgfpathlineto{\pgfqpoint{2.106046in}{1.737844in}}%
\pgfpathlineto{\pgfqpoint{2.173333in}{1.680611in}}%
\pgfpathlineto{\pgfqpoint{2.206977in}{1.650628in}}%
\pgfpathlineto{\pgfqpoint{2.240620in}{1.618420in}}%
\pgfpathlineto{\pgfqpoint{2.307907in}{1.547762in}}%
\pgfpathlineto{\pgfqpoint{2.341550in}{1.512113in}}%
\pgfpathlineto{\pgfqpoint{2.375194in}{1.478673in}}%
\pgfpathlineto{\pgfqpoint{2.408837in}{1.448259in}}%
\pgfpathlineto{\pgfqpoint{2.476124in}{1.391878in}}%
\pgfpathlineto{\pgfqpoint{2.509768in}{1.360791in}}%
\pgfpathlineto{\pgfqpoint{2.543411in}{1.324600in}}%
\pgfpathlineto{\pgfqpoint{2.577055in}{1.282762in}}%
\pgfpathlineto{\pgfqpoint{2.677985in}{1.144405in}}%
\pgfpathlineto{\pgfqpoint{2.711629in}{1.103640in}}%
\pgfpathlineto{\pgfqpoint{2.745272in}{1.067742in}}%
\pgfpathlineto{\pgfqpoint{2.879846in}{0.938031in}}%
\pgfpathlineto{\pgfqpoint{2.947133in}{0.867916in}}%
\pgfpathlineto{\pgfqpoint{2.980777in}{0.836123in}}%
\pgfpathlineto{\pgfqpoint{3.014420in}{0.808856in}}%
\pgfpathlineto{\pgfqpoint{3.048064in}{0.786275in}}%
\pgfpathlineto{\pgfqpoint{3.115351in}{0.749078in}}%
\pgfpathlineto{\pgfqpoint{3.148994in}{0.730265in}}%
\pgfpathlineto{\pgfqpoint{3.216281in}{0.687849in}}%
\pgfpathlineto{\pgfqpoint{3.249924in}{0.666773in}}%
\pgfpathlineto{\pgfqpoint{3.283568in}{0.649054in}}%
\pgfpathlineto{\pgfqpoint{3.317212in}{0.637000in}}%
\pgfpathlineto{\pgfqpoint{3.350855in}{0.631782in}}%
\pgfpathlineto{\pgfqpoint{3.384498in}{0.633051in}}%
\pgfpathlineto{\pgfqpoint{3.418142in}{0.639148in}}%
\pgfpathlineto{\pgfqpoint{3.552716in}{0.674533in}}%
\pgfpathlineto{\pgfqpoint{3.586359in}{0.684488in}}%
\pgfpathlineto{\pgfqpoint{3.620003in}{0.697371in}}%
\pgfpathlineto{\pgfqpoint{3.653646in}{0.714365in}}%
\pgfpathlineto{\pgfqpoint{3.687290in}{0.735555in}}%
\pgfpathlineto{\pgfqpoint{3.754577in}{0.785400in}}%
\pgfpathlineto{\pgfqpoint{3.821864in}{0.834412in}}%
\pgfpathlineto{\pgfqpoint{3.889151in}{0.882037in}}%
\pgfpathlineto{\pgfqpoint{3.922794in}{0.909679in}}%
\pgfpathlineto{\pgfqpoint{3.956438in}{0.942159in}}%
\pgfpathlineto{\pgfqpoint{3.990081in}{0.979651in}}%
\pgfpathlineto{\pgfqpoint{4.091012in}{1.103620in}}%
\pgfpathlineto{\pgfqpoint{4.124655in}{1.141246in}}%
\pgfpathlineto{\pgfqpoint{4.191942in}{1.208847in}}%
\pgfpathlineto{\pgfqpoint{4.225586in}{1.242642in}}%
\pgfpathlineto{\pgfqpoint{4.259229in}{1.279172in}}%
\pgfpathlineto{\pgfqpoint{4.292873in}{1.319215in}}%
\pgfpathlineto{\pgfqpoint{4.393803in}{1.448333in}}%
\pgfpathlineto{\pgfqpoint{4.427446in}{1.488030in}}%
\pgfpathlineto{\pgfqpoint{4.461090in}{1.524365in}}%
\pgfpathlineto{\pgfqpoint{4.494734in}{1.557800in}}%
\pgfpathlineto{\pgfqpoint{4.562020in}{1.619607in}}%
\pgfpathlineto{\pgfqpoint{4.629307in}{1.676873in}}%
\pgfpathlineto{\pgfqpoint{4.662951in}{1.702679in}}%
\pgfpathlineto{\pgfqpoint{4.696595in}{1.725851in}}%
\pgfpathlineto{\pgfqpoint{4.730238in}{1.746371in}}%
\pgfpathlineto{\pgfqpoint{4.763881in}{1.764684in}}%
\pgfpathlineto{\pgfqpoint{4.797525in}{1.781383in}}%
\pgfpathlineto{\pgfqpoint{4.831168in}{1.796823in}}%
\pgfpathlineto{\pgfqpoint{4.864812in}{1.810889in}}%
\pgfpathlineto{\pgfqpoint{4.898455in}{1.823054in}}%
\pgfpathlineto{\pgfqpoint{4.932098in}{1.832658in}}%
\pgfpathlineto{\pgfqpoint{4.965742in}{1.839213in}}%
\pgfpathlineto{\pgfqpoint{4.999386in}{1.842533in}}%
\pgfpathlineto{\pgfqpoint{5.033029in}{1.842618in}}%
\pgfpathlineto{\pgfqpoint{5.066673in}{1.839408in}}%
\pgfpathlineto{\pgfqpoint{5.100316in}{1.832610in}}%
\pgfpathlineto{\pgfqpoint{5.133959in}{1.821797in}}%
\pgfpathlineto{\pgfqpoint{5.167603in}{1.806756in}}%
\pgfpathlineto{\pgfqpoint{5.201247in}{1.787921in}}%
\pgfpathlineto{\pgfqpoint{5.302177in}{1.723756in}}%
\pgfpathlineto{\pgfqpoint{5.335820in}{1.704953in}}%
\pgfpathlineto{\pgfqpoint{5.403108in}{1.669559in}}%
\pgfpathlineto{\pgfqpoint{5.436751in}{1.648202in}}%
\pgfpathlineto{\pgfqpoint{5.470394in}{1.621214in}}%
\pgfpathlineto{\pgfqpoint{5.504038in}{1.587775in}}%
\pgfpathlineto{\pgfqpoint{5.537682in}{1.548925in}}%
\pgfpathlineto{\pgfqpoint{5.638612in}{1.425281in}}%
\pgfpathlineto{\pgfqpoint{5.705899in}{1.351600in}}%
\pgfpathlineto{\pgfqpoint{5.773186in}{1.277366in}}%
\pgfpathlineto{\pgfqpoint{5.874116in}{1.159648in}}%
\pgfpathlineto{\pgfqpoint{5.907760in}{1.123822in}}%
\pgfpathlineto{\pgfqpoint{6.042334in}{0.990235in}}%
\pgfpathlineto{\pgfqpoint{6.109621in}{0.911412in}}%
\pgfpathlineto{\pgfqpoint{6.143264in}{0.872106in}}%
\pgfpathlineto{\pgfqpoint{6.176907in}{0.836379in}}%
\pgfpathlineto{\pgfqpoint{6.210551in}{0.805735in}}%
\pgfpathlineto{\pgfqpoint{6.244195in}{0.779880in}}%
\pgfpathlineto{\pgfqpoint{6.345125in}{0.710621in}}%
\pgfpathlineto{\pgfqpoint{6.412412in}{0.659375in}}%
\pgfpathlineto{\pgfqpoint{6.446056in}{0.635723in}}%
\pgfpathlineto{\pgfqpoint{6.479699in}{0.616412in}}%
\pgfpathlineto{\pgfqpoint{6.513343in}{0.602569in}}%
\pgfpathlineto{\pgfqpoint{6.546986in}{0.593847in}}%
\pgfpathlineto{\pgfqpoint{6.580629in}{0.588791in}}%
\pgfpathlineto{\pgfqpoint{6.647916in}{0.583604in}}%
\pgfpathlineto{\pgfqpoint{6.681560in}{0.582552in}}%
\pgfpathlineto{\pgfqpoint{6.715203in}{0.583585in}}%
\pgfpathlineto{\pgfqpoint{6.748847in}{0.587955in}}%
\pgfpathlineto{\pgfqpoint{6.782490in}{0.596347in}}%
\pgfpathlineto{\pgfqpoint{6.816134in}{0.608422in}}%
\pgfpathlineto{\pgfqpoint{6.883421in}{0.638426in}}%
\pgfpathlineto{\pgfqpoint{6.950708in}{0.669568in}}%
\pgfpathlineto{\pgfqpoint{6.984351in}{0.686545in}}%
\pgfpathlineto{\pgfqpoint{7.017995in}{0.706489in}}%
\pgfpathlineto{\pgfqpoint{7.051638in}{0.730613in}}%
\pgfpathlineto{\pgfqpoint{7.085282in}{0.759005in}}%
\pgfpathlineto{\pgfqpoint{7.186212in}{0.854030in}}%
\pgfpathlineto{\pgfqpoint{7.219856in}{0.882989in}}%
\pgfpathlineto{\pgfqpoint{7.287143in}{0.936796in}}%
\pgfpathlineto{\pgfqpoint{7.320786in}{0.965644in}}%
\pgfpathlineto{\pgfqpoint{7.354429in}{0.998479in}}%
\pgfpathlineto{\pgfqpoint{7.388073in}{1.035963in}}%
\pgfpathlineto{\pgfqpoint{7.455360in}{1.119867in}}%
\pgfpathlineto{\pgfqpoint{7.489004in}{1.161455in}}%
\pgfpathlineto{\pgfqpoint{7.522647in}{1.199889in}}%
\pgfpathlineto{\pgfqpoint{7.556290in}{1.234475in}}%
\pgfpathlineto{\pgfqpoint{7.623577in}{1.296620in}}%
\pgfpathlineto{\pgfqpoint{7.623577in}{1.296620in}}%
\pgfusepath{stroke}%
\end{pgfscope}%
\begin{pgfscope}%
\pgfpathrectangle{\pgfqpoint{0.593772in}{0.517039in}}{\pgfqpoint{7.364558in}{1.441291in}}%
\pgfusepath{clip}%
\pgfsetrectcap%
\pgfsetroundjoin%
\pgfsetlinewidth{1.505625pt}%
\definecolor{currentstroke}{rgb}{1.000000,0.498039,0.054902}%
\pgfsetstrokecolor{currentstroke}%
\pgfsetdash{}{0pt}%
\pgfpathmoveto{\pgfqpoint{0.928524in}{1.325640in}}%
\pgfpathlineto{\pgfqpoint{0.995811in}{1.375983in}}%
\pgfpathlineto{\pgfqpoint{1.130385in}{1.478479in}}%
\pgfpathlineto{\pgfqpoint{1.197672in}{1.526577in}}%
\pgfpathlineto{\pgfqpoint{1.231316in}{1.548562in}}%
\pgfpathlineto{\pgfqpoint{1.264959in}{1.568046in}}%
\pgfpathlineto{\pgfqpoint{1.298603in}{1.584238in}}%
\pgfpathlineto{\pgfqpoint{1.332246in}{1.596821in}}%
\pgfpathlineto{\pgfqpoint{1.365889in}{1.606159in}}%
\pgfpathlineto{\pgfqpoint{1.433176in}{1.619182in}}%
\pgfpathlineto{\pgfqpoint{1.500463in}{1.632021in}}%
\pgfpathlineto{\pgfqpoint{1.534107in}{1.640006in}}%
\pgfpathlineto{\pgfqpoint{1.567750in}{1.649244in}}%
\pgfpathlineto{\pgfqpoint{1.635037in}{1.670878in}}%
\pgfpathlineto{\pgfqpoint{1.702324in}{1.692531in}}%
\pgfpathlineto{\pgfqpoint{1.735968in}{1.700059in}}%
\pgfpathlineto{\pgfqpoint{1.769611in}{1.703215in}}%
\pgfpathlineto{\pgfqpoint{1.803255in}{1.701081in}}%
\pgfpathlineto{\pgfqpoint{1.836898in}{1.694022in}}%
\pgfpathlineto{\pgfqpoint{1.937829in}{1.664361in}}%
\pgfpathlineto{\pgfqpoint{1.971472in}{1.658842in}}%
\pgfpathlineto{\pgfqpoint{2.038759in}{1.652847in}}%
\pgfpathlineto{\pgfqpoint{2.072403in}{1.646724in}}%
\pgfpathlineto{\pgfqpoint{2.106046in}{1.634885in}}%
\pgfpathlineto{\pgfqpoint{2.139690in}{1.616487in}}%
\pgfpathlineto{\pgfqpoint{2.173333in}{1.592675in}}%
\pgfpathlineto{\pgfqpoint{2.341550in}{1.458255in}}%
\pgfpathlineto{\pgfqpoint{2.375194in}{1.427179in}}%
\pgfpathlineto{\pgfqpoint{2.408837in}{1.391919in}}%
\pgfpathlineto{\pgfqpoint{2.509768in}{1.277133in}}%
\pgfpathlineto{\pgfqpoint{2.543411in}{1.244011in}}%
\pgfpathlineto{\pgfqpoint{2.577055in}{1.214988in}}%
\pgfpathlineto{\pgfqpoint{2.644342in}{1.161998in}}%
\pgfpathlineto{\pgfqpoint{2.677985in}{1.133734in}}%
\pgfpathlineto{\pgfqpoint{2.778916in}{1.041476in}}%
\pgfpathlineto{\pgfqpoint{2.812559in}{1.014433in}}%
\pgfpathlineto{\pgfqpoint{2.846203in}{0.991557in}}%
\pgfpathlineto{\pgfqpoint{2.879846in}{0.972380in}}%
\pgfpathlineto{\pgfqpoint{2.947133in}{0.939446in}}%
\pgfpathlineto{\pgfqpoint{3.014420in}{0.907850in}}%
\pgfpathlineto{\pgfqpoint{3.048064in}{0.893251in}}%
\pgfpathlineto{\pgfqpoint{3.081707in}{0.880218in}}%
\pgfpathlineto{\pgfqpoint{3.148994in}{0.856795in}}%
\pgfpathlineto{\pgfqpoint{3.182638in}{0.843732in}}%
\pgfpathlineto{\pgfqpoint{3.216281in}{0.828290in}}%
\pgfpathlineto{\pgfqpoint{3.317212in}{0.775996in}}%
\pgfpathlineto{\pgfqpoint{3.350855in}{0.763494in}}%
\pgfpathlineto{\pgfqpoint{3.384498in}{0.756363in}}%
\pgfpathlineto{\pgfqpoint{3.418142in}{0.754794in}}%
\pgfpathlineto{\pgfqpoint{3.451785in}{0.758011in}}%
\pgfpathlineto{\pgfqpoint{3.485429in}{0.764830in}}%
\pgfpathlineto{\pgfqpoint{3.519072in}{0.774198in}}%
\pgfpathlineto{\pgfqpoint{3.552716in}{0.785403in}}%
\pgfpathlineto{\pgfqpoint{3.620003in}{0.811213in}}%
\pgfpathlineto{\pgfqpoint{3.687290in}{0.836943in}}%
\pgfpathlineto{\pgfqpoint{3.788220in}{0.872137in}}%
\pgfpathlineto{\pgfqpoint{3.821864in}{0.887927in}}%
\pgfpathlineto{\pgfqpoint{3.855507in}{0.908538in}}%
\pgfpathlineto{\pgfqpoint{3.889151in}{0.934316in}}%
\pgfpathlineto{\pgfqpoint{3.922794in}{0.964251in}}%
\pgfpathlineto{\pgfqpoint{4.023725in}{1.062091in}}%
\pgfpathlineto{\pgfqpoint{4.057368in}{1.095249in}}%
\pgfpathlineto{\pgfqpoint{4.091012in}{1.130011in}}%
\pgfpathlineto{\pgfqpoint{4.158298in}{1.205867in}}%
\pgfpathlineto{\pgfqpoint{4.191942in}{1.243916in}}%
\pgfpathlineto{\pgfqpoint{4.225586in}{1.278064in}}%
\pgfpathlineto{\pgfqpoint{4.259229in}{1.305784in}}%
\pgfpathlineto{\pgfqpoint{4.292873in}{1.326692in}}%
\pgfpathlineto{\pgfqpoint{4.360159in}{1.359379in}}%
\pgfpathlineto{\pgfqpoint{4.393803in}{1.379962in}}%
\pgfpathlineto{\pgfqpoint{4.427446in}{1.407490in}}%
\pgfpathlineto{\pgfqpoint{4.461090in}{1.441474in}}%
\pgfpathlineto{\pgfqpoint{4.528377in}{1.513923in}}%
\pgfpathlineto{\pgfqpoint{4.562020in}{1.543917in}}%
\pgfpathlineto{\pgfqpoint{4.595664in}{1.567365in}}%
\pgfpathlineto{\pgfqpoint{4.696595in}{1.623640in}}%
\pgfpathlineto{\pgfqpoint{4.797525in}{1.692113in}}%
\pgfpathlineto{\pgfqpoint{4.831168in}{1.706425in}}%
\pgfpathlineto{\pgfqpoint{4.864812in}{1.711822in}}%
\pgfpathlineto{\pgfqpoint{4.898455in}{1.709569in}}%
\pgfpathlineto{\pgfqpoint{4.965742in}{1.698555in}}%
\pgfpathlineto{\pgfqpoint{4.999386in}{1.697518in}}%
\pgfpathlineto{\pgfqpoint{5.066673in}{1.704068in}}%
\pgfpathlineto{\pgfqpoint{5.100316in}{1.703801in}}%
\pgfpathlineto{\pgfqpoint{5.133959in}{1.695810in}}%
\pgfpathlineto{\pgfqpoint{5.167603in}{1.679135in}}%
\pgfpathlineto{\pgfqpoint{5.268534in}{1.611600in}}%
\pgfpathlineto{\pgfqpoint{5.302177in}{1.597988in}}%
\pgfpathlineto{\pgfqpoint{5.335820in}{1.590981in}}%
\pgfpathlineto{\pgfqpoint{5.369464in}{1.587289in}}%
\pgfpathlineto{\pgfqpoint{5.403108in}{1.582101in}}%
\pgfpathlineto{\pgfqpoint{5.436751in}{1.571156in}}%
\pgfpathlineto{\pgfqpoint{5.470394in}{1.552370in}}%
\pgfpathlineto{\pgfqpoint{5.504038in}{1.526349in}}%
\pgfpathlineto{\pgfqpoint{5.571325in}{1.463635in}}%
\pgfpathlineto{\pgfqpoint{5.604968in}{1.432513in}}%
\pgfpathlineto{\pgfqpoint{5.672255in}{1.374865in}}%
\pgfpathlineto{\pgfqpoint{5.739543in}{1.316663in}}%
\pgfpathlineto{\pgfqpoint{5.907760in}{1.162459in}}%
\pgfpathlineto{\pgfqpoint{5.975047in}{1.104330in}}%
\pgfpathlineto{\pgfqpoint{6.042334in}{1.042452in}}%
\pgfpathlineto{\pgfqpoint{6.109621in}{0.979946in}}%
\pgfpathlineto{\pgfqpoint{6.143264in}{0.951490in}}%
\pgfpathlineto{\pgfqpoint{6.176907in}{0.926275in}}%
\pgfpathlineto{\pgfqpoint{6.210551in}{0.904698in}}%
\pgfpathlineto{\pgfqpoint{6.244195in}{0.886663in}}%
\pgfpathlineto{\pgfqpoint{6.277838in}{0.871771in}}%
\pgfpathlineto{\pgfqpoint{6.311482in}{0.859498in}}%
\pgfpathlineto{\pgfqpoint{6.345125in}{0.849270in}}%
\pgfpathlineto{\pgfqpoint{6.479699in}{0.813272in}}%
\pgfpathlineto{\pgfqpoint{6.513343in}{0.801494in}}%
\pgfpathlineto{\pgfqpoint{6.580629in}{0.775033in}}%
\pgfpathlineto{\pgfqpoint{6.614273in}{0.763524in}}%
\pgfpathlineto{\pgfqpoint{6.647916in}{0.755877in}}%
\pgfpathlineto{\pgfqpoint{6.681560in}{0.753577in}}%
\pgfpathlineto{\pgfqpoint{6.715203in}{0.756898in}}%
\pgfpathlineto{\pgfqpoint{6.748847in}{0.764670in}}%
\pgfpathlineto{\pgfqpoint{6.816134in}{0.784027in}}%
\pgfpathlineto{\pgfqpoint{6.849777in}{0.791047in}}%
\pgfpathlineto{\pgfqpoint{6.883421in}{0.795234in}}%
\pgfpathlineto{\pgfqpoint{6.917064in}{0.797962in}}%
\pgfpathlineto{\pgfqpoint{6.950708in}{0.801936in}}%
\pgfpathlineto{\pgfqpoint{6.984351in}{0.810148in}}%
\pgfpathlineto{\pgfqpoint{7.017995in}{0.824657in}}%
\pgfpathlineto{\pgfqpoint{7.051638in}{0.845690in}}%
\pgfpathlineto{\pgfqpoint{7.085282in}{0.871491in}}%
\pgfpathlineto{\pgfqpoint{7.152568in}{0.925066in}}%
\pgfpathlineto{\pgfqpoint{7.186212in}{0.947825in}}%
\pgfpathlineto{\pgfqpoint{7.287143in}{1.006239in}}%
\pgfpathlineto{\pgfqpoint{7.320786in}{1.031380in}}%
\pgfpathlineto{\pgfqpoint{7.354429in}{1.062555in}}%
\pgfpathlineto{\pgfqpoint{7.388073in}{1.099050in}}%
\pgfpathlineto{\pgfqpoint{7.455360in}{1.177730in}}%
\pgfpathlineto{\pgfqpoint{7.489004in}{1.214203in}}%
\pgfpathlineto{\pgfqpoint{7.522647in}{1.246577in}}%
\pgfpathlineto{\pgfqpoint{7.556290in}{1.275046in}}%
\pgfpathlineto{\pgfqpoint{7.623577in}{1.325640in}}%
\pgfpathlineto{\pgfqpoint{7.623577in}{1.325640in}}%
\pgfusepath{stroke}%
\end{pgfscope}%
\begin{pgfscope}%
\pgfpathrectangle{\pgfqpoint{0.593772in}{0.517039in}}{\pgfqpoint{7.364558in}{1.441291in}}%
\pgfusepath{clip}%
\pgfsetrectcap%
\pgfsetroundjoin%
\pgfsetlinewidth{1.505625pt}%
\definecolor{currentstroke}{rgb}{0.172549,0.627451,0.172549}%
\pgfsetstrokecolor{currentstroke}%
\pgfsetdash{}{0pt}%
\pgfpathmoveto{\pgfqpoint{0.928524in}{1.302300in}}%
\pgfpathlineto{\pgfqpoint{0.962168in}{1.306771in}}%
\pgfpathlineto{\pgfqpoint{0.995811in}{1.307067in}}%
\pgfpathlineto{\pgfqpoint{1.029455in}{1.300935in}}%
\pgfpathlineto{\pgfqpoint{1.063098in}{1.287387in}}%
\pgfpathlineto{\pgfqpoint{1.096742in}{1.267318in}}%
\pgfpathlineto{\pgfqpoint{1.130385in}{1.243969in}}%
\pgfpathlineto{\pgfqpoint{1.164029in}{1.222835in}}%
\pgfpathlineto{\pgfqpoint{1.197672in}{1.210584in}}%
\pgfpathlineto{\pgfqpoint{1.231316in}{1.212962in}}%
\pgfpathlineto{\pgfqpoint{1.264959in}{1.232239in}}%
\pgfpathlineto{\pgfqpoint{1.298603in}{1.265317in}}%
\pgfpathlineto{\pgfqpoint{1.332246in}{1.303698in}}%
\pgfpathlineto{\pgfqpoint{1.365889in}{1.335843in}}%
\pgfpathlineto{\pgfqpoint{1.399533in}{1.351381in}}%
\pgfpathlineto{\pgfqpoint{1.433176in}{1.345521in}}%
\pgfpathlineto{\pgfqpoint{1.466820in}{1.321609in}}%
\pgfpathlineto{\pgfqpoint{1.500463in}{1.290353in}}%
\pgfpathlineto{\pgfqpoint{1.534107in}{1.265661in}}%
\pgfpathlineto{\pgfqpoint{1.567750in}{1.258721in}}%
\pgfpathlineto{\pgfqpoint{1.601394in}{1.272929in}}%
\pgfpathlineto{\pgfqpoint{1.635037in}{1.302140in}}%
\pgfpathlineto{\pgfqpoint{1.668681in}{1.333250in}}%
\pgfpathlineto{\pgfqpoint{1.702324in}{1.352089in}}%
\pgfpathlineto{\pgfqpoint{1.735968in}{1.349968in}}%
\pgfpathlineto{\pgfqpoint{1.769611in}{1.327771in}}%
\pgfpathlineto{\pgfqpoint{1.803255in}{1.295574in}}%
\pgfpathlineto{\pgfqpoint{1.836898in}{1.267900in}}%
\pgfpathlineto{\pgfqpoint{1.870542in}{1.256833in}}%
\pgfpathlineto{\pgfqpoint{1.904185in}{1.266273in}}%
\pgfpathlineto{\pgfqpoint{1.971472in}{1.315123in}}%
\pgfpathlineto{\pgfqpoint{2.005116in}{1.327348in}}%
\pgfpathlineto{\pgfqpoint{2.038759in}{1.318532in}}%
\pgfpathlineto{\pgfqpoint{2.072403in}{1.289875in}}%
\pgfpathlineto{\pgfqpoint{2.106046in}{1.251172in}}%
\pgfpathlineto{\pgfqpoint{2.139690in}{1.215992in}}%
\pgfpathlineto{\pgfqpoint{2.173333in}{1.195347in}}%
\pgfpathlineto{\pgfqpoint{2.206977in}{1.192951in}}%
\pgfpathlineto{\pgfqpoint{2.240620in}{1.204257in}}%
\pgfpathlineto{\pgfqpoint{2.274264in}{1.219469in}}%
\pgfpathlineto{\pgfqpoint{2.307907in}{1.228797in}}%
\pgfpathlineto{\pgfqpoint{2.341550in}{1.227204in}}%
\pgfpathlineto{\pgfqpoint{2.375194in}{1.216338in}}%
\pgfpathlineto{\pgfqpoint{2.408837in}{1.202926in}}%
\pgfpathlineto{\pgfqpoint{2.442481in}{1.194756in}}%
\pgfpathlineto{\pgfqpoint{2.476124in}{1.196508in}}%
\pgfpathlineto{\pgfqpoint{2.509768in}{1.207625in}}%
\pgfpathlineto{\pgfqpoint{2.543411in}{1.223173in}}%
\pgfpathlineto{\pgfqpoint{2.577055in}{1.236975in}}%
\pgfpathlineto{\pgfqpoint{2.610698in}{1.245147in}}%
\pgfpathlineto{\pgfqpoint{2.644342in}{1.248034in}}%
\pgfpathlineto{\pgfqpoint{2.677985in}{1.249532in}}%
\pgfpathlineto{\pgfqpoint{2.711629in}{1.254269in}}%
\pgfpathlineto{\pgfqpoint{2.745272in}{1.264309in}}%
\pgfpathlineto{\pgfqpoint{2.778916in}{1.277273in}}%
\pgfpathlineto{\pgfqpoint{2.812559in}{1.286993in}}%
\pgfpathlineto{\pgfqpoint{2.846203in}{1.286403in}}%
\pgfpathlineto{\pgfqpoint{2.879846in}{1.271181in}}%
\pgfpathlineto{\pgfqpoint{2.913490in}{1.242218in}}%
\pgfpathlineto{\pgfqpoint{2.980777in}{1.170240in}}%
\pgfpathlineto{\pgfqpoint{3.014420in}{1.143887in}}%
\pgfpathlineto{\pgfqpoint{3.048064in}{1.130228in}}%
\pgfpathlineto{\pgfqpoint{3.081707in}{1.127703in}}%
\pgfpathlineto{\pgfqpoint{3.148994in}{1.134036in}}%
\pgfpathlineto{\pgfqpoint{3.182638in}{1.133331in}}%
\pgfpathlineto{\pgfqpoint{3.249924in}{1.126238in}}%
\pgfpathlineto{\pgfqpoint{3.283568in}{1.128925in}}%
\pgfpathlineto{\pgfqpoint{3.317212in}{1.140646in}}%
\pgfpathlineto{\pgfqpoint{3.350855in}{1.160737in}}%
\pgfpathlineto{\pgfqpoint{3.384498in}{1.184595in}}%
\pgfpathlineto{\pgfqpoint{3.418142in}{1.205454in}}%
\pgfpathlineto{\pgfqpoint{3.451785in}{1.217134in}}%
\pgfpathlineto{\pgfqpoint{3.485429in}{1.216463in}}%
\pgfpathlineto{\pgfqpoint{3.519072in}{1.204352in}}%
\pgfpathlineto{\pgfqpoint{3.586359in}{1.164800in}}%
\pgfpathlineto{\pgfqpoint{3.620003in}{1.148524in}}%
\pgfpathlineto{\pgfqpoint{3.653646in}{1.139293in}}%
\pgfpathlineto{\pgfqpoint{3.687290in}{1.137338in}}%
\pgfpathlineto{\pgfqpoint{3.720933in}{1.140830in}}%
\pgfpathlineto{\pgfqpoint{3.788220in}{1.154517in}}%
\pgfpathlineto{\pgfqpoint{3.855507in}{1.170488in}}%
\pgfpathlineto{\pgfqpoint{3.889151in}{1.181112in}}%
\pgfpathlineto{\pgfqpoint{3.922794in}{1.195062in}}%
\pgfpathlineto{\pgfqpoint{3.956438in}{1.212546in}}%
\pgfpathlineto{\pgfqpoint{4.023725in}{1.252892in}}%
\pgfpathlineto{\pgfqpoint{4.057368in}{1.271074in}}%
\pgfpathlineto{\pgfqpoint{4.091012in}{1.284818in}}%
\pgfpathlineto{\pgfqpoint{4.124655in}{1.292943in}}%
\pgfpathlineto{\pgfqpoint{4.158298in}{1.295755in}}%
\pgfpathlineto{\pgfqpoint{4.259229in}{1.292985in}}%
\pgfpathlineto{\pgfqpoint{4.292873in}{1.295700in}}%
\pgfpathlineto{\pgfqpoint{4.393803in}{1.312404in}}%
\pgfpathlineto{\pgfqpoint{4.427446in}{1.313221in}}%
\pgfpathlineto{\pgfqpoint{4.461090in}{1.309310in}}%
\pgfpathlineto{\pgfqpoint{4.528377in}{1.294058in}}%
\pgfpathlineto{\pgfqpoint{4.562020in}{1.288911in}}%
\pgfpathlineto{\pgfqpoint{4.595664in}{1.288468in}}%
\pgfpathlineto{\pgfqpoint{4.662951in}{1.296248in}}%
\pgfpathlineto{\pgfqpoint{4.696595in}{1.295679in}}%
\pgfpathlineto{\pgfqpoint{4.730238in}{1.285440in}}%
\pgfpathlineto{\pgfqpoint{4.763881in}{1.263325in}}%
\pgfpathlineto{\pgfqpoint{4.797525in}{1.231504in}}%
\pgfpathlineto{\pgfqpoint{4.831168in}{1.196659in}}%
\pgfpathlineto{\pgfqpoint{4.864812in}{1.168351in}}%
\pgfpathlineto{\pgfqpoint{4.898455in}{1.156115in}}%
\pgfpathlineto{\pgfqpoint{4.932098in}{1.166236in}}%
\pgfpathlineto{\pgfqpoint{4.965742in}{1.199381in}}%
\pgfpathlineto{\pgfqpoint{4.999386in}{1.249947in}}%
\pgfpathlineto{\pgfqpoint{5.033029in}{1.307388in}}%
\pgfpathlineto{\pgfqpoint{5.066673in}{1.359081in}}%
\pgfpathlineto{\pgfqpoint{5.100316in}{1.393776in}}%
\pgfpathlineto{\pgfqpoint{5.133959in}{1.404520in}}%
\pgfpathlineto{\pgfqpoint{5.167603in}{1.390207in}}%
\pgfpathlineto{\pgfqpoint{5.201247in}{1.355412in}}%
\pgfpathlineto{\pgfqpoint{5.268534in}{1.260412in}}%
\pgfpathlineto{\pgfqpoint{5.302177in}{1.219652in}}%
\pgfpathlineto{\pgfqpoint{5.335820in}{1.192891in}}%
\pgfpathlineto{\pgfqpoint{5.369464in}{1.182751in}}%
\pgfpathlineto{\pgfqpoint{5.403108in}{1.188049in}}%
\pgfpathlineto{\pgfqpoint{5.436751in}{1.204564in}}%
\pgfpathlineto{\pgfqpoint{5.504038in}{1.246843in}}%
\pgfpathlineto{\pgfqpoint{5.537682in}{1.260969in}}%
\pgfpathlineto{\pgfqpoint{5.571325in}{1.265580in}}%
\pgfpathlineto{\pgfqpoint{5.604968in}{1.260395in}}%
\pgfpathlineto{\pgfqpoint{5.638612in}{1.247793in}}%
\pgfpathlineto{\pgfqpoint{5.705899in}{1.217125in}}%
\pgfpathlineto{\pgfqpoint{5.739543in}{1.206563in}}%
\pgfpathlineto{\pgfqpoint{5.773186in}{1.200978in}}%
\pgfpathlineto{\pgfqpoint{5.840473in}{1.196948in}}%
\pgfpathlineto{\pgfqpoint{5.874116in}{1.193087in}}%
\pgfpathlineto{\pgfqpoint{5.941403in}{1.180454in}}%
\pgfpathlineto{\pgfqpoint{5.975047in}{1.178437in}}%
\pgfpathlineto{\pgfqpoint{6.008690in}{1.184956in}}%
\pgfpathlineto{\pgfqpoint{6.042334in}{1.201582in}}%
\pgfpathlineto{\pgfqpoint{6.109621in}{1.250059in}}%
\pgfpathlineto{\pgfqpoint{6.143264in}{1.266233in}}%
\pgfpathlineto{\pgfqpoint{6.176907in}{1.266907in}}%
\pgfpathlineto{\pgfqpoint{6.210551in}{1.249800in}}%
\pgfpathlineto{\pgfqpoint{6.244195in}{1.219059in}}%
\pgfpathlineto{\pgfqpoint{6.277838in}{1.184106in}}%
\pgfpathlineto{\pgfqpoint{6.311482in}{1.156097in}}%
\pgfpathlineto{\pgfqpoint{6.345125in}{1.143444in}}%
\pgfpathlineto{\pgfqpoint{6.378768in}{1.148357in}}%
\pgfpathlineto{\pgfqpoint{6.412412in}{1.166006in}}%
\pgfpathlineto{\pgfqpoint{6.446056in}{1.186729in}}%
\pgfpathlineto{\pgfqpoint{6.479699in}{1.200360in}}%
\pgfpathlineto{\pgfqpoint{6.513343in}{1.200688in}}%
\pgfpathlineto{\pgfqpoint{6.546986in}{1.188043in}}%
\pgfpathlineto{\pgfqpoint{6.580629in}{1.168847in}}%
\pgfpathlineto{\pgfqpoint{6.614273in}{1.152407in}}%
\pgfpathlineto{\pgfqpoint{6.647916in}{1.146520in}}%
\pgfpathlineto{\pgfqpoint{6.681560in}{1.153991in}}%
\pgfpathlineto{\pgfqpoint{6.715203in}{1.171650in}}%
\pgfpathlineto{\pgfqpoint{6.748847in}{1.192206in}}%
\pgfpathlineto{\pgfqpoint{6.782490in}{1.207905in}}%
\pgfpathlineto{\pgfqpoint{6.816134in}{1.214112in}}%
\pgfpathlineto{\pgfqpoint{6.849777in}{1.211081in}}%
\pgfpathlineto{\pgfqpoint{6.917064in}{1.196133in}}%
\pgfpathlineto{\pgfqpoint{6.950708in}{1.193858in}}%
\pgfpathlineto{\pgfqpoint{6.984351in}{1.196528in}}%
\pgfpathlineto{\pgfqpoint{7.017995in}{1.200858in}}%
\pgfpathlineto{\pgfqpoint{7.051638in}{1.202312in}}%
\pgfpathlineto{\pgfqpoint{7.085282in}{1.197993in}}%
\pgfpathlineto{\pgfqpoint{7.152568in}{1.178551in}}%
\pgfpathlineto{\pgfqpoint{7.186212in}{1.173819in}}%
\pgfpathlineto{\pgfqpoint{7.219856in}{1.179264in}}%
\pgfpathlineto{\pgfqpoint{7.253499in}{1.196161in}}%
\pgfpathlineto{\pgfqpoint{7.287143in}{1.221553in}}%
\pgfpathlineto{\pgfqpoint{7.320786in}{1.249501in}}%
\pgfpathlineto{\pgfqpoint{7.354429in}{1.273600in}}%
\pgfpathlineto{\pgfqpoint{7.388073in}{1.289486in}}%
\pgfpathlineto{\pgfqpoint{7.421716in}{1.296222in}}%
\pgfpathlineto{\pgfqpoint{7.455360in}{1.296088in}}%
\pgfpathlineto{\pgfqpoint{7.522647in}{1.291056in}}%
\pgfpathlineto{\pgfqpoint{7.556290in}{1.292187in}}%
\pgfpathlineto{\pgfqpoint{7.589934in}{1.296522in}}%
\pgfpathlineto{\pgfqpoint{7.623577in}{1.302300in}}%
\pgfpathlineto{\pgfqpoint{7.623577in}{1.302300in}}%
\pgfusepath{stroke}%
\end{pgfscope}%
\begin{pgfscope}%
\pgfpathrectangle{\pgfqpoint{0.593772in}{0.517039in}}{\pgfqpoint{7.364558in}{1.441291in}}%
\pgfusepath{clip}%
\pgfsetrectcap%
\pgfsetroundjoin%
\pgfsetlinewidth{1.505625pt}%
\definecolor{currentstroke}{rgb}{0.839216,0.152941,0.156863}%
\pgfsetstrokecolor{currentstroke}%
\pgfsetdash{}{0pt}%
\pgfpathmoveto{\pgfqpoint{0.928524in}{1.235446in}}%
\pgfpathlineto{\pgfqpoint{0.962168in}{1.256645in}}%
\pgfpathlineto{\pgfqpoint{0.995811in}{1.294510in}}%
\pgfpathlineto{\pgfqpoint{1.029455in}{1.343899in}}%
\pgfpathlineto{\pgfqpoint{1.063098in}{1.396699in}}%
\pgfpathlineto{\pgfqpoint{1.096742in}{1.445367in}}%
\pgfpathlineto{\pgfqpoint{1.130385in}{1.485791in}}%
\pgfpathlineto{\pgfqpoint{1.164029in}{1.518289in}}%
\pgfpathlineto{\pgfqpoint{1.231316in}{1.574960in}}%
\pgfpathlineto{\pgfqpoint{1.264959in}{1.606584in}}%
\pgfpathlineto{\pgfqpoint{1.365889in}{1.709734in}}%
\pgfpathlineto{\pgfqpoint{1.399533in}{1.737258in}}%
\pgfpathlineto{\pgfqpoint{1.433176in}{1.759117in}}%
\pgfpathlineto{\pgfqpoint{1.466820in}{1.776916in}}%
\pgfpathlineto{\pgfqpoint{1.601394in}{1.842471in}}%
\pgfpathlineto{\pgfqpoint{1.635037in}{1.855428in}}%
\pgfpathlineto{\pgfqpoint{1.668681in}{1.863731in}}%
\pgfpathlineto{\pgfqpoint{1.702324in}{1.867081in}}%
\pgfpathlineto{\pgfqpoint{1.735968in}{1.866541in}}%
\pgfpathlineto{\pgfqpoint{1.803255in}{1.860225in}}%
\pgfpathlineto{\pgfqpoint{1.836898in}{1.856017in}}%
\pgfpathlineto{\pgfqpoint{1.870542in}{1.850300in}}%
\pgfpathlineto{\pgfqpoint{1.904185in}{1.841653in}}%
\pgfpathlineto{\pgfqpoint{1.937829in}{1.828992in}}%
\pgfpathlineto{\pgfqpoint{1.971472in}{1.812178in}}%
\pgfpathlineto{\pgfqpoint{2.005116in}{1.792067in}}%
\pgfpathlineto{\pgfqpoint{2.072403in}{1.746985in}}%
\pgfpathlineto{\pgfqpoint{2.139690in}{1.698410in}}%
\pgfpathlineto{\pgfqpoint{2.173333in}{1.671265in}}%
\pgfpathlineto{\pgfqpoint{2.206977in}{1.641167in}}%
\pgfpathlineto{\pgfqpoint{2.240620in}{1.608112in}}%
\pgfpathlineto{\pgfqpoint{2.307907in}{1.536468in}}%
\pgfpathlineto{\pgfqpoint{2.408837in}{1.426278in}}%
\pgfpathlineto{\pgfqpoint{2.476124in}{1.348677in}}%
\pgfpathlineto{\pgfqpoint{2.644342in}{1.143584in}}%
\pgfpathlineto{\pgfqpoint{2.778916in}{0.988773in}}%
\pgfpathlineto{\pgfqpoint{2.846203in}{0.914414in}}%
\pgfpathlineto{\pgfqpoint{2.879846in}{0.879897in}}%
\pgfpathlineto{\pgfqpoint{2.913490in}{0.847512in}}%
\pgfpathlineto{\pgfqpoint{2.947133in}{0.817040in}}%
\pgfpathlineto{\pgfqpoint{3.014420in}{0.760464in}}%
\pgfpathlineto{\pgfqpoint{3.048064in}{0.734228in}}%
\pgfpathlineto{\pgfqpoint{3.081707in}{0.709820in}}%
\pgfpathlineto{\pgfqpoint{3.115351in}{0.687780in}}%
\pgfpathlineto{\pgfqpoint{3.148994in}{0.668502in}}%
\pgfpathlineto{\pgfqpoint{3.182638in}{0.652065in}}%
\pgfpathlineto{\pgfqpoint{3.216281in}{0.638258in}}%
\pgfpathlineto{\pgfqpoint{3.249924in}{0.626759in}}%
\pgfpathlineto{\pgfqpoint{3.283568in}{0.617362in}}%
\pgfpathlineto{\pgfqpoint{3.317212in}{0.610117in}}%
\pgfpathlineto{\pgfqpoint{3.350855in}{0.605299in}}%
\pgfpathlineto{\pgfqpoint{3.384498in}{0.603253in}}%
\pgfpathlineto{\pgfqpoint{3.418142in}{0.604193in}}%
\pgfpathlineto{\pgfqpoint{3.451785in}{0.608096in}}%
\pgfpathlineto{\pgfqpoint{3.485429in}{0.614734in}}%
\pgfpathlineto{\pgfqpoint{3.519072in}{0.623809in}}%
\pgfpathlineto{\pgfqpoint{3.552716in}{0.635120in}}%
\pgfpathlineto{\pgfqpoint{3.586359in}{0.648645in}}%
\pgfpathlineto{\pgfqpoint{3.620003in}{0.664510in}}%
\pgfpathlineto{\pgfqpoint{3.653646in}{0.682867in}}%
\pgfpathlineto{\pgfqpoint{3.687290in}{0.703762in}}%
\pgfpathlineto{\pgfqpoint{3.720933in}{0.727078in}}%
\pgfpathlineto{\pgfqpoint{3.754577in}{0.752568in}}%
\pgfpathlineto{\pgfqpoint{3.788220in}{0.779958in}}%
\pgfpathlineto{\pgfqpoint{3.821864in}{0.809050in}}%
\pgfpathlineto{\pgfqpoint{3.855507in}{0.839760in}}%
\pgfpathlineto{\pgfqpoint{3.889151in}{0.872078in}}%
\pgfpathlineto{\pgfqpoint{3.956438in}{0.941397in}}%
\pgfpathlineto{\pgfqpoint{4.023725in}{1.015921in}}%
\pgfpathlineto{\pgfqpoint{4.091012in}{1.093888in}}%
\pgfpathlineto{\pgfqpoint{4.191942in}{1.214681in}}%
\pgfpathlineto{\pgfqpoint{4.326516in}{1.376213in}}%
\pgfpathlineto{\pgfqpoint{4.393803in}{1.454207in}}%
\pgfpathlineto{\pgfqpoint{4.461090in}{1.528773in}}%
\pgfpathlineto{\pgfqpoint{4.528377in}{1.598152in}}%
\pgfpathlineto{\pgfqpoint{4.562020in}{1.630502in}}%
\pgfpathlineto{\pgfqpoint{4.595664in}{1.661243in}}%
\pgfpathlineto{\pgfqpoint{4.629307in}{1.690369in}}%
\pgfpathlineto{\pgfqpoint{4.662951in}{1.717795in}}%
\pgfpathlineto{\pgfqpoint{4.696595in}{1.743326in}}%
\pgfpathlineto{\pgfqpoint{4.730238in}{1.766687in}}%
\pgfpathlineto{\pgfqpoint{4.763881in}{1.787633in}}%
\pgfpathlineto{\pgfqpoint{4.797525in}{1.806040in}}%
\pgfpathlineto{\pgfqpoint{4.831168in}{1.821954in}}%
\pgfpathlineto{\pgfqpoint{4.864812in}{1.835525in}}%
\pgfpathlineto{\pgfqpoint{4.898455in}{1.846879in}}%
\pgfpathlineto{\pgfqpoint{4.932098in}{1.856000in}}%
\pgfpathlineto{\pgfqpoint{4.965742in}{1.862689in}}%
\pgfpathlineto{\pgfqpoint{4.999386in}{1.866650in}}%
\pgfpathlineto{\pgfqpoint{5.033029in}{1.867650in}}%
\pgfpathlineto{\pgfqpoint{5.066673in}{1.865662in}}%
\pgfpathlineto{\pgfqpoint{5.100316in}{1.860897in}}%
\pgfpathlineto{\pgfqpoint{5.133959in}{1.853697in}}%
\pgfpathlineto{\pgfqpoint{5.167603in}{1.844342in}}%
\pgfpathlineto{\pgfqpoint{5.201247in}{1.832886in}}%
\pgfpathlineto{\pgfqpoint{5.234890in}{1.819128in}}%
\pgfpathlineto{\pgfqpoint{5.268534in}{1.802747in}}%
\pgfpathlineto{\pgfqpoint{5.302177in}{1.783526in}}%
\pgfpathlineto{\pgfqpoint{5.335820in}{1.761538in}}%
\pgfpathlineto{\pgfqpoint{5.369464in}{1.737172in}}%
\pgfpathlineto{\pgfqpoint{5.403108in}{1.710967in}}%
\pgfpathlineto{\pgfqpoint{5.470394in}{1.654445in}}%
\pgfpathlineto{\pgfqpoint{5.504038in}{1.624007in}}%
\pgfpathlineto{\pgfqpoint{5.537682in}{1.591663in}}%
\pgfpathlineto{\pgfqpoint{5.571325in}{1.557189in}}%
\pgfpathlineto{\pgfqpoint{5.638612in}{1.482885in}}%
\pgfpathlineto{\pgfqpoint{5.773186in}{1.328094in}}%
\pgfpathlineto{\pgfqpoint{5.840473in}{1.247541in}}%
\pgfpathlineto{\pgfqpoint{5.941403in}{1.123023in}}%
\pgfpathlineto{\pgfqpoint{6.008690in}{1.045363in}}%
\pgfpathlineto{\pgfqpoint{6.143264in}{0.898766in}}%
\pgfpathlineto{\pgfqpoint{6.176907in}{0.863468in}}%
\pgfpathlineto{\pgfqpoint{6.210551in}{0.830359in}}%
\pgfpathlineto{\pgfqpoint{6.244195in}{0.800199in}}%
\pgfpathlineto{\pgfqpoint{6.277838in}{0.773000in}}%
\pgfpathlineto{\pgfqpoint{6.345125in}{0.724373in}}%
\pgfpathlineto{\pgfqpoint{6.412412in}{0.679252in}}%
\pgfpathlineto{\pgfqpoint{6.446056in}{0.659088in}}%
\pgfpathlineto{\pgfqpoint{6.479699in}{0.642198in}}%
\pgfpathlineto{\pgfqpoint{6.513343in}{0.629451in}}%
\pgfpathlineto{\pgfqpoint{6.546986in}{0.620733in}}%
\pgfpathlineto{\pgfqpoint{6.580629in}{0.614971in}}%
\pgfpathlineto{\pgfqpoint{6.647916in}{0.607146in}}%
\pgfpathlineto{\pgfqpoint{6.681560in}{0.604390in}}%
\pgfpathlineto{\pgfqpoint{6.715203in}{0.603790in}}%
\pgfpathlineto{\pgfqpoint{6.748847in}{0.607048in}}%
\pgfpathlineto{\pgfqpoint{6.782490in}{0.615249in}}%
\pgfpathlineto{\pgfqpoint{6.816134in}{0.628126in}}%
\pgfpathlineto{\pgfqpoint{6.883421in}{0.660990in}}%
\pgfpathlineto{\pgfqpoint{6.984351in}{0.711388in}}%
\pgfpathlineto{\pgfqpoint{7.017995in}{0.733145in}}%
\pgfpathlineto{\pgfqpoint{7.051638in}{0.760552in}}%
\pgfpathlineto{\pgfqpoint{7.085282in}{0.793249in}}%
\pgfpathlineto{\pgfqpoint{7.152568in}{0.863639in}}%
\pgfpathlineto{\pgfqpoint{7.186212in}{0.895337in}}%
\pgfpathlineto{\pgfqpoint{7.253499in}{0.952013in}}%
\pgfpathlineto{\pgfqpoint{7.287143in}{0.984383in}}%
\pgfpathlineto{\pgfqpoint{7.320786in}{1.024630in}}%
\pgfpathlineto{\pgfqpoint{7.354429in}{1.073159in}}%
\pgfpathlineto{\pgfqpoint{7.388073in}{1.125944in}}%
\pgfpathlineto{\pgfqpoint{7.421716in}{1.175484in}}%
\pgfpathlineto{\pgfqpoint{7.455360in}{1.213645in}}%
\pgfpathlineto{\pgfqpoint{7.489004in}{1.235194in}}%
\pgfpathlineto{\pgfqpoint{7.522647in}{1.240415in}}%
\pgfpathlineto{\pgfqpoint{7.589934in}{1.230506in}}%
\pgfpathlineto{\pgfqpoint{7.623577in}{1.235446in}}%
\pgfpathlineto{\pgfqpoint{7.623577in}{1.235446in}}%
\pgfusepath{stroke}%
\end{pgfscope}%
\begin{pgfscope}%
\pgfsetrectcap%
\pgfsetmiterjoin%
\pgfsetlinewidth{0.803000pt}%
\definecolor{currentstroke}{rgb}{0.000000,0.000000,0.000000}%
\pgfsetstrokecolor{currentstroke}%
\pgfsetdash{}{0pt}%
\pgfpathmoveto{\pgfqpoint{0.593772in}{0.517039in}}%
\pgfpathlineto{\pgfqpoint{0.593772in}{1.958330in}}%
\pgfusepath{stroke}%
\end{pgfscope}%
\begin{pgfscope}%
\pgfsetrectcap%
\pgfsetmiterjoin%
\pgfsetlinewidth{0.803000pt}%
\definecolor{currentstroke}{rgb}{0.000000,0.000000,0.000000}%
\pgfsetstrokecolor{currentstroke}%
\pgfsetdash{}{0pt}%
\pgfpathmoveto{\pgfqpoint{7.958330in}{0.517039in}}%
\pgfpathlineto{\pgfqpoint{7.958330in}{1.958330in}}%
\pgfusepath{stroke}%
\end{pgfscope}%
\begin{pgfscope}%
\pgfsetrectcap%
\pgfsetmiterjoin%
\pgfsetlinewidth{0.803000pt}%
\definecolor{currentstroke}{rgb}{0.000000,0.000000,0.000000}%
\pgfsetstrokecolor{currentstroke}%
\pgfsetdash{}{0pt}%
\pgfpathmoveto{\pgfqpoint{0.593772in}{0.517039in}}%
\pgfpathlineto{\pgfqpoint{7.958330in}{0.517039in}}%
\pgfusepath{stroke}%
\end{pgfscope}%
\begin{pgfscope}%
\pgfsetrectcap%
\pgfsetmiterjoin%
\pgfsetlinewidth{0.803000pt}%
\definecolor{currentstroke}{rgb}{0.000000,0.000000,0.000000}%
\pgfsetstrokecolor{currentstroke}%
\pgfsetdash{}{0pt}%
\pgfpathmoveto{\pgfqpoint{0.593772in}{1.958330in}}%
\pgfpathlineto{\pgfqpoint{7.958330in}{1.958330in}}%
\pgfusepath{stroke}%
\end{pgfscope}%
\begin{pgfscope}%
\pgfsetbuttcap%
\pgfsetmiterjoin%
\definecolor{currentfill}{rgb}{1.000000,1.000000,1.000000}%
\pgfsetfillcolor{currentfill}%
\pgfsetfillopacity{0.800000}%
\pgfsetlinewidth{1.003750pt}%
\definecolor{currentstroke}{rgb}{0.800000,0.800000,0.800000}%
\pgfsetstrokecolor{currentstroke}%
\pgfsetstrokeopacity{0.800000}%
\pgfsetdash{}{0pt}%
\pgfpathmoveto{\pgfqpoint{5.916728in}{0.844122in}}%
\pgfpathlineto{\pgfqpoint{7.841663in}{0.844122in}}%
\pgfpathquadraticcurveto{\pgfqpoint{7.874997in}{0.844122in}}{\pgfqpoint{7.874997in}{0.877455in}}%
\pgfpathlineto{\pgfqpoint{7.874997in}{1.841663in}}%
\pgfpathquadraticcurveto{\pgfqpoint{7.874997in}{1.874997in}}{\pgfqpoint{7.841663in}{1.874997in}}%
\pgfpathlineto{\pgfqpoint{5.916728in}{1.874997in}}%
\pgfpathquadraticcurveto{\pgfqpoint{5.883395in}{1.874997in}}{\pgfqpoint{5.883395in}{1.841663in}}%
\pgfpathlineto{\pgfqpoint{5.883395in}{0.877455in}}%
\pgfpathquadraticcurveto{\pgfqpoint{5.883395in}{0.844122in}}{\pgfqpoint{5.916728in}{0.844122in}}%
\pgfpathlineto{\pgfqpoint{5.916728in}{0.844122in}}%
\pgfpathclose%
\pgfusepath{stroke,fill}%
\end{pgfscope}%
\begin{pgfscope}%
\pgfsetrectcap%
\pgfsetroundjoin%
\pgfsetlinewidth{1.505625pt}%
\definecolor{currentstroke}{rgb}{0.121569,0.466667,0.705882}%
\pgfsetstrokecolor{currentstroke}%
\pgfsetdash{}{0pt}%
\pgfpathmoveto{\pgfqpoint{5.950062in}{1.740036in}}%
\pgfpathlineto{\pgfqpoint{6.116728in}{1.740036in}}%
\pgfpathlineto{\pgfqpoint{6.283395in}{1.740036in}}%
\pgfusepath{stroke}%
\end{pgfscope}%
\begin{pgfscope}%
\definecolor{textcolor}{rgb}{0.000000,0.000000,0.000000}%
\pgfsetstrokecolor{textcolor}%
\pgfsetfillcolor{textcolor}%
\pgftext[x=6.416728in,y=1.681702in,left,base]{\color{textcolor}{\rmfamily\fontsize{12.000000}{14.400000}\selectfont\catcode`\^=\active\def^{\ifmmode\sp\else\^{}\fi}\catcode`\%=\active\def%{\%}5% Input Error}}%
\end{pgfscope}%
\begin{pgfscope}%
\pgfsetrectcap%
\pgfsetroundjoin%
\pgfsetlinewidth{1.505625pt}%
\definecolor{currentstroke}{rgb}{1.000000,0.498039,0.054902}%
\pgfsetstrokecolor{currentstroke}%
\pgfsetdash{}{0pt}%
\pgfpathmoveto{\pgfqpoint{5.950062in}{1.495407in}}%
\pgfpathlineto{\pgfqpoint{6.116728in}{1.495407in}}%
\pgfpathlineto{\pgfqpoint{6.283395in}{1.495407in}}%
\pgfusepath{stroke}%
\end{pgfscope}%
\begin{pgfscope}%
\definecolor{textcolor}{rgb}{0.000000,0.000000,0.000000}%
\pgfsetstrokecolor{textcolor}%
\pgfsetfillcolor{textcolor}%
\pgftext[x=6.416728in,y=1.437074in,left,base]{\color{textcolor}{\rmfamily\fontsize{12.000000}{14.400000}\selectfont\catcode`\^=\active\def^{\ifmmode\sp\else\^{}\fi}\catcode`\%=\active\def%{\%}10% Input Error}}%
\end{pgfscope}%
\begin{pgfscope}%
\pgfsetrectcap%
\pgfsetroundjoin%
\pgfsetlinewidth{1.505625pt}%
\definecolor{currentstroke}{rgb}{0.172549,0.627451,0.172549}%
\pgfsetstrokecolor{currentstroke}%
\pgfsetdash{}{0pt}%
\pgfpathmoveto{\pgfqpoint{5.950062in}{1.250778in}}%
\pgfpathlineto{\pgfqpoint{6.116728in}{1.250778in}}%
\pgfpathlineto{\pgfqpoint{6.283395in}{1.250778in}}%
\pgfusepath{stroke}%
\end{pgfscope}%
\begin{pgfscope}%
\definecolor{textcolor}{rgb}{0.000000,0.000000,0.000000}%
\pgfsetstrokecolor{textcolor}%
\pgfsetfillcolor{textcolor}%
\pgftext[x=6.416728in,y=1.192445in,left,base]{\color{textcolor}{\rmfamily\fontsize{12.000000}{14.400000}\selectfont\catcode`\^=\active\def^{\ifmmode\sp\else\^{}\fi}\catcode`\%=\active\def%{\%}50% Input Error}}%
\end{pgfscope}%
\begin{pgfscope}%
\pgfsetrectcap%
\pgfsetroundjoin%
\pgfsetlinewidth{1.505625pt}%
\definecolor{currentstroke}{rgb}{0.839216,0.152941,0.156863}%
\pgfsetstrokecolor{currentstroke}%
\pgfsetdash{}{0pt}%
\pgfpathmoveto{\pgfqpoint{5.950062in}{1.006149in}}%
\pgfpathlineto{\pgfqpoint{6.116728in}{1.006149in}}%
\pgfpathlineto{\pgfqpoint{6.283395in}{1.006149in}}%
\pgfusepath{stroke}%
\end{pgfscope}%
\begin{pgfscope}%
\definecolor{textcolor}{rgb}{0.000000,0.000000,0.000000}%
\pgfsetstrokecolor{textcolor}%
\pgfsetfillcolor{textcolor}%
\pgftext[x=6.416728in,y=0.947816in,left,base]{\color{textcolor}{\rmfamily\fontsize{12.000000}{14.400000}\selectfont\catcode`\^=\active\def^{\ifmmode\sp\else\^{}\fi}\catcode`\%=\active\def%{\%}Exact Target}}%
\end{pgfscope}%
\end{pgfpicture}%
\makeatother%
\endgroup%

    \end{adjustbox}
    \caption{}\label{fig:antiderivative_exact_prediction}
  \end{subfigure}
  \caption{(\subref{fig:antiderivative_exact_input}) The perturbed exact input function values from \lccref{eq:sine_derivative}.\ (\subref{fig:antiderivative_exact_prediction}) Prediction of antiderivative from the input function that was perturbed.}\label{fig:antiderivative_exact}
\end{figure}

Next, we have the correlation image and p-matrix for the model of each noise level. The results are shown in \lccref{fig:scenario_1_interpretation}. Each row in the figure represent the results of a model trained on a different noise level. The correlation image for 5\% and 10\% noise show clear lines on the coefficient corresponding to the output coefficient that was sorted. We can also see that the negative wave number reflection also being sorted. This is because of the reflection that occurs with complex Fourier coefficients for real-valued functions. There are other faint vertical lines you are able to see for other input coefficients. These faint lines are not sorted like the corresponding coefficients we mentioned before. Meaning, while they don't affect the particular output coefficients that were sorted, they do show that there is information embedded in the kernel matrix for these input coefficients. However, if we look at the 50\% noise correlation image, even the corresponding coefficient does not show a clear line or gradient. The lines are more noisy compared to the other noise levels. But the kernel still manages to embed some information. The p-matrices show that The model itself is able to still learn some relationship between the input coefficients and the output coefficients. For the 5\% and 10\% noise levels, the p-matrices show very clearly the contributions of input coefficients to the output coefficients. However, looking at the p-matrix for 50\% noise level, the lower wave numbers show lower contribution of input coefficients to the output coefficients.
\begin{figure}[H]
  \begin{adjustwidth}{-0.05\linewidth}{-0.05\linewidth}
    \centering
    \begin{subfigure}{0.49\linewidth}
      \begin{adjustbox}{width=\linewidth}
        \begingroup%
\makeatletter%
\begin{pgfpicture}%
\pgfpathrectangle{\pgfpointorigin}{\pgfqpoint{6.400000in}{4.800000in}}%
\pgfusepath{use as bounding box, clip}%
\begin{pgfscope}%
\pgfsetbuttcap%
\pgfsetmiterjoin%
\pgfsetlinewidth{0.000000pt}%
\definecolor{currentstroke}{rgb}{0.000000,0.000000,0.000000}%
\pgfsetstrokecolor{currentstroke}%
\pgfsetstrokeopacity{0.000000}%
\pgfsetdash{}{0pt}%
\pgfpathmoveto{\pgfqpoint{0.000000in}{0.000000in}}%
\pgfpathlineto{\pgfqpoint{6.400000in}{0.000000in}}%
\pgfpathlineto{\pgfqpoint{6.400000in}{4.800000in}}%
\pgfpathlineto{\pgfqpoint{0.000000in}{4.800000in}}%
\pgfpathlineto{\pgfqpoint{0.000000in}{0.000000in}}%
\pgfpathclose%
\pgfusepath{}%
\end{pgfscope}%
\begin{pgfscope}%
\pgfsetbuttcap%
\pgfsetmiterjoin%
\pgfsetlinewidth{0.000000pt}%
\definecolor{currentstroke}{rgb}{0.000000,0.000000,0.000000}%
\pgfsetstrokecolor{currentstroke}%
\pgfsetstrokeopacity{0.000000}%
\pgfsetdash{}{0pt}%
\pgfpathmoveto{\pgfqpoint{0.800000in}{0.528000in}}%
\pgfpathlineto{\pgfqpoint{4.768000in}{0.528000in}}%
\pgfpathlineto{\pgfqpoint{4.768000in}{4.224000in}}%
\pgfpathlineto{\pgfqpoint{0.800000in}{4.224000in}}%
\pgfpathlineto{\pgfqpoint{0.800000in}{0.528000in}}%
\pgfpathclose%
\pgfusepath{}%
\end{pgfscope}%
\begin{pgfscope}%
\pgfpathrectangle{\pgfqpoint{0.800000in}{0.528000in}}{\pgfqpoint{3.968000in}{3.696000in}}%
\pgfusepath{clip}%
\pgfsys@transformcm{3.968000}{0.000000}{0.000000}{-3.696000}{0.800000in}{4.224000in}%
\pgftext[left,bottom]{\includegraphics[interpolate=false,width=1.000000in,height=1.000000in]{antiderivative_ci_5-img0.png}}%
\end{pgfscope}%
\begin{pgfscope}%
\pgfsetbuttcap%
\pgfsetroundjoin%
\definecolor{currentfill}{rgb}{0.000000,0.000000,0.000000}%
\pgfsetfillcolor{currentfill}%
\pgfsetlinewidth{0.803000pt}%
\definecolor{currentstroke}{rgb}{0.000000,0.000000,0.000000}%
\pgfsetstrokecolor{currentstroke}%
\pgfsetdash{}{0pt}%
\pgfsys@defobject{currentmarker}{\pgfqpoint{0.000000in}{-0.048611in}}{\pgfqpoint{0.000000in}{0.000000in}}{%
\pgfpathmoveto{\pgfqpoint{0.000000in}{0.000000in}}%
\pgfpathlineto{\pgfqpoint{0.000000in}{-0.048611in}}%
\pgfusepath{stroke,fill}%
}%
\begin{pgfscope}%
\pgfsys@transformshift{0.800000in}{0.528000in}%
\pgfsys@useobject{currentmarker}{}%
\end{pgfscope}%
\end{pgfscope}%
\begin{pgfscope}%
\definecolor{textcolor}{rgb}{0.000000,0.000000,0.000000}%
\pgfsetstrokecolor{textcolor}%
\pgfsetfillcolor{textcolor}%
\pgftext[x=0.800000in,y=0.430778in,,top]{\color{textcolor}{\rmfamily\fontsize{12.000000}{14.400000}\selectfont\catcode`\^=\active\def^{\ifmmode\sp\else\^{}\fi}\catcode`\%=\active\def%{\%}0}}%
\end{pgfscope}%
\begin{pgfscope}%
\pgfsetbuttcap%
\pgfsetroundjoin%
\definecolor{currentfill}{rgb}{0.000000,0.000000,0.000000}%
\pgfsetfillcolor{currentfill}%
\pgfsetlinewidth{0.803000pt}%
\definecolor{currentstroke}{rgb}{0.000000,0.000000,0.000000}%
\pgfsetstrokecolor{currentstroke}%
\pgfsetdash{}{0pt}%
\pgfsys@defobject{currentmarker}{\pgfqpoint{0.000000in}{-0.048611in}}{\pgfqpoint{0.000000in}{0.000000in}}{%
\pgfpathmoveto{\pgfqpoint{0.000000in}{0.000000in}}%
\pgfpathlineto{\pgfqpoint{0.000000in}{-0.048611in}}%
\pgfusepath{stroke,fill}%
}%
\begin{pgfscope}%
\pgfsys@transformshift{1.593600in}{0.528000in}%
\pgfsys@useobject{currentmarker}{}%
\end{pgfscope}%
\end{pgfscope}%
\begin{pgfscope}%
\definecolor{textcolor}{rgb}{0.000000,0.000000,0.000000}%
\pgfsetstrokecolor{textcolor}%
\pgfsetfillcolor{textcolor}%
\pgftext[x=1.593600in,y=0.430778in,,top]{\color{textcolor}{\rmfamily\fontsize{12.000000}{14.400000}\selectfont\catcode`\^=\active\def^{\ifmmode\sp\else\^{}\fi}\catcode`\%=\active\def%{\%}10}}%
\end{pgfscope}%
\begin{pgfscope}%
\pgfsetbuttcap%
\pgfsetroundjoin%
\definecolor{currentfill}{rgb}{0.000000,0.000000,0.000000}%
\pgfsetfillcolor{currentfill}%
\pgfsetlinewidth{0.803000pt}%
\definecolor{currentstroke}{rgb}{0.000000,0.000000,0.000000}%
\pgfsetstrokecolor{currentstroke}%
\pgfsetdash{}{0pt}%
\pgfsys@defobject{currentmarker}{\pgfqpoint{0.000000in}{-0.048611in}}{\pgfqpoint{0.000000in}{0.000000in}}{%
\pgfpathmoveto{\pgfqpoint{0.000000in}{0.000000in}}%
\pgfpathlineto{\pgfqpoint{0.000000in}{-0.048611in}}%
\pgfusepath{stroke,fill}%
}%
\begin{pgfscope}%
\pgfsys@transformshift{2.387200in}{0.528000in}%
\pgfsys@useobject{currentmarker}{}%
\end{pgfscope}%
\end{pgfscope}%
\begin{pgfscope}%
\definecolor{textcolor}{rgb}{0.000000,0.000000,0.000000}%
\pgfsetstrokecolor{textcolor}%
\pgfsetfillcolor{textcolor}%
\pgftext[x=2.387200in,y=0.430778in,,top]{\color{textcolor}{\rmfamily\fontsize{12.000000}{14.400000}\selectfont\catcode`\^=\active\def^{\ifmmode\sp\else\^{}\fi}\catcode`\%=\active\def%{\%}20}}%
\end{pgfscope}%
\begin{pgfscope}%
\pgfsetbuttcap%
\pgfsetroundjoin%
\definecolor{currentfill}{rgb}{0.000000,0.000000,0.000000}%
\pgfsetfillcolor{currentfill}%
\pgfsetlinewidth{0.803000pt}%
\definecolor{currentstroke}{rgb}{0.000000,0.000000,0.000000}%
\pgfsetstrokecolor{currentstroke}%
\pgfsetdash{}{0pt}%
\pgfsys@defobject{currentmarker}{\pgfqpoint{0.000000in}{-0.048611in}}{\pgfqpoint{0.000000in}{0.000000in}}{%
\pgfpathmoveto{\pgfqpoint{0.000000in}{0.000000in}}%
\pgfpathlineto{\pgfqpoint{0.000000in}{-0.048611in}}%
\pgfusepath{stroke,fill}%
}%
\begin{pgfscope}%
\pgfsys@transformshift{3.180800in}{0.528000in}%
\pgfsys@useobject{currentmarker}{}%
\end{pgfscope}%
\end{pgfscope}%
\begin{pgfscope}%
\definecolor{textcolor}{rgb}{0.000000,0.000000,0.000000}%
\pgfsetstrokecolor{textcolor}%
\pgfsetfillcolor{textcolor}%
\pgftext[x=3.180800in,y=0.430778in,,top]{\color{textcolor}{\rmfamily\fontsize{12.000000}{14.400000}\selectfont\catcode`\^=\active\def^{\ifmmode\sp\else\^{}\fi}\catcode`\%=\active\def%{\%}30}}%
\end{pgfscope}%
\begin{pgfscope}%
\pgfsetbuttcap%
\pgfsetroundjoin%
\definecolor{currentfill}{rgb}{0.000000,0.000000,0.000000}%
\pgfsetfillcolor{currentfill}%
\pgfsetlinewidth{0.803000pt}%
\definecolor{currentstroke}{rgb}{0.000000,0.000000,0.000000}%
\pgfsetstrokecolor{currentstroke}%
\pgfsetdash{}{0pt}%
\pgfsys@defobject{currentmarker}{\pgfqpoint{0.000000in}{-0.048611in}}{\pgfqpoint{0.000000in}{0.000000in}}{%
\pgfpathmoveto{\pgfqpoint{0.000000in}{0.000000in}}%
\pgfpathlineto{\pgfqpoint{0.000000in}{-0.048611in}}%
\pgfusepath{stroke,fill}%
}%
\begin{pgfscope}%
\pgfsys@transformshift{3.974400in}{0.528000in}%
\pgfsys@useobject{currentmarker}{}%
\end{pgfscope}%
\end{pgfscope}%
\begin{pgfscope}%
\definecolor{textcolor}{rgb}{0.000000,0.000000,0.000000}%
\pgfsetstrokecolor{textcolor}%
\pgfsetfillcolor{textcolor}%
\pgftext[x=3.974400in,y=0.430778in,,top]{\color{textcolor}{\rmfamily\fontsize{12.000000}{14.400000}\selectfont\catcode`\^=\active\def^{\ifmmode\sp\else\^{}\fi}\catcode`\%=\active\def%{\%}40}}%
\end{pgfscope}%
\begin{pgfscope}%
\pgfsetbuttcap%
\pgfsetroundjoin%
\definecolor{currentfill}{rgb}{0.000000,0.000000,0.000000}%
\pgfsetfillcolor{currentfill}%
\pgfsetlinewidth{0.803000pt}%
\definecolor{currentstroke}{rgb}{0.000000,0.000000,0.000000}%
\pgfsetstrokecolor{currentstroke}%
\pgfsetdash{}{0pt}%
\pgfsys@defobject{currentmarker}{\pgfqpoint{0.000000in}{-0.048611in}}{\pgfqpoint{0.000000in}{0.000000in}}{%
\pgfpathmoveto{\pgfqpoint{0.000000in}{0.000000in}}%
\pgfpathlineto{\pgfqpoint{0.000000in}{-0.048611in}}%
\pgfusepath{stroke,fill}%
}%
\begin{pgfscope}%
\pgfsys@transformshift{4.768000in}{0.528000in}%
\pgfsys@useobject{currentmarker}{}%
\end{pgfscope}%
\end{pgfscope}%
\begin{pgfscope}%
\definecolor{textcolor}{rgb}{0.000000,0.000000,0.000000}%
\pgfsetstrokecolor{textcolor}%
\pgfsetfillcolor{textcolor}%
\pgftext[x=4.768000in,y=0.430778in,,top]{\color{textcolor}{\rmfamily\fontsize{12.000000}{14.400000}\selectfont\catcode`\^=\active\def^{\ifmmode\sp\else\^{}\fi}\catcode`\%=\active\def%{\%}50}}%
\end{pgfscope}%
\begin{pgfscope}%
\definecolor{textcolor}{rgb}{0.000000,0.000000,0.000000}%
\pgfsetstrokecolor{textcolor}%
\pgfsetfillcolor{textcolor}%
\pgftext[x=2.784000in,y=0.213927in,,top]{\color{textcolor}{\rmfamily\fontsize{12.000000}{14.400000}\selectfont\catcode`\^=\active\def^{\ifmmode\sp\else\^{}\fi}\catcode`\%=\active\def%{\%}input coefficients}}%
\end{pgfscope}%
\begin{pgfscope}%
\pgfsetbuttcap%
\pgfsetroundjoin%
\definecolor{currentfill}{rgb}{0.000000,0.000000,0.000000}%
\pgfsetfillcolor{currentfill}%
\pgfsetlinewidth{0.803000pt}%
\definecolor{currentstroke}{rgb}{0.000000,0.000000,0.000000}%
\pgfsetstrokecolor{currentstroke}%
\pgfsetdash{}{0pt}%
\pgfsys@defobject{currentmarker}{\pgfqpoint{-0.048611in}{0.000000in}}{\pgfqpoint{-0.000000in}{0.000000in}}{%
\pgfpathmoveto{\pgfqpoint{-0.000000in}{0.000000in}}%
\pgfpathlineto{\pgfqpoint{-0.048611in}{0.000000in}}%
\pgfusepath{stroke,fill}%
}%
\begin{pgfscope}%
\pgfsys@transformshift{0.800000in}{4.224000in}%
\pgfsys@useobject{currentmarker}{}%
\end{pgfscope}%
\end{pgfscope}%
\begin{pgfscope}%
\definecolor{textcolor}{rgb}{0.000000,0.000000,0.000000}%
\pgfsetstrokecolor{textcolor}%
\pgfsetfillcolor{textcolor}%
\pgftext[x=0.596739in, y=4.160686in, left, base]{\color{textcolor}{\rmfamily\fontsize{12.000000}{14.400000}\selectfont\catcode`\^=\active\def^{\ifmmode\sp\else\^{}\fi}\catcode`\%=\active\def%{\%}0}}%
\end{pgfscope}%
\begin{pgfscope}%
\pgfsetbuttcap%
\pgfsetroundjoin%
\definecolor{currentfill}{rgb}{0.000000,0.000000,0.000000}%
\pgfsetfillcolor{currentfill}%
\pgfsetlinewidth{0.803000pt}%
\definecolor{currentstroke}{rgb}{0.000000,0.000000,0.000000}%
\pgfsetstrokecolor{currentstroke}%
\pgfsetdash{}{0pt}%
\pgfsys@defobject{currentmarker}{\pgfqpoint{-0.048611in}{0.000000in}}{\pgfqpoint{-0.000000in}{0.000000in}}{%
\pgfpathmoveto{\pgfqpoint{-0.000000in}{0.000000in}}%
\pgfpathlineto{\pgfqpoint{-0.048611in}{0.000000in}}%
\pgfusepath{stroke,fill}%
}%
\begin{pgfscope}%
\pgfsys@transformshift{0.800000in}{3.762000in}%
\pgfsys@useobject{currentmarker}{}%
\end{pgfscope}%
\end{pgfscope}%
\begin{pgfscope}%
\definecolor{textcolor}{rgb}{0.000000,0.000000,0.000000}%
\pgfsetstrokecolor{textcolor}%
\pgfsetfillcolor{textcolor}%
\pgftext[x=0.384662in, y=3.698686in, left, base]{\color{textcolor}{\rmfamily\fontsize{12.000000}{14.400000}\selectfont\catcode`\^=\active\def^{\ifmmode\sp\else\^{}\fi}\catcode`\%=\active\def%{\%}500}}%
\end{pgfscope}%
\begin{pgfscope}%
\pgfsetbuttcap%
\pgfsetroundjoin%
\definecolor{currentfill}{rgb}{0.000000,0.000000,0.000000}%
\pgfsetfillcolor{currentfill}%
\pgfsetlinewidth{0.803000pt}%
\definecolor{currentstroke}{rgb}{0.000000,0.000000,0.000000}%
\pgfsetstrokecolor{currentstroke}%
\pgfsetdash{}{0pt}%
\pgfsys@defobject{currentmarker}{\pgfqpoint{-0.048611in}{0.000000in}}{\pgfqpoint{-0.000000in}{0.000000in}}{%
\pgfpathmoveto{\pgfqpoint{-0.000000in}{0.000000in}}%
\pgfpathlineto{\pgfqpoint{-0.048611in}{0.000000in}}%
\pgfusepath{stroke,fill}%
}%
\begin{pgfscope}%
\pgfsys@transformshift{0.800000in}{3.300000in}%
\pgfsys@useobject{currentmarker}{}%
\end{pgfscope}%
\end{pgfscope}%
\begin{pgfscope}%
\definecolor{textcolor}{rgb}{0.000000,0.000000,0.000000}%
\pgfsetstrokecolor{textcolor}%
\pgfsetfillcolor{textcolor}%
\pgftext[x=0.278624in, y=3.236686in, left, base]{\color{textcolor}{\rmfamily\fontsize{12.000000}{14.400000}\selectfont\catcode`\^=\active\def^{\ifmmode\sp\else\^{}\fi}\catcode`\%=\active\def%{\%}1000}}%
\end{pgfscope}%
\begin{pgfscope}%
\pgfsetbuttcap%
\pgfsetroundjoin%
\definecolor{currentfill}{rgb}{0.000000,0.000000,0.000000}%
\pgfsetfillcolor{currentfill}%
\pgfsetlinewidth{0.803000pt}%
\definecolor{currentstroke}{rgb}{0.000000,0.000000,0.000000}%
\pgfsetstrokecolor{currentstroke}%
\pgfsetdash{}{0pt}%
\pgfsys@defobject{currentmarker}{\pgfqpoint{-0.048611in}{0.000000in}}{\pgfqpoint{-0.000000in}{0.000000in}}{%
\pgfpathmoveto{\pgfqpoint{-0.000000in}{0.000000in}}%
\pgfpathlineto{\pgfqpoint{-0.048611in}{0.000000in}}%
\pgfusepath{stroke,fill}%
}%
\begin{pgfscope}%
\pgfsys@transformshift{0.800000in}{2.838000in}%
\pgfsys@useobject{currentmarker}{}%
\end{pgfscope}%
\end{pgfscope}%
\begin{pgfscope}%
\definecolor{textcolor}{rgb}{0.000000,0.000000,0.000000}%
\pgfsetstrokecolor{textcolor}%
\pgfsetfillcolor{textcolor}%
\pgftext[x=0.278624in, y=2.774686in, left, base]{\color{textcolor}{\rmfamily\fontsize{12.000000}{14.400000}\selectfont\catcode`\^=\active\def^{\ifmmode\sp\else\^{}\fi}\catcode`\%=\active\def%{\%}1500}}%
\end{pgfscope}%
\begin{pgfscope}%
\pgfsetbuttcap%
\pgfsetroundjoin%
\definecolor{currentfill}{rgb}{0.000000,0.000000,0.000000}%
\pgfsetfillcolor{currentfill}%
\pgfsetlinewidth{0.803000pt}%
\definecolor{currentstroke}{rgb}{0.000000,0.000000,0.000000}%
\pgfsetstrokecolor{currentstroke}%
\pgfsetdash{}{0pt}%
\pgfsys@defobject{currentmarker}{\pgfqpoint{-0.048611in}{0.000000in}}{\pgfqpoint{-0.000000in}{0.000000in}}{%
\pgfpathmoveto{\pgfqpoint{-0.000000in}{0.000000in}}%
\pgfpathlineto{\pgfqpoint{-0.048611in}{0.000000in}}%
\pgfusepath{stroke,fill}%
}%
\begin{pgfscope}%
\pgfsys@transformshift{0.800000in}{2.376000in}%
\pgfsys@useobject{currentmarker}{}%
\end{pgfscope}%
\end{pgfscope}%
\begin{pgfscope}%
\definecolor{textcolor}{rgb}{0.000000,0.000000,0.000000}%
\pgfsetstrokecolor{textcolor}%
\pgfsetfillcolor{textcolor}%
\pgftext[x=0.278624in, y=2.312686in, left, base]{\color{textcolor}{\rmfamily\fontsize{12.000000}{14.400000}\selectfont\catcode`\^=\active\def^{\ifmmode\sp\else\^{}\fi}\catcode`\%=\active\def%{\%}2000}}%
\end{pgfscope}%
\begin{pgfscope}%
\pgfsetbuttcap%
\pgfsetroundjoin%
\definecolor{currentfill}{rgb}{0.000000,0.000000,0.000000}%
\pgfsetfillcolor{currentfill}%
\pgfsetlinewidth{0.803000pt}%
\definecolor{currentstroke}{rgb}{0.000000,0.000000,0.000000}%
\pgfsetstrokecolor{currentstroke}%
\pgfsetdash{}{0pt}%
\pgfsys@defobject{currentmarker}{\pgfqpoint{-0.048611in}{0.000000in}}{\pgfqpoint{-0.000000in}{0.000000in}}{%
\pgfpathmoveto{\pgfqpoint{-0.000000in}{0.000000in}}%
\pgfpathlineto{\pgfqpoint{-0.048611in}{0.000000in}}%
\pgfusepath{stroke,fill}%
}%
\begin{pgfscope}%
\pgfsys@transformshift{0.800000in}{1.914000in}%
\pgfsys@useobject{currentmarker}{}%
\end{pgfscope}%
\end{pgfscope}%
\begin{pgfscope}%
\definecolor{textcolor}{rgb}{0.000000,0.000000,0.000000}%
\pgfsetstrokecolor{textcolor}%
\pgfsetfillcolor{textcolor}%
\pgftext[x=0.278624in, y=1.850686in, left, base]{\color{textcolor}{\rmfamily\fontsize{12.000000}{14.400000}\selectfont\catcode`\^=\active\def^{\ifmmode\sp\else\^{}\fi}\catcode`\%=\active\def%{\%}2500}}%
\end{pgfscope}%
\begin{pgfscope}%
\pgfsetbuttcap%
\pgfsetroundjoin%
\definecolor{currentfill}{rgb}{0.000000,0.000000,0.000000}%
\pgfsetfillcolor{currentfill}%
\pgfsetlinewidth{0.803000pt}%
\definecolor{currentstroke}{rgb}{0.000000,0.000000,0.000000}%
\pgfsetstrokecolor{currentstroke}%
\pgfsetdash{}{0pt}%
\pgfsys@defobject{currentmarker}{\pgfqpoint{-0.048611in}{0.000000in}}{\pgfqpoint{-0.000000in}{0.000000in}}{%
\pgfpathmoveto{\pgfqpoint{-0.000000in}{0.000000in}}%
\pgfpathlineto{\pgfqpoint{-0.048611in}{0.000000in}}%
\pgfusepath{stroke,fill}%
}%
\begin{pgfscope}%
\pgfsys@transformshift{0.800000in}{1.452000in}%
\pgfsys@useobject{currentmarker}{}%
\end{pgfscope}%
\end{pgfscope}%
\begin{pgfscope}%
\definecolor{textcolor}{rgb}{0.000000,0.000000,0.000000}%
\pgfsetstrokecolor{textcolor}%
\pgfsetfillcolor{textcolor}%
\pgftext[x=0.278624in, y=1.388686in, left, base]{\color{textcolor}{\rmfamily\fontsize{12.000000}{14.400000}\selectfont\catcode`\^=\active\def^{\ifmmode\sp\else\^{}\fi}\catcode`\%=\active\def%{\%}3000}}%
\end{pgfscope}%
\begin{pgfscope}%
\pgfsetbuttcap%
\pgfsetroundjoin%
\definecolor{currentfill}{rgb}{0.000000,0.000000,0.000000}%
\pgfsetfillcolor{currentfill}%
\pgfsetlinewidth{0.803000pt}%
\definecolor{currentstroke}{rgb}{0.000000,0.000000,0.000000}%
\pgfsetstrokecolor{currentstroke}%
\pgfsetdash{}{0pt}%
\pgfsys@defobject{currentmarker}{\pgfqpoint{-0.048611in}{0.000000in}}{\pgfqpoint{-0.000000in}{0.000000in}}{%
\pgfpathmoveto{\pgfqpoint{-0.000000in}{0.000000in}}%
\pgfpathlineto{\pgfqpoint{-0.048611in}{0.000000in}}%
\pgfusepath{stroke,fill}%
}%
\begin{pgfscope}%
\pgfsys@transformshift{0.800000in}{0.990000in}%
\pgfsys@useobject{currentmarker}{}%
\end{pgfscope}%
\end{pgfscope}%
\begin{pgfscope}%
\definecolor{textcolor}{rgb}{0.000000,0.000000,0.000000}%
\pgfsetstrokecolor{textcolor}%
\pgfsetfillcolor{textcolor}%
\pgftext[x=0.278624in, y=0.926686in, left, base]{\color{textcolor}{\rmfamily\fontsize{12.000000}{14.400000}\selectfont\catcode`\^=\active\def^{\ifmmode\sp\else\^{}\fi}\catcode`\%=\active\def%{\%}3500}}%
\end{pgfscope}%
\begin{pgfscope}%
\pgfsetbuttcap%
\pgfsetroundjoin%
\definecolor{currentfill}{rgb}{0.000000,0.000000,0.000000}%
\pgfsetfillcolor{currentfill}%
\pgfsetlinewidth{0.803000pt}%
\definecolor{currentstroke}{rgb}{0.000000,0.000000,0.000000}%
\pgfsetstrokecolor{currentstroke}%
\pgfsetdash{}{0pt}%
\pgfsys@defobject{currentmarker}{\pgfqpoint{-0.048611in}{0.000000in}}{\pgfqpoint{-0.000000in}{0.000000in}}{%
\pgfpathmoveto{\pgfqpoint{-0.000000in}{0.000000in}}%
\pgfpathlineto{\pgfqpoint{-0.048611in}{0.000000in}}%
\pgfusepath{stroke,fill}%
}%
\begin{pgfscope}%
\pgfsys@transformshift{0.800000in}{0.528000in}%
\pgfsys@useobject{currentmarker}{}%
\end{pgfscope}%
\end{pgfscope}%
\begin{pgfscope}%
\definecolor{textcolor}{rgb}{0.000000,0.000000,0.000000}%
\pgfsetstrokecolor{textcolor}%
\pgfsetfillcolor{textcolor}%
\pgftext[x=0.278624in, y=0.464686in, left, base]{\color{textcolor}{\rmfamily\fontsize{12.000000}{14.400000}\selectfont\catcode`\^=\active\def^{\ifmmode\sp\else\^{}\fi}\catcode`\%=\active\def%{\%}4000}}%
\end{pgfscope}%
\begin{pgfscope}%
\definecolor{textcolor}{rgb}{0.000000,0.000000,0.000000}%
\pgfsetstrokecolor{textcolor}%
\pgfsetfillcolor{textcolor}%
\pgftext[x=0.223069in,y=2.376000in,,bottom,rotate=90.000000]{\color{textcolor}{\rmfamily\fontsize{12.000000}{14.400000}\selectfont\catcode`\^=\active\def^{\ifmmode\sp\else\^{}\fi}\catcode`\%=\active\def%{\%}samples}}%
\end{pgfscope}%
\begin{pgfscope}%
\pgfsetrectcap%
\pgfsetmiterjoin%
\pgfsetlinewidth{0.803000pt}%
\definecolor{currentstroke}{rgb}{0.000000,0.000000,0.000000}%
\pgfsetstrokecolor{currentstroke}%
\pgfsetdash{}{0pt}%
\pgfpathmoveto{\pgfqpoint{0.800000in}{0.528000in}}%
\pgfpathlineto{\pgfqpoint{0.800000in}{4.224000in}}%
\pgfusepath{stroke}%
\end{pgfscope}%
\begin{pgfscope}%
\pgfsetrectcap%
\pgfsetmiterjoin%
\pgfsetlinewidth{0.803000pt}%
\definecolor{currentstroke}{rgb}{0.000000,0.000000,0.000000}%
\pgfsetstrokecolor{currentstroke}%
\pgfsetdash{}{0pt}%
\pgfpathmoveto{\pgfqpoint{4.768000in}{0.528000in}}%
\pgfpathlineto{\pgfqpoint{4.768000in}{4.224000in}}%
\pgfusepath{stroke}%
\end{pgfscope}%
\begin{pgfscope}%
\pgfsetrectcap%
\pgfsetmiterjoin%
\pgfsetlinewidth{0.803000pt}%
\definecolor{currentstroke}{rgb}{0.000000,0.000000,0.000000}%
\pgfsetstrokecolor{currentstroke}%
\pgfsetdash{}{0pt}%
\pgfpathmoveto{\pgfqpoint{0.800000in}{0.528000in}}%
\pgfpathlineto{\pgfqpoint{4.768000in}{0.528000in}}%
\pgfusepath{stroke}%
\end{pgfscope}%
\begin{pgfscope}%
\pgfsetrectcap%
\pgfsetmiterjoin%
\pgfsetlinewidth{0.803000pt}%
\definecolor{currentstroke}{rgb}{0.000000,0.000000,0.000000}%
\pgfsetstrokecolor{currentstroke}%
\pgfsetdash{}{0pt}%
\pgfpathmoveto{\pgfqpoint{0.800000in}{4.224000in}}%
\pgfpathlineto{\pgfqpoint{4.768000in}{4.224000in}}%
\pgfusepath{stroke}%
\end{pgfscope}%
\begin{pgfscope}%
\pgfsetbuttcap%
\pgfsetmiterjoin%
\pgfsetlinewidth{0.000000pt}%
\definecolor{currentstroke}{rgb}{0.000000,0.000000,0.000000}%
\pgfsetstrokecolor{currentstroke}%
\pgfsetstrokeopacity{0.000000}%
\pgfsetdash{}{0pt}%
\pgfpathmoveto{\pgfqpoint{5.016000in}{0.528000in}}%
\pgfpathlineto{\pgfqpoint{5.200800in}{0.528000in}}%
\pgfpathlineto{\pgfqpoint{5.200800in}{4.224000in}}%
\pgfpathlineto{\pgfqpoint{5.016000in}{4.224000in}}%
\pgfpathlineto{\pgfqpoint{5.016000in}{0.528000in}}%
\pgfpathclose%
\pgfusepath{}%
\end{pgfscope}%
\begin{pgfscope}%
\pgfsys@transformshift{5.020000in}{0.530000in}%
\pgftext[left,bottom]{\includegraphics[interpolate=true,width=0.180000in,height=3.690000in]{antiderivative_ci_5-img1.png}}%
\end{pgfscope}%
\begin{pgfscope}%
\pgfsetbuttcap%
\pgfsetroundjoin%
\definecolor{currentfill}{rgb}{0.000000,0.000000,0.000000}%
\pgfsetfillcolor{currentfill}%
\pgfsetlinewidth{0.803000pt}%
\definecolor{currentstroke}{rgb}{0.000000,0.000000,0.000000}%
\pgfsetstrokecolor{currentstroke}%
\pgfsetdash{}{0pt}%
\pgfsys@defobject{currentmarker}{\pgfqpoint{0.000000in}{0.000000in}}{\pgfqpoint{0.048611in}{0.000000in}}{%
\pgfpathmoveto{\pgfqpoint{0.000000in}{0.000000in}}%
\pgfpathlineto{\pgfqpoint{0.048611in}{0.000000in}}%
\pgfusepath{stroke,fill}%
}%
\begin{pgfscope}%
\pgfsys@transformshift{5.200800in}{0.713682in}%
\pgfsys@useobject{currentmarker}{}%
\end{pgfscope}%
\end{pgfscope}%
\begin{pgfscope}%
\definecolor{textcolor}{rgb}{0.000000,0.000000,0.000000}%
\pgfsetstrokecolor{textcolor}%
\pgfsetfillcolor{textcolor}%
\pgftext[x=5.298022in, y=0.650368in, left, base]{\color{textcolor}{\rmfamily\fontsize{12.000000}{14.400000}\selectfont\catcode`\^=\active\def^{\ifmmode\sp\else\^{}\fi}\catcode`\%=\active\def%{\%}\ensuremath{-}100}}%
\end{pgfscope}%
\begin{pgfscope}%
\pgfsetbuttcap%
\pgfsetroundjoin%
\definecolor{currentfill}{rgb}{0.000000,0.000000,0.000000}%
\pgfsetfillcolor{currentfill}%
\pgfsetlinewidth{0.803000pt}%
\definecolor{currentstroke}{rgb}{0.000000,0.000000,0.000000}%
\pgfsetstrokecolor{currentstroke}%
\pgfsetdash{}{0pt}%
\pgfsys@defobject{currentmarker}{\pgfqpoint{0.000000in}{0.000000in}}{\pgfqpoint{0.048611in}{0.000000in}}{%
\pgfpathmoveto{\pgfqpoint{0.000000in}{0.000000in}}%
\pgfpathlineto{\pgfqpoint{0.048611in}{0.000000in}}%
\pgfusepath{stroke,fill}%
}%
\begin{pgfscope}%
\pgfsys@transformshift{5.200800in}{1.129261in}%
\pgfsys@useobject{currentmarker}{}%
\end{pgfscope}%
\end{pgfscope}%
\begin{pgfscope}%
\definecolor{textcolor}{rgb}{0.000000,0.000000,0.000000}%
\pgfsetstrokecolor{textcolor}%
\pgfsetfillcolor{textcolor}%
\pgftext[x=5.298022in, y=1.065948in, left, base]{\color{textcolor}{\rmfamily\fontsize{12.000000}{14.400000}\selectfont\catcode`\^=\active\def^{\ifmmode\sp\else\^{}\fi}\catcode`\%=\active\def%{\%}\ensuremath{-}75}}%
\end{pgfscope}%
\begin{pgfscope}%
\pgfsetbuttcap%
\pgfsetroundjoin%
\definecolor{currentfill}{rgb}{0.000000,0.000000,0.000000}%
\pgfsetfillcolor{currentfill}%
\pgfsetlinewidth{0.803000pt}%
\definecolor{currentstroke}{rgb}{0.000000,0.000000,0.000000}%
\pgfsetstrokecolor{currentstroke}%
\pgfsetdash{}{0pt}%
\pgfsys@defobject{currentmarker}{\pgfqpoint{0.000000in}{0.000000in}}{\pgfqpoint{0.048611in}{0.000000in}}{%
\pgfpathmoveto{\pgfqpoint{0.000000in}{0.000000in}}%
\pgfpathlineto{\pgfqpoint{0.048611in}{0.000000in}}%
\pgfusepath{stroke,fill}%
}%
\begin{pgfscope}%
\pgfsys@transformshift{5.200800in}{1.544841in}%
\pgfsys@useobject{currentmarker}{}%
\end{pgfscope}%
\end{pgfscope}%
\begin{pgfscope}%
\definecolor{textcolor}{rgb}{0.000000,0.000000,0.000000}%
\pgfsetstrokecolor{textcolor}%
\pgfsetfillcolor{textcolor}%
\pgftext[x=5.298022in, y=1.481527in, left, base]{\color{textcolor}{\rmfamily\fontsize{12.000000}{14.400000}\selectfont\catcode`\^=\active\def^{\ifmmode\sp\else\^{}\fi}\catcode`\%=\active\def%{\%}\ensuremath{-}50}}%
\end{pgfscope}%
\begin{pgfscope}%
\pgfsetbuttcap%
\pgfsetroundjoin%
\definecolor{currentfill}{rgb}{0.000000,0.000000,0.000000}%
\pgfsetfillcolor{currentfill}%
\pgfsetlinewidth{0.803000pt}%
\definecolor{currentstroke}{rgb}{0.000000,0.000000,0.000000}%
\pgfsetstrokecolor{currentstroke}%
\pgfsetdash{}{0pt}%
\pgfsys@defobject{currentmarker}{\pgfqpoint{0.000000in}{0.000000in}}{\pgfqpoint{0.048611in}{0.000000in}}{%
\pgfpathmoveto{\pgfqpoint{0.000000in}{0.000000in}}%
\pgfpathlineto{\pgfqpoint{0.048611in}{0.000000in}}%
\pgfusepath{stroke,fill}%
}%
\begin{pgfscope}%
\pgfsys@transformshift{5.200800in}{1.960421in}%
\pgfsys@useobject{currentmarker}{}%
\end{pgfscope}%
\end{pgfscope}%
\begin{pgfscope}%
\definecolor{textcolor}{rgb}{0.000000,0.000000,0.000000}%
\pgfsetstrokecolor{textcolor}%
\pgfsetfillcolor{textcolor}%
\pgftext[x=5.298022in, y=1.897107in, left, base]{\color{textcolor}{\rmfamily\fontsize{12.000000}{14.400000}\selectfont\catcode`\^=\active\def^{\ifmmode\sp\else\^{}\fi}\catcode`\%=\active\def%{\%}\ensuremath{-}25}}%
\end{pgfscope}%
\begin{pgfscope}%
\pgfsetbuttcap%
\pgfsetroundjoin%
\definecolor{currentfill}{rgb}{0.000000,0.000000,0.000000}%
\pgfsetfillcolor{currentfill}%
\pgfsetlinewidth{0.803000pt}%
\definecolor{currentstroke}{rgb}{0.000000,0.000000,0.000000}%
\pgfsetstrokecolor{currentstroke}%
\pgfsetdash{}{0pt}%
\pgfsys@defobject{currentmarker}{\pgfqpoint{0.000000in}{0.000000in}}{\pgfqpoint{0.048611in}{0.000000in}}{%
\pgfpathmoveto{\pgfqpoint{0.000000in}{0.000000in}}%
\pgfpathlineto{\pgfqpoint{0.048611in}{0.000000in}}%
\pgfusepath{stroke,fill}%
}%
\begin{pgfscope}%
\pgfsys@transformshift{5.200800in}{2.376000in}%
\pgfsys@useobject{currentmarker}{}%
\end{pgfscope}%
\end{pgfscope}%
\begin{pgfscope}%
\definecolor{textcolor}{rgb}{0.000000,0.000000,0.000000}%
\pgfsetstrokecolor{textcolor}%
\pgfsetfillcolor{textcolor}%
\pgftext[x=5.298022in, y=2.312687in, left, base]{\color{textcolor}{\rmfamily\fontsize{12.000000}{14.400000}\selectfont\catcode`\^=\active\def^{\ifmmode\sp\else\^{}\fi}\catcode`\%=\active\def%{\%}0}}%
\end{pgfscope}%
\begin{pgfscope}%
\pgfsetbuttcap%
\pgfsetroundjoin%
\definecolor{currentfill}{rgb}{0.000000,0.000000,0.000000}%
\pgfsetfillcolor{currentfill}%
\pgfsetlinewidth{0.803000pt}%
\definecolor{currentstroke}{rgb}{0.000000,0.000000,0.000000}%
\pgfsetstrokecolor{currentstroke}%
\pgfsetdash{}{0pt}%
\pgfsys@defobject{currentmarker}{\pgfqpoint{0.000000in}{0.000000in}}{\pgfqpoint{0.048611in}{0.000000in}}{%
\pgfpathmoveto{\pgfqpoint{0.000000in}{0.000000in}}%
\pgfpathlineto{\pgfqpoint{0.048611in}{0.000000in}}%
\pgfusepath{stroke,fill}%
}%
\begin{pgfscope}%
\pgfsys@transformshift{5.200800in}{2.791580in}%
\pgfsys@useobject{currentmarker}{}%
\end{pgfscope}%
\end{pgfscope}%
\begin{pgfscope}%
\definecolor{textcolor}{rgb}{0.000000,0.000000,0.000000}%
\pgfsetstrokecolor{textcolor}%
\pgfsetfillcolor{textcolor}%
\pgftext[x=5.298022in, y=2.728266in, left, base]{\color{textcolor}{\rmfamily\fontsize{12.000000}{14.400000}\selectfont\catcode`\^=\active\def^{\ifmmode\sp\else\^{}\fi}\catcode`\%=\active\def%{\%}25}}%
\end{pgfscope}%
\begin{pgfscope}%
\pgfsetbuttcap%
\pgfsetroundjoin%
\definecolor{currentfill}{rgb}{0.000000,0.000000,0.000000}%
\pgfsetfillcolor{currentfill}%
\pgfsetlinewidth{0.803000pt}%
\definecolor{currentstroke}{rgb}{0.000000,0.000000,0.000000}%
\pgfsetstrokecolor{currentstroke}%
\pgfsetdash{}{0pt}%
\pgfsys@defobject{currentmarker}{\pgfqpoint{0.000000in}{0.000000in}}{\pgfqpoint{0.048611in}{0.000000in}}{%
\pgfpathmoveto{\pgfqpoint{0.000000in}{0.000000in}}%
\pgfpathlineto{\pgfqpoint{0.048611in}{0.000000in}}%
\pgfusepath{stroke,fill}%
}%
\begin{pgfscope}%
\pgfsys@transformshift{5.200800in}{3.207160in}%
\pgfsys@useobject{currentmarker}{}%
\end{pgfscope}%
\end{pgfscope}%
\begin{pgfscope}%
\definecolor{textcolor}{rgb}{0.000000,0.000000,0.000000}%
\pgfsetstrokecolor{textcolor}%
\pgfsetfillcolor{textcolor}%
\pgftext[x=5.298022in, y=3.143846in, left, base]{\color{textcolor}{\rmfamily\fontsize{12.000000}{14.400000}\selectfont\catcode`\^=\active\def^{\ifmmode\sp\else\^{}\fi}\catcode`\%=\active\def%{\%}50}}%
\end{pgfscope}%
\begin{pgfscope}%
\pgfsetbuttcap%
\pgfsetroundjoin%
\definecolor{currentfill}{rgb}{0.000000,0.000000,0.000000}%
\pgfsetfillcolor{currentfill}%
\pgfsetlinewidth{0.803000pt}%
\definecolor{currentstroke}{rgb}{0.000000,0.000000,0.000000}%
\pgfsetstrokecolor{currentstroke}%
\pgfsetdash{}{0pt}%
\pgfsys@defobject{currentmarker}{\pgfqpoint{0.000000in}{0.000000in}}{\pgfqpoint{0.048611in}{0.000000in}}{%
\pgfpathmoveto{\pgfqpoint{0.000000in}{0.000000in}}%
\pgfpathlineto{\pgfqpoint{0.048611in}{0.000000in}}%
\pgfusepath{stroke,fill}%
}%
\begin{pgfscope}%
\pgfsys@transformshift{5.200800in}{3.622739in}%
\pgfsys@useobject{currentmarker}{}%
\end{pgfscope}%
\end{pgfscope}%
\begin{pgfscope}%
\definecolor{textcolor}{rgb}{0.000000,0.000000,0.000000}%
\pgfsetstrokecolor{textcolor}%
\pgfsetfillcolor{textcolor}%
\pgftext[x=5.298022in, y=3.559425in, left, base]{\color{textcolor}{\rmfamily\fontsize{12.000000}{14.400000}\selectfont\catcode`\^=\active\def^{\ifmmode\sp\else\^{}\fi}\catcode`\%=\active\def%{\%}75}}%
\end{pgfscope}%
\begin{pgfscope}%
\pgfsetbuttcap%
\pgfsetroundjoin%
\definecolor{currentfill}{rgb}{0.000000,0.000000,0.000000}%
\pgfsetfillcolor{currentfill}%
\pgfsetlinewidth{0.803000pt}%
\definecolor{currentstroke}{rgb}{0.000000,0.000000,0.000000}%
\pgfsetstrokecolor{currentstroke}%
\pgfsetdash{}{0pt}%
\pgfsys@defobject{currentmarker}{\pgfqpoint{0.000000in}{0.000000in}}{\pgfqpoint{0.048611in}{0.000000in}}{%
\pgfpathmoveto{\pgfqpoint{0.000000in}{0.000000in}}%
\pgfpathlineto{\pgfqpoint{0.048611in}{0.000000in}}%
\pgfusepath{stroke,fill}%
}%
\begin{pgfscope}%
\pgfsys@transformshift{5.200800in}{4.038319in}%
\pgfsys@useobject{currentmarker}{}%
\end{pgfscope}%
\end{pgfscope}%
\begin{pgfscope}%
\definecolor{textcolor}{rgb}{0.000000,0.000000,0.000000}%
\pgfsetstrokecolor{textcolor}%
\pgfsetfillcolor{textcolor}%
\pgftext[x=5.298022in, y=3.975005in, left, base]{\color{textcolor}{\rmfamily\fontsize{12.000000}{14.400000}\selectfont\catcode`\^=\active\def^{\ifmmode\sp\else\^{}\fi}\catcode`\%=\active\def%{\%}100}}%
\end{pgfscope}%
\begin{pgfscope}%
\pgfsetrectcap%
\pgfsetmiterjoin%
\pgfsetlinewidth{0.803000pt}%
\definecolor{currentstroke}{rgb}{0.000000,0.000000,0.000000}%
\pgfsetstrokecolor{currentstroke}%
\pgfsetdash{}{0pt}%
\pgfpathmoveto{\pgfqpoint{5.016000in}{0.528000in}}%
\pgfpathlineto{\pgfqpoint{5.108400in}{0.528000in}}%
\pgfpathlineto{\pgfqpoint{5.200800in}{0.528000in}}%
\pgfpathlineto{\pgfqpoint{5.200800in}{4.224000in}}%
\pgfpathlineto{\pgfqpoint{5.108400in}{4.224000in}}%
\pgfpathlineto{\pgfqpoint{5.016000in}{4.224000in}}%
\pgfpathlineto{\pgfqpoint{5.016000in}{0.528000in}}%
\pgfpathclose%
\pgfusepath{stroke}%
\end{pgfscope}%
\end{pgfpicture}%
\makeatother%
\endgroup%

      \end{adjustbox}
      \caption{Correlation image 5\% noise.}\label{fig:sc1_ci_5}
    \end{subfigure}
    \begin{subfigure}{0.49\linewidth}
      \begin{adjustbox}{width=\linewidth}
        \begingroup%
\makeatletter%
\begin{pgfpicture}%
\pgfpathrectangle{\pgfpointorigin}{\pgfqpoint{6.400000in}{4.800000in}}%
\pgfusepath{use as bounding box, clip}%
\begin{pgfscope}%
\pgfsetbuttcap%
\pgfsetmiterjoin%
\pgfsetlinewidth{0.000000pt}%
\definecolor{currentstroke}{rgb}{0.000000,0.000000,0.000000}%
\pgfsetstrokecolor{currentstroke}%
\pgfsetstrokeopacity{0.000000}%
\pgfsetdash{}{0pt}%
\pgfpathmoveto{\pgfqpoint{0.000000in}{0.000000in}}%
\pgfpathlineto{\pgfqpoint{6.400000in}{0.000000in}}%
\pgfpathlineto{\pgfqpoint{6.400000in}{4.800000in}}%
\pgfpathlineto{\pgfqpoint{0.000000in}{4.800000in}}%
\pgfpathlineto{\pgfqpoint{0.000000in}{0.000000in}}%
\pgfpathclose%
\pgfusepath{}%
\end{pgfscope}%
\begin{pgfscope}%
\pgfsetbuttcap%
\pgfsetmiterjoin%
\pgfsetlinewidth{0.000000pt}%
\definecolor{currentstroke}{rgb}{0.000000,0.000000,0.000000}%
\pgfsetstrokecolor{currentstroke}%
\pgfsetstrokeopacity{0.000000}%
\pgfsetdash{}{0pt}%
\pgfpathmoveto{\pgfqpoint{1.072000in}{0.528000in}}%
\pgfpathlineto{\pgfqpoint{4.768000in}{0.528000in}}%
\pgfpathlineto{\pgfqpoint{4.768000in}{4.224000in}}%
\pgfpathlineto{\pgfqpoint{1.072000in}{4.224000in}}%
\pgfpathlineto{\pgfqpoint{1.072000in}{0.528000in}}%
\pgfpathclose%
\pgfusepath{}%
\end{pgfscope}%
\begin{pgfscope}%
\pgfpathrectangle{\pgfqpoint{1.072000in}{0.528000in}}{\pgfqpoint{3.696000in}{3.696000in}}%
\pgfusepath{clip}%
\pgfsys@transformcm{3.696000}{0.000000}{0.000000}{-3.696000}{1.072000in}{4.224000in}%
\pgftext[left,bottom]{\includegraphics[interpolate=false,width=1.000000in,height=1.000000in]{antiderivative_pm_5-img0.png}}%
\end{pgfscope}%
\begin{pgfscope}%
\pgfsetbuttcap%
\pgfsetroundjoin%
\definecolor{currentfill}{rgb}{0.000000,0.000000,0.000000}%
\pgfsetfillcolor{currentfill}%
\pgfsetlinewidth{0.803000pt}%
\definecolor{currentstroke}{rgb}{0.000000,0.000000,0.000000}%
\pgfsetstrokecolor{currentstroke}%
\pgfsetdash{}{0pt}%
\pgfsys@defobject{currentmarker}{\pgfqpoint{0.000000in}{-0.048611in}}{\pgfqpoint{0.000000in}{0.000000in}}{%
\pgfpathmoveto{\pgfqpoint{0.000000in}{0.000000in}}%
\pgfpathlineto{\pgfqpoint{0.000000in}{-0.048611in}}%
\pgfusepath{stroke,fill}%
}%
\begin{pgfscope}%
\pgfsys@transformshift{1.090480in}{0.528000in}%
\pgfsys@useobject{currentmarker}{}%
\end{pgfscope}%
\end{pgfscope}%
\begin{pgfscope}%
\definecolor{textcolor}{rgb}{0.000000,0.000000,0.000000}%
\pgfsetstrokecolor{textcolor}%
\pgfsetfillcolor{textcolor}%
\pgftext[x=1.090480in,y=0.430778in,,top]{\color{textcolor}{\rmfamily\fontsize{12.000000}{14.400000}\selectfont\catcode`\^=\active\def^{\ifmmode\sp\else\^{}\fi}\catcode`\%=\active\def%{\%}0}}%
\end{pgfscope}%
\begin{pgfscope}%
\pgfsetbuttcap%
\pgfsetroundjoin%
\definecolor{currentfill}{rgb}{0.000000,0.000000,0.000000}%
\pgfsetfillcolor{currentfill}%
\pgfsetlinewidth{0.803000pt}%
\definecolor{currentstroke}{rgb}{0.000000,0.000000,0.000000}%
\pgfsetstrokecolor{currentstroke}%
\pgfsetdash{}{0pt}%
\pgfsys@defobject{currentmarker}{\pgfqpoint{0.000000in}{-0.048611in}}{\pgfqpoint{0.000000in}{0.000000in}}{%
\pgfpathmoveto{\pgfqpoint{0.000000in}{0.000000in}}%
\pgfpathlineto{\pgfqpoint{0.000000in}{-0.048611in}}%
\pgfusepath{stroke,fill}%
}%
\begin{pgfscope}%
\pgfsys@transformshift{1.829680in}{0.528000in}%
\pgfsys@useobject{currentmarker}{}%
\end{pgfscope}%
\end{pgfscope}%
\begin{pgfscope}%
\definecolor{textcolor}{rgb}{0.000000,0.000000,0.000000}%
\pgfsetstrokecolor{textcolor}%
\pgfsetfillcolor{textcolor}%
\pgftext[x=1.829680in,y=0.430778in,,top]{\color{textcolor}{\rmfamily\fontsize{12.000000}{14.400000}\selectfont\catcode`\^=\active\def^{\ifmmode\sp\else\^{}\fi}\catcode`\%=\active\def%{\%}20}}%
\end{pgfscope}%
\begin{pgfscope}%
\pgfsetbuttcap%
\pgfsetroundjoin%
\definecolor{currentfill}{rgb}{0.000000,0.000000,0.000000}%
\pgfsetfillcolor{currentfill}%
\pgfsetlinewidth{0.803000pt}%
\definecolor{currentstroke}{rgb}{0.000000,0.000000,0.000000}%
\pgfsetstrokecolor{currentstroke}%
\pgfsetdash{}{0pt}%
\pgfsys@defobject{currentmarker}{\pgfqpoint{0.000000in}{-0.048611in}}{\pgfqpoint{0.000000in}{0.000000in}}{%
\pgfpathmoveto{\pgfqpoint{0.000000in}{0.000000in}}%
\pgfpathlineto{\pgfqpoint{0.000000in}{-0.048611in}}%
\pgfusepath{stroke,fill}%
}%
\begin{pgfscope}%
\pgfsys@transformshift{2.568880in}{0.528000in}%
\pgfsys@useobject{currentmarker}{}%
\end{pgfscope}%
\end{pgfscope}%
\begin{pgfscope}%
\definecolor{textcolor}{rgb}{0.000000,0.000000,0.000000}%
\pgfsetstrokecolor{textcolor}%
\pgfsetfillcolor{textcolor}%
\pgftext[x=2.568880in,y=0.430778in,,top]{\color{textcolor}{\rmfamily\fontsize{12.000000}{14.400000}\selectfont\catcode`\^=\active\def^{\ifmmode\sp\else\^{}\fi}\catcode`\%=\active\def%{\%}40}}%
\end{pgfscope}%
\begin{pgfscope}%
\pgfsetbuttcap%
\pgfsetroundjoin%
\definecolor{currentfill}{rgb}{0.000000,0.000000,0.000000}%
\pgfsetfillcolor{currentfill}%
\pgfsetlinewidth{0.803000pt}%
\definecolor{currentstroke}{rgb}{0.000000,0.000000,0.000000}%
\pgfsetstrokecolor{currentstroke}%
\pgfsetdash{}{0pt}%
\pgfsys@defobject{currentmarker}{\pgfqpoint{0.000000in}{-0.048611in}}{\pgfqpoint{0.000000in}{0.000000in}}{%
\pgfpathmoveto{\pgfqpoint{0.000000in}{0.000000in}}%
\pgfpathlineto{\pgfqpoint{0.000000in}{-0.048611in}}%
\pgfusepath{stroke,fill}%
}%
\begin{pgfscope}%
\pgfsys@transformshift{3.308080in}{0.528000in}%
\pgfsys@useobject{currentmarker}{}%
\end{pgfscope}%
\end{pgfscope}%
\begin{pgfscope}%
\definecolor{textcolor}{rgb}{0.000000,0.000000,0.000000}%
\pgfsetstrokecolor{textcolor}%
\pgfsetfillcolor{textcolor}%
\pgftext[x=3.308080in,y=0.430778in,,top]{\color{textcolor}{\rmfamily\fontsize{12.000000}{14.400000}\selectfont\catcode`\^=\active\def^{\ifmmode\sp\else\^{}\fi}\catcode`\%=\active\def%{\%}60}}%
\end{pgfscope}%
\begin{pgfscope}%
\pgfsetbuttcap%
\pgfsetroundjoin%
\definecolor{currentfill}{rgb}{0.000000,0.000000,0.000000}%
\pgfsetfillcolor{currentfill}%
\pgfsetlinewidth{0.803000pt}%
\definecolor{currentstroke}{rgb}{0.000000,0.000000,0.000000}%
\pgfsetstrokecolor{currentstroke}%
\pgfsetdash{}{0pt}%
\pgfsys@defobject{currentmarker}{\pgfqpoint{0.000000in}{-0.048611in}}{\pgfqpoint{0.000000in}{0.000000in}}{%
\pgfpathmoveto{\pgfqpoint{0.000000in}{0.000000in}}%
\pgfpathlineto{\pgfqpoint{0.000000in}{-0.048611in}}%
\pgfusepath{stroke,fill}%
}%
\begin{pgfscope}%
\pgfsys@transformshift{4.047280in}{0.528000in}%
\pgfsys@useobject{currentmarker}{}%
\end{pgfscope}%
\end{pgfscope}%
\begin{pgfscope}%
\definecolor{textcolor}{rgb}{0.000000,0.000000,0.000000}%
\pgfsetstrokecolor{textcolor}%
\pgfsetfillcolor{textcolor}%
\pgftext[x=4.047280in,y=0.430778in,,top]{\color{textcolor}{\rmfamily\fontsize{12.000000}{14.400000}\selectfont\catcode`\^=\active\def^{\ifmmode\sp\else\^{}\fi}\catcode`\%=\active\def%{\%}80}}%
\end{pgfscope}%
\begin{pgfscope}%
\definecolor{textcolor}{rgb}{0.000000,0.000000,0.000000}%
\pgfsetstrokecolor{textcolor}%
\pgfsetfillcolor{textcolor}%
\pgftext[x=2.920000in,y=0.213927in,,top]{\color{textcolor}{\rmfamily\fontsize{12.000000}{14.400000}\selectfont\catcode`\^=\active\def^{\ifmmode\sp\else\^{}\fi}\catcode`\%=\active\def%{\%}output coefficients}}%
\end{pgfscope}%
\begin{pgfscope}%
\pgfsetbuttcap%
\pgfsetroundjoin%
\definecolor{currentfill}{rgb}{0.000000,0.000000,0.000000}%
\pgfsetfillcolor{currentfill}%
\pgfsetlinewidth{0.803000pt}%
\definecolor{currentstroke}{rgb}{0.000000,0.000000,0.000000}%
\pgfsetstrokecolor{currentstroke}%
\pgfsetdash{}{0pt}%
\pgfsys@defobject{currentmarker}{\pgfqpoint{-0.048611in}{0.000000in}}{\pgfqpoint{-0.000000in}{0.000000in}}{%
\pgfpathmoveto{\pgfqpoint{-0.000000in}{0.000000in}}%
\pgfpathlineto{\pgfqpoint{-0.048611in}{0.000000in}}%
\pgfusepath{stroke,fill}%
}%
\begin{pgfscope}%
\pgfsys@transformshift{1.072000in}{4.205520in}%
\pgfsys@useobject{currentmarker}{}%
\end{pgfscope}%
\end{pgfscope}%
\begin{pgfscope}%
\definecolor{textcolor}{rgb}{0.000000,0.000000,0.000000}%
\pgfsetstrokecolor{textcolor}%
\pgfsetfillcolor{textcolor}%
\pgftext[x=0.868739in, y=4.142206in, left, base]{\color{textcolor}{\rmfamily\fontsize{12.000000}{14.400000}\selectfont\catcode`\^=\active\def^{\ifmmode\sp\else\^{}\fi}\catcode`\%=\active\def%{\%}0}}%
\end{pgfscope}%
\begin{pgfscope}%
\pgfsetbuttcap%
\pgfsetroundjoin%
\definecolor{currentfill}{rgb}{0.000000,0.000000,0.000000}%
\pgfsetfillcolor{currentfill}%
\pgfsetlinewidth{0.803000pt}%
\definecolor{currentstroke}{rgb}{0.000000,0.000000,0.000000}%
\pgfsetstrokecolor{currentstroke}%
\pgfsetdash{}{0pt}%
\pgfsys@defobject{currentmarker}{\pgfqpoint{-0.048611in}{0.000000in}}{\pgfqpoint{-0.000000in}{0.000000in}}{%
\pgfpathmoveto{\pgfqpoint{-0.000000in}{0.000000in}}%
\pgfpathlineto{\pgfqpoint{-0.048611in}{0.000000in}}%
\pgfusepath{stroke,fill}%
}%
\begin{pgfscope}%
\pgfsys@transformshift{1.072000in}{3.466320in}%
\pgfsys@useobject{currentmarker}{}%
\end{pgfscope}%
\end{pgfscope}%
\begin{pgfscope}%
\definecolor{textcolor}{rgb}{0.000000,0.000000,0.000000}%
\pgfsetstrokecolor{textcolor}%
\pgfsetfillcolor{textcolor}%
\pgftext[x=0.762701in, y=3.403006in, left, base]{\color{textcolor}{\rmfamily\fontsize{12.000000}{14.400000}\selectfont\catcode`\^=\active\def^{\ifmmode\sp\else\^{}\fi}\catcode`\%=\active\def%{\%}20}}%
\end{pgfscope}%
\begin{pgfscope}%
\pgfsetbuttcap%
\pgfsetroundjoin%
\definecolor{currentfill}{rgb}{0.000000,0.000000,0.000000}%
\pgfsetfillcolor{currentfill}%
\pgfsetlinewidth{0.803000pt}%
\definecolor{currentstroke}{rgb}{0.000000,0.000000,0.000000}%
\pgfsetstrokecolor{currentstroke}%
\pgfsetdash{}{0pt}%
\pgfsys@defobject{currentmarker}{\pgfqpoint{-0.048611in}{0.000000in}}{\pgfqpoint{-0.000000in}{0.000000in}}{%
\pgfpathmoveto{\pgfqpoint{-0.000000in}{0.000000in}}%
\pgfpathlineto{\pgfqpoint{-0.048611in}{0.000000in}}%
\pgfusepath{stroke,fill}%
}%
\begin{pgfscope}%
\pgfsys@transformshift{1.072000in}{2.727120in}%
\pgfsys@useobject{currentmarker}{}%
\end{pgfscope}%
\end{pgfscope}%
\begin{pgfscope}%
\definecolor{textcolor}{rgb}{0.000000,0.000000,0.000000}%
\pgfsetstrokecolor{textcolor}%
\pgfsetfillcolor{textcolor}%
\pgftext[x=0.762701in, y=2.663806in, left, base]{\color{textcolor}{\rmfamily\fontsize{12.000000}{14.400000}\selectfont\catcode`\^=\active\def^{\ifmmode\sp\else\^{}\fi}\catcode`\%=\active\def%{\%}40}}%
\end{pgfscope}%
\begin{pgfscope}%
\pgfsetbuttcap%
\pgfsetroundjoin%
\definecolor{currentfill}{rgb}{0.000000,0.000000,0.000000}%
\pgfsetfillcolor{currentfill}%
\pgfsetlinewidth{0.803000pt}%
\definecolor{currentstroke}{rgb}{0.000000,0.000000,0.000000}%
\pgfsetstrokecolor{currentstroke}%
\pgfsetdash{}{0pt}%
\pgfsys@defobject{currentmarker}{\pgfqpoint{-0.048611in}{0.000000in}}{\pgfqpoint{-0.000000in}{0.000000in}}{%
\pgfpathmoveto{\pgfqpoint{-0.000000in}{0.000000in}}%
\pgfpathlineto{\pgfqpoint{-0.048611in}{0.000000in}}%
\pgfusepath{stroke,fill}%
}%
\begin{pgfscope}%
\pgfsys@transformshift{1.072000in}{1.987920in}%
\pgfsys@useobject{currentmarker}{}%
\end{pgfscope}%
\end{pgfscope}%
\begin{pgfscope}%
\definecolor{textcolor}{rgb}{0.000000,0.000000,0.000000}%
\pgfsetstrokecolor{textcolor}%
\pgfsetfillcolor{textcolor}%
\pgftext[x=0.762701in, y=1.924606in, left, base]{\color{textcolor}{\rmfamily\fontsize{12.000000}{14.400000}\selectfont\catcode`\^=\active\def^{\ifmmode\sp\else\^{}\fi}\catcode`\%=\active\def%{\%}60}}%
\end{pgfscope}%
\begin{pgfscope}%
\pgfsetbuttcap%
\pgfsetroundjoin%
\definecolor{currentfill}{rgb}{0.000000,0.000000,0.000000}%
\pgfsetfillcolor{currentfill}%
\pgfsetlinewidth{0.803000pt}%
\definecolor{currentstroke}{rgb}{0.000000,0.000000,0.000000}%
\pgfsetstrokecolor{currentstroke}%
\pgfsetdash{}{0pt}%
\pgfsys@defobject{currentmarker}{\pgfqpoint{-0.048611in}{0.000000in}}{\pgfqpoint{-0.000000in}{0.000000in}}{%
\pgfpathmoveto{\pgfqpoint{-0.000000in}{0.000000in}}%
\pgfpathlineto{\pgfqpoint{-0.048611in}{0.000000in}}%
\pgfusepath{stroke,fill}%
}%
\begin{pgfscope}%
\pgfsys@transformshift{1.072000in}{1.248720in}%
\pgfsys@useobject{currentmarker}{}%
\end{pgfscope}%
\end{pgfscope}%
\begin{pgfscope}%
\definecolor{textcolor}{rgb}{0.000000,0.000000,0.000000}%
\pgfsetstrokecolor{textcolor}%
\pgfsetfillcolor{textcolor}%
\pgftext[x=0.762701in, y=1.185406in, left, base]{\color{textcolor}{\rmfamily\fontsize{12.000000}{14.400000}\selectfont\catcode`\^=\active\def^{\ifmmode\sp\else\^{}\fi}\catcode`\%=\active\def%{\%}80}}%
\end{pgfscope}%
\begin{pgfscope}%
\definecolor{textcolor}{rgb}{0.000000,0.000000,0.000000}%
\pgfsetstrokecolor{textcolor}%
\pgfsetfillcolor{textcolor}%
\pgftext[x=0.707145in,y=2.376000in,,bottom,rotate=90.000000]{\color{textcolor}{\rmfamily\fontsize{12.000000}{14.400000}\selectfont\catcode`\^=\active\def^{\ifmmode\sp\else\^{}\fi}\catcode`\%=\active\def%{\%}input coefficients}}%
\end{pgfscope}%
\begin{pgfscope}%
\pgfsetrectcap%
\pgfsetmiterjoin%
\pgfsetlinewidth{0.803000pt}%
\definecolor{currentstroke}{rgb}{0.000000,0.000000,0.000000}%
\pgfsetstrokecolor{currentstroke}%
\pgfsetdash{}{0pt}%
\pgfpathmoveto{\pgfqpoint{1.072000in}{0.528000in}}%
\pgfpathlineto{\pgfqpoint{1.072000in}{4.224000in}}%
\pgfusepath{stroke}%
\end{pgfscope}%
\begin{pgfscope}%
\pgfsetrectcap%
\pgfsetmiterjoin%
\pgfsetlinewidth{0.803000pt}%
\definecolor{currentstroke}{rgb}{0.000000,0.000000,0.000000}%
\pgfsetstrokecolor{currentstroke}%
\pgfsetdash{}{0pt}%
\pgfpathmoveto{\pgfqpoint{4.768000in}{0.528000in}}%
\pgfpathlineto{\pgfqpoint{4.768000in}{4.224000in}}%
\pgfusepath{stroke}%
\end{pgfscope}%
\begin{pgfscope}%
\pgfsetrectcap%
\pgfsetmiterjoin%
\pgfsetlinewidth{0.803000pt}%
\definecolor{currentstroke}{rgb}{0.000000,0.000000,0.000000}%
\pgfsetstrokecolor{currentstroke}%
\pgfsetdash{}{0pt}%
\pgfpathmoveto{\pgfqpoint{1.072000in}{0.528000in}}%
\pgfpathlineto{\pgfqpoint{4.768000in}{0.528000in}}%
\pgfusepath{stroke}%
\end{pgfscope}%
\begin{pgfscope}%
\pgfsetrectcap%
\pgfsetmiterjoin%
\pgfsetlinewidth{0.803000pt}%
\definecolor{currentstroke}{rgb}{0.000000,0.000000,0.000000}%
\pgfsetstrokecolor{currentstroke}%
\pgfsetdash{}{0pt}%
\pgfpathmoveto{\pgfqpoint{1.072000in}{4.224000in}}%
\pgfpathlineto{\pgfqpoint{4.768000in}{4.224000in}}%
\pgfusepath{stroke}%
\end{pgfscope}%
\begin{pgfscope}%
\pgfsetbuttcap%
\pgfsetmiterjoin%
\pgfsetlinewidth{0.000000pt}%
\definecolor{currentstroke}{rgb}{0.000000,0.000000,0.000000}%
\pgfsetstrokecolor{currentstroke}%
\pgfsetstrokeopacity{0.000000}%
\pgfsetdash{}{0pt}%
\pgfpathmoveto{\pgfqpoint{5.016000in}{0.528000in}}%
\pgfpathlineto{\pgfqpoint{5.200800in}{0.528000in}}%
\pgfpathlineto{\pgfqpoint{5.200800in}{4.224000in}}%
\pgfpathlineto{\pgfqpoint{5.016000in}{4.224000in}}%
\pgfpathlineto{\pgfqpoint{5.016000in}{0.528000in}}%
\pgfpathclose%
\pgfusepath{}%
\end{pgfscope}%
\begin{pgfscope}%
\pgfsys@transformshift{5.020000in}{0.530000in}%
\pgftext[left,bottom]{\includegraphics[interpolate=true,width=0.180000in,height=3.690000in]{antiderivative_pm_5-img1.png}}%
\end{pgfscope}%
\begin{pgfscope}%
\pgfsetbuttcap%
\pgfsetroundjoin%
\definecolor{currentfill}{rgb}{0.000000,0.000000,0.000000}%
\pgfsetfillcolor{currentfill}%
\pgfsetlinewidth{0.803000pt}%
\definecolor{currentstroke}{rgb}{0.000000,0.000000,0.000000}%
\pgfsetstrokecolor{currentstroke}%
\pgfsetdash{}{0pt}%
\pgfsys@defobject{currentmarker}{\pgfqpoint{0.000000in}{0.000000in}}{\pgfqpoint{0.048611in}{0.000000in}}{%
\pgfpathmoveto{\pgfqpoint{0.000000in}{0.000000in}}%
\pgfpathlineto{\pgfqpoint{0.048611in}{0.000000in}}%
\pgfusepath{stroke,fill}%
}%
\begin{pgfscope}%
\pgfsys@transformshift{5.200800in}{0.806760in}%
\pgfsys@useobject{currentmarker}{}%
\end{pgfscope}%
\end{pgfscope}%
\begin{pgfscope}%
\definecolor{textcolor}{rgb}{0.000000,0.000000,0.000000}%
\pgfsetstrokecolor{textcolor}%
\pgfsetfillcolor{textcolor}%
\pgftext[x=5.298022in, y=0.743447in, left, base]{\color{textcolor}{\rmfamily\fontsize{12.000000}{14.400000}\selectfont\catcode`\^=\active\def^{\ifmmode\sp\else\^{}\fi}\catcode`\%=\active\def%{\%}\ensuremath{-}2000}}%
\end{pgfscope}%
\begin{pgfscope}%
\pgfsetbuttcap%
\pgfsetroundjoin%
\definecolor{currentfill}{rgb}{0.000000,0.000000,0.000000}%
\pgfsetfillcolor{currentfill}%
\pgfsetlinewidth{0.803000pt}%
\definecolor{currentstroke}{rgb}{0.000000,0.000000,0.000000}%
\pgfsetstrokecolor{currentstroke}%
\pgfsetdash{}{0pt}%
\pgfsys@defobject{currentmarker}{\pgfqpoint{0.000000in}{0.000000in}}{\pgfqpoint{0.048611in}{0.000000in}}{%
\pgfpathmoveto{\pgfqpoint{0.000000in}{0.000000in}}%
\pgfpathlineto{\pgfqpoint{0.048611in}{0.000000in}}%
\pgfusepath{stroke,fill}%
}%
\begin{pgfscope}%
\pgfsys@transformshift{5.200800in}{1.404820in}%
\pgfsys@useobject{currentmarker}{}%
\end{pgfscope}%
\end{pgfscope}%
\begin{pgfscope}%
\definecolor{textcolor}{rgb}{0.000000,0.000000,0.000000}%
\pgfsetstrokecolor{textcolor}%
\pgfsetfillcolor{textcolor}%
\pgftext[x=5.298022in, y=1.341506in, left, base]{\color{textcolor}{\rmfamily\fontsize{12.000000}{14.400000}\selectfont\catcode`\^=\active\def^{\ifmmode\sp\else\^{}\fi}\catcode`\%=\active\def%{\%}\ensuremath{-}1500}}%
\end{pgfscope}%
\begin{pgfscope}%
\pgfsetbuttcap%
\pgfsetroundjoin%
\definecolor{currentfill}{rgb}{0.000000,0.000000,0.000000}%
\pgfsetfillcolor{currentfill}%
\pgfsetlinewidth{0.803000pt}%
\definecolor{currentstroke}{rgb}{0.000000,0.000000,0.000000}%
\pgfsetstrokecolor{currentstroke}%
\pgfsetdash{}{0pt}%
\pgfsys@defobject{currentmarker}{\pgfqpoint{0.000000in}{0.000000in}}{\pgfqpoint{0.048611in}{0.000000in}}{%
\pgfpathmoveto{\pgfqpoint{0.000000in}{0.000000in}}%
\pgfpathlineto{\pgfqpoint{0.048611in}{0.000000in}}%
\pgfusepath{stroke,fill}%
}%
\begin{pgfscope}%
\pgfsys@transformshift{5.200800in}{2.002880in}%
\pgfsys@useobject{currentmarker}{}%
\end{pgfscope}%
\end{pgfscope}%
\begin{pgfscope}%
\definecolor{textcolor}{rgb}{0.000000,0.000000,0.000000}%
\pgfsetstrokecolor{textcolor}%
\pgfsetfillcolor{textcolor}%
\pgftext[x=5.298022in, y=1.939566in, left, base]{\color{textcolor}{\rmfamily\fontsize{12.000000}{14.400000}\selectfont\catcode`\^=\active\def^{\ifmmode\sp\else\^{}\fi}\catcode`\%=\active\def%{\%}\ensuremath{-}1000}}%
\end{pgfscope}%
\begin{pgfscope}%
\pgfsetbuttcap%
\pgfsetroundjoin%
\definecolor{currentfill}{rgb}{0.000000,0.000000,0.000000}%
\pgfsetfillcolor{currentfill}%
\pgfsetlinewidth{0.803000pt}%
\definecolor{currentstroke}{rgb}{0.000000,0.000000,0.000000}%
\pgfsetstrokecolor{currentstroke}%
\pgfsetdash{}{0pt}%
\pgfsys@defobject{currentmarker}{\pgfqpoint{0.000000in}{0.000000in}}{\pgfqpoint{0.048611in}{0.000000in}}{%
\pgfpathmoveto{\pgfqpoint{0.000000in}{0.000000in}}%
\pgfpathlineto{\pgfqpoint{0.048611in}{0.000000in}}%
\pgfusepath{stroke,fill}%
}%
\begin{pgfscope}%
\pgfsys@transformshift{5.200800in}{2.600940in}%
\pgfsys@useobject{currentmarker}{}%
\end{pgfscope}%
\end{pgfscope}%
\begin{pgfscope}%
\definecolor{textcolor}{rgb}{0.000000,0.000000,0.000000}%
\pgfsetstrokecolor{textcolor}%
\pgfsetfillcolor{textcolor}%
\pgftext[x=5.298022in, y=2.537626in, left, base]{\color{textcolor}{\rmfamily\fontsize{12.000000}{14.400000}\selectfont\catcode`\^=\active\def^{\ifmmode\sp\else\^{}\fi}\catcode`\%=\active\def%{\%}\ensuremath{-}500}}%
\end{pgfscope}%
\begin{pgfscope}%
\pgfsetbuttcap%
\pgfsetroundjoin%
\definecolor{currentfill}{rgb}{0.000000,0.000000,0.000000}%
\pgfsetfillcolor{currentfill}%
\pgfsetlinewidth{0.803000pt}%
\definecolor{currentstroke}{rgb}{0.000000,0.000000,0.000000}%
\pgfsetstrokecolor{currentstroke}%
\pgfsetdash{}{0pt}%
\pgfsys@defobject{currentmarker}{\pgfqpoint{0.000000in}{0.000000in}}{\pgfqpoint{0.048611in}{0.000000in}}{%
\pgfpathmoveto{\pgfqpoint{0.000000in}{0.000000in}}%
\pgfpathlineto{\pgfqpoint{0.048611in}{0.000000in}}%
\pgfusepath{stroke,fill}%
}%
\begin{pgfscope}%
\pgfsys@transformshift{5.200800in}{3.199000in}%
\pgfsys@useobject{currentmarker}{}%
\end{pgfscope}%
\end{pgfscope}%
\begin{pgfscope}%
\definecolor{textcolor}{rgb}{0.000000,0.000000,0.000000}%
\pgfsetstrokecolor{textcolor}%
\pgfsetfillcolor{textcolor}%
\pgftext[x=5.298022in, y=3.135686in, left, base]{\color{textcolor}{\rmfamily\fontsize{12.000000}{14.400000}\selectfont\catcode`\^=\active\def^{\ifmmode\sp\else\^{}\fi}\catcode`\%=\active\def%{\%}0}}%
\end{pgfscope}%
\begin{pgfscope}%
\pgfsetbuttcap%
\pgfsetroundjoin%
\definecolor{currentfill}{rgb}{0.000000,0.000000,0.000000}%
\pgfsetfillcolor{currentfill}%
\pgfsetlinewidth{0.803000pt}%
\definecolor{currentstroke}{rgb}{0.000000,0.000000,0.000000}%
\pgfsetstrokecolor{currentstroke}%
\pgfsetdash{}{0pt}%
\pgfsys@defobject{currentmarker}{\pgfqpoint{0.000000in}{0.000000in}}{\pgfqpoint{0.048611in}{0.000000in}}{%
\pgfpathmoveto{\pgfqpoint{0.000000in}{0.000000in}}%
\pgfpathlineto{\pgfqpoint{0.048611in}{0.000000in}}%
\pgfusepath{stroke,fill}%
}%
\begin{pgfscope}%
\pgfsys@transformshift{5.200800in}{3.797059in}%
\pgfsys@useobject{currentmarker}{}%
\end{pgfscope}%
\end{pgfscope}%
\begin{pgfscope}%
\definecolor{textcolor}{rgb}{0.000000,0.000000,0.000000}%
\pgfsetstrokecolor{textcolor}%
\pgfsetfillcolor{textcolor}%
\pgftext[x=5.298022in, y=3.733746in, left, base]{\color{textcolor}{\rmfamily\fontsize{12.000000}{14.400000}\selectfont\catcode`\^=\active\def^{\ifmmode\sp\else\^{}\fi}\catcode`\%=\active\def%{\%}500}}%
\end{pgfscope}%
\begin{pgfscope}%
\pgfsetrectcap%
\pgfsetmiterjoin%
\pgfsetlinewidth{0.803000pt}%
\definecolor{currentstroke}{rgb}{0.000000,0.000000,0.000000}%
\pgfsetstrokecolor{currentstroke}%
\pgfsetdash{}{0pt}%
\pgfpathmoveto{\pgfqpoint{5.016000in}{0.528000in}}%
\pgfpathlineto{\pgfqpoint{5.108400in}{0.528000in}}%
\pgfpathlineto{\pgfqpoint{5.200800in}{0.528000in}}%
\pgfpathlineto{\pgfqpoint{5.200800in}{4.224000in}}%
\pgfpathlineto{\pgfqpoint{5.108400in}{4.224000in}}%
\pgfpathlineto{\pgfqpoint{5.016000in}{4.224000in}}%
\pgfpathlineto{\pgfqpoint{5.016000in}{0.528000in}}%
\pgfpathclose%
\pgfusepath{stroke}%
\end{pgfscope}%
\end{pgfpicture}%
\makeatother%
\endgroup%

      \end{adjustbox}
      \caption{The p-matrix for 5\% noise.}\label{fig:sc1_pm_5}
    \end{subfigure}
    % \\[\baselineskip]
    \begin{subfigure}{0.49\linewidth}
      \begin{adjustbox}{width=\linewidth}
        \begingroup%
\makeatletter%
\begin{pgfpicture}%
\pgfpathrectangle{\pgfpointorigin}{\pgfqpoint{4.000000in}{3.000000in}}%
\pgfusepath{use as bounding box, clip}%
\begin{pgfscope}%
\pgfsetbuttcap%
\pgfsetmiterjoin%
\pgfsetlinewidth{0.000000pt}%
\definecolor{currentstroke}{rgb}{0.000000,0.000000,0.000000}%
\pgfsetstrokecolor{currentstroke}%
\pgfsetstrokeopacity{0.000000}%
\pgfsetdash{}{0pt}%
\pgfpathmoveto{\pgfqpoint{0.000000in}{0.000000in}}%
\pgfpathlineto{\pgfqpoint{4.000000in}{0.000000in}}%
\pgfpathlineto{\pgfqpoint{4.000000in}{3.000000in}}%
\pgfpathlineto{\pgfqpoint{0.000000in}{3.000000in}}%
\pgfpathlineto{\pgfqpoint{0.000000in}{0.000000in}}%
\pgfpathclose%
\pgfusepath{}%
\end{pgfscope}%
\begin{pgfscope}%
\pgfsetbuttcap%
\pgfsetmiterjoin%
\pgfsetlinewidth{0.000000pt}%
\definecolor{currentstroke}{rgb}{0.000000,0.000000,0.000000}%
\pgfsetstrokecolor{currentstroke}%
\pgfsetstrokeopacity{0.000000}%
\pgfsetdash{}{0pt}%
\pgfpathmoveto{\pgfqpoint{0.779897in}{0.517039in}}%
\pgfpathlineto{\pgfqpoint{3.062697in}{0.517039in}}%
\pgfpathlineto{\pgfqpoint{3.062697in}{2.895016in}}%
\pgfpathlineto{\pgfqpoint{0.779897in}{2.895016in}}%
\pgfpathlineto{\pgfqpoint{0.779897in}{0.517039in}}%
\pgfpathclose%
\pgfusepath{}%
\end{pgfscope}%
\begin{pgfscope}%
\pgfpathrectangle{\pgfqpoint{0.779897in}{0.517039in}}{\pgfqpoint{2.282800in}{2.377978in}}%
\pgfusepath{clip}%
\pgfsys@transformcm{2.282800}{0.000000}{0.000000}{-2.377978}{0.779897in}{2.895016in}%
\pgftext[left,bottom]{\includegraphics[interpolate=false,width=1.000000in,height=1.000000in]{antiderivative_ci_10-img0.png}}%
\end{pgfscope}%
\begin{pgfscope}%
\pgfsetbuttcap%
\pgfsetroundjoin%
\definecolor{currentfill}{rgb}{0.000000,0.000000,0.000000}%
\pgfsetfillcolor{currentfill}%
\pgfsetlinewidth{0.803000pt}%
\definecolor{currentstroke}{rgb}{0.000000,0.000000,0.000000}%
\pgfsetstrokecolor{currentstroke}%
\pgfsetdash{}{0pt}%
\pgfsys@defobject{currentmarker}{\pgfqpoint{0.000000in}{-0.048611in}}{\pgfqpoint{0.000000in}{0.000000in}}{%
\pgfpathmoveto{\pgfqpoint{0.000000in}{0.000000in}}%
\pgfpathlineto{\pgfqpoint{0.000000in}{-0.048611in}}%
\pgfusepath{stroke,fill}%
}%
\begin{pgfscope}%
\pgfsys@transformshift{0.779897in}{0.517039in}%
\pgfsys@useobject{currentmarker}{}%
\end{pgfscope}%
\end{pgfscope}%
\begin{pgfscope}%
\definecolor{textcolor}{rgb}{0.000000,0.000000,0.000000}%
\pgfsetstrokecolor{textcolor}%
\pgfsetfillcolor{textcolor}%
\pgftext[x=0.779897in,y=0.419816in,,top]{\color{textcolor}{\rmfamily\fontsize{12.000000}{14.400000}\selectfont\catcode`\^=\active\def^{\ifmmode\sp\else\^{}\fi}\catcode`\%=\active\def%{\%}0}}%
\end{pgfscope}%
\begin{pgfscope}%
\pgfsetbuttcap%
\pgfsetroundjoin%
\definecolor{currentfill}{rgb}{0.000000,0.000000,0.000000}%
\pgfsetfillcolor{currentfill}%
\pgfsetlinewidth{0.803000pt}%
\definecolor{currentstroke}{rgb}{0.000000,0.000000,0.000000}%
\pgfsetstrokecolor{currentstroke}%
\pgfsetdash{}{0pt}%
\pgfsys@defobject{currentmarker}{\pgfqpoint{0.000000in}{-0.048611in}}{\pgfqpoint{0.000000in}{0.000000in}}{%
\pgfpathmoveto{\pgfqpoint{0.000000in}{0.000000in}}%
\pgfpathlineto{\pgfqpoint{0.000000in}{-0.048611in}}%
\pgfusepath{stroke,fill}%
}%
\begin{pgfscope}%
\pgfsys@transformshift{1.693017in}{0.517039in}%
\pgfsys@useobject{currentmarker}{}%
\end{pgfscope}%
\end{pgfscope}%
\begin{pgfscope}%
\definecolor{textcolor}{rgb}{0.000000,0.000000,0.000000}%
\pgfsetstrokecolor{textcolor}%
\pgfsetfillcolor{textcolor}%
\pgftext[x=1.693017in,y=0.419816in,,top]{\color{textcolor}{\rmfamily\fontsize{12.000000}{14.400000}\selectfont\catcode`\^=\active\def^{\ifmmode\sp\else\^{}\fi}\catcode`\%=\active\def%{\%}20}}%
\end{pgfscope}%
\begin{pgfscope}%
\pgfsetbuttcap%
\pgfsetroundjoin%
\definecolor{currentfill}{rgb}{0.000000,0.000000,0.000000}%
\pgfsetfillcolor{currentfill}%
\pgfsetlinewidth{0.803000pt}%
\definecolor{currentstroke}{rgb}{0.000000,0.000000,0.000000}%
\pgfsetstrokecolor{currentstroke}%
\pgfsetdash{}{0pt}%
\pgfsys@defobject{currentmarker}{\pgfqpoint{0.000000in}{-0.048611in}}{\pgfqpoint{0.000000in}{0.000000in}}{%
\pgfpathmoveto{\pgfqpoint{0.000000in}{0.000000in}}%
\pgfpathlineto{\pgfqpoint{0.000000in}{-0.048611in}}%
\pgfusepath{stroke,fill}%
}%
\begin{pgfscope}%
\pgfsys@transformshift{2.606137in}{0.517039in}%
\pgfsys@useobject{currentmarker}{}%
\end{pgfscope}%
\end{pgfscope}%
\begin{pgfscope}%
\definecolor{textcolor}{rgb}{0.000000,0.000000,0.000000}%
\pgfsetstrokecolor{textcolor}%
\pgfsetfillcolor{textcolor}%
\pgftext[x=2.606137in,y=0.419816in,,top]{\color{textcolor}{\rmfamily\fontsize{12.000000}{14.400000}\selectfont\catcode`\^=\active\def^{\ifmmode\sp\else\^{}\fi}\catcode`\%=\active\def%{\%}40}}%
\end{pgfscope}%
\begin{pgfscope}%
\definecolor{textcolor}{rgb}{0.000000,0.000000,0.000000}%
\pgfsetstrokecolor{textcolor}%
\pgfsetfillcolor{textcolor}%
\pgftext[x=1.921297in,y=0.202965in,,top]{\color{textcolor}{\rmfamily\fontsize{12.000000}{14.400000}\selectfont\catcode`\^=\active\def^{\ifmmode\sp\else\^{}\fi}\catcode`\%=\active\def%{\%}input coefficients}}%
\end{pgfscope}%
\begin{pgfscope}%
\pgfsetbuttcap%
\pgfsetroundjoin%
\definecolor{currentfill}{rgb}{0.000000,0.000000,0.000000}%
\pgfsetfillcolor{currentfill}%
\pgfsetlinewidth{0.803000pt}%
\definecolor{currentstroke}{rgb}{0.000000,0.000000,0.000000}%
\pgfsetstrokecolor{currentstroke}%
\pgfsetdash{}{0pt}%
\pgfsys@defobject{currentmarker}{\pgfqpoint{-0.048611in}{0.000000in}}{\pgfqpoint{-0.000000in}{0.000000in}}{%
\pgfpathmoveto{\pgfqpoint{-0.000000in}{0.000000in}}%
\pgfpathlineto{\pgfqpoint{-0.048611in}{0.000000in}}%
\pgfusepath{stroke,fill}%
}%
\begin{pgfscope}%
\pgfsys@transformshift{0.779897in}{2.895016in}%
\pgfsys@useobject{currentmarker}{}%
\end{pgfscope}%
\end{pgfscope}%
\begin{pgfscope}%
\definecolor{textcolor}{rgb}{0.000000,0.000000,0.000000}%
\pgfsetstrokecolor{textcolor}%
\pgfsetfillcolor{textcolor}%
\pgftext[x=0.576636in, y=2.831702in, left, base]{\color{textcolor}{\rmfamily\fontsize{12.000000}{14.400000}\selectfont\catcode`\^=\active\def^{\ifmmode\sp\else\^{}\fi}\catcode`\%=\active\def%{\%}0}}%
\end{pgfscope}%
\begin{pgfscope}%
\pgfsetbuttcap%
\pgfsetroundjoin%
\definecolor{currentfill}{rgb}{0.000000,0.000000,0.000000}%
\pgfsetfillcolor{currentfill}%
\pgfsetlinewidth{0.803000pt}%
\definecolor{currentstroke}{rgb}{0.000000,0.000000,0.000000}%
\pgfsetstrokecolor{currentstroke}%
\pgfsetdash{}{0pt}%
\pgfsys@defobject{currentmarker}{\pgfqpoint{-0.048611in}{0.000000in}}{\pgfqpoint{-0.000000in}{0.000000in}}{%
\pgfpathmoveto{\pgfqpoint{-0.000000in}{0.000000in}}%
\pgfpathlineto{\pgfqpoint{-0.048611in}{0.000000in}}%
\pgfusepath{stroke,fill}%
}%
\begin{pgfscope}%
\pgfsys@transformshift{0.779897in}{2.300522in}%
\pgfsys@useobject{currentmarker}{}%
\end{pgfscope}%
\end{pgfscope}%
\begin{pgfscope}%
\definecolor{textcolor}{rgb}{0.000000,0.000000,0.000000}%
\pgfsetstrokecolor{textcolor}%
\pgfsetfillcolor{textcolor}%
\pgftext[x=0.258521in, y=2.237208in, left, base]{\color{textcolor}{\rmfamily\fontsize{12.000000}{14.400000}\selectfont\catcode`\^=\active\def^{\ifmmode\sp\else\^{}\fi}\catcode`\%=\active\def%{\%}1000}}%
\end{pgfscope}%
\begin{pgfscope}%
\pgfsetbuttcap%
\pgfsetroundjoin%
\definecolor{currentfill}{rgb}{0.000000,0.000000,0.000000}%
\pgfsetfillcolor{currentfill}%
\pgfsetlinewidth{0.803000pt}%
\definecolor{currentstroke}{rgb}{0.000000,0.000000,0.000000}%
\pgfsetstrokecolor{currentstroke}%
\pgfsetdash{}{0pt}%
\pgfsys@defobject{currentmarker}{\pgfqpoint{-0.048611in}{0.000000in}}{\pgfqpoint{-0.000000in}{0.000000in}}{%
\pgfpathmoveto{\pgfqpoint{-0.000000in}{0.000000in}}%
\pgfpathlineto{\pgfqpoint{-0.048611in}{0.000000in}}%
\pgfusepath{stroke,fill}%
}%
\begin{pgfscope}%
\pgfsys@transformshift{0.779897in}{1.706027in}%
\pgfsys@useobject{currentmarker}{}%
\end{pgfscope}%
\end{pgfscope}%
\begin{pgfscope}%
\definecolor{textcolor}{rgb}{0.000000,0.000000,0.000000}%
\pgfsetstrokecolor{textcolor}%
\pgfsetfillcolor{textcolor}%
\pgftext[x=0.258521in, y=1.642714in, left, base]{\color{textcolor}{\rmfamily\fontsize{12.000000}{14.400000}\selectfont\catcode`\^=\active\def^{\ifmmode\sp\else\^{}\fi}\catcode`\%=\active\def%{\%}2000}}%
\end{pgfscope}%
\begin{pgfscope}%
\pgfsetbuttcap%
\pgfsetroundjoin%
\definecolor{currentfill}{rgb}{0.000000,0.000000,0.000000}%
\pgfsetfillcolor{currentfill}%
\pgfsetlinewidth{0.803000pt}%
\definecolor{currentstroke}{rgb}{0.000000,0.000000,0.000000}%
\pgfsetstrokecolor{currentstroke}%
\pgfsetdash{}{0pt}%
\pgfsys@defobject{currentmarker}{\pgfqpoint{-0.048611in}{0.000000in}}{\pgfqpoint{-0.000000in}{0.000000in}}{%
\pgfpathmoveto{\pgfqpoint{-0.000000in}{0.000000in}}%
\pgfpathlineto{\pgfqpoint{-0.048611in}{0.000000in}}%
\pgfusepath{stroke,fill}%
}%
\begin{pgfscope}%
\pgfsys@transformshift{0.779897in}{1.111533in}%
\pgfsys@useobject{currentmarker}{}%
\end{pgfscope}%
\end{pgfscope}%
\begin{pgfscope}%
\definecolor{textcolor}{rgb}{0.000000,0.000000,0.000000}%
\pgfsetstrokecolor{textcolor}%
\pgfsetfillcolor{textcolor}%
\pgftext[x=0.258521in, y=1.048219in, left, base]{\color{textcolor}{\rmfamily\fontsize{12.000000}{14.400000}\selectfont\catcode`\^=\active\def^{\ifmmode\sp\else\^{}\fi}\catcode`\%=\active\def%{\%}3000}}%
\end{pgfscope}%
\begin{pgfscope}%
\pgfsetbuttcap%
\pgfsetroundjoin%
\definecolor{currentfill}{rgb}{0.000000,0.000000,0.000000}%
\pgfsetfillcolor{currentfill}%
\pgfsetlinewidth{0.803000pt}%
\definecolor{currentstroke}{rgb}{0.000000,0.000000,0.000000}%
\pgfsetstrokecolor{currentstroke}%
\pgfsetdash{}{0pt}%
\pgfsys@defobject{currentmarker}{\pgfqpoint{-0.048611in}{0.000000in}}{\pgfqpoint{-0.000000in}{0.000000in}}{%
\pgfpathmoveto{\pgfqpoint{-0.000000in}{0.000000in}}%
\pgfpathlineto{\pgfqpoint{-0.048611in}{0.000000in}}%
\pgfusepath{stroke,fill}%
}%
\begin{pgfscope}%
\pgfsys@transformshift{0.779897in}{0.517039in}%
\pgfsys@useobject{currentmarker}{}%
\end{pgfscope}%
\end{pgfscope}%
\begin{pgfscope}%
\definecolor{textcolor}{rgb}{0.000000,0.000000,0.000000}%
\pgfsetstrokecolor{textcolor}%
\pgfsetfillcolor{textcolor}%
\pgftext[x=0.258521in, y=0.453725in, left, base]{\color{textcolor}{\rmfamily\fontsize{12.000000}{14.400000}\selectfont\catcode`\^=\active\def^{\ifmmode\sp\else\^{}\fi}\catcode`\%=\active\def%{\%}4000}}%
\end{pgfscope}%
\begin{pgfscope}%
\definecolor{textcolor}{rgb}{0.000000,0.000000,0.000000}%
\pgfsetstrokecolor{textcolor}%
\pgfsetfillcolor{textcolor}%
\pgftext[x=0.202965in,y=1.706027in,,bottom,rotate=90.000000]{\color{textcolor}{\rmfamily\fontsize{12.000000}{14.400000}\selectfont\catcode`\^=\active\def^{\ifmmode\sp\else\^{}\fi}\catcode`\%=\active\def%{\%}samples}}%
\end{pgfscope}%
\begin{pgfscope}%
\pgfsetrectcap%
\pgfsetmiterjoin%
\pgfsetlinewidth{0.803000pt}%
\definecolor{currentstroke}{rgb}{0.000000,0.000000,0.000000}%
\pgfsetstrokecolor{currentstroke}%
\pgfsetdash{}{0pt}%
\pgfpathmoveto{\pgfqpoint{0.779897in}{0.517039in}}%
\pgfpathlineto{\pgfqpoint{0.779897in}{2.895016in}}%
\pgfusepath{stroke}%
\end{pgfscope}%
\begin{pgfscope}%
\pgfsetrectcap%
\pgfsetmiterjoin%
\pgfsetlinewidth{0.803000pt}%
\definecolor{currentstroke}{rgb}{0.000000,0.000000,0.000000}%
\pgfsetstrokecolor{currentstroke}%
\pgfsetdash{}{0pt}%
\pgfpathmoveto{\pgfqpoint{3.062697in}{0.517039in}}%
\pgfpathlineto{\pgfqpoint{3.062697in}{2.895016in}}%
\pgfusepath{stroke}%
\end{pgfscope}%
\begin{pgfscope}%
\pgfsetrectcap%
\pgfsetmiterjoin%
\pgfsetlinewidth{0.803000pt}%
\definecolor{currentstroke}{rgb}{0.000000,0.000000,0.000000}%
\pgfsetstrokecolor{currentstroke}%
\pgfsetdash{}{0pt}%
\pgfpathmoveto{\pgfqpoint{0.779897in}{0.517039in}}%
\pgfpathlineto{\pgfqpoint{3.062697in}{0.517039in}}%
\pgfusepath{stroke}%
\end{pgfscope}%
\begin{pgfscope}%
\pgfsetrectcap%
\pgfsetmiterjoin%
\pgfsetlinewidth{0.803000pt}%
\definecolor{currentstroke}{rgb}{0.000000,0.000000,0.000000}%
\pgfsetstrokecolor{currentstroke}%
\pgfsetdash{}{0pt}%
\pgfpathmoveto{\pgfqpoint{0.779897in}{2.895016in}}%
\pgfpathlineto{\pgfqpoint{3.062697in}{2.895016in}}%
\pgfusepath{stroke}%
\end{pgfscope}%
\begin{pgfscope}%
\pgfsetbuttcap%
\pgfsetmiterjoin%
\pgfsetlinewidth{0.000000pt}%
\definecolor{currentstroke}{rgb}{0.000000,0.000000,0.000000}%
\pgfsetstrokecolor{currentstroke}%
\pgfsetstrokeopacity{0.000000}%
\pgfsetdash{}{0pt}%
\pgfpathmoveto{\pgfqpoint{3.282875in}{0.517039in}}%
\pgfpathlineto{\pgfqpoint{3.401774in}{0.517039in}}%
\pgfpathlineto{\pgfqpoint{3.401774in}{2.895016in}}%
\pgfpathlineto{\pgfqpoint{3.282875in}{2.895016in}}%
\pgfpathlineto{\pgfqpoint{3.282875in}{0.517039in}}%
\pgfpathclose%
\pgfusepath{}%
\end{pgfscope}%
\begin{pgfscope}%
\pgfsys@transformshift{3.280000in}{0.520000in}%
\pgftext[left,bottom]{\includegraphics[interpolate=true,width=0.120000in,height=2.380000in]{antiderivative_ci_10-img1.png}}%
\end{pgfscope}%
\begin{pgfscope}%
\pgfsetbuttcap%
\pgfsetroundjoin%
\definecolor{currentfill}{rgb}{0.000000,0.000000,0.000000}%
\pgfsetfillcolor{currentfill}%
\pgfsetlinewidth{0.803000pt}%
\definecolor{currentstroke}{rgb}{0.000000,0.000000,0.000000}%
\pgfsetstrokecolor{currentstroke}%
\pgfsetdash{}{0pt}%
\pgfsys@defobject{currentmarker}{\pgfqpoint{0.000000in}{0.000000in}}{\pgfqpoint{0.048611in}{0.000000in}}{%
\pgfpathmoveto{\pgfqpoint{0.000000in}{0.000000in}}%
\pgfpathlineto{\pgfqpoint{0.048611in}{0.000000in}}%
\pgfusepath{stroke,fill}%
}%
\begin{pgfscope}%
\pgfsys@transformshift{3.401774in}{0.529884in}%
\pgfsys@useobject{currentmarker}{}%
\end{pgfscope}%
\end{pgfscope}%
\begin{pgfscope}%
\definecolor{textcolor}{rgb}{0.000000,0.000000,0.000000}%
\pgfsetstrokecolor{textcolor}%
\pgfsetfillcolor{textcolor}%
\pgftext[x=3.498996in, y=0.466570in, left, base]{\color{textcolor}{\rmfamily\fontsize{12.000000}{14.400000}\selectfont\catcode`\^=\active\def^{\ifmmode\sp\else\^{}\fi}\catcode`\%=\active\def%{\%}\ensuremath{-}100}}%
\end{pgfscope}%
\begin{pgfscope}%
\pgfsetbuttcap%
\pgfsetroundjoin%
\definecolor{currentfill}{rgb}{0.000000,0.000000,0.000000}%
\pgfsetfillcolor{currentfill}%
\pgfsetlinewidth{0.803000pt}%
\definecolor{currentstroke}{rgb}{0.000000,0.000000,0.000000}%
\pgfsetstrokecolor{currentstroke}%
\pgfsetdash{}{0pt}%
\pgfsys@defobject{currentmarker}{\pgfqpoint{0.000000in}{0.000000in}}{\pgfqpoint{0.048611in}{0.000000in}}{%
\pgfpathmoveto{\pgfqpoint{0.000000in}{0.000000in}}%
\pgfpathlineto{\pgfqpoint{0.048611in}{0.000000in}}%
\pgfusepath{stroke,fill}%
}%
\begin{pgfscope}%
\pgfsys@transformshift{3.401774in}{1.109279in}%
\pgfsys@useobject{currentmarker}{}%
\end{pgfscope}%
\end{pgfscope}%
\begin{pgfscope}%
\definecolor{textcolor}{rgb}{0.000000,0.000000,0.000000}%
\pgfsetstrokecolor{textcolor}%
\pgfsetfillcolor{textcolor}%
\pgftext[x=3.498996in, y=1.045965in, left, base]{\color{textcolor}{\rmfamily\fontsize{12.000000}{14.400000}\selectfont\catcode`\^=\active\def^{\ifmmode\sp\else\^{}\fi}\catcode`\%=\active\def%{\%}\ensuremath{-}50}}%
\end{pgfscope}%
\begin{pgfscope}%
\pgfsetbuttcap%
\pgfsetroundjoin%
\definecolor{currentfill}{rgb}{0.000000,0.000000,0.000000}%
\pgfsetfillcolor{currentfill}%
\pgfsetlinewidth{0.803000pt}%
\definecolor{currentstroke}{rgb}{0.000000,0.000000,0.000000}%
\pgfsetstrokecolor{currentstroke}%
\pgfsetdash{}{0pt}%
\pgfsys@defobject{currentmarker}{\pgfqpoint{0.000000in}{0.000000in}}{\pgfqpoint{0.048611in}{0.000000in}}{%
\pgfpathmoveto{\pgfqpoint{0.000000in}{0.000000in}}%
\pgfpathlineto{\pgfqpoint{0.048611in}{0.000000in}}%
\pgfusepath{stroke,fill}%
}%
\begin{pgfscope}%
\pgfsys@transformshift{3.401774in}{1.688674in}%
\pgfsys@useobject{currentmarker}{}%
\end{pgfscope}%
\end{pgfscope}%
\begin{pgfscope}%
\definecolor{textcolor}{rgb}{0.000000,0.000000,0.000000}%
\pgfsetstrokecolor{textcolor}%
\pgfsetfillcolor{textcolor}%
\pgftext[x=3.498996in, y=1.625360in, left, base]{\color{textcolor}{\rmfamily\fontsize{12.000000}{14.400000}\selectfont\catcode`\^=\active\def^{\ifmmode\sp\else\^{}\fi}\catcode`\%=\active\def%{\%}0}}%
\end{pgfscope}%
\begin{pgfscope}%
\pgfsetbuttcap%
\pgfsetroundjoin%
\definecolor{currentfill}{rgb}{0.000000,0.000000,0.000000}%
\pgfsetfillcolor{currentfill}%
\pgfsetlinewidth{0.803000pt}%
\definecolor{currentstroke}{rgb}{0.000000,0.000000,0.000000}%
\pgfsetstrokecolor{currentstroke}%
\pgfsetdash{}{0pt}%
\pgfsys@defobject{currentmarker}{\pgfqpoint{0.000000in}{0.000000in}}{\pgfqpoint{0.048611in}{0.000000in}}{%
\pgfpathmoveto{\pgfqpoint{0.000000in}{0.000000in}}%
\pgfpathlineto{\pgfqpoint{0.048611in}{0.000000in}}%
\pgfusepath{stroke,fill}%
}%
\begin{pgfscope}%
\pgfsys@transformshift{3.401774in}{2.268068in}%
\pgfsys@useobject{currentmarker}{}%
\end{pgfscope}%
\end{pgfscope}%
\begin{pgfscope}%
\definecolor{textcolor}{rgb}{0.000000,0.000000,0.000000}%
\pgfsetstrokecolor{textcolor}%
\pgfsetfillcolor{textcolor}%
\pgftext[x=3.498996in, y=2.204754in, left, base]{\color{textcolor}{\rmfamily\fontsize{12.000000}{14.400000}\selectfont\catcode`\^=\active\def^{\ifmmode\sp\else\^{}\fi}\catcode`\%=\active\def%{\%}50}}%
\end{pgfscope}%
\begin{pgfscope}%
\pgfsetbuttcap%
\pgfsetroundjoin%
\definecolor{currentfill}{rgb}{0.000000,0.000000,0.000000}%
\pgfsetfillcolor{currentfill}%
\pgfsetlinewidth{0.803000pt}%
\definecolor{currentstroke}{rgb}{0.000000,0.000000,0.000000}%
\pgfsetstrokecolor{currentstroke}%
\pgfsetdash{}{0pt}%
\pgfsys@defobject{currentmarker}{\pgfqpoint{0.000000in}{0.000000in}}{\pgfqpoint{0.048611in}{0.000000in}}{%
\pgfpathmoveto{\pgfqpoint{0.000000in}{0.000000in}}%
\pgfpathlineto{\pgfqpoint{0.048611in}{0.000000in}}%
\pgfusepath{stroke,fill}%
}%
\begin{pgfscope}%
\pgfsys@transformshift{3.401774in}{2.847463in}%
\pgfsys@useobject{currentmarker}{}%
\end{pgfscope}%
\end{pgfscope}%
\begin{pgfscope}%
\definecolor{textcolor}{rgb}{0.000000,0.000000,0.000000}%
\pgfsetstrokecolor{textcolor}%
\pgfsetfillcolor{textcolor}%
\pgftext[x=3.498996in, y=2.784149in, left, base]{\color{textcolor}{\rmfamily\fontsize{12.000000}{14.400000}\selectfont\catcode`\^=\active\def^{\ifmmode\sp\else\^{}\fi}\catcode`\%=\active\def%{\%}100}}%
\end{pgfscope}%
\begin{pgfscope}%
\pgfsetrectcap%
\pgfsetmiterjoin%
\pgfsetlinewidth{0.803000pt}%
\definecolor{currentstroke}{rgb}{0.000000,0.000000,0.000000}%
\pgfsetstrokecolor{currentstroke}%
\pgfsetdash{}{0pt}%
\pgfpathmoveto{\pgfqpoint{3.282875in}{0.517039in}}%
\pgfpathlineto{\pgfqpoint{3.342325in}{0.517039in}}%
\pgfpathlineto{\pgfqpoint{3.401774in}{0.517039in}}%
\pgfpathlineto{\pgfqpoint{3.401774in}{2.895016in}}%
\pgfpathlineto{\pgfqpoint{3.342325in}{2.895016in}}%
\pgfpathlineto{\pgfqpoint{3.282875in}{2.895016in}}%
\pgfpathlineto{\pgfqpoint{3.282875in}{0.517039in}}%
\pgfpathclose%
\pgfusepath{stroke}%
\end{pgfscope}%
\end{pgfpicture}%
\makeatother%
\endgroup%

      \end{adjustbox}
      \caption{Correlation image 10\% noise.}\label{fig:sc1_ci_10}
    \end{subfigure}
    \begin{subfigure}{0.49\linewidth}
      \begin{adjustbox}{width=\linewidth}
        \input{figures/antiderivative_pm_10.pgf}
      \end{adjustbox}
      \caption{The p-matrix for 10\% noise.}\label{fig:sc1_pm_10}
    \end{subfigure}
    % \\[\baselineskip]
    \begin{subfigure}{0.49\linewidth}
      \begin{adjustbox}{width=\linewidth}
        \input{figures/antiderivative_ci_50.pgf}
      \end{adjustbox}
      \caption{Correlation image 50\% noise.}\label{fig:sc1_ci_50}
    \end{subfigure}
    \begin{subfigure}{0.49\linewidth}
      \begin{adjustbox}{width=\linewidth}
        \input{figures/antiderivative_pm_50.pgf}
      \end{adjustbox}
      \caption{The p-matrix for 50\% noise.}\label{fig:sc1_pm_50}
    \end{subfigure}
  \end{adjustwidth}
  \caption{Correlation image (left column) and p-matrix (right column) for each model trained on a different noise level (row). The correlation image was sorted same order the values of the real component of wave number \(k=2\) were sorted in descending order.}\label{fig:scenario_1_interpretation}
\end{figure}

The final observation we make is how for all the p-matrices the pronounced contribution of the input wave number to their corresponding output wave number. This means that the majority of contributions to each output coefficients come from the corresponding input coefficients of the same wave number. Our knowledge on how the simple derivative equation for Fourier series relate the coefficients of the derivative function and the antiderivative function is shown in \lccref{eq:derivative_coeff}. This aligns with the contributions shown by the p-matrices. This confirms that the model is indeed learning the relations that is defined by the derivative equation.

\subsection{Scenario 2: Burgers' Equation}
The function value evaluated from the predictions is shown in \lccref{table:scenario_2_function_metrics}. The metrics across the board shows that the function value relative metrics are slightly better compared to the coefficients. Comparing the absolute metrics for the same table, we see the same general trend that the higher viscosity show the model performing worse. For reference, the maximum amplitude of the functions on average are \num{2.56e-02}. This puts the error in an order of magnitude less than the average maximum amplitude.
\begin{table}[H]
  \caption{Performance metrics of function values evaluated from coefficient prediction of next time step in scenario 2 by viscosity.}\label{table:scenario_2_function_metrics}
  \centering
  \begin{tabular}{lrrrrr}
    \toprule
    \(\nu \) & MSE      & RMSE     & MAE      & R\textsuperscript{2} & sMAPE \\
    \midrule
    0.0      & 5.58e-05 & 7.47e-03 & 5.89e-03 & 0.82                 & 0.70  \\
    0.01     & 5.74e-05 & 7.58e-03 & 6.01e-03 & 0.81                 & 0.71  \\
    0.1      & 1.58e-04 & 1.26e-02 & 9.98e-03 & 0.51                 & 1.04  \\
    \bottomrule
  \end{tabular}
\end{table}

The testing results are complimented with a rollout experiment. The results from the rollout are shown in \lccref{fig:scenario_2_rollout}. The rollout starts from time \(t=0\). In all the plots, this is at the bottom. We have included this initial condition in the plots. From this initial condition in combination with the forcing term at the initial time step, the model predicts the first time step solution. The model then uses the predicted solution combined with the corresponding forcing term to predict the second time step. To see how well the model is performing relative to the actual target values, we compute the difference between the two as shown in \lccrefs{fig:sc2_rollout_diff_0.0,fig:sc2_rollout_diff_0.01,fig:sc2_rollout_diff_0.1}. Observe that errors are introduced relatively early around time \(t=1\). However, the model rollout stays stable until the end despite this error. This results in the model being able to finish the rollout for this sample.
\begin{figure}[H]
  \centering
  \begin{adjustwidth}{-0.1\linewidth}{-0.1\linewidth}
    \begin{subfigure}{0.33\linewidth}
      \begin{adjustbox}{width=\linewidth}
        \input{figures/burgers_rollout_target_0.0.pgf}
      \end{adjustbox}
      \caption{The target for \(\nu=0.0\)}\label{fig:sc2_rollout_target_0.0}
    \end{subfigure}
    \begin{subfigure}{0.33\linewidth}
      \begin{adjustbox}{width=\linewidth}
        \input{figures/burgers_rollout_pred_0.0.pgf}
      \end{adjustbox}
      \caption{The prediction for \(\nu=0.0\)}\label{fig:sc2_rollout_pred_0.0}
    \end{subfigure}
    \begin{subfigure}{0.32\linewidth}
      \begin{adjustbox}{width=\linewidth}
        \input{figures/burgers_rollout_diff_0.0.pgf}
      \end{adjustbox}
      \caption{The difference for \(\nu=0.0\)}\label{fig:sc2_rollout_diff_0.0}
    \end{subfigure}
    \\[0.7\baselineskip]
    \begin{subfigure}{0.33\linewidth}
      \begin{adjustbox}{width=\linewidth}
        \input{figures/burgers_rollout_target_0.01.pgf}
      \end{adjustbox}
      \caption{The target for \(\nu=0.01\)}\label{fig:sc2_rollout_target_0.01}
    \end{subfigure}
    \begin{subfigure}{0.33\linewidth}
      \begin{adjustbox}{width=\linewidth}
        \input{figures/burgers_rollout_pred_0.01.pgf}
      \end{adjustbox}
      \caption{The prediction for \(\nu=0.01\)}\label{fig:sc2_rollout_pred_0.01}
    \end{subfigure}
    \begin{subfigure}{0.32\linewidth}
      \begin{adjustbox}{width=\linewidth}
        \input{figures/burgers_rollout_diff_0.01.pgf}
      \end{adjustbox}
      \caption{The difference for \(\nu=0.01\)}\label{fig:sc2_rollout_diff_0.01}
    \end{subfigure}
    \\[0.7\baselineskip]
    \begin{subfigure}{0.33\linewidth}
      \begin{adjustbox}{width=\linewidth}
        \input{figures/burgers_rollout_target_0.1.pgf}
      \end{adjustbox}
      \caption{The target for \(\nu=0.1\)}\label{fig:sc2_rollout_target_0.1}
    \end{subfigure}
    \begin{subfigure}{0.33\linewidth}
      \begin{adjustbox}{width=\linewidth}
        \input{figures/burgers_rollout_pred_0.1.pgf}
      \end{adjustbox}
      \caption{The prediction for \(\nu=0.1\)}\label{fig:sc2_rollout_pred_0.1}
    \end{subfigure}
    \begin{subfigure}{0.32\linewidth}
      \begin{adjustbox}{width=\linewidth}
        \input{figures/burgers_rollout_diff_0.1.pgf}
      \end{adjustbox}
      \caption{The difference for \(\nu=0.1\)}\label{fig:sc2_rollout_diff_0.1}
    \end{subfigure}
    % \\[0.7\baselineskip]
  \end{adjustwidth}
  \caption{The rollout predictions for one of the test function. The difference is calculated as the targets subtracted by the predictions.}\label{fig:scenario_2_rollout}
\end{figure}

Another observation we can see is that the difference between the target and rollout predictions is much more pronounced for lower viscosity values. This result is aligned with metrics shown in \lccrefs{table:scenario_2_rollout_function_metrics}. The absolute metrics show that the models are performing several times worse in rollout compared to the single time step tests shown in \lccrefs{table:scenario_2_function_metrics}. This seems to be the same challenge that traditional solvers also face with the shocks that may be present with lower viscosity values. This means that for rollout scenarios, the model has a much harder time with lower viscosity values. The observation here is that for higher viscosity values, the error is dominated by the global error, and the lower viscosity value, the error is dominated by local errors and as such accumulates.
\begin{table}[H]
  \caption{Performance metrics of function values evaluated from coefficient rollout in scenario 2 by viscosity.}\label{table:scenario_2_rollout_function_metrics}
  \centering
  \begin{tabular}{lrrrr}
    \toprule
    \(\nu \) & MSE      & RMSE     & MAE      & sMAPE \\
    \midrule
    0.0      & 5.81e-02 & 2.41e-01 & 1.89e-01 & 1.04  \\
    0.01     & 3.17e-02 & 1.78e-01 & 1.40e-01 & 0.94  \\
    0.1      & 2.91e-02 & 1.71e-01 & 1.33e-01 & 0.92  \\
    \bottomrule
  \end{tabular}
\end{table}

To test the generalization capability of the model on other function families, we use an exact solution provided in the literature shown in \lccref{eq:burgers_exact_solution} \citep{woodExactSolutionBurgers2006,wazwazPartialDifferentialEquations2010,bentonTableSolutionsOnedimensional1972}.
\begin{equation}
  u(x,t) = \frac{2\nu\pi e^{-\pi^2\nu t}\sin(\pi x)}{a+e^{-\pi^2\nu t}\cos(\pi x)} \label{eq:burgers_exact_solution}
\end{equation}

The exact function \cref{eq:burgers_exact_solution} is computed for each viscosity value. The discrete Fourier transform of the values are then used for doing rollout with the model. For the forcing term, we just use a constant function with a value of zero. The rollout predictions are shown in \lccref{fig:scenario_2_exact}. The model is not successful in the rollout for the inviscid equation and for the viscosity of \(\nu=0.01\). Both of these cases, the model produces too much error that the original function is no longer recognizable in the predictions. For the higher viscosity value of \(\nu=0.1\) the rollout produces a recognizable prediction. The error stays to about half the maximum function value. We do see some more noticeable error in the flatter parts of the function. The error also take a very similar shape across the different viscosity values. Since we essentially used the same solutions for training the model for each viscosity, the similarity in error means that it can be alleviated with more diverse training samples. The error for the flatter functions may also be alleviated by including similar flatter functions during training.
\begin{figure}[H]
  \centering
  \begin{adjustwidth}{-0.1\linewidth}{-0.1\linewidth}
    \begin{subfigure}{0.33\linewidth}
      \begin{adjustbox}{width=\linewidth}
        \input{figures/burgers_exact_target_0.0.pgf}
      \end{adjustbox}
      \caption{The target for \(\nu=0.0\)}\label{fig:sc2_exact_target_0.0}
    \end{subfigure}
    \begin{subfigure}{0.33\linewidth}
      \begin{adjustbox}{width=\linewidth}
        \input{figures/burgers_exact_pred_0.0.pgf}
      \end{adjustbox}
      \caption{The prediction for \(\nu=0.0\)}\label{fig:sc2_exact_pred_0.0}
    \end{subfigure}
    \begin{subfigure}{0.32\linewidth}
      \begin{adjustbox}{width=\linewidth}
        \input{figures/burgers_exact_diff_0.0.pgf}
      \end{adjustbox}
      \caption{The difference for \(\nu=0.0\)}\label{fig:sc2_exact_diff_0.0}
    \end{subfigure}
    \\[0.7\baselineskip]
    \begin{subfigure}{0.33\linewidth}
      \begin{adjustbox}{width=\linewidth}
        \input{figures/burgers_exact_target_0.01.pgf}
      \end{adjustbox}
      \caption{The target for \(\nu=0.01\)}\label{fig:sc2_exact_target_0.01}
    \end{subfigure}
    \begin{subfigure}{0.33\linewidth}
      \begin{adjustbox}{width=\linewidth}
        \input{figures/burgers_exact_pred_0.01.pgf}
      \end{adjustbox}
      \caption{The prediction for \(\nu=0.01\)}\label{fig:sc2_exact_pred_0.01}
    \end{subfigure}
    \begin{subfigure}{0.32\linewidth}
      \begin{adjustbox}{width=\linewidth}
        \input{figures/burgers_exact_diff_0.01.pgf}
      \end{adjustbox}
      \caption{The difference for \(\nu=0.01\)}\label{fig:sc2_exact_diff_0.01}
    \end{subfigure}
    \\[0.7\baselineskip]
    \begin{subfigure}{0.33\linewidth}
      \begin{adjustbox}{width=\linewidth}
        \input{figures/burgers_exact_target_0.1.pgf}
      \end{adjustbox}
      \caption{The target for \(\nu=0.1\)}\label{fig:sc2_exact_target_0.1}
    \end{subfigure}
    \begin{subfigure}{0.33\linewidth}
      \begin{adjustbox}{width=\linewidth}
        \input{figures/burgers_exact_pred_0.1.pgf}
      \end{adjustbox}
      \caption{The prediction for \(\nu=0.1\)}\label{fig:sc2_exact_pred_0.1}
    \end{subfigure}
    \begin{subfigure}{0.32\linewidth}
      \begin{adjustbox}{width=\linewidth}
        \input{figures/burgers_exact_diff_0.1.pgf}
      \end{adjustbox}
      \caption{The difference for \(\nu=0.1\)}\label{fig:sc2_exact_diff_0.1}
    \end{subfigure}
    % \\[0.7\baselineskip]
  \end{adjustwidth}
  \caption{The rollout predictions for one of the exact function in \lccref{eq:burgers_exact_solution}. The difference is calculated as the targets subtracted by the predictions.}\label{fig:scenario_2_exact}
\end{figure}

The correlation image and p-matrix for each viscosity can be seen in \lccref{fig:scenario_2_interpretation}. Analyzing the correlation images first, we see that some columns are sorted in correlation to how the samples are sorted based on the real component of the 4th output wave number. For the inviscid equation, this correlation is the strongest with multiple columns exhibiting the correlation. The highest and visually most sorted values are from the forcing term portion of the inputs. Different wave numbers in the inputs are also sorted differently, this indicates an inverse correlation. For a higher viscosity of \(\nu=0.01\), a weaker correlation is still present.
\begin{figure}[H]
  \centering
  \begin{subfigure}{0.49\linewidth}
    \begin{adjustbox}{width=\linewidth}
      \begingroup%
\makeatletter%
\begin{pgfpicture}%
\pgfpathrectangle{\pgfpointorigin}{\pgfqpoint{4.000000in}{3.000000in}}%
\pgfusepath{use as bounding box, clip}%
\begin{pgfscope}%
\pgfsetbuttcap%
\pgfsetmiterjoin%
\pgfsetlinewidth{0.000000pt}%
\definecolor{currentstroke}{rgb}{0.000000,0.000000,0.000000}%
\pgfsetstrokecolor{currentstroke}%
\pgfsetstrokeopacity{0.000000}%
\pgfsetdash{}{0pt}%
\pgfpathmoveto{\pgfqpoint{0.000000in}{0.000000in}}%
\pgfpathlineto{\pgfqpoint{4.000000in}{0.000000in}}%
\pgfpathlineto{\pgfqpoint{4.000000in}{3.000000in}}%
\pgfpathlineto{\pgfqpoint{0.000000in}{3.000000in}}%
\pgfpathlineto{\pgfqpoint{0.000000in}{0.000000in}}%
\pgfpathclose%
\pgfusepath{}%
\end{pgfscope}%
\begin{pgfscope}%
\pgfsetbuttcap%
\pgfsetmiterjoin%
\pgfsetlinewidth{0.000000pt}%
\definecolor{currentstroke}{rgb}{0.000000,0.000000,0.000000}%
\pgfsetstrokecolor{currentstroke}%
\pgfsetstrokeopacity{0.000000}%
\pgfsetdash{}{0pt}%
\pgfpathmoveto{\pgfqpoint{0.779897in}{0.517039in}}%
\pgfpathlineto{\pgfqpoint{3.118367in}{0.517039in}}%
\pgfpathlineto{\pgfqpoint{3.118367in}{2.895016in}}%
\pgfpathlineto{\pgfqpoint{0.779897in}{2.895016in}}%
\pgfpathlineto{\pgfqpoint{0.779897in}{0.517039in}}%
\pgfpathclose%
\pgfusepath{}%
\end{pgfscope}%
\begin{pgfscope}%
\pgfpathrectangle{\pgfqpoint{0.779897in}{0.517039in}}{\pgfqpoint{2.338470in}{2.377978in}}%
\pgfusepath{clip}%
\pgfsys@transformcm{2.338470}{0.000000}{0.000000}{-2.377978}{0.779897in}{2.895016in}%
\pgftext[left,bottom]{\includegraphics[interpolate=false,width=1.000000in,height=1.000000in]{burgers_ci_0.0-img0.png}}%
\end{pgfscope}%
\begin{pgfscope}%
\pgfsetbuttcap%
\pgfsetroundjoin%
\definecolor{currentfill}{rgb}{0.000000,0.000000,0.000000}%
\pgfsetfillcolor{currentfill}%
\pgfsetlinewidth{0.803000pt}%
\definecolor{currentstroke}{rgb}{0.000000,0.000000,0.000000}%
\pgfsetstrokecolor{currentstroke}%
\pgfsetdash{}{0pt}%
\pgfsys@defobject{currentmarker}{\pgfqpoint{0.000000in}{-0.048611in}}{\pgfqpoint{0.000000in}{0.000000in}}{%
\pgfpathmoveto{\pgfqpoint{0.000000in}{0.000000in}}%
\pgfpathlineto{\pgfqpoint{0.000000in}{-0.048611in}}%
\pgfusepath{stroke,fill}%
}%
\begin{pgfscope}%
\pgfsys@transformshift{0.779897in}{0.517039in}%
\pgfsys@useobject{currentmarker}{}%
\end{pgfscope}%
\end{pgfscope}%
\begin{pgfscope}%
\definecolor{textcolor}{rgb}{0.000000,0.000000,0.000000}%
\pgfsetstrokecolor{textcolor}%
\pgfsetfillcolor{textcolor}%
\pgftext[x=0.779897in,y=0.419816in,,top]{\color{textcolor}{\rmfamily\fontsize{12.000000}{14.400000}\selectfont\catcode`\^=\active\def^{\ifmmode\sp\else\^{}\fi}\catcode`\%=\active\def%{\%}0}}%
\end{pgfscope}%
\begin{pgfscope}%
\pgfsetbuttcap%
\pgfsetroundjoin%
\definecolor{currentfill}{rgb}{0.000000,0.000000,0.000000}%
\pgfsetfillcolor{currentfill}%
\pgfsetlinewidth{0.803000pt}%
\definecolor{currentstroke}{rgb}{0.000000,0.000000,0.000000}%
\pgfsetstrokecolor{currentstroke}%
\pgfsetdash{}{0pt}%
\pgfsys@defobject{currentmarker}{\pgfqpoint{0.000000in}{-0.048611in}}{\pgfqpoint{0.000000in}{0.000000in}}{%
\pgfpathmoveto{\pgfqpoint{0.000000in}{0.000000in}}%
\pgfpathlineto{\pgfqpoint{0.000000in}{-0.048611in}}%
\pgfusepath{stroke,fill}%
}%
\begin{pgfscope}%
\pgfsys@transformshift{1.364514in}{0.517039in}%
\pgfsys@useobject{currentmarker}{}%
\end{pgfscope}%
\end{pgfscope}%
\begin{pgfscope}%
\definecolor{textcolor}{rgb}{0.000000,0.000000,0.000000}%
\pgfsetstrokecolor{textcolor}%
\pgfsetfillcolor{textcolor}%
\pgftext[x=1.364514in,y=0.419816in,,top]{\color{textcolor}{\rmfamily\fontsize{12.000000}{14.400000}\selectfont\catcode`\^=\active\def^{\ifmmode\sp\else\^{}\fi}\catcode`\%=\active\def%{\%}2}}%
\end{pgfscope}%
\begin{pgfscope}%
\pgfsetbuttcap%
\pgfsetroundjoin%
\definecolor{currentfill}{rgb}{0.000000,0.000000,0.000000}%
\pgfsetfillcolor{currentfill}%
\pgfsetlinewidth{0.803000pt}%
\definecolor{currentstroke}{rgb}{0.000000,0.000000,0.000000}%
\pgfsetstrokecolor{currentstroke}%
\pgfsetdash{}{0pt}%
\pgfsys@defobject{currentmarker}{\pgfqpoint{0.000000in}{-0.048611in}}{\pgfqpoint{0.000000in}{0.000000in}}{%
\pgfpathmoveto{\pgfqpoint{0.000000in}{0.000000in}}%
\pgfpathlineto{\pgfqpoint{0.000000in}{-0.048611in}}%
\pgfusepath{stroke,fill}%
}%
\begin{pgfscope}%
\pgfsys@transformshift{1.949132in}{0.517039in}%
\pgfsys@useobject{currentmarker}{}%
\end{pgfscope}%
\end{pgfscope}%
\begin{pgfscope}%
\definecolor{textcolor}{rgb}{0.000000,0.000000,0.000000}%
\pgfsetstrokecolor{textcolor}%
\pgfsetfillcolor{textcolor}%
\pgftext[x=1.949132in,y=0.419816in,,top]{\color{textcolor}{\rmfamily\fontsize{12.000000}{14.400000}\selectfont\catcode`\^=\active\def^{\ifmmode\sp\else\^{}\fi}\catcode`\%=\active\def%{\%}4}}%
\end{pgfscope}%
\begin{pgfscope}%
\pgfsetbuttcap%
\pgfsetroundjoin%
\definecolor{currentfill}{rgb}{0.000000,0.000000,0.000000}%
\pgfsetfillcolor{currentfill}%
\pgfsetlinewidth{0.803000pt}%
\definecolor{currentstroke}{rgb}{0.000000,0.000000,0.000000}%
\pgfsetstrokecolor{currentstroke}%
\pgfsetdash{}{0pt}%
\pgfsys@defobject{currentmarker}{\pgfqpoint{0.000000in}{-0.048611in}}{\pgfqpoint{0.000000in}{0.000000in}}{%
\pgfpathmoveto{\pgfqpoint{0.000000in}{0.000000in}}%
\pgfpathlineto{\pgfqpoint{0.000000in}{-0.048611in}}%
\pgfusepath{stroke,fill}%
}%
\begin{pgfscope}%
\pgfsys@transformshift{2.533749in}{0.517039in}%
\pgfsys@useobject{currentmarker}{}%
\end{pgfscope}%
\end{pgfscope}%
\begin{pgfscope}%
\definecolor{textcolor}{rgb}{0.000000,0.000000,0.000000}%
\pgfsetstrokecolor{textcolor}%
\pgfsetfillcolor{textcolor}%
\pgftext[x=2.533749in,y=0.419816in,,top]{\color{textcolor}{\rmfamily\fontsize{12.000000}{14.400000}\selectfont\catcode`\^=\active\def^{\ifmmode\sp\else\^{}\fi}\catcode`\%=\active\def%{\%}6}}%
\end{pgfscope}%
\begin{pgfscope}%
\pgfsetbuttcap%
\pgfsetroundjoin%
\definecolor{currentfill}{rgb}{0.000000,0.000000,0.000000}%
\pgfsetfillcolor{currentfill}%
\pgfsetlinewidth{0.803000pt}%
\definecolor{currentstroke}{rgb}{0.000000,0.000000,0.000000}%
\pgfsetstrokecolor{currentstroke}%
\pgfsetdash{}{0pt}%
\pgfsys@defobject{currentmarker}{\pgfqpoint{0.000000in}{-0.048611in}}{\pgfqpoint{0.000000in}{0.000000in}}{%
\pgfpathmoveto{\pgfqpoint{0.000000in}{0.000000in}}%
\pgfpathlineto{\pgfqpoint{0.000000in}{-0.048611in}}%
\pgfusepath{stroke,fill}%
}%
\begin{pgfscope}%
\pgfsys@transformshift{3.118367in}{0.517039in}%
\pgfsys@useobject{currentmarker}{}%
\end{pgfscope}%
\end{pgfscope}%
\begin{pgfscope}%
\definecolor{textcolor}{rgb}{0.000000,0.000000,0.000000}%
\pgfsetstrokecolor{textcolor}%
\pgfsetfillcolor{textcolor}%
\pgftext[x=3.118367in,y=0.419816in,,top]{\color{textcolor}{\rmfamily\fontsize{12.000000}{14.400000}\selectfont\catcode`\^=\active\def^{\ifmmode\sp\else\^{}\fi}\catcode`\%=\active\def%{\%}8}}%
\end{pgfscope}%
\begin{pgfscope}%
\definecolor{textcolor}{rgb}{0.000000,0.000000,0.000000}%
\pgfsetstrokecolor{textcolor}%
\pgfsetfillcolor{textcolor}%
\pgftext[x=1.949132in,y=0.202965in,,top]{\color{textcolor}{\rmfamily\fontsize{12.000000}{14.400000}\selectfont\catcode`\^=\active\def^{\ifmmode\sp\else\^{}\fi}\catcode`\%=\active\def%{\%}input coefficients}}%
\end{pgfscope}%
\begin{pgfscope}%
\pgfsetbuttcap%
\pgfsetroundjoin%
\definecolor{currentfill}{rgb}{0.000000,0.000000,0.000000}%
\pgfsetfillcolor{currentfill}%
\pgfsetlinewidth{0.803000pt}%
\definecolor{currentstroke}{rgb}{0.000000,0.000000,0.000000}%
\pgfsetstrokecolor{currentstroke}%
\pgfsetdash{}{0pt}%
\pgfsys@defobject{currentmarker}{\pgfqpoint{-0.048611in}{0.000000in}}{\pgfqpoint{-0.000000in}{0.000000in}}{%
\pgfpathmoveto{\pgfqpoint{-0.000000in}{0.000000in}}%
\pgfpathlineto{\pgfqpoint{-0.048611in}{0.000000in}}%
\pgfusepath{stroke,fill}%
}%
\begin{pgfscope}%
\pgfsys@transformshift{0.779897in}{2.895016in}%
\pgfsys@useobject{currentmarker}{}%
\end{pgfscope}%
\end{pgfscope}%
\begin{pgfscope}%
\definecolor{textcolor}{rgb}{0.000000,0.000000,0.000000}%
\pgfsetstrokecolor{textcolor}%
\pgfsetfillcolor{textcolor}%
\pgftext[x=0.576636in, y=2.831702in, left, base]{\color{textcolor}{\rmfamily\fontsize{12.000000}{14.400000}\selectfont\catcode`\^=\active\def^{\ifmmode\sp\else\^{}\fi}\catcode`\%=\active\def%{\%}0}}%
\end{pgfscope}%
\begin{pgfscope}%
\pgfsetbuttcap%
\pgfsetroundjoin%
\definecolor{currentfill}{rgb}{0.000000,0.000000,0.000000}%
\pgfsetfillcolor{currentfill}%
\pgfsetlinewidth{0.803000pt}%
\definecolor{currentstroke}{rgb}{0.000000,0.000000,0.000000}%
\pgfsetstrokecolor{currentstroke}%
\pgfsetdash{}{0pt}%
\pgfsys@defobject{currentmarker}{\pgfqpoint{-0.048611in}{0.000000in}}{\pgfqpoint{-0.000000in}{0.000000in}}{%
\pgfpathmoveto{\pgfqpoint{-0.000000in}{0.000000in}}%
\pgfpathlineto{\pgfqpoint{-0.048611in}{0.000000in}}%
\pgfusepath{stroke,fill}%
}%
\begin{pgfscope}%
\pgfsys@transformshift{0.779897in}{2.300522in}%
\pgfsys@useobject{currentmarker}{}%
\end{pgfscope}%
\end{pgfscope}%
\begin{pgfscope}%
\definecolor{textcolor}{rgb}{0.000000,0.000000,0.000000}%
\pgfsetstrokecolor{textcolor}%
\pgfsetfillcolor{textcolor}%
\pgftext[x=0.258521in, y=2.237208in, left, base]{\color{textcolor}{\rmfamily\fontsize{12.000000}{14.400000}\selectfont\catcode`\^=\active\def^{\ifmmode\sp\else\^{}\fi}\catcode`\%=\active\def%{\%}2000}}%
\end{pgfscope}%
\begin{pgfscope}%
\pgfsetbuttcap%
\pgfsetroundjoin%
\definecolor{currentfill}{rgb}{0.000000,0.000000,0.000000}%
\pgfsetfillcolor{currentfill}%
\pgfsetlinewidth{0.803000pt}%
\definecolor{currentstroke}{rgb}{0.000000,0.000000,0.000000}%
\pgfsetstrokecolor{currentstroke}%
\pgfsetdash{}{0pt}%
\pgfsys@defobject{currentmarker}{\pgfqpoint{-0.048611in}{0.000000in}}{\pgfqpoint{-0.000000in}{0.000000in}}{%
\pgfpathmoveto{\pgfqpoint{-0.000000in}{0.000000in}}%
\pgfpathlineto{\pgfqpoint{-0.048611in}{0.000000in}}%
\pgfusepath{stroke,fill}%
}%
\begin{pgfscope}%
\pgfsys@transformshift{0.779897in}{1.706027in}%
\pgfsys@useobject{currentmarker}{}%
\end{pgfscope}%
\end{pgfscope}%
\begin{pgfscope}%
\definecolor{textcolor}{rgb}{0.000000,0.000000,0.000000}%
\pgfsetstrokecolor{textcolor}%
\pgfsetfillcolor{textcolor}%
\pgftext[x=0.258521in, y=1.642714in, left, base]{\color{textcolor}{\rmfamily\fontsize{12.000000}{14.400000}\selectfont\catcode`\^=\active\def^{\ifmmode\sp\else\^{}\fi}\catcode`\%=\active\def%{\%}4000}}%
\end{pgfscope}%
\begin{pgfscope}%
\pgfsetbuttcap%
\pgfsetroundjoin%
\definecolor{currentfill}{rgb}{0.000000,0.000000,0.000000}%
\pgfsetfillcolor{currentfill}%
\pgfsetlinewidth{0.803000pt}%
\definecolor{currentstroke}{rgb}{0.000000,0.000000,0.000000}%
\pgfsetstrokecolor{currentstroke}%
\pgfsetdash{}{0pt}%
\pgfsys@defobject{currentmarker}{\pgfqpoint{-0.048611in}{0.000000in}}{\pgfqpoint{-0.000000in}{0.000000in}}{%
\pgfpathmoveto{\pgfqpoint{-0.000000in}{0.000000in}}%
\pgfpathlineto{\pgfqpoint{-0.048611in}{0.000000in}}%
\pgfusepath{stroke,fill}%
}%
\begin{pgfscope}%
\pgfsys@transformshift{0.779897in}{1.111533in}%
\pgfsys@useobject{currentmarker}{}%
\end{pgfscope}%
\end{pgfscope}%
\begin{pgfscope}%
\definecolor{textcolor}{rgb}{0.000000,0.000000,0.000000}%
\pgfsetstrokecolor{textcolor}%
\pgfsetfillcolor{textcolor}%
\pgftext[x=0.258521in, y=1.048219in, left, base]{\color{textcolor}{\rmfamily\fontsize{12.000000}{14.400000}\selectfont\catcode`\^=\active\def^{\ifmmode\sp\else\^{}\fi}\catcode`\%=\active\def%{\%}6000}}%
\end{pgfscope}%
\begin{pgfscope}%
\pgfsetbuttcap%
\pgfsetroundjoin%
\definecolor{currentfill}{rgb}{0.000000,0.000000,0.000000}%
\pgfsetfillcolor{currentfill}%
\pgfsetlinewidth{0.803000pt}%
\definecolor{currentstroke}{rgb}{0.000000,0.000000,0.000000}%
\pgfsetstrokecolor{currentstroke}%
\pgfsetdash{}{0pt}%
\pgfsys@defobject{currentmarker}{\pgfqpoint{-0.048611in}{0.000000in}}{\pgfqpoint{-0.000000in}{0.000000in}}{%
\pgfpathmoveto{\pgfqpoint{-0.000000in}{0.000000in}}%
\pgfpathlineto{\pgfqpoint{-0.048611in}{0.000000in}}%
\pgfusepath{stroke,fill}%
}%
\begin{pgfscope}%
\pgfsys@transformshift{0.779897in}{0.517039in}%
\pgfsys@useobject{currentmarker}{}%
\end{pgfscope}%
\end{pgfscope}%
\begin{pgfscope}%
\definecolor{textcolor}{rgb}{0.000000,0.000000,0.000000}%
\pgfsetstrokecolor{textcolor}%
\pgfsetfillcolor{textcolor}%
\pgftext[x=0.258521in, y=0.453725in, left, base]{\color{textcolor}{\rmfamily\fontsize{12.000000}{14.400000}\selectfont\catcode`\^=\active\def^{\ifmmode\sp\else\^{}\fi}\catcode`\%=\active\def%{\%}8000}}%
\end{pgfscope}%
\begin{pgfscope}%
\definecolor{textcolor}{rgb}{0.000000,0.000000,0.000000}%
\pgfsetstrokecolor{textcolor}%
\pgfsetfillcolor{textcolor}%
\pgftext[x=0.202965in,y=1.706027in,,bottom,rotate=90.000000]{\color{textcolor}{\rmfamily\fontsize{12.000000}{14.400000}\selectfont\catcode`\^=\active\def^{\ifmmode\sp\else\^{}\fi}\catcode`\%=\active\def%{\%}samples}}%
\end{pgfscope}%
\begin{pgfscope}%
\pgfsetrectcap%
\pgfsetmiterjoin%
\pgfsetlinewidth{0.803000pt}%
\definecolor{currentstroke}{rgb}{0.000000,0.000000,0.000000}%
\pgfsetstrokecolor{currentstroke}%
\pgfsetdash{}{0pt}%
\pgfpathmoveto{\pgfqpoint{0.779897in}{0.517039in}}%
\pgfpathlineto{\pgfqpoint{0.779897in}{2.895016in}}%
\pgfusepath{stroke}%
\end{pgfscope}%
\begin{pgfscope}%
\pgfsetrectcap%
\pgfsetmiterjoin%
\pgfsetlinewidth{0.803000pt}%
\definecolor{currentstroke}{rgb}{0.000000,0.000000,0.000000}%
\pgfsetstrokecolor{currentstroke}%
\pgfsetdash{}{0pt}%
\pgfpathmoveto{\pgfqpoint{3.118367in}{0.517039in}}%
\pgfpathlineto{\pgfqpoint{3.118367in}{2.895016in}}%
\pgfusepath{stroke}%
\end{pgfscope}%
\begin{pgfscope}%
\pgfsetrectcap%
\pgfsetmiterjoin%
\pgfsetlinewidth{0.803000pt}%
\definecolor{currentstroke}{rgb}{0.000000,0.000000,0.000000}%
\pgfsetstrokecolor{currentstroke}%
\pgfsetdash{}{0pt}%
\pgfpathmoveto{\pgfqpoint{0.779897in}{0.517039in}}%
\pgfpathlineto{\pgfqpoint{3.118367in}{0.517039in}}%
\pgfusepath{stroke}%
\end{pgfscope}%
\begin{pgfscope}%
\pgfsetrectcap%
\pgfsetmiterjoin%
\pgfsetlinewidth{0.803000pt}%
\definecolor{currentstroke}{rgb}{0.000000,0.000000,0.000000}%
\pgfsetstrokecolor{currentstroke}%
\pgfsetdash{}{0pt}%
\pgfpathmoveto{\pgfqpoint{0.779897in}{2.895016in}}%
\pgfpathlineto{\pgfqpoint{3.118367in}{2.895016in}}%
\pgfusepath{stroke}%
\end{pgfscope}%
\begin{pgfscope}%
\pgfsetbuttcap%
\pgfsetmiterjoin%
\pgfsetlinewidth{0.000000pt}%
\definecolor{currentstroke}{rgb}{0.000000,0.000000,0.000000}%
\pgfsetstrokecolor{currentstroke}%
\pgfsetstrokeopacity{0.000000}%
\pgfsetdash{}{0pt}%
\pgfpathmoveto{\pgfqpoint{3.288310in}{0.517039in}}%
\pgfpathlineto{\pgfqpoint{3.407209in}{0.517039in}}%
\pgfpathlineto{\pgfqpoint{3.407209in}{2.895016in}}%
\pgfpathlineto{\pgfqpoint{3.288310in}{2.895016in}}%
\pgfpathlineto{\pgfqpoint{3.288310in}{0.517039in}}%
\pgfpathclose%
\pgfusepath{}%
\end{pgfscope}%
\begin{pgfscope}%
\pgfsys@transformshift{3.290000in}{0.520000in}%
\pgftext[left,bottom]{\includegraphics[interpolate=true,width=0.120000in,height=2.380000in]{burgers_ci_0.0-img1.png}}%
\end{pgfscope}%
\begin{pgfscope}%
\pgfsetbuttcap%
\pgfsetroundjoin%
\definecolor{currentfill}{rgb}{0.000000,0.000000,0.000000}%
\pgfsetfillcolor{currentfill}%
\pgfsetlinewidth{0.803000pt}%
\definecolor{currentstroke}{rgb}{0.000000,0.000000,0.000000}%
\pgfsetstrokecolor{currentstroke}%
\pgfsetdash{}{0pt}%
\pgfsys@defobject{currentmarker}{\pgfqpoint{0.000000in}{0.000000in}}{\pgfqpoint{0.048611in}{0.000000in}}{%
\pgfpathmoveto{\pgfqpoint{0.000000in}{0.000000in}}%
\pgfpathlineto{\pgfqpoint{0.048611in}{0.000000in}}%
\pgfusepath{stroke,fill}%
}%
\begin{pgfscope}%
\pgfsys@transformshift{3.407209in}{0.639045in}%
\pgfsys@useobject{currentmarker}{}%
\end{pgfscope}%
\end{pgfscope}%
\begin{pgfscope}%
\definecolor{textcolor}{rgb}{0.000000,0.000000,0.000000}%
\pgfsetstrokecolor{textcolor}%
\pgfsetfillcolor{textcolor}%
\pgftext[x=3.504431in, y=0.575731in, left, base]{\color{textcolor}{\rmfamily\fontsize{12.000000}{14.400000}\selectfont\catcode`\^=\active\def^{\ifmmode\sp\else\^{}\fi}\catcode`\%=\active\def%{\%}\ensuremath{-}600}}%
\end{pgfscope}%
\begin{pgfscope}%
\pgfsetbuttcap%
\pgfsetroundjoin%
\definecolor{currentfill}{rgb}{0.000000,0.000000,0.000000}%
\pgfsetfillcolor{currentfill}%
\pgfsetlinewidth{0.803000pt}%
\definecolor{currentstroke}{rgb}{0.000000,0.000000,0.000000}%
\pgfsetstrokecolor{currentstroke}%
\pgfsetdash{}{0pt}%
\pgfsys@defobject{currentmarker}{\pgfqpoint{0.000000in}{0.000000in}}{\pgfqpoint{0.048611in}{0.000000in}}{%
\pgfpathmoveto{\pgfqpoint{0.000000in}{0.000000in}}%
\pgfpathlineto{\pgfqpoint{0.048611in}{0.000000in}}%
\pgfusepath{stroke,fill}%
}%
\begin{pgfscope}%
\pgfsys@transformshift{3.407209in}{0.980231in}%
\pgfsys@useobject{currentmarker}{}%
\end{pgfscope}%
\end{pgfscope}%
\begin{pgfscope}%
\definecolor{textcolor}{rgb}{0.000000,0.000000,0.000000}%
\pgfsetstrokecolor{textcolor}%
\pgfsetfillcolor{textcolor}%
\pgftext[x=3.504431in, y=0.916917in, left, base]{\color{textcolor}{\rmfamily\fontsize{12.000000}{14.400000}\selectfont\catcode`\^=\active\def^{\ifmmode\sp\else\^{}\fi}\catcode`\%=\active\def%{\%}\ensuremath{-}400}}%
\end{pgfscope}%
\begin{pgfscope}%
\pgfsetbuttcap%
\pgfsetroundjoin%
\definecolor{currentfill}{rgb}{0.000000,0.000000,0.000000}%
\pgfsetfillcolor{currentfill}%
\pgfsetlinewidth{0.803000pt}%
\definecolor{currentstroke}{rgb}{0.000000,0.000000,0.000000}%
\pgfsetstrokecolor{currentstroke}%
\pgfsetdash{}{0pt}%
\pgfsys@defobject{currentmarker}{\pgfqpoint{0.000000in}{0.000000in}}{\pgfqpoint{0.048611in}{0.000000in}}{%
\pgfpathmoveto{\pgfqpoint{0.000000in}{0.000000in}}%
\pgfpathlineto{\pgfqpoint{0.048611in}{0.000000in}}%
\pgfusepath{stroke,fill}%
}%
\begin{pgfscope}%
\pgfsys@transformshift{3.407209in}{1.321417in}%
\pgfsys@useobject{currentmarker}{}%
\end{pgfscope}%
\end{pgfscope}%
\begin{pgfscope}%
\definecolor{textcolor}{rgb}{0.000000,0.000000,0.000000}%
\pgfsetstrokecolor{textcolor}%
\pgfsetfillcolor{textcolor}%
\pgftext[x=3.504431in, y=1.258103in, left, base]{\color{textcolor}{\rmfamily\fontsize{12.000000}{14.400000}\selectfont\catcode`\^=\active\def^{\ifmmode\sp\else\^{}\fi}\catcode`\%=\active\def%{\%}\ensuremath{-}200}}%
\end{pgfscope}%
\begin{pgfscope}%
\pgfsetbuttcap%
\pgfsetroundjoin%
\definecolor{currentfill}{rgb}{0.000000,0.000000,0.000000}%
\pgfsetfillcolor{currentfill}%
\pgfsetlinewidth{0.803000pt}%
\definecolor{currentstroke}{rgb}{0.000000,0.000000,0.000000}%
\pgfsetstrokecolor{currentstroke}%
\pgfsetdash{}{0pt}%
\pgfsys@defobject{currentmarker}{\pgfqpoint{0.000000in}{0.000000in}}{\pgfqpoint{0.048611in}{0.000000in}}{%
\pgfpathmoveto{\pgfqpoint{0.000000in}{0.000000in}}%
\pgfpathlineto{\pgfqpoint{0.048611in}{0.000000in}}%
\pgfusepath{stroke,fill}%
}%
\begin{pgfscope}%
\pgfsys@transformshift{3.407209in}{1.662604in}%
\pgfsys@useobject{currentmarker}{}%
\end{pgfscope}%
\end{pgfscope}%
\begin{pgfscope}%
\definecolor{textcolor}{rgb}{0.000000,0.000000,0.000000}%
\pgfsetstrokecolor{textcolor}%
\pgfsetfillcolor{textcolor}%
\pgftext[x=3.504431in, y=1.599290in, left, base]{\color{textcolor}{\rmfamily\fontsize{12.000000}{14.400000}\selectfont\catcode`\^=\active\def^{\ifmmode\sp\else\^{}\fi}\catcode`\%=\active\def%{\%}0}}%
\end{pgfscope}%
\begin{pgfscope}%
\pgfsetbuttcap%
\pgfsetroundjoin%
\definecolor{currentfill}{rgb}{0.000000,0.000000,0.000000}%
\pgfsetfillcolor{currentfill}%
\pgfsetlinewidth{0.803000pt}%
\definecolor{currentstroke}{rgb}{0.000000,0.000000,0.000000}%
\pgfsetstrokecolor{currentstroke}%
\pgfsetdash{}{0pt}%
\pgfsys@defobject{currentmarker}{\pgfqpoint{0.000000in}{0.000000in}}{\pgfqpoint{0.048611in}{0.000000in}}{%
\pgfpathmoveto{\pgfqpoint{0.000000in}{0.000000in}}%
\pgfpathlineto{\pgfqpoint{0.048611in}{0.000000in}}%
\pgfusepath{stroke,fill}%
}%
\begin{pgfscope}%
\pgfsys@transformshift{3.407209in}{2.003790in}%
\pgfsys@useobject{currentmarker}{}%
\end{pgfscope}%
\end{pgfscope}%
\begin{pgfscope}%
\definecolor{textcolor}{rgb}{0.000000,0.000000,0.000000}%
\pgfsetstrokecolor{textcolor}%
\pgfsetfillcolor{textcolor}%
\pgftext[x=3.504431in, y=1.940476in, left, base]{\color{textcolor}{\rmfamily\fontsize{12.000000}{14.400000}\selectfont\catcode`\^=\active\def^{\ifmmode\sp\else\^{}\fi}\catcode`\%=\active\def%{\%}200}}%
\end{pgfscope}%
\begin{pgfscope}%
\pgfsetbuttcap%
\pgfsetroundjoin%
\definecolor{currentfill}{rgb}{0.000000,0.000000,0.000000}%
\pgfsetfillcolor{currentfill}%
\pgfsetlinewidth{0.803000pt}%
\definecolor{currentstroke}{rgb}{0.000000,0.000000,0.000000}%
\pgfsetstrokecolor{currentstroke}%
\pgfsetdash{}{0pt}%
\pgfsys@defobject{currentmarker}{\pgfqpoint{0.000000in}{0.000000in}}{\pgfqpoint{0.048611in}{0.000000in}}{%
\pgfpathmoveto{\pgfqpoint{0.000000in}{0.000000in}}%
\pgfpathlineto{\pgfqpoint{0.048611in}{0.000000in}}%
\pgfusepath{stroke,fill}%
}%
\begin{pgfscope}%
\pgfsys@transformshift{3.407209in}{2.344976in}%
\pgfsys@useobject{currentmarker}{}%
\end{pgfscope}%
\end{pgfscope}%
\begin{pgfscope}%
\definecolor{textcolor}{rgb}{0.000000,0.000000,0.000000}%
\pgfsetstrokecolor{textcolor}%
\pgfsetfillcolor{textcolor}%
\pgftext[x=3.504431in, y=2.281663in, left, base]{\color{textcolor}{\rmfamily\fontsize{12.000000}{14.400000}\selectfont\catcode`\^=\active\def^{\ifmmode\sp\else\^{}\fi}\catcode`\%=\active\def%{\%}400}}%
\end{pgfscope}%
\begin{pgfscope}%
\pgfsetbuttcap%
\pgfsetroundjoin%
\definecolor{currentfill}{rgb}{0.000000,0.000000,0.000000}%
\pgfsetfillcolor{currentfill}%
\pgfsetlinewidth{0.803000pt}%
\definecolor{currentstroke}{rgb}{0.000000,0.000000,0.000000}%
\pgfsetstrokecolor{currentstroke}%
\pgfsetdash{}{0pt}%
\pgfsys@defobject{currentmarker}{\pgfqpoint{0.000000in}{0.000000in}}{\pgfqpoint{0.048611in}{0.000000in}}{%
\pgfpathmoveto{\pgfqpoint{0.000000in}{0.000000in}}%
\pgfpathlineto{\pgfqpoint{0.048611in}{0.000000in}}%
\pgfusepath{stroke,fill}%
}%
\begin{pgfscope}%
\pgfsys@transformshift{3.407209in}{2.686163in}%
\pgfsys@useobject{currentmarker}{}%
\end{pgfscope}%
\end{pgfscope}%
\begin{pgfscope}%
\definecolor{textcolor}{rgb}{0.000000,0.000000,0.000000}%
\pgfsetstrokecolor{textcolor}%
\pgfsetfillcolor{textcolor}%
\pgftext[x=3.504431in, y=2.622849in, left, base]{\color{textcolor}{\rmfamily\fontsize{12.000000}{14.400000}\selectfont\catcode`\^=\active\def^{\ifmmode\sp\else\^{}\fi}\catcode`\%=\active\def%{\%}600}}%
\end{pgfscope}%
\begin{pgfscope}%
\pgfsetrectcap%
\pgfsetmiterjoin%
\pgfsetlinewidth{0.803000pt}%
\definecolor{currentstroke}{rgb}{0.000000,0.000000,0.000000}%
\pgfsetstrokecolor{currentstroke}%
\pgfsetdash{}{0pt}%
\pgfpathmoveto{\pgfqpoint{3.288310in}{0.517039in}}%
\pgfpathlineto{\pgfqpoint{3.347759in}{0.517039in}}%
\pgfpathlineto{\pgfqpoint{3.407209in}{0.517039in}}%
\pgfpathlineto{\pgfqpoint{3.407209in}{2.895016in}}%
\pgfpathlineto{\pgfqpoint{3.347759in}{2.895016in}}%
\pgfpathlineto{\pgfqpoint{3.288310in}{2.895016in}}%
\pgfpathlineto{\pgfqpoint{3.288310in}{0.517039in}}%
\pgfpathclose%
\pgfusepath{stroke}%
\end{pgfscope}%
\end{pgfpicture}%
\makeatother%
\endgroup%

    \end{adjustbox}
    \caption{Correlation image \(\nu=0.0\).}\label{fig:sc2_ci_0.0}
  \end{subfigure}
  \begin{subfigure}{0.49\linewidth}
    \begin{adjustbox}{width=\linewidth}
      \input{figures/burgers_pm_0.0.pgf}
    \end{adjustbox}
    \caption{The p-matrix for \(\nu=0.0\).}\label{fig:sc2_pm_0.0}
  \end{subfigure}
  % \\[-0.7\baselineskip]
  \begin{subfigure}{0.49\linewidth}
    \begin{adjustbox}{width=\linewidth}
      \begingroup%
\makeatletter%
\begin{pgfpicture}%
\pgfpathrectangle{\pgfpointorigin}{\pgfqpoint{4.110195in}{3.116660in}}%
\pgfusepath{use as bounding box, clip}%
\begin{pgfscope}%
\pgfsetbuttcap%
\pgfsetmiterjoin%
\pgfsetlinewidth{0.000000pt}%
\definecolor{currentstroke}{rgb}{0.000000,0.000000,0.000000}%
\pgfsetstrokecolor{currentstroke}%
\pgfsetstrokeopacity{0.000000}%
\pgfsetdash{}{0pt}%
\pgfpathmoveto{\pgfqpoint{0.000000in}{0.000000in}}%
\pgfpathlineto{\pgfqpoint{4.110195in}{0.000000in}}%
\pgfpathlineto{\pgfqpoint{4.110195in}{3.116660in}}%
\pgfpathlineto{\pgfqpoint{0.000000in}{3.116660in}}%
\pgfpathlineto{\pgfqpoint{0.000000in}{0.000000in}}%
\pgfpathclose%
\pgfusepath{}%
\end{pgfscope}%
\begin{pgfscope}%
\pgfsetbuttcap%
\pgfsetmiterjoin%
\pgfsetlinewidth{0.000000pt}%
\definecolor{currentstroke}{rgb}{0.000000,0.000000,0.000000}%
\pgfsetstrokecolor{currentstroke}%
\pgfsetstrokeopacity{0.000000}%
\pgfsetdash{}{0pt}%
\pgfpathmoveto{\pgfqpoint{0.838227in}{0.575369in}}%
\pgfpathlineto{\pgfqpoint{3.226888in}{0.575369in}}%
\pgfpathlineto{\pgfqpoint{3.226888in}{2.953495in}}%
\pgfpathlineto{\pgfqpoint{0.838227in}{2.953495in}}%
\pgfpathlineto{\pgfqpoint{0.838227in}{0.575369in}}%
\pgfpathclose%
\pgfusepath{}%
\end{pgfscope}%
\begin{pgfscope}%
\pgfpathrectangle{\pgfqpoint{0.838227in}{0.575369in}}{\pgfqpoint{2.388661in}{2.378126in}}%
\pgfusepath{clip}%
\pgfsys@transformcm{2.388661}{0.000000}{0.000000}{-2.378126}{0.838227in}{2.953495in}%
\pgftext[left,bottom]{\includegraphics[interpolate=false,width=1.000000in,height=1.000000in]{burgers_ci_0.01-img0.png}}%
\end{pgfscope}%
\begin{pgfscope}%
\pgfsetbuttcap%
\pgfsetroundjoin%
\definecolor{currentfill}{rgb}{0.000000,0.000000,0.000000}%
\pgfsetfillcolor{currentfill}%
\pgfsetlinewidth{0.803000pt}%
\definecolor{currentstroke}{rgb}{0.000000,0.000000,0.000000}%
\pgfsetstrokecolor{currentstroke}%
\pgfsetdash{}{0pt}%
\pgfsys@defobject{currentmarker}{\pgfqpoint{0.000000in}{-0.048611in}}{\pgfqpoint{0.000000in}{0.000000in}}{%
\pgfpathmoveto{\pgfqpoint{0.000000in}{0.000000in}}%
\pgfpathlineto{\pgfqpoint{0.000000in}{-0.048611in}}%
\pgfusepath{stroke,fill}%
}%
\begin{pgfscope}%
\pgfsys@transformshift{0.875550in}{0.575369in}%
\pgfsys@useobject{currentmarker}{}%
\end{pgfscope}%
\end{pgfscope}%
\begin{pgfscope}%
\definecolor{textcolor}{rgb}{0.000000,0.000000,0.000000}%
\pgfsetstrokecolor{textcolor}%
\pgfsetfillcolor{textcolor}%
\pgftext[x=0.875550in,y=0.478146in,,top]{\color{textcolor}\rmfamily\fontsize{12.000000}{14.400000}\selectfont 0}%
\end{pgfscope}%
\begin{pgfscope}%
\pgfsetbuttcap%
\pgfsetroundjoin%
\definecolor{currentfill}{rgb}{0.000000,0.000000,0.000000}%
\pgfsetfillcolor{currentfill}%
\pgfsetlinewidth{0.803000pt}%
\definecolor{currentstroke}{rgb}{0.000000,0.000000,0.000000}%
\pgfsetstrokecolor{currentstroke}%
\pgfsetdash{}{0pt}%
\pgfsys@defobject{currentmarker}{\pgfqpoint{0.000000in}{-0.048611in}}{\pgfqpoint{0.000000in}{0.000000in}}{%
\pgfpathmoveto{\pgfqpoint{0.000000in}{0.000000in}}%
\pgfpathlineto{\pgfqpoint{0.000000in}{-0.048611in}}%
\pgfusepath{stroke,fill}%
}%
\begin{pgfscope}%
\pgfsys@transformshift{1.622006in}{0.575369in}%
\pgfsys@useobject{currentmarker}{}%
\end{pgfscope}%
\end{pgfscope}%
\begin{pgfscope}%
\definecolor{textcolor}{rgb}{0.000000,0.000000,0.000000}%
\pgfsetstrokecolor{textcolor}%
\pgfsetfillcolor{textcolor}%
\pgftext[x=1.622006in,y=0.478146in,,top]{\color{textcolor}\rmfamily\fontsize{12.000000}{14.400000}\selectfont 10}%
\end{pgfscope}%
\begin{pgfscope}%
\pgfsetbuttcap%
\pgfsetroundjoin%
\definecolor{currentfill}{rgb}{0.000000,0.000000,0.000000}%
\pgfsetfillcolor{currentfill}%
\pgfsetlinewidth{0.803000pt}%
\definecolor{currentstroke}{rgb}{0.000000,0.000000,0.000000}%
\pgfsetstrokecolor{currentstroke}%
\pgfsetdash{}{0pt}%
\pgfsys@defobject{currentmarker}{\pgfqpoint{0.000000in}{-0.048611in}}{\pgfqpoint{0.000000in}{0.000000in}}{%
\pgfpathmoveto{\pgfqpoint{0.000000in}{0.000000in}}%
\pgfpathlineto{\pgfqpoint{0.000000in}{-0.048611in}}%
\pgfusepath{stroke,fill}%
}%
\begin{pgfscope}%
\pgfsys@transformshift{2.368463in}{0.575369in}%
\pgfsys@useobject{currentmarker}{}%
\end{pgfscope}%
\end{pgfscope}%
\begin{pgfscope}%
\definecolor{textcolor}{rgb}{0.000000,0.000000,0.000000}%
\pgfsetstrokecolor{textcolor}%
\pgfsetfillcolor{textcolor}%
\pgftext[x=2.368463in,y=0.478146in,,top]{\color{textcolor}\rmfamily\fontsize{12.000000}{14.400000}\selectfont 20}%
\end{pgfscope}%
\begin{pgfscope}%
\pgfsetbuttcap%
\pgfsetroundjoin%
\definecolor{currentfill}{rgb}{0.000000,0.000000,0.000000}%
\pgfsetfillcolor{currentfill}%
\pgfsetlinewidth{0.803000pt}%
\definecolor{currentstroke}{rgb}{0.000000,0.000000,0.000000}%
\pgfsetstrokecolor{currentstroke}%
\pgfsetdash{}{0pt}%
\pgfsys@defobject{currentmarker}{\pgfqpoint{0.000000in}{-0.048611in}}{\pgfqpoint{0.000000in}{0.000000in}}{%
\pgfpathmoveto{\pgfqpoint{0.000000in}{0.000000in}}%
\pgfpathlineto{\pgfqpoint{0.000000in}{-0.048611in}}%
\pgfusepath{stroke,fill}%
}%
\begin{pgfscope}%
\pgfsys@transformshift{3.114920in}{0.575369in}%
\pgfsys@useobject{currentmarker}{}%
\end{pgfscope}%
\end{pgfscope}%
\begin{pgfscope}%
\definecolor{textcolor}{rgb}{0.000000,0.000000,0.000000}%
\pgfsetstrokecolor{textcolor}%
\pgfsetfillcolor{textcolor}%
\pgftext[x=3.114920in,y=0.478146in,,top]{\color{textcolor}\rmfamily\fontsize{12.000000}{14.400000}\selectfont 30}%
\end{pgfscope}%
\begin{pgfscope}%
\definecolor{textcolor}{rgb}{0.000000,0.000000,0.000000}%
\pgfsetstrokecolor{textcolor}%
\pgfsetfillcolor{textcolor}%
\pgftext[x=2.032557in,y=0.261295in,,top]{\color{textcolor}\rmfamily\fontsize{12.000000}{14.400000}\selectfont input coefficients}%
\end{pgfscope}%
\begin{pgfscope}%
\pgfsetbuttcap%
\pgfsetroundjoin%
\definecolor{currentfill}{rgb}{0.000000,0.000000,0.000000}%
\pgfsetfillcolor{currentfill}%
\pgfsetlinewidth{0.803000pt}%
\definecolor{currentstroke}{rgb}{0.000000,0.000000,0.000000}%
\pgfsetstrokecolor{currentstroke}%
\pgfsetdash{}{0pt}%
\pgfsys@defobject{currentmarker}{\pgfqpoint{-0.048611in}{0.000000in}}{\pgfqpoint{-0.000000in}{0.000000in}}{%
\pgfpathmoveto{\pgfqpoint{-0.000000in}{0.000000in}}%
\pgfpathlineto{\pgfqpoint{-0.048611in}{0.000000in}}%
\pgfusepath{stroke,fill}%
}%
\begin{pgfscope}%
\pgfsys@transformshift{0.838227in}{2.953346in}%
\pgfsys@useobject{currentmarker}{}%
\end{pgfscope}%
\end{pgfscope}%
\begin{pgfscope}%
\definecolor{textcolor}{rgb}{0.000000,0.000000,0.000000}%
\pgfsetstrokecolor{textcolor}%
\pgfsetfillcolor{textcolor}%
\pgftext[x=0.634966in, y=2.890032in, left, base]{\color{textcolor}\rmfamily\fontsize{12.000000}{14.400000}\selectfont 0}%
\end{pgfscope}%
\begin{pgfscope}%
\pgfsetbuttcap%
\pgfsetroundjoin%
\definecolor{currentfill}{rgb}{0.000000,0.000000,0.000000}%
\pgfsetfillcolor{currentfill}%
\pgfsetlinewidth{0.803000pt}%
\definecolor{currentstroke}{rgb}{0.000000,0.000000,0.000000}%
\pgfsetstrokecolor{currentstroke}%
\pgfsetdash{}{0pt}%
\pgfsys@defobject{currentmarker}{\pgfqpoint{-0.048611in}{0.000000in}}{\pgfqpoint{-0.000000in}{0.000000in}}{%
\pgfpathmoveto{\pgfqpoint{-0.000000in}{0.000000in}}%
\pgfpathlineto{\pgfqpoint{-0.048611in}{0.000000in}}%
\pgfusepath{stroke,fill}%
}%
\begin{pgfscope}%
\pgfsys@transformshift{0.838227in}{2.358815in}%
\pgfsys@useobject{currentmarker}{}%
\end{pgfscope}%
\end{pgfscope}%
\begin{pgfscope}%
\definecolor{textcolor}{rgb}{0.000000,0.000000,0.000000}%
\pgfsetstrokecolor{textcolor}%
\pgfsetfillcolor{textcolor}%
\pgftext[x=0.316851in, y=2.295501in, left, base]{\color{textcolor}\rmfamily\fontsize{12.000000}{14.400000}\selectfont 2000}%
\end{pgfscope}%
\begin{pgfscope}%
\pgfsetbuttcap%
\pgfsetroundjoin%
\definecolor{currentfill}{rgb}{0.000000,0.000000,0.000000}%
\pgfsetfillcolor{currentfill}%
\pgfsetlinewidth{0.803000pt}%
\definecolor{currentstroke}{rgb}{0.000000,0.000000,0.000000}%
\pgfsetstrokecolor{currentstroke}%
\pgfsetdash{}{0pt}%
\pgfsys@defobject{currentmarker}{\pgfqpoint{-0.048611in}{0.000000in}}{\pgfqpoint{-0.000000in}{0.000000in}}{%
\pgfpathmoveto{\pgfqpoint{-0.000000in}{0.000000in}}%
\pgfpathlineto{\pgfqpoint{-0.048611in}{0.000000in}}%
\pgfusepath{stroke,fill}%
}%
\begin{pgfscope}%
\pgfsys@transformshift{0.838227in}{1.764283in}%
\pgfsys@useobject{currentmarker}{}%
\end{pgfscope}%
\end{pgfscope}%
\begin{pgfscope}%
\definecolor{textcolor}{rgb}{0.000000,0.000000,0.000000}%
\pgfsetstrokecolor{textcolor}%
\pgfsetfillcolor{textcolor}%
\pgftext[x=0.316851in, y=1.700969in, left, base]{\color{textcolor}\rmfamily\fontsize{12.000000}{14.400000}\selectfont 4000}%
\end{pgfscope}%
\begin{pgfscope}%
\pgfsetbuttcap%
\pgfsetroundjoin%
\definecolor{currentfill}{rgb}{0.000000,0.000000,0.000000}%
\pgfsetfillcolor{currentfill}%
\pgfsetlinewidth{0.803000pt}%
\definecolor{currentstroke}{rgb}{0.000000,0.000000,0.000000}%
\pgfsetstrokecolor{currentstroke}%
\pgfsetdash{}{0pt}%
\pgfsys@defobject{currentmarker}{\pgfqpoint{-0.048611in}{0.000000in}}{\pgfqpoint{-0.000000in}{0.000000in}}{%
\pgfpathmoveto{\pgfqpoint{-0.000000in}{0.000000in}}%
\pgfpathlineto{\pgfqpoint{-0.048611in}{0.000000in}}%
\pgfusepath{stroke,fill}%
}%
\begin{pgfscope}%
\pgfsys@transformshift{0.838227in}{1.169752in}%
\pgfsys@useobject{currentmarker}{}%
\end{pgfscope}%
\end{pgfscope}%
\begin{pgfscope}%
\definecolor{textcolor}{rgb}{0.000000,0.000000,0.000000}%
\pgfsetstrokecolor{textcolor}%
\pgfsetfillcolor{textcolor}%
\pgftext[x=0.316851in, y=1.106438in, left, base]{\color{textcolor}\rmfamily\fontsize{12.000000}{14.400000}\selectfont 6000}%
\end{pgfscope}%
\begin{pgfscope}%
\definecolor{textcolor}{rgb}{0.000000,0.000000,0.000000}%
\pgfsetstrokecolor{textcolor}%
\pgfsetfillcolor{textcolor}%
\pgftext[x=0.261295in,y=1.764432in,,bottom,rotate=90.000000]{\color{textcolor}\rmfamily\fontsize{12.000000}{14.400000}\selectfont samples}%
\end{pgfscope}%
\begin{pgfscope}%
\pgfsetrectcap%
\pgfsetmiterjoin%
\pgfsetlinewidth{0.803000pt}%
\definecolor{currentstroke}{rgb}{0.000000,0.000000,0.000000}%
\pgfsetstrokecolor{currentstroke}%
\pgfsetdash{}{0pt}%
\pgfpathmoveto{\pgfqpoint{0.838227in}{0.575369in}}%
\pgfpathlineto{\pgfqpoint{0.838227in}{2.953495in}}%
\pgfusepath{stroke}%
\end{pgfscope}%
\begin{pgfscope}%
\pgfsetrectcap%
\pgfsetmiterjoin%
\pgfsetlinewidth{0.803000pt}%
\definecolor{currentstroke}{rgb}{0.000000,0.000000,0.000000}%
\pgfsetstrokecolor{currentstroke}%
\pgfsetdash{}{0pt}%
\pgfpathmoveto{\pgfqpoint{3.226888in}{0.575369in}}%
\pgfpathlineto{\pgfqpoint{3.226888in}{2.953495in}}%
\pgfusepath{stroke}%
\end{pgfscope}%
\begin{pgfscope}%
\pgfsetrectcap%
\pgfsetmiterjoin%
\pgfsetlinewidth{0.803000pt}%
\definecolor{currentstroke}{rgb}{0.000000,0.000000,0.000000}%
\pgfsetstrokecolor{currentstroke}%
\pgfsetdash{}{0pt}%
\pgfpathmoveto{\pgfqpoint{0.838227in}{0.575369in}}%
\pgfpathlineto{\pgfqpoint{3.226888in}{0.575369in}}%
\pgfusepath{stroke}%
\end{pgfscope}%
\begin{pgfscope}%
\pgfsetrectcap%
\pgfsetmiterjoin%
\pgfsetlinewidth{0.803000pt}%
\definecolor{currentstroke}{rgb}{0.000000,0.000000,0.000000}%
\pgfsetstrokecolor{currentstroke}%
\pgfsetdash{}{0pt}%
\pgfpathmoveto{\pgfqpoint{0.838227in}{2.953495in}}%
\pgfpathlineto{\pgfqpoint{3.226888in}{2.953495in}}%
\pgfusepath{stroke}%
\end{pgfscope}%
\begin{pgfscope}%
\pgfsetbuttcap%
\pgfsetmiterjoin%
\pgfsetlinewidth{0.000000pt}%
\definecolor{currentstroke}{rgb}{0.000000,0.000000,0.000000}%
\pgfsetstrokecolor{currentstroke}%
\pgfsetstrokeopacity{0.000000}%
\pgfsetdash{}{0pt}%
\pgfpathmoveto{\pgfqpoint{3.346321in}{0.575369in}}%
\pgfpathlineto{\pgfqpoint{3.465227in}{0.575369in}}%
\pgfpathlineto{\pgfqpoint{3.465227in}{2.953495in}}%
\pgfpathlineto{\pgfqpoint{3.346321in}{2.953495in}}%
\pgfpathlineto{\pgfqpoint{3.346321in}{0.575369in}}%
\pgfpathclose%
\pgfusepath{}%
\end{pgfscope}%
\begin{pgfscope}%
\pgfsys@transformshift{3.350000in}{0.586660in}%
\pgftext[left,bottom]{\includegraphics[interpolate=true,width=0.120000in,height=2.370000in]{burgers_ci_0.01-img1.png}}%
\end{pgfscope}%
\begin{pgfscope}%
\pgfsetbuttcap%
\pgfsetroundjoin%
\definecolor{currentfill}{rgb}{0.000000,0.000000,0.000000}%
\pgfsetfillcolor{currentfill}%
\pgfsetlinewidth{0.803000pt}%
\definecolor{currentstroke}{rgb}{0.000000,0.000000,0.000000}%
\pgfsetstrokecolor{currentstroke}%
\pgfsetdash{}{0pt}%
\pgfsys@defobject{currentmarker}{\pgfqpoint{0.000000in}{0.000000in}}{\pgfqpoint{0.048611in}{0.000000in}}{%
\pgfpathmoveto{\pgfqpoint{0.000000in}{0.000000in}}%
\pgfpathlineto{\pgfqpoint{0.048611in}{0.000000in}}%
\pgfusepath{stroke,fill}%
}%
\begin{pgfscope}%
\pgfsys@transformshift{3.465227in}{0.682459in}%
\pgfsys@useobject{currentmarker}{}%
\end{pgfscope}%
\end{pgfscope}%
\begin{pgfscope}%
\definecolor{textcolor}{rgb}{0.000000,0.000000,0.000000}%
\pgfsetstrokecolor{textcolor}%
\pgfsetfillcolor{textcolor}%
\pgftext[x=3.562450in, y=0.619145in, left, base]{\color{textcolor}\rmfamily\fontsize{12.000000}{14.400000}\selectfont \ensuremath{-}600}%
\end{pgfscope}%
\begin{pgfscope}%
\pgfsetbuttcap%
\pgfsetroundjoin%
\definecolor{currentfill}{rgb}{0.000000,0.000000,0.000000}%
\pgfsetfillcolor{currentfill}%
\pgfsetlinewidth{0.803000pt}%
\definecolor{currentstroke}{rgb}{0.000000,0.000000,0.000000}%
\pgfsetstrokecolor{currentstroke}%
\pgfsetdash{}{0pt}%
\pgfsys@defobject{currentmarker}{\pgfqpoint{0.000000in}{0.000000in}}{\pgfqpoint{0.048611in}{0.000000in}}{%
\pgfpathmoveto{\pgfqpoint{0.000000in}{0.000000in}}%
\pgfpathlineto{\pgfqpoint{0.048611in}{0.000000in}}%
\pgfusepath{stroke,fill}%
}%
\begin{pgfscope}%
\pgfsys@transformshift{3.465227in}{1.043116in}%
\pgfsys@useobject{currentmarker}{}%
\end{pgfscope}%
\end{pgfscope}%
\begin{pgfscope}%
\definecolor{textcolor}{rgb}{0.000000,0.000000,0.000000}%
\pgfsetstrokecolor{textcolor}%
\pgfsetfillcolor{textcolor}%
\pgftext[x=3.562450in, y=0.979803in, left, base]{\color{textcolor}\rmfamily\fontsize{12.000000}{14.400000}\selectfont \ensuremath{-}400}%
\end{pgfscope}%
\begin{pgfscope}%
\pgfsetbuttcap%
\pgfsetroundjoin%
\definecolor{currentfill}{rgb}{0.000000,0.000000,0.000000}%
\pgfsetfillcolor{currentfill}%
\pgfsetlinewidth{0.803000pt}%
\definecolor{currentstroke}{rgb}{0.000000,0.000000,0.000000}%
\pgfsetstrokecolor{currentstroke}%
\pgfsetdash{}{0pt}%
\pgfsys@defobject{currentmarker}{\pgfqpoint{0.000000in}{0.000000in}}{\pgfqpoint{0.048611in}{0.000000in}}{%
\pgfpathmoveto{\pgfqpoint{0.000000in}{0.000000in}}%
\pgfpathlineto{\pgfqpoint{0.048611in}{0.000000in}}%
\pgfusepath{stroke,fill}%
}%
\begin{pgfscope}%
\pgfsys@transformshift{3.465227in}{1.403774in}%
\pgfsys@useobject{currentmarker}{}%
\end{pgfscope}%
\end{pgfscope}%
\begin{pgfscope}%
\definecolor{textcolor}{rgb}{0.000000,0.000000,0.000000}%
\pgfsetstrokecolor{textcolor}%
\pgfsetfillcolor{textcolor}%
\pgftext[x=3.562450in, y=1.340460in, left, base]{\color{textcolor}\rmfamily\fontsize{12.000000}{14.400000}\selectfont \ensuremath{-}200}%
\end{pgfscope}%
\begin{pgfscope}%
\pgfsetbuttcap%
\pgfsetroundjoin%
\definecolor{currentfill}{rgb}{0.000000,0.000000,0.000000}%
\pgfsetfillcolor{currentfill}%
\pgfsetlinewidth{0.803000pt}%
\definecolor{currentstroke}{rgb}{0.000000,0.000000,0.000000}%
\pgfsetstrokecolor{currentstroke}%
\pgfsetdash{}{0pt}%
\pgfsys@defobject{currentmarker}{\pgfqpoint{0.000000in}{0.000000in}}{\pgfqpoint{0.048611in}{0.000000in}}{%
\pgfpathmoveto{\pgfqpoint{0.000000in}{0.000000in}}%
\pgfpathlineto{\pgfqpoint{0.048611in}{0.000000in}}%
\pgfusepath{stroke,fill}%
}%
\begin{pgfscope}%
\pgfsys@transformshift{3.465227in}{1.764432in}%
\pgfsys@useobject{currentmarker}{}%
\end{pgfscope}%
\end{pgfscope}%
\begin{pgfscope}%
\definecolor{textcolor}{rgb}{0.000000,0.000000,0.000000}%
\pgfsetstrokecolor{textcolor}%
\pgfsetfillcolor{textcolor}%
\pgftext[x=3.562450in, y=1.701118in, left, base]{\color{textcolor}\rmfamily\fontsize{12.000000}{14.400000}\selectfont 0}%
\end{pgfscope}%
\begin{pgfscope}%
\pgfsetbuttcap%
\pgfsetroundjoin%
\definecolor{currentfill}{rgb}{0.000000,0.000000,0.000000}%
\pgfsetfillcolor{currentfill}%
\pgfsetlinewidth{0.803000pt}%
\definecolor{currentstroke}{rgb}{0.000000,0.000000,0.000000}%
\pgfsetstrokecolor{currentstroke}%
\pgfsetdash{}{0pt}%
\pgfsys@defobject{currentmarker}{\pgfqpoint{0.000000in}{0.000000in}}{\pgfqpoint{0.048611in}{0.000000in}}{%
\pgfpathmoveto{\pgfqpoint{0.000000in}{0.000000in}}%
\pgfpathlineto{\pgfqpoint{0.048611in}{0.000000in}}%
\pgfusepath{stroke,fill}%
}%
\begin{pgfscope}%
\pgfsys@transformshift{3.465227in}{2.125089in}%
\pgfsys@useobject{currentmarker}{}%
\end{pgfscope}%
\end{pgfscope}%
\begin{pgfscope}%
\definecolor{textcolor}{rgb}{0.000000,0.000000,0.000000}%
\pgfsetstrokecolor{textcolor}%
\pgfsetfillcolor{textcolor}%
\pgftext[x=3.562450in, y=2.061776in, left, base]{\color{textcolor}\rmfamily\fontsize{12.000000}{14.400000}\selectfont 200}%
\end{pgfscope}%
\begin{pgfscope}%
\pgfsetbuttcap%
\pgfsetroundjoin%
\definecolor{currentfill}{rgb}{0.000000,0.000000,0.000000}%
\pgfsetfillcolor{currentfill}%
\pgfsetlinewidth{0.803000pt}%
\definecolor{currentstroke}{rgb}{0.000000,0.000000,0.000000}%
\pgfsetstrokecolor{currentstroke}%
\pgfsetdash{}{0pt}%
\pgfsys@defobject{currentmarker}{\pgfqpoint{0.000000in}{0.000000in}}{\pgfqpoint{0.048611in}{0.000000in}}{%
\pgfpathmoveto{\pgfqpoint{0.000000in}{0.000000in}}%
\pgfpathlineto{\pgfqpoint{0.048611in}{0.000000in}}%
\pgfusepath{stroke,fill}%
}%
\begin{pgfscope}%
\pgfsys@transformshift{3.465227in}{2.485747in}%
\pgfsys@useobject{currentmarker}{}%
\end{pgfscope}%
\end{pgfscope}%
\begin{pgfscope}%
\definecolor{textcolor}{rgb}{0.000000,0.000000,0.000000}%
\pgfsetstrokecolor{textcolor}%
\pgfsetfillcolor{textcolor}%
\pgftext[x=3.562450in, y=2.422433in, left, base]{\color{textcolor}\rmfamily\fontsize{12.000000}{14.400000}\selectfont 400}%
\end{pgfscope}%
\begin{pgfscope}%
\pgfsetbuttcap%
\pgfsetroundjoin%
\definecolor{currentfill}{rgb}{0.000000,0.000000,0.000000}%
\pgfsetfillcolor{currentfill}%
\pgfsetlinewidth{0.803000pt}%
\definecolor{currentstroke}{rgb}{0.000000,0.000000,0.000000}%
\pgfsetstrokecolor{currentstroke}%
\pgfsetdash{}{0pt}%
\pgfsys@defobject{currentmarker}{\pgfqpoint{0.000000in}{0.000000in}}{\pgfqpoint{0.048611in}{0.000000in}}{%
\pgfpathmoveto{\pgfqpoint{0.000000in}{0.000000in}}%
\pgfpathlineto{\pgfqpoint{0.048611in}{0.000000in}}%
\pgfusepath{stroke,fill}%
}%
\begin{pgfscope}%
\pgfsys@transformshift{3.465227in}{2.846405in}%
\pgfsys@useobject{currentmarker}{}%
\end{pgfscope}%
\end{pgfscope}%
\begin{pgfscope}%
\definecolor{textcolor}{rgb}{0.000000,0.000000,0.000000}%
\pgfsetstrokecolor{textcolor}%
\pgfsetfillcolor{textcolor}%
\pgftext[x=3.562450in, y=2.783091in, left, base]{\color{textcolor}\rmfamily\fontsize{12.000000}{14.400000}\selectfont 600}%
\end{pgfscope}%
\begin{pgfscope}%
\pgfsetrectcap%
\pgfsetmiterjoin%
\pgfsetlinewidth{0.803000pt}%
\definecolor{currentstroke}{rgb}{0.000000,0.000000,0.000000}%
\pgfsetstrokecolor{currentstroke}%
\pgfsetdash{}{0pt}%
\pgfpathmoveto{\pgfqpoint{3.346321in}{0.575369in}}%
\pgfpathlineto{\pgfqpoint{3.405774in}{0.575369in}}%
\pgfpathlineto{\pgfqpoint{3.465227in}{0.575369in}}%
\pgfpathlineto{\pgfqpoint{3.465227in}{2.953495in}}%
\pgfpathlineto{\pgfqpoint{3.405774in}{2.953495in}}%
\pgfpathlineto{\pgfqpoint{3.346321in}{2.953495in}}%
\pgfpathlineto{\pgfqpoint{3.346321in}{0.575369in}}%
\pgfpathclose%
\pgfusepath{stroke}%
\end{pgfscope}%
\end{pgfpicture}%
\makeatother%
\endgroup%

    \end{adjustbox}
    \caption{Correlation image \(\nu=0.01\).}\label{fig:sc2_ci_0.01}
  \end{subfigure}
  \begin{subfigure}{0.49\linewidth}
    \begin{adjustbox}{width=\linewidth}
      \begingroup%
\makeatletter%
\begin{pgfpicture}%
\pgfpathrectangle{\pgfpointorigin}{\pgfqpoint{4.000000in}{3.000000in}}%
\pgfusepath{use as bounding box, clip}%
\begin{pgfscope}%
\pgfsetbuttcap%
\pgfsetmiterjoin%
\pgfsetlinewidth{0.000000pt}%
\definecolor{currentstroke}{rgb}{0.000000,0.000000,0.000000}%
\pgfsetstrokecolor{currentstroke}%
\pgfsetstrokeopacity{0.000000}%
\pgfsetdash{}{0pt}%
\pgfpathmoveto{\pgfqpoint{0.000000in}{0.000000in}}%
\pgfpathlineto{\pgfqpoint{4.000000in}{0.000000in}}%
\pgfpathlineto{\pgfqpoint{4.000000in}{3.000000in}}%
\pgfpathlineto{\pgfqpoint{0.000000in}{3.000000in}}%
\pgfpathlineto{\pgfqpoint{0.000000in}{0.000000in}}%
\pgfpathclose%
\pgfusepath{}%
\end{pgfscope}%
\begin{pgfscope}%
\pgfsetbuttcap%
\pgfsetmiterjoin%
\pgfsetlinewidth{0.000000pt}%
\definecolor{currentstroke}{rgb}{0.000000,0.000000,0.000000}%
\pgfsetstrokecolor{currentstroke}%
\pgfsetstrokeopacity{0.000000}%
\pgfsetdash{}{0pt}%
\pgfpathmoveto{\pgfqpoint{2.123642in}{0.517039in}}%
\pgfpathlineto{\pgfqpoint{3.331504in}{0.517039in}}%
\pgfpathlineto{\pgfqpoint{3.331504in}{2.932762in}}%
\pgfpathlineto{\pgfqpoint{2.123642in}{2.932762in}}%
\pgfpathlineto{\pgfqpoint{2.123642in}{0.517039in}}%
\pgfpathclose%
\pgfusepath{}%
\end{pgfscope}%
\begin{pgfscope}%
\pgfpathrectangle{\pgfqpoint{2.123642in}{0.517039in}}{\pgfqpoint{1.207862in}{2.415723in}}%
\pgfusepath{clip}%
\pgfsys@transformcm{1.207862}{0.000000}{0.000000}{-2.415723}{2.123642in}{2.932762in}%
\pgftext[left,bottom]{\includegraphics[interpolate=false,width=1.000000in,height=1.000000in]{burgers_pm_0.01-img0.png}}%
\end{pgfscope}%
\begin{pgfscope}%
\pgfsetbuttcap%
\pgfsetroundjoin%
\definecolor{currentfill}{rgb}{0.000000,0.000000,0.000000}%
\pgfsetfillcolor{currentfill}%
\pgfsetlinewidth{0.803000pt}%
\definecolor{currentstroke}{rgb}{0.000000,0.000000,0.000000}%
\pgfsetstrokecolor{currentstroke}%
\pgfsetdash{}{0pt}%
\pgfsys@defobject{currentmarker}{\pgfqpoint{0.000000in}{-0.048611in}}{\pgfqpoint{0.000000in}{0.000000in}}{%
\pgfpathmoveto{\pgfqpoint{0.000000in}{0.000000in}}%
\pgfpathlineto{\pgfqpoint{0.000000in}{-0.048611in}}%
\pgfusepath{stroke,fill}%
}%
\begin{pgfscope}%
\pgfsys@transformshift{2.161388in}{0.517039in}%
\pgfsys@useobject{currentmarker}{}%
\end{pgfscope}%
\end{pgfscope}%
\begin{pgfscope}%
\definecolor{textcolor}{rgb}{0.000000,0.000000,0.000000}%
\pgfsetstrokecolor{textcolor}%
\pgfsetfillcolor{textcolor}%
\pgftext[x=2.161388in,y=0.419816in,,top]{\color{textcolor}{\rmfamily\fontsize{12.000000}{14.400000}\selectfont\catcode`\^=\active\def^{\ifmmode\sp\else\^{}\fi}\catcode`\%=\active\def%{\%}0}}%
\end{pgfscope}%
\begin{pgfscope}%
\pgfsetbuttcap%
\pgfsetroundjoin%
\definecolor{currentfill}{rgb}{0.000000,0.000000,0.000000}%
\pgfsetfillcolor{currentfill}%
\pgfsetlinewidth{0.803000pt}%
\definecolor{currentstroke}{rgb}{0.000000,0.000000,0.000000}%
\pgfsetstrokecolor{currentstroke}%
\pgfsetdash{}{0pt}%
\pgfsys@defobject{currentmarker}{\pgfqpoint{0.000000in}{-0.048611in}}{\pgfqpoint{0.000000in}{0.000000in}}{%
\pgfpathmoveto{\pgfqpoint{0.000000in}{0.000000in}}%
\pgfpathlineto{\pgfqpoint{0.000000in}{-0.048611in}}%
\pgfusepath{stroke,fill}%
}%
\begin{pgfscope}%
\pgfsys@transformshift{2.916302in}{0.517039in}%
\pgfsys@useobject{currentmarker}{}%
\end{pgfscope}%
\end{pgfscope}%
\begin{pgfscope}%
\definecolor{textcolor}{rgb}{0.000000,0.000000,0.000000}%
\pgfsetstrokecolor{textcolor}%
\pgfsetfillcolor{textcolor}%
\pgftext[x=2.916302in,y=0.419816in,,top]{\color{textcolor}{\rmfamily\fontsize{12.000000}{14.400000}\selectfont\catcode`\^=\active\def^{\ifmmode\sp\else\^{}\fi}\catcode`\%=\active\def%{\%}10}}%
\end{pgfscope}%
\begin{pgfscope}%
\definecolor{textcolor}{rgb}{0.000000,0.000000,0.000000}%
\pgfsetstrokecolor{textcolor}%
\pgfsetfillcolor{textcolor}%
\pgftext[x=2.727573in,y=0.202965in,,top]{\color{textcolor}{\rmfamily\fontsize{12.000000}{14.400000}\selectfont\catcode`\^=\active\def^{\ifmmode\sp\else\^{}\fi}\catcode`\%=\active\def%{\%}output coefficients}}%
\end{pgfscope}%
\begin{pgfscope}%
\pgfsetbuttcap%
\pgfsetroundjoin%
\definecolor{currentfill}{rgb}{0.000000,0.000000,0.000000}%
\pgfsetfillcolor{currentfill}%
\pgfsetlinewidth{0.803000pt}%
\definecolor{currentstroke}{rgb}{0.000000,0.000000,0.000000}%
\pgfsetstrokecolor{currentstroke}%
\pgfsetdash{}{0pt}%
\pgfsys@defobject{currentmarker}{\pgfqpoint{-0.048611in}{0.000000in}}{\pgfqpoint{-0.000000in}{0.000000in}}{%
\pgfpathmoveto{\pgfqpoint{-0.000000in}{0.000000in}}%
\pgfpathlineto{\pgfqpoint{-0.048611in}{0.000000in}}%
\pgfusepath{stroke,fill}%
}%
\begin{pgfscope}%
\pgfsys@transformshift{2.123642in}{2.895016in}%
\pgfsys@useobject{currentmarker}{}%
\end{pgfscope}%
\end{pgfscope}%
\begin{pgfscope}%
\definecolor{textcolor}{rgb}{0.000000,0.000000,0.000000}%
\pgfsetstrokecolor{textcolor}%
\pgfsetfillcolor{textcolor}%
\pgftext[x=1.920382in, y=2.831702in, left, base]{\color{textcolor}{\rmfamily\fontsize{12.000000}{14.400000}\selectfont\catcode`\^=\active\def^{\ifmmode\sp\else\^{}\fi}\catcode`\%=\active\def%{\%}0}}%
\end{pgfscope}%
\begin{pgfscope}%
\pgfsetbuttcap%
\pgfsetroundjoin%
\definecolor{currentfill}{rgb}{0.000000,0.000000,0.000000}%
\pgfsetfillcolor{currentfill}%
\pgfsetlinewidth{0.803000pt}%
\definecolor{currentstroke}{rgb}{0.000000,0.000000,0.000000}%
\pgfsetstrokecolor{currentstroke}%
\pgfsetdash{}{0pt}%
\pgfsys@defobject{currentmarker}{\pgfqpoint{-0.048611in}{0.000000in}}{\pgfqpoint{-0.000000in}{0.000000in}}{%
\pgfpathmoveto{\pgfqpoint{-0.000000in}{0.000000in}}%
\pgfpathlineto{\pgfqpoint{-0.048611in}{0.000000in}}%
\pgfusepath{stroke,fill}%
}%
\begin{pgfscope}%
\pgfsys@transformshift{2.123642in}{2.517559in}%
\pgfsys@useobject{currentmarker}{}%
\end{pgfscope}%
\end{pgfscope}%
\begin{pgfscope}%
\definecolor{textcolor}{rgb}{0.000000,0.000000,0.000000}%
\pgfsetstrokecolor{textcolor}%
\pgfsetfillcolor{textcolor}%
\pgftext[x=1.920382in, y=2.454246in, left, base]{\color{textcolor}{\rmfamily\fontsize{12.000000}{14.400000}\selectfont\catcode`\^=\active\def^{\ifmmode\sp\else\^{}\fi}\catcode`\%=\active\def%{\%}5}}%
\end{pgfscope}%
\begin{pgfscope}%
\pgfsetbuttcap%
\pgfsetroundjoin%
\definecolor{currentfill}{rgb}{0.000000,0.000000,0.000000}%
\pgfsetfillcolor{currentfill}%
\pgfsetlinewidth{0.803000pt}%
\definecolor{currentstroke}{rgb}{0.000000,0.000000,0.000000}%
\pgfsetstrokecolor{currentstroke}%
\pgfsetdash{}{0pt}%
\pgfsys@defobject{currentmarker}{\pgfqpoint{-0.048611in}{0.000000in}}{\pgfqpoint{-0.000000in}{0.000000in}}{%
\pgfpathmoveto{\pgfqpoint{-0.000000in}{0.000000in}}%
\pgfpathlineto{\pgfqpoint{-0.048611in}{0.000000in}}%
\pgfusepath{stroke,fill}%
}%
\begin{pgfscope}%
\pgfsys@transformshift{2.123642in}{2.140103in}%
\pgfsys@useobject{currentmarker}{}%
\end{pgfscope}%
\end{pgfscope}%
\begin{pgfscope}%
\definecolor{textcolor}{rgb}{0.000000,0.000000,0.000000}%
\pgfsetstrokecolor{textcolor}%
\pgfsetfillcolor{textcolor}%
\pgftext[x=1.814343in, y=2.076789in, left, base]{\color{textcolor}{\rmfamily\fontsize{12.000000}{14.400000}\selectfont\catcode`\^=\active\def^{\ifmmode\sp\else\^{}\fi}\catcode`\%=\active\def%{\%}10}}%
\end{pgfscope}%
\begin{pgfscope}%
\pgfsetbuttcap%
\pgfsetroundjoin%
\definecolor{currentfill}{rgb}{0.000000,0.000000,0.000000}%
\pgfsetfillcolor{currentfill}%
\pgfsetlinewidth{0.803000pt}%
\definecolor{currentstroke}{rgb}{0.000000,0.000000,0.000000}%
\pgfsetstrokecolor{currentstroke}%
\pgfsetdash{}{0pt}%
\pgfsys@defobject{currentmarker}{\pgfqpoint{-0.048611in}{0.000000in}}{\pgfqpoint{-0.000000in}{0.000000in}}{%
\pgfpathmoveto{\pgfqpoint{-0.000000in}{0.000000in}}%
\pgfpathlineto{\pgfqpoint{-0.048611in}{0.000000in}}%
\pgfusepath{stroke,fill}%
}%
\begin{pgfscope}%
\pgfsys@transformshift{2.123642in}{1.762646in}%
\pgfsys@useobject{currentmarker}{}%
\end{pgfscope}%
\end{pgfscope}%
\begin{pgfscope}%
\definecolor{textcolor}{rgb}{0.000000,0.000000,0.000000}%
\pgfsetstrokecolor{textcolor}%
\pgfsetfillcolor{textcolor}%
\pgftext[x=1.814343in, y=1.699332in, left, base]{\color{textcolor}{\rmfamily\fontsize{12.000000}{14.400000}\selectfont\catcode`\^=\active\def^{\ifmmode\sp\else\^{}\fi}\catcode`\%=\active\def%{\%}15}}%
\end{pgfscope}%
\begin{pgfscope}%
\pgfsetbuttcap%
\pgfsetroundjoin%
\definecolor{currentfill}{rgb}{0.000000,0.000000,0.000000}%
\pgfsetfillcolor{currentfill}%
\pgfsetlinewidth{0.803000pt}%
\definecolor{currentstroke}{rgb}{0.000000,0.000000,0.000000}%
\pgfsetstrokecolor{currentstroke}%
\pgfsetdash{}{0pt}%
\pgfsys@defobject{currentmarker}{\pgfqpoint{-0.048611in}{0.000000in}}{\pgfqpoint{-0.000000in}{0.000000in}}{%
\pgfpathmoveto{\pgfqpoint{-0.000000in}{0.000000in}}%
\pgfpathlineto{\pgfqpoint{-0.048611in}{0.000000in}}%
\pgfusepath{stroke,fill}%
}%
\begin{pgfscope}%
\pgfsys@transformshift{2.123642in}{1.385189in}%
\pgfsys@useobject{currentmarker}{}%
\end{pgfscope}%
\end{pgfscope}%
\begin{pgfscope}%
\definecolor{textcolor}{rgb}{0.000000,0.000000,0.000000}%
\pgfsetstrokecolor{textcolor}%
\pgfsetfillcolor{textcolor}%
\pgftext[x=1.814343in, y=1.321875in, left, base]{\color{textcolor}{\rmfamily\fontsize{12.000000}{14.400000}\selectfont\catcode`\^=\active\def^{\ifmmode\sp\else\^{}\fi}\catcode`\%=\active\def%{\%}20}}%
\end{pgfscope}%
\begin{pgfscope}%
\pgfsetbuttcap%
\pgfsetroundjoin%
\definecolor{currentfill}{rgb}{0.000000,0.000000,0.000000}%
\pgfsetfillcolor{currentfill}%
\pgfsetlinewidth{0.803000pt}%
\definecolor{currentstroke}{rgb}{0.000000,0.000000,0.000000}%
\pgfsetstrokecolor{currentstroke}%
\pgfsetdash{}{0pt}%
\pgfsys@defobject{currentmarker}{\pgfqpoint{-0.048611in}{0.000000in}}{\pgfqpoint{-0.000000in}{0.000000in}}{%
\pgfpathmoveto{\pgfqpoint{-0.000000in}{0.000000in}}%
\pgfpathlineto{\pgfqpoint{-0.048611in}{0.000000in}}%
\pgfusepath{stroke,fill}%
}%
\begin{pgfscope}%
\pgfsys@transformshift{2.123642in}{1.007732in}%
\pgfsys@useobject{currentmarker}{}%
\end{pgfscope}%
\end{pgfscope}%
\begin{pgfscope}%
\definecolor{textcolor}{rgb}{0.000000,0.000000,0.000000}%
\pgfsetstrokecolor{textcolor}%
\pgfsetfillcolor{textcolor}%
\pgftext[x=1.814343in, y=0.944419in, left, base]{\color{textcolor}{\rmfamily\fontsize{12.000000}{14.400000}\selectfont\catcode`\^=\active\def^{\ifmmode\sp\else\^{}\fi}\catcode`\%=\active\def%{\%}25}}%
\end{pgfscope}%
\begin{pgfscope}%
\pgfsetbuttcap%
\pgfsetroundjoin%
\definecolor{currentfill}{rgb}{0.000000,0.000000,0.000000}%
\pgfsetfillcolor{currentfill}%
\pgfsetlinewidth{0.803000pt}%
\definecolor{currentstroke}{rgb}{0.000000,0.000000,0.000000}%
\pgfsetstrokecolor{currentstroke}%
\pgfsetdash{}{0pt}%
\pgfsys@defobject{currentmarker}{\pgfqpoint{-0.048611in}{0.000000in}}{\pgfqpoint{-0.000000in}{0.000000in}}{%
\pgfpathmoveto{\pgfqpoint{-0.000000in}{0.000000in}}%
\pgfpathlineto{\pgfqpoint{-0.048611in}{0.000000in}}%
\pgfusepath{stroke,fill}%
}%
\begin{pgfscope}%
\pgfsys@transformshift{2.123642in}{0.630276in}%
\pgfsys@useobject{currentmarker}{}%
\end{pgfscope}%
\end{pgfscope}%
\begin{pgfscope}%
\definecolor{textcolor}{rgb}{0.000000,0.000000,0.000000}%
\pgfsetstrokecolor{textcolor}%
\pgfsetfillcolor{textcolor}%
\pgftext[x=1.814343in, y=0.566962in, left, base]{\color{textcolor}{\rmfamily\fontsize{12.000000}{14.400000}\selectfont\catcode`\^=\active\def^{\ifmmode\sp\else\^{}\fi}\catcode`\%=\active\def%{\%}30}}%
\end{pgfscope}%
\begin{pgfscope}%
\definecolor{textcolor}{rgb}{0.000000,0.000000,0.000000}%
\pgfsetstrokecolor{textcolor}%
\pgfsetfillcolor{textcolor}%
\pgftext[x=1.758788in,y=1.724900in,,bottom,rotate=90.000000]{\color{textcolor}{\rmfamily\fontsize{12.000000}{14.400000}\selectfont\catcode`\^=\active\def^{\ifmmode\sp\else\^{}\fi}\catcode`\%=\active\def%{\%}input coefficients}}%
\end{pgfscope}%
\begin{pgfscope}%
\pgfsetrectcap%
\pgfsetmiterjoin%
\pgfsetlinewidth{0.803000pt}%
\definecolor{currentstroke}{rgb}{0.000000,0.000000,0.000000}%
\pgfsetstrokecolor{currentstroke}%
\pgfsetdash{}{0pt}%
\pgfpathmoveto{\pgfqpoint{2.123642in}{0.517039in}}%
\pgfpathlineto{\pgfqpoint{2.123642in}{2.932762in}}%
\pgfusepath{stroke}%
\end{pgfscope}%
\begin{pgfscope}%
\pgfsetrectcap%
\pgfsetmiterjoin%
\pgfsetlinewidth{0.803000pt}%
\definecolor{currentstroke}{rgb}{0.000000,0.000000,0.000000}%
\pgfsetstrokecolor{currentstroke}%
\pgfsetdash{}{0pt}%
\pgfpathmoveto{\pgfqpoint{3.331504in}{0.517039in}}%
\pgfpathlineto{\pgfqpoint{3.331504in}{2.932762in}}%
\pgfusepath{stroke}%
\end{pgfscope}%
\begin{pgfscope}%
\pgfsetrectcap%
\pgfsetmiterjoin%
\pgfsetlinewidth{0.803000pt}%
\definecolor{currentstroke}{rgb}{0.000000,0.000000,0.000000}%
\pgfsetstrokecolor{currentstroke}%
\pgfsetdash{}{0pt}%
\pgfpathmoveto{\pgfqpoint{2.123642in}{0.517039in}}%
\pgfpathlineto{\pgfqpoint{3.331504in}{0.517039in}}%
\pgfusepath{stroke}%
\end{pgfscope}%
\begin{pgfscope}%
\pgfsetrectcap%
\pgfsetmiterjoin%
\pgfsetlinewidth{0.803000pt}%
\definecolor{currentstroke}{rgb}{0.000000,0.000000,0.000000}%
\pgfsetstrokecolor{currentstroke}%
\pgfsetdash{}{0pt}%
\pgfpathmoveto{\pgfqpoint{2.123642in}{2.932762in}}%
\pgfpathlineto{\pgfqpoint{3.331504in}{2.932762in}}%
\pgfusepath{stroke}%
\end{pgfscope}%
\begin{pgfscope}%
\pgfsetbuttcap%
\pgfsetmiterjoin%
\pgfsetlinewidth{0.000000pt}%
\definecolor{currentstroke}{rgb}{0.000000,0.000000,0.000000}%
\pgfsetstrokecolor{currentstroke}%
\pgfsetstrokeopacity{0.000000}%
\pgfsetdash{}{0pt}%
\pgfpathmoveto{\pgfqpoint{3.495996in}{0.517039in}}%
\pgfpathlineto{\pgfqpoint{3.616782in}{0.517039in}}%
\pgfpathlineto{\pgfqpoint{3.616782in}{2.932762in}}%
\pgfpathlineto{\pgfqpoint{3.495996in}{2.932762in}}%
\pgfpathlineto{\pgfqpoint{3.495996in}{0.517039in}}%
\pgfpathclose%
\pgfusepath{}%
\end{pgfscope}%
\begin{pgfscope}%
\pgfsys@transformshift{3.500000in}{0.520000in}%
\pgftext[left,bottom]{\includegraphics[interpolate=true,width=0.120000in,height=2.410000in]{burgers_pm_0.01-img1.png}}%
\end{pgfscope}%
\begin{pgfscope}%
\pgfsetbuttcap%
\pgfsetroundjoin%
\definecolor{currentfill}{rgb}{0.000000,0.000000,0.000000}%
\pgfsetfillcolor{currentfill}%
\pgfsetlinewidth{0.803000pt}%
\definecolor{currentstroke}{rgb}{0.000000,0.000000,0.000000}%
\pgfsetstrokecolor{currentstroke}%
\pgfsetdash{}{0pt}%
\pgfsys@defobject{currentmarker}{\pgfqpoint{0.000000in}{0.000000in}}{\pgfqpoint{0.048611in}{0.000000in}}{%
\pgfpathmoveto{\pgfqpoint{0.000000in}{0.000000in}}%
\pgfpathlineto{\pgfqpoint{0.048611in}{0.000000in}}%
\pgfusepath{stroke,fill}%
}%
\begin{pgfscope}%
\pgfsys@transformshift{3.616782in}{0.643972in}%
\pgfsys@useobject{currentmarker}{}%
\end{pgfscope}%
\end{pgfscope}%
\begin{pgfscope}%
\definecolor{textcolor}{rgb}{0.000000,0.000000,0.000000}%
\pgfsetstrokecolor{textcolor}%
\pgfsetfillcolor{textcolor}%
\pgftext[x=3.714004in, y=0.580658in, left, base]{\color{textcolor}{\rmfamily\fontsize{12.000000}{14.400000}\selectfont\catcode`\^=\active\def^{\ifmmode\sp\else\^{}\fi}\catcode`\%=\active\def%{\%}\ensuremath{-}2}}%
\end{pgfscope}%
\begin{pgfscope}%
\pgfsetbuttcap%
\pgfsetroundjoin%
\definecolor{currentfill}{rgb}{0.000000,0.000000,0.000000}%
\pgfsetfillcolor{currentfill}%
\pgfsetlinewidth{0.803000pt}%
\definecolor{currentstroke}{rgb}{0.000000,0.000000,0.000000}%
\pgfsetstrokecolor{currentstroke}%
\pgfsetdash{}{0pt}%
\pgfsys@defobject{currentmarker}{\pgfqpoint{0.000000in}{0.000000in}}{\pgfqpoint{0.048611in}{0.000000in}}{%
\pgfpathmoveto{\pgfqpoint{0.000000in}{0.000000in}}%
\pgfpathlineto{\pgfqpoint{0.048611in}{0.000000in}}%
\pgfusepath{stroke,fill}%
}%
\begin{pgfscope}%
\pgfsys@transformshift{3.616782in}{0.991417in}%
\pgfsys@useobject{currentmarker}{}%
\end{pgfscope}%
\end{pgfscope}%
\begin{pgfscope}%
\definecolor{textcolor}{rgb}{0.000000,0.000000,0.000000}%
\pgfsetstrokecolor{textcolor}%
\pgfsetfillcolor{textcolor}%
\pgftext[x=3.714004in, y=0.928103in, left, base]{\color{textcolor}{\rmfamily\fontsize{12.000000}{14.400000}\selectfont\catcode`\^=\active\def^{\ifmmode\sp\else\^{}\fi}\catcode`\%=\active\def%{\%}\ensuremath{-}1}}%
\end{pgfscope}%
\begin{pgfscope}%
\pgfsetbuttcap%
\pgfsetroundjoin%
\definecolor{currentfill}{rgb}{0.000000,0.000000,0.000000}%
\pgfsetfillcolor{currentfill}%
\pgfsetlinewidth{0.803000pt}%
\definecolor{currentstroke}{rgb}{0.000000,0.000000,0.000000}%
\pgfsetstrokecolor{currentstroke}%
\pgfsetdash{}{0pt}%
\pgfsys@defobject{currentmarker}{\pgfqpoint{0.000000in}{0.000000in}}{\pgfqpoint{0.048611in}{0.000000in}}{%
\pgfpathmoveto{\pgfqpoint{0.000000in}{0.000000in}}%
\pgfpathlineto{\pgfqpoint{0.048611in}{0.000000in}}%
\pgfusepath{stroke,fill}%
}%
\begin{pgfscope}%
\pgfsys@transformshift{3.616782in}{1.338862in}%
\pgfsys@useobject{currentmarker}{}%
\end{pgfscope}%
\end{pgfscope}%
\begin{pgfscope}%
\definecolor{textcolor}{rgb}{0.000000,0.000000,0.000000}%
\pgfsetstrokecolor{textcolor}%
\pgfsetfillcolor{textcolor}%
\pgftext[x=3.714004in, y=1.275548in, left, base]{\color{textcolor}{\rmfamily\fontsize{12.000000}{14.400000}\selectfont\catcode`\^=\active\def^{\ifmmode\sp\else\^{}\fi}\catcode`\%=\active\def%{\%}0}}%
\end{pgfscope}%
\begin{pgfscope}%
\pgfsetbuttcap%
\pgfsetroundjoin%
\definecolor{currentfill}{rgb}{0.000000,0.000000,0.000000}%
\pgfsetfillcolor{currentfill}%
\pgfsetlinewidth{0.803000pt}%
\definecolor{currentstroke}{rgb}{0.000000,0.000000,0.000000}%
\pgfsetstrokecolor{currentstroke}%
\pgfsetdash{}{0pt}%
\pgfsys@defobject{currentmarker}{\pgfqpoint{0.000000in}{0.000000in}}{\pgfqpoint{0.048611in}{0.000000in}}{%
\pgfpathmoveto{\pgfqpoint{0.000000in}{0.000000in}}%
\pgfpathlineto{\pgfqpoint{0.048611in}{0.000000in}}%
\pgfusepath{stroke,fill}%
}%
\begin{pgfscope}%
\pgfsys@transformshift{3.616782in}{1.686307in}%
\pgfsys@useobject{currentmarker}{}%
\end{pgfscope}%
\end{pgfscope}%
\begin{pgfscope}%
\definecolor{textcolor}{rgb}{0.000000,0.000000,0.000000}%
\pgfsetstrokecolor{textcolor}%
\pgfsetfillcolor{textcolor}%
\pgftext[x=3.714004in, y=1.622993in, left, base]{\color{textcolor}{\rmfamily\fontsize{12.000000}{14.400000}\selectfont\catcode`\^=\active\def^{\ifmmode\sp\else\^{}\fi}\catcode`\%=\active\def%{\%}1}}%
\end{pgfscope}%
\begin{pgfscope}%
\pgfsetbuttcap%
\pgfsetroundjoin%
\definecolor{currentfill}{rgb}{0.000000,0.000000,0.000000}%
\pgfsetfillcolor{currentfill}%
\pgfsetlinewidth{0.803000pt}%
\definecolor{currentstroke}{rgb}{0.000000,0.000000,0.000000}%
\pgfsetstrokecolor{currentstroke}%
\pgfsetdash{}{0pt}%
\pgfsys@defobject{currentmarker}{\pgfqpoint{0.000000in}{0.000000in}}{\pgfqpoint{0.048611in}{0.000000in}}{%
\pgfpathmoveto{\pgfqpoint{0.000000in}{0.000000in}}%
\pgfpathlineto{\pgfqpoint{0.048611in}{0.000000in}}%
\pgfusepath{stroke,fill}%
}%
\begin{pgfscope}%
\pgfsys@transformshift{3.616782in}{2.033752in}%
\pgfsys@useobject{currentmarker}{}%
\end{pgfscope}%
\end{pgfscope}%
\begin{pgfscope}%
\definecolor{textcolor}{rgb}{0.000000,0.000000,0.000000}%
\pgfsetstrokecolor{textcolor}%
\pgfsetfillcolor{textcolor}%
\pgftext[x=3.714004in, y=1.970438in, left, base]{\color{textcolor}{\rmfamily\fontsize{12.000000}{14.400000}\selectfont\catcode`\^=\active\def^{\ifmmode\sp\else\^{}\fi}\catcode`\%=\active\def%{\%}2}}%
\end{pgfscope}%
\begin{pgfscope}%
\pgfsetbuttcap%
\pgfsetroundjoin%
\definecolor{currentfill}{rgb}{0.000000,0.000000,0.000000}%
\pgfsetfillcolor{currentfill}%
\pgfsetlinewidth{0.803000pt}%
\definecolor{currentstroke}{rgb}{0.000000,0.000000,0.000000}%
\pgfsetstrokecolor{currentstroke}%
\pgfsetdash{}{0pt}%
\pgfsys@defobject{currentmarker}{\pgfqpoint{0.000000in}{0.000000in}}{\pgfqpoint{0.048611in}{0.000000in}}{%
\pgfpathmoveto{\pgfqpoint{0.000000in}{0.000000in}}%
\pgfpathlineto{\pgfqpoint{0.048611in}{0.000000in}}%
\pgfusepath{stroke,fill}%
}%
\begin{pgfscope}%
\pgfsys@transformshift{3.616782in}{2.381197in}%
\pgfsys@useobject{currentmarker}{}%
\end{pgfscope}%
\end{pgfscope}%
\begin{pgfscope}%
\definecolor{textcolor}{rgb}{0.000000,0.000000,0.000000}%
\pgfsetstrokecolor{textcolor}%
\pgfsetfillcolor{textcolor}%
\pgftext[x=3.714004in, y=2.317883in, left, base]{\color{textcolor}{\rmfamily\fontsize{12.000000}{14.400000}\selectfont\catcode`\^=\active\def^{\ifmmode\sp\else\^{}\fi}\catcode`\%=\active\def%{\%}3}}%
\end{pgfscope}%
\begin{pgfscope}%
\pgfsetbuttcap%
\pgfsetroundjoin%
\definecolor{currentfill}{rgb}{0.000000,0.000000,0.000000}%
\pgfsetfillcolor{currentfill}%
\pgfsetlinewidth{0.803000pt}%
\definecolor{currentstroke}{rgb}{0.000000,0.000000,0.000000}%
\pgfsetstrokecolor{currentstroke}%
\pgfsetdash{}{0pt}%
\pgfsys@defobject{currentmarker}{\pgfqpoint{0.000000in}{0.000000in}}{\pgfqpoint{0.048611in}{0.000000in}}{%
\pgfpathmoveto{\pgfqpoint{0.000000in}{0.000000in}}%
\pgfpathlineto{\pgfqpoint{0.048611in}{0.000000in}}%
\pgfusepath{stroke,fill}%
}%
\begin{pgfscope}%
\pgfsys@transformshift{3.616782in}{2.728642in}%
\pgfsys@useobject{currentmarker}{}%
\end{pgfscope}%
\end{pgfscope}%
\begin{pgfscope}%
\definecolor{textcolor}{rgb}{0.000000,0.000000,0.000000}%
\pgfsetstrokecolor{textcolor}%
\pgfsetfillcolor{textcolor}%
\pgftext[x=3.714004in, y=2.665328in, left, base]{\color{textcolor}{\rmfamily\fontsize{12.000000}{14.400000}\selectfont\catcode`\^=\active\def^{\ifmmode\sp\else\^{}\fi}\catcode`\%=\active\def%{\%}4}}%
\end{pgfscope}%
\begin{pgfscope}%
\pgfsetrectcap%
\pgfsetmiterjoin%
\pgfsetlinewidth{0.803000pt}%
\definecolor{currentstroke}{rgb}{0.000000,0.000000,0.000000}%
\pgfsetstrokecolor{currentstroke}%
\pgfsetdash{}{0pt}%
\pgfpathmoveto{\pgfqpoint{3.495996in}{0.517039in}}%
\pgfpathlineto{\pgfqpoint{3.556389in}{0.517039in}}%
\pgfpathlineto{\pgfqpoint{3.616782in}{0.517039in}}%
\pgfpathlineto{\pgfqpoint{3.616782in}{2.932762in}}%
\pgfpathlineto{\pgfqpoint{3.556389in}{2.932762in}}%
\pgfpathlineto{\pgfqpoint{3.495996in}{2.932762in}}%
\pgfpathlineto{\pgfqpoint{3.495996in}{0.517039in}}%
\pgfpathclose%
\pgfusepath{stroke}%
\end{pgfscope}%
\end{pgfpicture}%
\makeatother%
\endgroup%

    \end{adjustbox}
    \caption{The p-matrix for \(\nu=0.01\).}\label{fig:sc2_pm_0.01}
  \end{subfigure}
  % \\[-0.7\baselineskip]
  \begin{subfigure}{0.49\linewidth}
    \begin{adjustbox}{width=\linewidth}
      \begingroup%
\makeatletter%
\begin{pgfpicture}%
\pgfpathrectangle{\pgfpointorigin}{\pgfqpoint{4.000000in}{3.000000in}}%
\pgfusepath{use as bounding box, clip}%
\begin{pgfscope}%
\pgfsetbuttcap%
\pgfsetmiterjoin%
\pgfsetlinewidth{0.000000pt}%
\definecolor{currentstroke}{rgb}{0.000000,0.000000,0.000000}%
\pgfsetstrokecolor{currentstroke}%
\pgfsetstrokeopacity{0.000000}%
\pgfsetdash{}{0pt}%
\pgfpathmoveto{\pgfqpoint{0.000000in}{0.000000in}}%
\pgfpathlineto{\pgfqpoint{4.000000in}{0.000000in}}%
\pgfpathlineto{\pgfqpoint{4.000000in}{3.000000in}}%
\pgfpathlineto{\pgfqpoint{0.000000in}{3.000000in}}%
\pgfpathlineto{\pgfqpoint{0.000000in}{0.000000in}}%
\pgfpathclose%
\pgfusepath{}%
\end{pgfscope}%
\begin{pgfscope}%
\pgfsetbuttcap%
\pgfsetmiterjoin%
\pgfsetlinewidth{0.000000pt}%
\definecolor{currentstroke}{rgb}{0.000000,0.000000,0.000000}%
\pgfsetstrokecolor{currentstroke}%
\pgfsetstrokeopacity{0.000000}%
\pgfsetdash{}{0pt}%
\pgfpathmoveto{\pgfqpoint{0.779897in}{0.517039in}}%
\pgfpathlineto{\pgfqpoint{3.062697in}{0.517039in}}%
\pgfpathlineto{\pgfqpoint{3.062697in}{2.895016in}}%
\pgfpathlineto{\pgfqpoint{0.779897in}{2.895016in}}%
\pgfpathlineto{\pgfqpoint{0.779897in}{0.517039in}}%
\pgfpathclose%
\pgfusepath{}%
\end{pgfscope}%
\begin{pgfscope}%
\pgfpathrectangle{\pgfqpoint{0.779897in}{0.517039in}}{\pgfqpoint{2.282800in}{2.377978in}}%
\pgfusepath{clip}%
\pgfsys@transformcm{2.282800}{0.000000}{0.000000}{-2.377978}{0.779897in}{2.895016in}%
\pgftext[left,bottom]{\includegraphics[interpolate=false,width=1.000000in,height=1.000000in]{burgers_ci_0.1-img0.png}}%
\end{pgfscope}%
\begin{pgfscope}%
\pgfsetbuttcap%
\pgfsetroundjoin%
\definecolor{currentfill}{rgb}{0.000000,0.000000,0.000000}%
\pgfsetfillcolor{currentfill}%
\pgfsetlinewidth{0.803000pt}%
\definecolor{currentstroke}{rgb}{0.000000,0.000000,0.000000}%
\pgfsetstrokecolor{currentstroke}%
\pgfsetdash{}{0pt}%
\pgfsys@defobject{currentmarker}{\pgfqpoint{0.000000in}{-0.048611in}}{\pgfqpoint{0.000000in}{0.000000in}}{%
\pgfpathmoveto{\pgfqpoint{0.000000in}{0.000000in}}%
\pgfpathlineto{\pgfqpoint{0.000000in}{-0.048611in}}%
\pgfusepath{stroke,fill}%
}%
\begin{pgfscope}%
\pgfsys@transformshift{0.779897in}{0.517039in}%
\pgfsys@useobject{currentmarker}{}%
\end{pgfscope}%
\end{pgfscope}%
\begin{pgfscope}%
\definecolor{textcolor}{rgb}{0.000000,0.000000,0.000000}%
\pgfsetstrokecolor{textcolor}%
\pgfsetfillcolor{textcolor}%
\pgftext[x=0.779897in,y=0.419816in,,top]{\color{textcolor}{\rmfamily\fontsize{12.000000}{14.400000}\selectfont\catcode`\^=\active\def^{\ifmmode\sp\else\^{}\fi}\catcode`\%=\active\def%{\%}0}}%
\end{pgfscope}%
\begin{pgfscope}%
\pgfsetbuttcap%
\pgfsetroundjoin%
\definecolor{currentfill}{rgb}{0.000000,0.000000,0.000000}%
\pgfsetfillcolor{currentfill}%
\pgfsetlinewidth{0.803000pt}%
\definecolor{currentstroke}{rgb}{0.000000,0.000000,0.000000}%
\pgfsetstrokecolor{currentstroke}%
\pgfsetdash{}{0pt}%
\pgfsys@defobject{currentmarker}{\pgfqpoint{0.000000in}{-0.048611in}}{\pgfqpoint{0.000000in}{0.000000in}}{%
\pgfpathmoveto{\pgfqpoint{0.000000in}{0.000000in}}%
\pgfpathlineto{\pgfqpoint{0.000000in}{-0.048611in}}%
\pgfusepath{stroke,fill}%
}%
\begin{pgfscope}%
\pgfsys@transformshift{1.693017in}{0.517039in}%
\pgfsys@useobject{currentmarker}{}%
\end{pgfscope}%
\end{pgfscope}%
\begin{pgfscope}%
\definecolor{textcolor}{rgb}{0.000000,0.000000,0.000000}%
\pgfsetstrokecolor{textcolor}%
\pgfsetfillcolor{textcolor}%
\pgftext[x=1.693017in,y=0.419816in,,top]{\color{textcolor}{\rmfamily\fontsize{12.000000}{14.400000}\selectfont\catcode`\^=\active\def^{\ifmmode\sp\else\^{}\fi}\catcode`\%=\active\def%{\%}10}}%
\end{pgfscope}%
\begin{pgfscope}%
\pgfsetbuttcap%
\pgfsetroundjoin%
\definecolor{currentfill}{rgb}{0.000000,0.000000,0.000000}%
\pgfsetfillcolor{currentfill}%
\pgfsetlinewidth{0.803000pt}%
\definecolor{currentstroke}{rgb}{0.000000,0.000000,0.000000}%
\pgfsetstrokecolor{currentstroke}%
\pgfsetdash{}{0pt}%
\pgfsys@defobject{currentmarker}{\pgfqpoint{0.000000in}{-0.048611in}}{\pgfqpoint{0.000000in}{0.000000in}}{%
\pgfpathmoveto{\pgfqpoint{0.000000in}{0.000000in}}%
\pgfpathlineto{\pgfqpoint{0.000000in}{-0.048611in}}%
\pgfusepath{stroke,fill}%
}%
\begin{pgfscope}%
\pgfsys@transformshift{2.606137in}{0.517039in}%
\pgfsys@useobject{currentmarker}{}%
\end{pgfscope}%
\end{pgfscope}%
\begin{pgfscope}%
\definecolor{textcolor}{rgb}{0.000000,0.000000,0.000000}%
\pgfsetstrokecolor{textcolor}%
\pgfsetfillcolor{textcolor}%
\pgftext[x=2.606137in,y=0.419816in,,top]{\color{textcolor}{\rmfamily\fontsize{12.000000}{14.400000}\selectfont\catcode`\^=\active\def^{\ifmmode\sp\else\^{}\fi}\catcode`\%=\active\def%{\%}20}}%
\end{pgfscope}%
\begin{pgfscope}%
\definecolor{textcolor}{rgb}{0.000000,0.000000,0.000000}%
\pgfsetstrokecolor{textcolor}%
\pgfsetfillcolor{textcolor}%
\pgftext[x=1.921297in,y=0.202965in,,top]{\color{textcolor}{\rmfamily\fontsize{12.000000}{14.400000}\selectfont\catcode`\^=\active\def^{\ifmmode\sp\else\^{}\fi}\catcode`\%=\active\def%{\%}input coefficients}}%
\end{pgfscope}%
\begin{pgfscope}%
\pgfsetbuttcap%
\pgfsetroundjoin%
\definecolor{currentfill}{rgb}{0.000000,0.000000,0.000000}%
\pgfsetfillcolor{currentfill}%
\pgfsetlinewidth{0.803000pt}%
\definecolor{currentstroke}{rgb}{0.000000,0.000000,0.000000}%
\pgfsetstrokecolor{currentstroke}%
\pgfsetdash{}{0pt}%
\pgfsys@defobject{currentmarker}{\pgfqpoint{-0.048611in}{0.000000in}}{\pgfqpoint{-0.000000in}{0.000000in}}{%
\pgfpathmoveto{\pgfqpoint{-0.000000in}{0.000000in}}%
\pgfpathlineto{\pgfqpoint{-0.048611in}{0.000000in}}%
\pgfusepath{stroke,fill}%
}%
\begin{pgfscope}%
\pgfsys@transformshift{0.779897in}{2.895016in}%
\pgfsys@useobject{currentmarker}{}%
\end{pgfscope}%
\end{pgfscope}%
\begin{pgfscope}%
\definecolor{textcolor}{rgb}{0.000000,0.000000,0.000000}%
\pgfsetstrokecolor{textcolor}%
\pgfsetfillcolor{textcolor}%
\pgftext[x=0.576636in, y=2.831702in, left, base]{\color{textcolor}{\rmfamily\fontsize{12.000000}{14.400000}\selectfont\catcode`\^=\active\def^{\ifmmode\sp\else\^{}\fi}\catcode`\%=\active\def%{\%}0}}%
\end{pgfscope}%
\begin{pgfscope}%
\pgfsetbuttcap%
\pgfsetroundjoin%
\definecolor{currentfill}{rgb}{0.000000,0.000000,0.000000}%
\pgfsetfillcolor{currentfill}%
\pgfsetlinewidth{0.803000pt}%
\definecolor{currentstroke}{rgb}{0.000000,0.000000,0.000000}%
\pgfsetstrokecolor{currentstroke}%
\pgfsetdash{}{0pt}%
\pgfsys@defobject{currentmarker}{\pgfqpoint{-0.048611in}{0.000000in}}{\pgfqpoint{-0.000000in}{0.000000in}}{%
\pgfpathmoveto{\pgfqpoint{-0.000000in}{0.000000in}}%
\pgfpathlineto{\pgfqpoint{-0.048611in}{0.000000in}}%
\pgfusepath{stroke,fill}%
}%
\begin{pgfscope}%
\pgfsys@transformshift{0.779897in}{2.521590in}%
\pgfsys@useobject{currentmarker}{}%
\end{pgfscope}%
\end{pgfscope}%
\begin{pgfscope}%
\definecolor{textcolor}{rgb}{0.000000,0.000000,0.000000}%
\pgfsetstrokecolor{textcolor}%
\pgfsetfillcolor{textcolor}%
\pgftext[x=0.258521in, y=2.458276in, left, base]{\color{textcolor}{\rmfamily\fontsize{12.000000}{14.400000}\selectfont\catcode`\^=\active\def^{\ifmmode\sp\else\^{}\fi}\catcode`\%=\active\def%{\%}1000}}%
\end{pgfscope}%
\begin{pgfscope}%
\pgfsetbuttcap%
\pgfsetroundjoin%
\definecolor{currentfill}{rgb}{0.000000,0.000000,0.000000}%
\pgfsetfillcolor{currentfill}%
\pgfsetlinewidth{0.803000pt}%
\definecolor{currentstroke}{rgb}{0.000000,0.000000,0.000000}%
\pgfsetstrokecolor{currentstroke}%
\pgfsetdash{}{0pt}%
\pgfsys@defobject{currentmarker}{\pgfqpoint{-0.048611in}{0.000000in}}{\pgfqpoint{-0.000000in}{0.000000in}}{%
\pgfpathmoveto{\pgfqpoint{-0.000000in}{0.000000in}}%
\pgfpathlineto{\pgfqpoint{-0.048611in}{0.000000in}}%
\pgfusepath{stroke,fill}%
}%
\begin{pgfscope}%
\pgfsys@transformshift{0.779897in}{2.148164in}%
\pgfsys@useobject{currentmarker}{}%
\end{pgfscope}%
\end{pgfscope}%
\begin{pgfscope}%
\definecolor{textcolor}{rgb}{0.000000,0.000000,0.000000}%
\pgfsetstrokecolor{textcolor}%
\pgfsetfillcolor{textcolor}%
\pgftext[x=0.258521in, y=2.084850in, left, base]{\color{textcolor}{\rmfamily\fontsize{12.000000}{14.400000}\selectfont\catcode`\^=\active\def^{\ifmmode\sp\else\^{}\fi}\catcode`\%=\active\def%{\%}2000}}%
\end{pgfscope}%
\begin{pgfscope}%
\pgfsetbuttcap%
\pgfsetroundjoin%
\definecolor{currentfill}{rgb}{0.000000,0.000000,0.000000}%
\pgfsetfillcolor{currentfill}%
\pgfsetlinewidth{0.803000pt}%
\definecolor{currentstroke}{rgb}{0.000000,0.000000,0.000000}%
\pgfsetstrokecolor{currentstroke}%
\pgfsetdash{}{0pt}%
\pgfsys@defobject{currentmarker}{\pgfqpoint{-0.048611in}{0.000000in}}{\pgfqpoint{-0.000000in}{0.000000in}}{%
\pgfpathmoveto{\pgfqpoint{-0.000000in}{0.000000in}}%
\pgfpathlineto{\pgfqpoint{-0.048611in}{0.000000in}}%
\pgfusepath{stroke,fill}%
}%
\begin{pgfscope}%
\pgfsys@transformshift{0.779897in}{1.774738in}%
\pgfsys@useobject{currentmarker}{}%
\end{pgfscope}%
\end{pgfscope}%
\begin{pgfscope}%
\definecolor{textcolor}{rgb}{0.000000,0.000000,0.000000}%
\pgfsetstrokecolor{textcolor}%
\pgfsetfillcolor{textcolor}%
\pgftext[x=0.258521in, y=1.711424in, left, base]{\color{textcolor}{\rmfamily\fontsize{12.000000}{14.400000}\selectfont\catcode`\^=\active\def^{\ifmmode\sp\else\^{}\fi}\catcode`\%=\active\def%{\%}3000}}%
\end{pgfscope}%
\begin{pgfscope}%
\pgfsetbuttcap%
\pgfsetroundjoin%
\definecolor{currentfill}{rgb}{0.000000,0.000000,0.000000}%
\pgfsetfillcolor{currentfill}%
\pgfsetlinewidth{0.803000pt}%
\definecolor{currentstroke}{rgb}{0.000000,0.000000,0.000000}%
\pgfsetstrokecolor{currentstroke}%
\pgfsetdash{}{0pt}%
\pgfsys@defobject{currentmarker}{\pgfqpoint{-0.048611in}{0.000000in}}{\pgfqpoint{-0.000000in}{0.000000in}}{%
\pgfpathmoveto{\pgfqpoint{-0.000000in}{0.000000in}}%
\pgfpathlineto{\pgfqpoint{-0.048611in}{0.000000in}}%
\pgfusepath{stroke,fill}%
}%
\begin{pgfscope}%
\pgfsys@transformshift{0.779897in}{1.401312in}%
\pgfsys@useobject{currentmarker}{}%
\end{pgfscope}%
\end{pgfscope}%
\begin{pgfscope}%
\definecolor{textcolor}{rgb}{0.000000,0.000000,0.000000}%
\pgfsetstrokecolor{textcolor}%
\pgfsetfillcolor{textcolor}%
\pgftext[x=0.258521in, y=1.337998in, left, base]{\color{textcolor}{\rmfamily\fontsize{12.000000}{14.400000}\selectfont\catcode`\^=\active\def^{\ifmmode\sp\else\^{}\fi}\catcode`\%=\active\def%{\%}4000}}%
\end{pgfscope}%
\begin{pgfscope}%
\pgfsetbuttcap%
\pgfsetroundjoin%
\definecolor{currentfill}{rgb}{0.000000,0.000000,0.000000}%
\pgfsetfillcolor{currentfill}%
\pgfsetlinewidth{0.803000pt}%
\definecolor{currentstroke}{rgb}{0.000000,0.000000,0.000000}%
\pgfsetstrokecolor{currentstroke}%
\pgfsetdash{}{0pt}%
\pgfsys@defobject{currentmarker}{\pgfqpoint{-0.048611in}{0.000000in}}{\pgfqpoint{-0.000000in}{0.000000in}}{%
\pgfpathmoveto{\pgfqpoint{-0.000000in}{0.000000in}}%
\pgfpathlineto{\pgfqpoint{-0.048611in}{0.000000in}}%
\pgfusepath{stroke,fill}%
}%
\begin{pgfscope}%
\pgfsys@transformshift{0.779897in}{1.027886in}%
\pgfsys@useobject{currentmarker}{}%
\end{pgfscope}%
\end{pgfscope}%
\begin{pgfscope}%
\definecolor{textcolor}{rgb}{0.000000,0.000000,0.000000}%
\pgfsetstrokecolor{textcolor}%
\pgfsetfillcolor{textcolor}%
\pgftext[x=0.258521in, y=0.964572in, left, base]{\color{textcolor}{\rmfamily\fontsize{12.000000}{14.400000}\selectfont\catcode`\^=\active\def^{\ifmmode\sp\else\^{}\fi}\catcode`\%=\active\def%{\%}5000}}%
\end{pgfscope}%
\begin{pgfscope}%
\pgfsetbuttcap%
\pgfsetroundjoin%
\definecolor{currentfill}{rgb}{0.000000,0.000000,0.000000}%
\pgfsetfillcolor{currentfill}%
\pgfsetlinewidth{0.803000pt}%
\definecolor{currentstroke}{rgb}{0.000000,0.000000,0.000000}%
\pgfsetstrokecolor{currentstroke}%
\pgfsetdash{}{0pt}%
\pgfsys@defobject{currentmarker}{\pgfqpoint{-0.048611in}{0.000000in}}{\pgfqpoint{-0.000000in}{0.000000in}}{%
\pgfpathmoveto{\pgfqpoint{-0.000000in}{0.000000in}}%
\pgfpathlineto{\pgfqpoint{-0.048611in}{0.000000in}}%
\pgfusepath{stroke,fill}%
}%
\begin{pgfscope}%
\pgfsys@transformshift{0.779897in}{0.654459in}%
\pgfsys@useobject{currentmarker}{}%
\end{pgfscope}%
\end{pgfscope}%
\begin{pgfscope}%
\definecolor{textcolor}{rgb}{0.000000,0.000000,0.000000}%
\pgfsetstrokecolor{textcolor}%
\pgfsetfillcolor{textcolor}%
\pgftext[x=0.258521in, y=0.591146in, left, base]{\color{textcolor}{\rmfamily\fontsize{12.000000}{14.400000}\selectfont\catcode`\^=\active\def^{\ifmmode\sp\else\^{}\fi}\catcode`\%=\active\def%{\%}6000}}%
\end{pgfscope}%
\begin{pgfscope}%
\definecolor{textcolor}{rgb}{0.000000,0.000000,0.000000}%
\pgfsetstrokecolor{textcolor}%
\pgfsetfillcolor{textcolor}%
\pgftext[x=0.202965in,y=1.706027in,,bottom,rotate=90.000000]{\color{textcolor}{\rmfamily\fontsize{12.000000}{14.400000}\selectfont\catcode`\^=\active\def^{\ifmmode\sp\else\^{}\fi}\catcode`\%=\active\def%{\%}samples}}%
\end{pgfscope}%
\begin{pgfscope}%
\pgfsetrectcap%
\pgfsetmiterjoin%
\pgfsetlinewidth{0.803000pt}%
\definecolor{currentstroke}{rgb}{0.000000,0.000000,0.000000}%
\pgfsetstrokecolor{currentstroke}%
\pgfsetdash{}{0pt}%
\pgfpathmoveto{\pgfqpoint{0.779897in}{0.517039in}}%
\pgfpathlineto{\pgfqpoint{0.779897in}{2.895016in}}%
\pgfusepath{stroke}%
\end{pgfscope}%
\begin{pgfscope}%
\pgfsetrectcap%
\pgfsetmiterjoin%
\pgfsetlinewidth{0.803000pt}%
\definecolor{currentstroke}{rgb}{0.000000,0.000000,0.000000}%
\pgfsetstrokecolor{currentstroke}%
\pgfsetdash{}{0pt}%
\pgfpathmoveto{\pgfqpoint{3.062697in}{0.517039in}}%
\pgfpathlineto{\pgfqpoint{3.062697in}{2.895016in}}%
\pgfusepath{stroke}%
\end{pgfscope}%
\begin{pgfscope}%
\pgfsetrectcap%
\pgfsetmiterjoin%
\pgfsetlinewidth{0.803000pt}%
\definecolor{currentstroke}{rgb}{0.000000,0.000000,0.000000}%
\pgfsetstrokecolor{currentstroke}%
\pgfsetdash{}{0pt}%
\pgfpathmoveto{\pgfqpoint{0.779897in}{0.517039in}}%
\pgfpathlineto{\pgfqpoint{3.062697in}{0.517039in}}%
\pgfusepath{stroke}%
\end{pgfscope}%
\begin{pgfscope}%
\pgfsetrectcap%
\pgfsetmiterjoin%
\pgfsetlinewidth{0.803000pt}%
\definecolor{currentstroke}{rgb}{0.000000,0.000000,0.000000}%
\pgfsetstrokecolor{currentstroke}%
\pgfsetdash{}{0pt}%
\pgfpathmoveto{\pgfqpoint{0.779897in}{2.895016in}}%
\pgfpathlineto{\pgfqpoint{3.062697in}{2.895016in}}%
\pgfusepath{stroke}%
\end{pgfscope}%
\begin{pgfscope}%
\pgfsetbuttcap%
\pgfsetmiterjoin%
\pgfsetlinewidth{0.000000pt}%
\definecolor{currentstroke}{rgb}{0.000000,0.000000,0.000000}%
\pgfsetstrokecolor{currentstroke}%
\pgfsetstrokeopacity{0.000000}%
\pgfsetdash{}{0pt}%
\pgfpathmoveto{\pgfqpoint{3.282875in}{0.517039in}}%
\pgfpathlineto{\pgfqpoint{3.401774in}{0.517039in}}%
\pgfpathlineto{\pgfqpoint{3.401774in}{2.895016in}}%
\pgfpathlineto{\pgfqpoint{3.282875in}{2.895016in}}%
\pgfpathlineto{\pgfqpoint{3.282875in}{0.517039in}}%
\pgfpathclose%
\pgfusepath{}%
\end{pgfscope}%
\begin{pgfscope}%
\pgfsys@transformshift{3.280000in}{0.520000in}%
\pgftext[left,bottom]{\includegraphics[interpolate=true,width=0.120000in,height=2.380000in]{burgers_ci_0.1-img1.png}}%
\end{pgfscope}%
\begin{pgfscope}%
\pgfsetbuttcap%
\pgfsetroundjoin%
\definecolor{currentfill}{rgb}{0.000000,0.000000,0.000000}%
\pgfsetfillcolor{currentfill}%
\pgfsetlinewidth{0.803000pt}%
\definecolor{currentstroke}{rgb}{0.000000,0.000000,0.000000}%
\pgfsetstrokecolor{currentstroke}%
\pgfsetdash{}{0pt}%
\pgfsys@defobject{currentmarker}{\pgfqpoint{0.000000in}{0.000000in}}{\pgfqpoint{0.048611in}{0.000000in}}{%
\pgfpathmoveto{\pgfqpoint{0.000000in}{0.000000in}}%
\pgfpathlineto{\pgfqpoint{0.048611in}{0.000000in}}%
\pgfusepath{stroke,fill}%
}%
\begin{pgfscope}%
\pgfsys@transformshift{3.401774in}{0.561629in}%
\pgfsys@useobject{currentmarker}{}%
\end{pgfscope}%
\end{pgfscope}%
\begin{pgfscope}%
\definecolor{textcolor}{rgb}{0.000000,0.000000,0.000000}%
\pgfsetstrokecolor{textcolor}%
\pgfsetfillcolor{textcolor}%
\pgftext[x=3.498996in, y=0.498315in, left, base]{\color{textcolor}{\rmfamily\fontsize{12.000000}{14.400000}\selectfont\catcode`\^=\active\def^{\ifmmode\sp\else\^{}\fi}\catcode`\%=\active\def%{\%}\ensuremath{-}750}}%
\end{pgfscope}%
\begin{pgfscope}%
\pgfsetbuttcap%
\pgfsetroundjoin%
\definecolor{currentfill}{rgb}{0.000000,0.000000,0.000000}%
\pgfsetfillcolor{currentfill}%
\pgfsetlinewidth{0.803000pt}%
\definecolor{currentstroke}{rgb}{0.000000,0.000000,0.000000}%
\pgfsetstrokecolor{currentstroke}%
\pgfsetdash{}{0pt}%
\pgfsys@defobject{currentmarker}{\pgfqpoint{0.000000in}{0.000000in}}{\pgfqpoint{0.048611in}{0.000000in}}{%
\pgfpathmoveto{\pgfqpoint{0.000000in}{0.000000in}}%
\pgfpathlineto{\pgfqpoint{0.048611in}{0.000000in}}%
\pgfusepath{stroke,fill}%
}%
\begin{pgfscope}%
\pgfsys@transformshift{3.401774in}{0.946916in}%
\pgfsys@useobject{currentmarker}{}%
\end{pgfscope}%
\end{pgfscope}%
\begin{pgfscope}%
\definecolor{textcolor}{rgb}{0.000000,0.000000,0.000000}%
\pgfsetstrokecolor{textcolor}%
\pgfsetfillcolor{textcolor}%
\pgftext[x=3.498996in, y=0.883603in, left, base]{\color{textcolor}{\rmfamily\fontsize{12.000000}{14.400000}\selectfont\catcode`\^=\active\def^{\ifmmode\sp\else\^{}\fi}\catcode`\%=\active\def%{\%}\ensuremath{-}500}}%
\end{pgfscope}%
\begin{pgfscope}%
\pgfsetbuttcap%
\pgfsetroundjoin%
\definecolor{currentfill}{rgb}{0.000000,0.000000,0.000000}%
\pgfsetfillcolor{currentfill}%
\pgfsetlinewidth{0.803000pt}%
\definecolor{currentstroke}{rgb}{0.000000,0.000000,0.000000}%
\pgfsetstrokecolor{currentstroke}%
\pgfsetdash{}{0pt}%
\pgfsys@defobject{currentmarker}{\pgfqpoint{0.000000in}{0.000000in}}{\pgfqpoint{0.048611in}{0.000000in}}{%
\pgfpathmoveto{\pgfqpoint{0.000000in}{0.000000in}}%
\pgfpathlineto{\pgfqpoint{0.048611in}{0.000000in}}%
\pgfusepath{stroke,fill}%
}%
\begin{pgfscope}%
\pgfsys@transformshift{3.401774in}{1.332204in}%
\pgfsys@useobject{currentmarker}{}%
\end{pgfscope}%
\end{pgfscope}%
\begin{pgfscope}%
\definecolor{textcolor}{rgb}{0.000000,0.000000,0.000000}%
\pgfsetstrokecolor{textcolor}%
\pgfsetfillcolor{textcolor}%
\pgftext[x=3.498996in, y=1.268890in, left, base]{\color{textcolor}{\rmfamily\fontsize{12.000000}{14.400000}\selectfont\catcode`\^=\active\def^{\ifmmode\sp\else\^{}\fi}\catcode`\%=\active\def%{\%}\ensuremath{-}250}}%
\end{pgfscope}%
\begin{pgfscope}%
\pgfsetbuttcap%
\pgfsetroundjoin%
\definecolor{currentfill}{rgb}{0.000000,0.000000,0.000000}%
\pgfsetfillcolor{currentfill}%
\pgfsetlinewidth{0.803000pt}%
\definecolor{currentstroke}{rgb}{0.000000,0.000000,0.000000}%
\pgfsetstrokecolor{currentstroke}%
\pgfsetdash{}{0pt}%
\pgfsys@defobject{currentmarker}{\pgfqpoint{0.000000in}{0.000000in}}{\pgfqpoint{0.048611in}{0.000000in}}{%
\pgfpathmoveto{\pgfqpoint{0.000000in}{0.000000in}}%
\pgfpathlineto{\pgfqpoint{0.048611in}{0.000000in}}%
\pgfusepath{stroke,fill}%
}%
\begin{pgfscope}%
\pgfsys@transformshift{3.401774in}{1.717492in}%
\pgfsys@useobject{currentmarker}{}%
\end{pgfscope}%
\end{pgfscope}%
\begin{pgfscope}%
\definecolor{textcolor}{rgb}{0.000000,0.000000,0.000000}%
\pgfsetstrokecolor{textcolor}%
\pgfsetfillcolor{textcolor}%
\pgftext[x=3.498996in, y=1.654178in, left, base]{\color{textcolor}{\rmfamily\fontsize{12.000000}{14.400000}\selectfont\catcode`\^=\active\def^{\ifmmode\sp\else\^{}\fi}\catcode`\%=\active\def%{\%}0}}%
\end{pgfscope}%
\begin{pgfscope}%
\pgfsetbuttcap%
\pgfsetroundjoin%
\definecolor{currentfill}{rgb}{0.000000,0.000000,0.000000}%
\pgfsetfillcolor{currentfill}%
\pgfsetlinewidth{0.803000pt}%
\definecolor{currentstroke}{rgb}{0.000000,0.000000,0.000000}%
\pgfsetstrokecolor{currentstroke}%
\pgfsetdash{}{0pt}%
\pgfsys@defobject{currentmarker}{\pgfqpoint{0.000000in}{0.000000in}}{\pgfqpoint{0.048611in}{0.000000in}}{%
\pgfpathmoveto{\pgfqpoint{0.000000in}{0.000000in}}%
\pgfpathlineto{\pgfqpoint{0.048611in}{0.000000in}}%
\pgfusepath{stroke,fill}%
}%
\begin{pgfscope}%
\pgfsys@transformshift{3.401774in}{2.102780in}%
\pgfsys@useobject{currentmarker}{}%
\end{pgfscope}%
\end{pgfscope}%
\begin{pgfscope}%
\definecolor{textcolor}{rgb}{0.000000,0.000000,0.000000}%
\pgfsetstrokecolor{textcolor}%
\pgfsetfillcolor{textcolor}%
\pgftext[x=3.498996in, y=2.039466in, left, base]{\color{textcolor}{\rmfamily\fontsize{12.000000}{14.400000}\selectfont\catcode`\^=\active\def^{\ifmmode\sp\else\^{}\fi}\catcode`\%=\active\def%{\%}250}}%
\end{pgfscope}%
\begin{pgfscope}%
\pgfsetbuttcap%
\pgfsetroundjoin%
\definecolor{currentfill}{rgb}{0.000000,0.000000,0.000000}%
\pgfsetfillcolor{currentfill}%
\pgfsetlinewidth{0.803000pt}%
\definecolor{currentstroke}{rgb}{0.000000,0.000000,0.000000}%
\pgfsetstrokecolor{currentstroke}%
\pgfsetdash{}{0pt}%
\pgfsys@defobject{currentmarker}{\pgfqpoint{0.000000in}{0.000000in}}{\pgfqpoint{0.048611in}{0.000000in}}{%
\pgfpathmoveto{\pgfqpoint{0.000000in}{0.000000in}}%
\pgfpathlineto{\pgfqpoint{0.048611in}{0.000000in}}%
\pgfusepath{stroke,fill}%
}%
\begin{pgfscope}%
\pgfsys@transformshift{3.401774in}{2.488067in}%
\pgfsys@useobject{currentmarker}{}%
\end{pgfscope}%
\end{pgfscope}%
\begin{pgfscope}%
\definecolor{textcolor}{rgb}{0.000000,0.000000,0.000000}%
\pgfsetstrokecolor{textcolor}%
\pgfsetfillcolor{textcolor}%
\pgftext[x=3.498996in, y=2.424754in, left, base]{\color{textcolor}{\rmfamily\fontsize{12.000000}{14.400000}\selectfont\catcode`\^=\active\def^{\ifmmode\sp\else\^{}\fi}\catcode`\%=\active\def%{\%}500}}%
\end{pgfscope}%
\begin{pgfscope}%
\pgfsetbuttcap%
\pgfsetroundjoin%
\definecolor{currentfill}{rgb}{0.000000,0.000000,0.000000}%
\pgfsetfillcolor{currentfill}%
\pgfsetlinewidth{0.803000pt}%
\definecolor{currentstroke}{rgb}{0.000000,0.000000,0.000000}%
\pgfsetstrokecolor{currentstroke}%
\pgfsetdash{}{0pt}%
\pgfsys@defobject{currentmarker}{\pgfqpoint{0.000000in}{0.000000in}}{\pgfqpoint{0.048611in}{0.000000in}}{%
\pgfpathmoveto{\pgfqpoint{0.000000in}{0.000000in}}%
\pgfpathlineto{\pgfqpoint{0.048611in}{0.000000in}}%
\pgfusepath{stroke,fill}%
}%
\begin{pgfscope}%
\pgfsys@transformshift{3.401774in}{2.873355in}%
\pgfsys@useobject{currentmarker}{}%
\end{pgfscope}%
\end{pgfscope}%
\begin{pgfscope}%
\definecolor{textcolor}{rgb}{0.000000,0.000000,0.000000}%
\pgfsetstrokecolor{textcolor}%
\pgfsetfillcolor{textcolor}%
\pgftext[x=3.498996in, y=2.810041in, left, base]{\color{textcolor}{\rmfamily\fontsize{12.000000}{14.400000}\selectfont\catcode`\^=\active\def^{\ifmmode\sp\else\^{}\fi}\catcode`\%=\active\def%{\%}750}}%
\end{pgfscope}%
\begin{pgfscope}%
\pgfsetrectcap%
\pgfsetmiterjoin%
\pgfsetlinewidth{0.803000pt}%
\definecolor{currentstroke}{rgb}{0.000000,0.000000,0.000000}%
\pgfsetstrokecolor{currentstroke}%
\pgfsetdash{}{0pt}%
\pgfpathmoveto{\pgfqpoint{3.282875in}{0.517039in}}%
\pgfpathlineto{\pgfqpoint{3.342325in}{0.517039in}}%
\pgfpathlineto{\pgfqpoint{3.401774in}{0.517039in}}%
\pgfpathlineto{\pgfqpoint{3.401774in}{2.895016in}}%
\pgfpathlineto{\pgfqpoint{3.342325in}{2.895016in}}%
\pgfpathlineto{\pgfqpoint{3.282875in}{2.895016in}}%
\pgfpathlineto{\pgfqpoint{3.282875in}{0.517039in}}%
\pgfpathclose%
\pgfusepath{stroke}%
\end{pgfscope}%
\end{pgfpicture}%
\makeatother%
\endgroup%

    \end{adjustbox}
    \caption{Correlation image \(\nu=0.1\).}\label{fig:sc2_ci_0.1}
  \end{subfigure}
  \begin{subfigure}{0.49\linewidth}
    \begin{adjustbox}{width=\linewidth}
      \begingroup%
\makeatletter%
\begin{pgfpicture}%
\pgfpathrectangle{\pgfpointorigin}{\pgfqpoint{2.552157in}{3.116635in}}%
\pgfusepath{use as bounding box, clip}%
\begin{pgfscope}%
\pgfsetbuttcap%
\pgfsetmiterjoin%
\pgfsetlinewidth{0.000000pt}%
\definecolor{currentstroke}{rgb}{0.000000,0.000000,0.000000}%
\pgfsetstrokecolor{currentstroke}%
\pgfsetstrokeopacity{0.000000}%
\pgfsetdash{}{0pt}%
\pgfpathmoveto{\pgfqpoint{0.000000in}{0.000000in}}%
\pgfpathlineto{\pgfqpoint{2.552157in}{0.000000in}}%
\pgfpathlineto{\pgfqpoint{2.552157in}{3.116635in}}%
\pgfpathlineto{\pgfqpoint{0.000000in}{3.116635in}}%
\pgfpathlineto{\pgfqpoint{0.000000in}{0.000000in}}%
\pgfpathclose%
\pgfusepath{}%
\end{pgfscope}%
\begin{pgfscope}%
\pgfsetbuttcap%
\pgfsetmiterjoin%
\pgfsetlinewidth{0.000000pt}%
\definecolor{currentstroke}{rgb}{0.000000,0.000000,0.000000}%
\pgfsetstrokecolor{currentstroke}%
\pgfsetstrokeopacity{0.000000}%
\pgfsetdash{}{0pt}%
\pgfpathmoveto{\pgfqpoint{0.626150in}{0.575369in}}%
\pgfpathlineto{\pgfqpoint{1.833999in}{0.575369in}}%
\pgfpathlineto{\pgfqpoint{1.833999in}{2.991066in}}%
\pgfpathlineto{\pgfqpoint{0.626150in}{2.991066in}}%
\pgfpathlineto{\pgfqpoint{0.626150in}{0.575369in}}%
\pgfpathclose%
\pgfusepath{}%
\end{pgfscope}%
\begin{pgfscope}%
\pgfpathrectangle{\pgfqpoint{0.626150in}{0.575369in}}{\pgfqpoint{1.207849in}{2.415697in}}%
\pgfusepath{clip}%
\pgfsys@transformcm{1.207849}{0.000000}{0.000000}{-2.415697}{0.626150in}{2.991066in}%
\pgftext[left,bottom]{\includegraphics[interpolate=false,width=1.000000in,height=1.000000in]{burgers_pm_0.1-img0.png}}%
\end{pgfscope}%
\begin{pgfscope}%
\pgfsetbuttcap%
\pgfsetroundjoin%
\definecolor{currentfill}{rgb}{0.000000,0.000000,0.000000}%
\pgfsetfillcolor{currentfill}%
\pgfsetlinewidth{0.803000pt}%
\definecolor{currentstroke}{rgb}{0.000000,0.000000,0.000000}%
\pgfsetstrokecolor{currentstroke}%
\pgfsetdash{}{0pt}%
\pgfsys@defobject{currentmarker}{\pgfqpoint{0.000000in}{-0.048611in}}{\pgfqpoint{0.000000in}{0.000000in}}{%
\pgfpathmoveto{\pgfqpoint{0.000000in}{0.000000in}}%
\pgfpathlineto{\pgfqpoint{0.000000in}{-0.048611in}}%
\pgfusepath{stroke,fill}%
}%
\begin{pgfscope}%
\pgfsys@transformshift{0.663895in}{0.575369in}%
\pgfsys@useobject{currentmarker}{}%
\end{pgfscope}%
\end{pgfscope}%
\begin{pgfscope}%
\definecolor{textcolor}{rgb}{0.000000,0.000000,0.000000}%
\pgfsetstrokecolor{textcolor}%
\pgfsetfillcolor{textcolor}%
\pgftext[x=0.663895in,y=0.478146in,,top]{\color{textcolor}\rmfamily\fontsize{12.000000}{14.400000}\selectfont 0}%
\end{pgfscope}%
\begin{pgfscope}%
\pgfsetbuttcap%
\pgfsetroundjoin%
\definecolor{currentfill}{rgb}{0.000000,0.000000,0.000000}%
\pgfsetfillcolor{currentfill}%
\pgfsetlinewidth{0.803000pt}%
\definecolor{currentstroke}{rgb}{0.000000,0.000000,0.000000}%
\pgfsetstrokecolor{currentstroke}%
\pgfsetdash{}{0pt}%
\pgfsys@defobject{currentmarker}{\pgfqpoint{0.000000in}{-0.048611in}}{\pgfqpoint{0.000000in}{0.000000in}}{%
\pgfpathmoveto{\pgfqpoint{0.000000in}{0.000000in}}%
\pgfpathlineto{\pgfqpoint{0.000000in}{-0.048611in}}%
\pgfusepath{stroke,fill}%
}%
\begin{pgfscope}%
\pgfsys@transformshift{1.418801in}{0.575369in}%
\pgfsys@useobject{currentmarker}{}%
\end{pgfscope}%
\end{pgfscope}%
\begin{pgfscope}%
\definecolor{textcolor}{rgb}{0.000000,0.000000,0.000000}%
\pgfsetstrokecolor{textcolor}%
\pgfsetfillcolor{textcolor}%
\pgftext[x=1.418801in,y=0.478146in,,top]{\color{textcolor}\rmfamily\fontsize{12.000000}{14.400000}\selectfont 10}%
\end{pgfscope}%
\begin{pgfscope}%
\definecolor{textcolor}{rgb}{0.000000,0.000000,0.000000}%
\pgfsetstrokecolor{textcolor}%
\pgfsetfillcolor{textcolor}%
\pgftext[x=1.230074in,y=0.261295in,,top]{\color{textcolor}\rmfamily\fontsize{12.000000}{14.400000}\selectfont output coefficients}%
\end{pgfscope}%
\begin{pgfscope}%
\pgfsetbuttcap%
\pgfsetroundjoin%
\definecolor{currentfill}{rgb}{0.000000,0.000000,0.000000}%
\pgfsetfillcolor{currentfill}%
\pgfsetlinewidth{0.803000pt}%
\definecolor{currentstroke}{rgb}{0.000000,0.000000,0.000000}%
\pgfsetstrokecolor{currentstroke}%
\pgfsetdash{}{0pt}%
\pgfsys@defobject{currentmarker}{\pgfqpoint{-0.048611in}{0.000000in}}{\pgfqpoint{-0.000000in}{0.000000in}}{%
\pgfpathmoveto{\pgfqpoint{-0.000000in}{0.000000in}}%
\pgfpathlineto{\pgfqpoint{-0.048611in}{0.000000in}}%
\pgfusepath{stroke,fill}%
}%
\begin{pgfscope}%
\pgfsys@transformshift{0.626150in}{2.953321in}%
\pgfsys@useobject{currentmarker}{}%
\end{pgfscope}%
\end{pgfscope}%
\begin{pgfscope}%
\definecolor{textcolor}{rgb}{0.000000,0.000000,0.000000}%
\pgfsetstrokecolor{textcolor}%
\pgfsetfillcolor{textcolor}%
\pgftext[x=0.422889in, y=2.890007in, left, base]{\color{textcolor}\rmfamily\fontsize{12.000000}{14.400000}\selectfont 0}%
\end{pgfscope}%
\begin{pgfscope}%
\pgfsetbuttcap%
\pgfsetroundjoin%
\definecolor{currentfill}{rgb}{0.000000,0.000000,0.000000}%
\pgfsetfillcolor{currentfill}%
\pgfsetlinewidth{0.803000pt}%
\definecolor{currentstroke}{rgb}{0.000000,0.000000,0.000000}%
\pgfsetstrokecolor{currentstroke}%
\pgfsetdash{}{0pt}%
\pgfsys@defobject{currentmarker}{\pgfqpoint{-0.048611in}{0.000000in}}{\pgfqpoint{-0.000000in}{0.000000in}}{%
\pgfpathmoveto{\pgfqpoint{-0.000000in}{0.000000in}}%
\pgfpathlineto{\pgfqpoint{-0.048611in}{0.000000in}}%
\pgfusepath{stroke,fill}%
}%
\begin{pgfscope}%
\pgfsys@transformshift{0.626150in}{2.575868in}%
\pgfsys@useobject{currentmarker}{}%
\end{pgfscope}%
\end{pgfscope}%
\begin{pgfscope}%
\definecolor{textcolor}{rgb}{0.000000,0.000000,0.000000}%
\pgfsetstrokecolor{textcolor}%
\pgfsetfillcolor{textcolor}%
\pgftext[x=0.422889in, y=2.512554in, left, base]{\color{textcolor}\rmfamily\fontsize{12.000000}{14.400000}\selectfont 5}%
\end{pgfscope}%
\begin{pgfscope}%
\pgfsetbuttcap%
\pgfsetroundjoin%
\definecolor{currentfill}{rgb}{0.000000,0.000000,0.000000}%
\pgfsetfillcolor{currentfill}%
\pgfsetlinewidth{0.803000pt}%
\definecolor{currentstroke}{rgb}{0.000000,0.000000,0.000000}%
\pgfsetstrokecolor{currentstroke}%
\pgfsetdash{}{0pt}%
\pgfsys@defobject{currentmarker}{\pgfqpoint{-0.048611in}{0.000000in}}{\pgfqpoint{-0.000000in}{0.000000in}}{%
\pgfpathmoveto{\pgfqpoint{-0.000000in}{0.000000in}}%
\pgfpathlineto{\pgfqpoint{-0.048611in}{0.000000in}}%
\pgfusepath{stroke,fill}%
}%
\begin{pgfscope}%
\pgfsys@transformshift{0.626150in}{2.198415in}%
\pgfsys@useobject{currentmarker}{}%
\end{pgfscope}%
\end{pgfscope}%
\begin{pgfscope}%
\definecolor{textcolor}{rgb}{0.000000,0.000000,0.000000}%
\pgfsetstrokecolor{textcolor}%
\pgfsetfillcolor{textcolor}%
\pgftext[x=0.316851in, y=2.135102in, left, base]{\color{textcolor}\rmfamily\fontsize{12.000000}{14.400000}\selectfont 10}%
\end{pgfscope}%
\begin{pgfscope}%
\pgfsetbuttcap%
\pgfsetroundjoin%
\definecolor{currentfill}{rgb}{0.000000,0.000000,0.000000}%
\pgfsetfillcolor{currentfill}%
\pgfsetlinewidth{0.803000pt}%
\definecolor{currentstroke}{rgb}{0.000000,0.000000,0.000000}%
\pgfsetstrokecolor{currentstroke}%
\pgfsetdash{}{0pt}%
\pgfsys@defobject{currentmarker}{\pgfqpoint{-0.048611in}{0.000000in}}{\pgfqpoint{-0.000000in}{0.000000in}}{%
\pgfpathmoveto{\pgfqpoint{-0.000000in}{0.000000in}}%
\pgfpathlineto{\pgfqpoint{-0.048611in}{0.000000in}}%
\pgfusepath{stroke,fill}%
}%
\begin{pgfscope}%
\pgfsys@transformshift{0.626150in}{1.820963in}%
\pgfsys@useobject{currentmarker}{}%
\end{pgfscope}%
\end{pgfscope}%
\begin{pgfscope}%
\definecolor{textcolor}{rgb}{0.000000,0.000000,0.000000}%
\pgfsetstrokecolor{textcolor}%
\pgfsetfillcolor{textcolor}%
\pgftext[x=0.316851in, y=1.757649in, left, base]{\color{textcolor}\rmfamily\fontsize{12.000000}{14.400000}\selectfont 15}%
\end{pgfscope}%
\begin{pgfscope}%
\pgfsetbuttcap%
\pgfsetroundjoin%
\definecolor{currentfill}{rgb}{0.000000,0.000000,0.000000}%
\pgfsetfillcolor{currentfill}%
\pgfsetlinewidth{0.803000pt}%
\definecolor{currentstroke}{rgb}{0.000000,0.000000,0.000000}%
\pgfsetstrokecolor{currentstroke}%
\pgfsetdash{}{0pt}%
\pgfsys@defobject{currentmarker}{\pgfqpoint{-0.048611in}{0.000000in}}{\pgfqpoint{-0.000000in}{0.000000in}}{%
\pgfpathmoveto{\pgfqpoint{-0.000000in}{0.000000in}}%
\pgfpathlineto{\pgfqpoint{-0.048611in}{0.000000in}}%
\pgfusepath{stroke,fill}%
}%
\begin{pgfscope}%
\pgfsys@transformshift{0.626150in}{1.443510in}%
\pgfsys@useobject{currentmarker}{}%
\end{pgfscope}%
\end{pgfscope}%
\begin{pgfscope}%
\definecolor{textcolor}{rgb}{0.000000,0.000000,0.000000}%
\pgfsetstrokecolor{textcolor}%
\pgfsetfillcolor{textcolor}%
\pgftext[x=0.316851in, y=1.380196in, left, base]{\color{textcolor}\rmfamily\fontsize{12.000000}{14.400000}\selectfont 20}%
\end{pgfscope}%
\begin{pgfscope}%
\pgfsetbuttcap%
\pgfsetroundjoin%
\definecolor{currentfill}{rgb}{0.000000,0.000000,0.000000}%
\pgfsetfillcolor{currentfill}%
\pgfsetlinewidth{0.803000pt}%
\definecolor{currentstroke}{rgb}{0.000000,0.000000,0.000000}%
\pgfsetstrokecolor{currentstroke}%
\pgfsetdash{}{0pt}%
\pgfsys@defobject{currentmarker}{\pgfqpoint{-0.048611in}{0.000000in}}{\pgfqpoint{-0.000000in}{0.000000in}}{%
\pgfpathmoveto{\pgfqpoint{-0.000000in}{0.000000in}}%
\pgfpathlineto{\pgfqpoint{-0.048611in}{0.000000in}}%
\pgfusepath{stroke,fill}%
}%
\begin{pgfscope}%
\pgfsys@transformshift{0.626150in}{1.066057in}%
\pgfsys@useobject{currentmarker}{}%
\end{pgfscope}%
\end{pgfscope}%
\begin{pgfscope}%
\definecolor{textcolor}{rgb}{0.000000,0.000000,0.000000}%
\pgfsetstrokecolor{textcolor}%
\pgfsetfillcolor{textcolor}%
\pgftext[x=0.316851in, y=1.002743in, left, base]{\color{textcolor}\rmfamily\fontsize{12.000000}{14.400000}\selectfont 25}%
\end{pgfscope}%
\begin{pgfscope}%
\pgfsetbuttcap%
\pgfsetroundjoin%
\definecolor{currentfill}{rgb}{0.000000,0.000000,0.000000}%
\pgfsetfillcolor{currentfill}%
\pgfsetlinewidth{0.803000pt}%
\definecolor{currentstroke}{rgb}{0.000000,0.000000,0.000000}%
\pgfsetstrokecolor{currentstroke}%
\pgfsetdash{}{0pt}%
\pgfsys@defobject{currentmarker}{\pgfqpoint{-0.048611in}{0.000000in}}{\pgfqpoint{-0.000000in}{0.000000in}}{%
\pgfpathmoveto{\pgfqpoint{-0.000000in}{0.000000in}}%
\pgfpathlineto{\pgfqpoint{-0.048611in}{0.000000in}}%
\pgfusepath{stroke,fill}%
}%
\begin{pgfscope}%
\pgfsys@transformshift{0.626150in}{0.688604in}%
\pgfsys@useobject{currentmarker}{}%
\end{pgfscope}%
\end{pgfscope}%
\begin{pgfscope}%
\definecolor{textcolor}{rgb}{0.000000,0.000000,0.000000}%
\pgfsetstrokecolor{textcolor}%
\pgfsetfillcolor{textcolor}%
\pgftext[x=0.316851in, y=0.625291in, left, base]{\color{textcolor}\rmfamily\fontsize{12.000000}{14.400000}\selectfont 30}%
\end{pgfscope}%
\begin{pgfscope}%
\definecolor{textcolor}{rgb}{0.000000,0.000000,0.000000}%
\pgfsetstrokecolor{textcolor}%
\pgfsetfillcolor{textcolor}%
\pgftext[x=0.261295in,y=1.783217in,,bottom,rotate=90.000000]{\color{textcolor}\rmfamily\fontsize{12.000000}{14.400000}\selectfont input coefficients}%
\end{pgfscope}%
\begin{pgfscope}%
\pgfsetrectcap%
\pgfsetmiterjoin%
\pgfsetlinewidth{0.803000pt}%
\definecolor{currentstroke}{rgb}{0.000000,0.000000,0.000000}%
\pgfsetstrokecolor{currentstroke}%
\pgfsetdash{}{0pt}%
\pgfpathmoveto{\pgfqpoint{0.626150in}{0.575369in}}%
\pgfpathlineto{\pgfqpoint{0.626150in}{2.991066in}}%
\pgfusepath{stroke}%
\end{pgfscope}%
\begin{pgfscope}%
\pgfsetrectcap%
\pgfsetmiterjoin%
\pgfsetlinewidth{0.803000pt}%
\definecolor{currentstroke}{rgb}{0.000000,0.000000,0.000000}%
\pgfsetstrokecolor{currentstroke}%
\pgfsetdash{}{0pt}%
\pgfpathmoveto{\pgfqpoint{1.833999in}{0.575369in}}%
\pgfpathlineto{\pgfqpoint{1.833999in}{2.991066in}}%
\pgfusepath{stroke}%
\end{pgfscope}%
\begin{pgfscope}%
\pgfsetrectcap%
\pgfsetmiterjoin%
\pgfsetlinewidth{0.803000pt}%
\definecolor{currentstroke}{rgb}{0.000000,0.000000,0.000000}%
\pgfsetstrokecolor{currentstroke}%
\pgfsetdash{}{0pt}%
\pgfpathmoveto{\pgfqpoint{0.626150in}{0.575369in}}%
\pgfpathlineto{\pgfqpoint{1.833999in}{0.575369in}}%
\pgfusepath{stroke}%
\end{pgfscope}%
\begin{pgfscope}%
\pgfsetrectcap%
\pgfsetmiterjoin%
\pgfsetlinewidth{0.803000pt}%
\definecolor{currentstroke}{rgb}{0.000000,0.000000,0.000000}%
\pgfsetstrokecolor{currentstroke}%
\pgfsetdash{}{0pt}%
\pgfpathmoveto{\pgfqpoint{0.626150in}{2.991066in}}%
\pgfpathlineto{\pgfqpoint{1.833999in}{2.991066in}}%
\pgfusepath{stroke}%
\end{pgfscope}%
\begin{pgfscope}%
\pgfsetbuttcap%
\pgfsetmiterjoin%
\pgfsetlinewidth{0.000000pt}%
\definecolor{currentstroke}{rgb}{0.000000,0.000000,0.000000}%
\pgfsetstrokecolor{currentstroke}%
\pgfsetstrokeopacity{0.000000}%
\pgfsetdash{}{0pt}%
\pgfpathmoveto{\pgfqpoint{1.998481in}{0.575369in}}%
\pgfpathlineto{\pgfqpoint{2.119266in}{0.575369in}}%
\pgfpathlineto{\pgfqpoint{2.119266in}{2.991066in}}%
\pgfpathlineto{\pgfqpoint{1.998481in}{2.991066in}}%
\pgfpathlineto{\pgfqpoint{1.998481in}{0.575369in}}%
\pgfpathclose%
\pgfusepath{}%
\end{pgfscope}%
\begin{pgfscope}%
\pgfsys@transformshift{2.000000in}{0.586635in}%
\pgftext[left,bottom]{\includegraphics[interpolate=true,width=0.120000in,height=2.410000in]{burgers_pm_0.1-img1.png}}%
\end{pgfscope}%
\begin{pgfscope}%
\pgfsetbuttcap%
\pgfsetroundjoin%
\definecolor{currentfill}{rgb}{0.000000,0.000000,0.000000}%
\pgfsetfillcolor{currentfill}%
\pgfsetlinewidth{0.803000pt}%
\definecolor{currentstroke}{rgb}{0.000000,0.000000,0.000000}%
\pgfsetstrokecolor{currentstroke}%
\pgfsetdash{}{0pt}%
\pgfsys@defobject{currentmarker}{\pgfqpoint{0.000000in}{0.000000in}}{\pgfqpoint{0.048611in}{0.000000in}}{%
\pgfpathmoveto{\pgfqpoint{0.000000in}{0.000000in}}%
\pgfpathlineto{\pgfqpoint{0.048611in}{0.000000in}}%
\pgfusepath{stroke,fill}%
}%
\begin{pgfscope}%
\pgfsys@transformshift{2.119266in}{0.657532in}%
\pgfsys@useobject{currentmarker}{}%
\end{pgfscope}%
\end{pgfscope}%
\begin{pgfscope}%
\definecolor{textcolor}{rgb}{0.000000,0.000000,0.000000}%
\pgfsetstrokecolor{textcolor}%
\pgfsetfillcolor{textcolor}%
\pgftext[x=2.216488in, y=0.594218in, left, base]{\color{textcolor}\rmfamily\fontsize{12.000000}{14.400000}\selectfont \ensuremath{-}4}%
\end{pgfscope}%
\begin{pgfscope}%
\pgfsetbuttcap%
\pgfsetroundjoin%
\definecolor{currentfill}{rgb}{0.000000,0.000000,0.000000}%
\pgfsetfillcolor{currentfill}%
\pgfsetlinewidth{0.803000pt}%
\definecolor{currentstroke}{rgb}{0.000000,0.000000,0.000000}%
\pgfsetstrokecolor{currentstroke}%
\pgfsetdash{}{0pt}%
\pgfsys@defobject{currentmarker}{\pgfqpoint{0.000000in}{0.000000in}}{\pgfqpoint{0.048611in}{0.000000in}}{%
\pgfpathmoveto{\pgfqpoint{0.000000in}{0.000000in}}%
\pgfpathlineto{\pgfqpoint{0.048611in}{0.000000in}}%
\pgfusepath{stroke,fill}%
}%
\begin{pgfscope}%
\pgfsys@transformshift{2.119266in}{1.220375in}%
\pgfsys@useobject{currentmarker}{}%
\end{pgfscope}%
\end{pgfscope}%
\begin{pgfscope}%
\definecolor{textcolor}{rgb}{0.000000,0.000000,0.000000}%
\pgfsetstrokecolor{textcolor}%
\pgfsetfillcolor{textcolor}%
\pgftext[x=2.216488in, y=1.157061in, left, base]{\color{textcolor}\rmfamily\fontsize{12.000000}{14.400000}\selectfont \ensuremath{-}2}%
\end{pgfscope}%
\begin{pgfscope}%
\pgfsetbuttcap%
\pgfsetroundjoin%
\definecolor{currentfill}{rgb}{0.000000,0.000000,0.000000}%
\pgfsetfillcolor{currentfill}%
\pgfsetlinewidth{0.803000pt}%
\definecolor{currentstroke}{rgb}{0.000000,0.000000,0.000000}%
\pgfsetstrokecolor{currentstroke}%
\pgfsetdash{}{0pt}%
\pgfsys@defobject{currentmarker}{\pgfqpoint{0.000000in}{0.000000in}}{\pgfqpoint{0.048611in}{0.000000in}}{%
\pgfpathmoveto{\pgfqpoint{0.000000in}{0.000000in}}%
\pgfpathlineto{\pgfqpoint{0.048611in}{0.000000in}}%
\pgfusepath{stroke,fill}%
}%
\begin{pgfscope}%
\pgfsys@transformshift{2.119266in}{1.783217in}%
\pgfsys@useobject{currentmarker}{}%
\end{pgfscope}%
\end{pgfscope}%
\begin{pgfscope}%
\definecolor{textcolor}{rgb}{0.000000,0.000000,0.000000}%
\pgfsetstrokecolor{textcolor}%
\pgfsetfillcolor{textcolor}%
\pgftext[x=2.216488in, y=1.719904in, left, base]{\color{textcolor}\rmfamily\fontsize{12.000000}{14.400000}\selectfont 0}%
\end{pgfscope}%
\begin{pgfscope}%
\pgfsetbuttcap%
\pgfsetroundjoin%
\definecolor{currentfill}{rgb}{0.000000,0.000000,0.000000}%
\pgfsetfillcolor{currentfill}%
\pgfsetlinewidth{0.803000pt}%
\definecolor{currentstroke}{rgb}{0.000000,0.000000,0.000000}%
\pgfsetstrokecolor{currentstroke}%
\pgfsetdash{}{0pt}%
\pgfsys@defobject{currentmarker}{\pgfqpoint{0.000000in}{0.000000in}}{\pgfqpoint{0.048611in}{0.000000in}}{%
\pgfpathmoveto{\pgfqpoint{0.000000in}{0.000000in}}%
\pgfpathlineto{\pgfqpoint{0.048611in}{0.000000in}}%
\pgfusepath{stroke,fill}%
}%
\begin{pgfscope}%
\pgfsys@transformshift{2.119266in}{2.346060in}%
\pgfsys@useobject{currentmarker}{}%
\end{pgfscope}%
\end{pgfscope}%
\begin{pgfscope}%
\definecolor{textcolor}{rgb}{0.000000,0.000000,0.000000}%
\pgfsetstrokecolor{textcolor}%
\pgfsetfillcolor{textcolor}%
\pgftext[x=2.216488in, y=2.282746in, left, base]{\color{textcolor}\rmfamily\fontsize{12.000000}{14.400000}\selectfont 2}%
\end{pgfscope}%
\begin{pgfscope}%
\pgfsetbuttcap%
\pgfsetroundjoin%
\definecolor{currentfill}{rgb}{0.000000,0.000000,0.000000}%
\pgfsetfillcolor{currentfill}%
\pgfsetlinewidth{0.803000pt}%
\definecolor{currentstroke}{rgb}{0.000000,0.000000,0.000000}%
\pgfsetstrokecolor{currentstroke}%
\pgfsetdash{}{0pt}%
\pgfsys@defobject{currentmarker}{\pgfqpoint{0.000000in}{0.000000in}}{\pgfqpoint{0.048611in}{0.000000in}}{%
\pgfpathmoveto{\pgfqpoint{0.000000in}{0.000000in}}%
\pgfpathlineto{\pgfqpoint{0.048611in}{0.000000in}}%
\pgfusepath{stroke,fill}%
}%
\begin{pgfscope}%
\pgfsys@transformshift{2.119266in}{2.908903in}%
\pgfsys@useobject{currentmarker}{}%
\end{pgfscope}%
\end{pgfscope}%
\begin{pgfscope}%
\definecolor{textcolor}{rgb}{0.000000,0.000000,0.000000}%
\pgfsetstrokecolor{textcolor}%
\pgfsetfillcolor{textcolor}%
\pgftext[x=2.216488in, y=2.845589in, left, base]{\color{textcolor}\rmfamily\fontsize{12.000000}{14.400000}\selectfont 4}%
\end{pgfscope}%
\begin{pgfscope}%
\pgfsetrectcap%
\pgfsetmiterjoin%
\pgfsetlinewidth{0.803000pt}%
\definecolor{currentstroke}{rgb}{0.000000,0.000000,0.000000}%
\pgfsetstrokecolor{currentstroke}%
\pgfsetdash{}{0pt}%
\pgfpathmoveto{\pgfqpoint{1.998481in}{0.575369in}}%
\pgfpathlineto{\pgfqpoint{2.058874in}{0.575369in}}%
\pgfpathlineto{\pgfqpoint{2.119266in}{0.575369in}}%
\pgfpathlineto{\pgfqpoint{2.119266in}{2.991066in}}%
\pgfpathlineto{\pgfqpoint{2.058874in}{2.991066in}}%
\pgfpathlineto{\pgfqpoint{1.998481in}{2.991066in}}%
\pgfpathlineto{\pgfqpoint{1.998481in}{0.575369in}}%
\pgfpathclose%
\pgfusepath{stroke}%
\end{pgfscope}%
\end{pgfpicture}%
\makeatother%
\endgroup%

    \end{adjustbox}
    \caption{The p-matrix for \(\nu=0.1\).}\label{fig:sc2_pm_0.1}
  \end{subfigure}
  \caption{Correlation image (left column) and p-matrix (right column) for each model trained on a different viscosity (row). The correlation image was sorted same order the values of the real component of wave number \(k=2\) were sorted in descending order.}\label{fig:scenario_2_interpretation}
\end{figure}

Once we get to the highest viscosity, while the order is still present, we see a more random arrangement of the correlation image values. This observation and the lower scores can both be explained by the level of temporal discretization. This is the same issue we encounter when we use some traditional methods for higher viscosity values with the same time step for lower viscosity values \citep{kassamFourthOrderTimeSteppingStiff2005,seydaogluNumericalSolutionBurgers2016}. The issue arises in stiff differential equations. Stiff differential equations refer to the presence of a rapidly varying component. We can see that the larger amplitudes of high frequency components in the forcing term to represent the rapidly varying components as shown in \lccrefs{fig:burgers_forcing_0.0,fig:burgers_forcing_0.01,fig:burgers_forcing_0.1}. Traditionally, stiff equations would require specific numerical methods purpose built to efficiently solve the problem. Otherwise, the using a solver like FDM would require much finer time steps than the one we are using here.%TODO: add the data figures to the appendix

The p-matrix shows the difference in contributions of both input functions. For the inviscid equation, the contributions mainly come from the forcing term. The current solution contributes a smaller amount in comparison. There is similarity with the antiderivative p-matrices in this case. The contributions to each output wave number mainly come from input coefficients with the same wave number. However, there is a noticeable contribution from other wave numbers too unlike the antiderivative case. The biggest contribution from the current solution is the constant/bias term. For higher viscosity values, notice that the contribution of the previous solution increases. The increase is more pronounced for higher frequencies. This aligns with the quadratic scaling from the wave number on the coefficients \citep{canutoSpectralMethodsEvolution2007}. The p-matrix for lower viscosity values show more structured contribution of other wave numbers in comparison to the higher wave numbers. This structure comes in the form of a grid pattern. For the real components of the output, the contribution of other wave numbers oscillate between higher and lower values depending on whether the input component is real or not. The imaginary output components on the other hand have a more constant contribution across the input components. The contribution once again become more pronounced the higher the absolute value of the wave number.
\section{Conclusion}
\noindent Based on the results of the research into \MakeLowercase{\judul}, several conclusions can be drawn.
\begin{enumerate}
  \item The proposed SpectralSVR model is able to learn the relationships defined by PDEs. This is done using a model that learns in the Fourier domain with Fourier Transforms and Inverse Transform bridging the Fourier domain and the physical domain.
  \item Interpretation of the proposed model can be done to a certain extent using two tools which are the correlation image and p-matrix developed by \citet{ustunVisualisationInterpretationSupport2007}. This interpretation shows that the model correctly attributes the relationship of the input coefficients to the output coefficients.
  \item The proposed model is also capable in dealing with noise and partial data as shown in scenario 1.
        % \item The model's capabilities are also verified using exact solution. This revealed that for some configurations the model is able to successfully generalize to exact solutions.
\end{enumerate}

To improve the model's predictive capabilities, several key areas of enhancement have been identified for future works. First, the synthetic data used in scenarios 1 and 2 lack realism, even though they are mathematically correct when generated using the method of manufactured solutions. Incorporating data derived from traditional numerical methods or other reliable sources could help the model produce more accurate and realistic predictions. Evaluating the performance of auto-regression in weather data compared to target data is another critical step. Additionally, the inverse prediction results should be compared with traditional methods, such as the spectral method or finite difference method (FDM), to assess accuracy and reliability.

Further improvements include benchmarking the proposed method against comparable approaches and measuring computational efficiency in terms of time, relative to both traditional and machine learning-based methods. Expanding the range of basis functions, such as spherical harmonics or wavelet bases, and incorporating additional regression models, specifically support vector regression models, could enhance the utility of tools like the correlation image and p-matrix. The use of better performance metrics, such as the standard error of regression, is recommended, along with a detailed analysis of performance across varying hyperparameters, data sizes, and error trends by wave number. Addressing the issue of double penalties, which lead to overly smooth solutions \citep{brownForecastsSpatialFields2011,NIPS2017_44a2e080}, requires exploring stronger regularization for higher wave numbers. Finally, incorporating data assimilation techniques, such as the Lomb-Scargle periodogram \citep{vanderplasUnderstandingLombScargle2018} and 4D-Var assimilation \citep{puNumericalWeatherPrediction2018,parkDataAssimilationAtmospheric2013}, could enable support for sparse data. Finally, a study of contribution patterns for different terms in partial differential equations (PDEs) using methods like the p-matrix would provide deeper insights into model dynamics.

These future directions build on the strengths of the proposed SpectralSVR model, which has demonstrated its ability to learn relationships defined by PDEs, interpret results using tools like the correlation image and p-matrix, and handle noise and partial data effectively. By addressing these identified areas of improvement, the model's robustness, accuracy, and applicability could be further enhanced, especially in real-world scenarios.

Through these developments, the SpectralSVR model has the potential to not only refine its generalization capabilities, as evidenced by its partial success with exact solutions, but also to expand its utility across diverse and complex problems. These steps will contribute to advancing computational approaches for SciML-based modelling and establish the proposed framework as a versatile tool in this domain.

% !TEX root = ./skripsi.tex
\appendix

\chapter{EXAMPLE COMPUTATION OF LSSVR}
% Example computation
In this example we will be using the function \(2x^2+4\). The values of this function can be seen in \lccref{table:lssvr_example_function_values}.

\begin{table}[H]
  \centering
  \begin{tabular}{@{}lll@{}}
    \toprule
    No & x          & y          \\ \midrule
    1  & 0.0        & 4.0        \\
    2  & 0.33\ldots & 4.22\ldots \\
    3  & 0.66\ldots & 4.88\ldots \\
    4  & 1.0        & 6.0        \\
    \bottomrule
  \end{tabular}
  \caption{Example data of function \(2x^2+4\)}\label{table:lssvr_example_function_values}
\end{table}
For \(\Omega_{i,j}=K(x_i,x_j)=\exp\left(-\frac{\norm*{x_i-x_j}^2}{2\sigma^2}\right)\)

For example, with \(\sigma=1\), \(x_i=0.0\), \& \(x_j = 0.33\dots \)
\begin{equation}
  \begin{split}
    K(0.0,0.33) & = \exp\left(-\frac{\norm*{0.0-0.33}^2}{2{(1)}^2}\right) \\
                & = \exp\left(-\frac{0.33^2}{2}\right)                  \\
                & = 0.9460
  \end{split}
\end{equation}

\begin{equation}
  \Omega\gets
  \begin{bmatrix}
    1.0000 & 0.9460 & 0.8007 & 0.6065 \\
    0.9460 & 1.0000 & 0.9460 & 0.8007 \\
    0.8007 & 0.9460 & 1.0000 & 0.9460 \\
    0.6065 & 0.8007 & 0.9460 & 1.0000 \\
  \end{bmatrix}
\end{equation}

\begin{equation}
  \vb{I}\frac{1}{C}\to\vb{I}\frac{1}{5}\to
  \begin{bmatrix}
    0.2000 & 0.0000 & 0.0000 & 0.0000 \\
    0.0000 & 0.2000 & 0.0000 & 0.0000 \\
    0.0000 & 0.0000 & 0.2000 & 0.0000 \\
    0.0000 & 0.0000 & 0.0000 & 0.2000
  \end{bmatrix}
\end{equation}

\begin{equation}
  \Omega+\vb{I}\frac{1}{5}\to H\to
  \begin{bmatrix}
    1.2000 & 0.9460 & 0.8007 & 0.6065 \\
    0.9460 & 1.2000 & 0.9460 & 0.8007 \\
    0.8007 & 0.9460 & 1.2000 & 0.9460 \\
    0.6065 & 0.8007 & 0.9460 & 1.2000 \\
  \end{bmatrix}
\end{equation}

\begin{equation}
  A\to
  \begin{bmatrix}
    0.0000 & 1.0000 & 1.0000 & 1.0000 & 1.0000 \\
    1.0000 & 1.2000 & 0.9460 & 0.8007 & 0.6065 \\
    1.0000 & 0.9460 & 1.2000 & 0.9460 & 0.8007 \\
    1.0000 & 0.8007 & 0.9460 & 1.2000 & 0.9460 \\
    1.0000 & 0.6065 & 0.8007 & 0.9460 & 1.2000 \\
  \end{bmatrix}
\end{equation}

\begin{equation}
  B\to
  \begin{bmatrix}
    0.0000 \\
    4.0000 \\
    4.2222 \\
    4.8889 \\
    6.0000 \\
  \end{bmatrix}
\end{equation}

\begin{equation}
  A^{\dag}\to
  \begin{bmatrix}
    -0.8994 & 0.4348  & 0.0652  & 0.0652  & 0.4348  \\
    0.4348  & 2.0686  & -1.6490 & -0.5292 & 0.1096  \\
    0.0652  & -1.6490 & 3.3774  & -1.1992 & -0.5292 \\
    0.0652  & -0.5292 & -1.1992 & 3.3774  & -1.6490 \\
    0.4348  & 0.1096  & -0.5292 & -1.6490 & 2.0686  \\
  \end{bmatrix}
\end{equation}

\begin{equation}
  A^{\dag}B\to S\to
  \begin{bmatrix}
    4.9421  \\
    -0.6177 \\
    -1.3737 \\
    -0.5625 \\
    2.5538  \\
  \end{bmatrix}
\end{equation}

\begin{equation}
  b\to 4.9421
\end{equation}

\begin{equation}
  \alpha \to
  \begin{bmatrix}
    -0.6177 \\
    -1.3737 \\
    -0.5625 \\
    2.5538  \\
  \end{bmatrix}
\end{equation}

Prediction
\begin{equation}
  U\to
  \begin{bmatrix}
    0.3 \\
    0.2 \\
    0.5
  \end{bmatrix}
\end{equation}

\begin{equation}
  \Omega\to
  \begin{bmatrix}
    0.9560 & 0.9994 & 0.9350 & 0.7827 \\
    0.9802 & 0.9912 & 0.8968 & 0.7261 \\
    0.8825 & 0.9862 & 0.9862 & 0.8825 \\
  \end{bmatrix}
\end{equation}

\begin{equation}
  \Omega\alpha\to
  \begin{bmatrix}
    -0.4904 \\
    -0.6170 \\
    -0.2008 \\
  \end{bmatrix}
\end{equation}

\begin{equation}
  \Omega\alpha+b\vb{1}_{m}\to v\to
  \begin{bmatrix}
    4.4516 \\
    4.3251 \\
    4.7413 \\
  \end{bmatrix}
\end{equation}
Where \(\vb{1}_{m}\) is a vector of 1s with the length of \(U\).

\chapter{BURGERS' EQUATION COMPARISON}\label{sec:burgers_comparison}
Comparison of SpectralSVR against the Method of Lines with finite differences. The numerical methods used for comparison are lsoda and Implicit Adams-Bashforth-Moulton methods.
\begin{figure}[H]
  \centering
  \begin{adjustwidth}{-0.05\linewidth}{-0.05\linewidth}
    \begin{subfigure}{0.49\linewidth}
      \begin{adjustbox}{width=\linewidth}
        \input{figures/comparisons/burgers_rollout_model_pred_0.0.pgf}
      \end{adjustbox}
      \caption{SpectralSVR prediction.}\label{fig:comp_model_pred_0.0}
    \end{subfigure}
    \begin{subfigure}{0.49\linewidth}
      \begin{adjustbox}{width=\linewidth}
        \input{figures/comparisons/burgers_rollout_model_diff_0.0.pgf}
      \end{adjustbox}
      \caption{SpectralSVR difference with target.}\label{fig:comp_model_diff_0.0}
    \end{subfigure}
    \begin{subfigure}{0.49\linewidth}
      \begin{adjustbox}{width=\linewidth}
        \input{figures/comparisons/burgers_rollout_spo_pred_0.0.pgf}
      \end{adjustbox}
      \caption{SciPy \& NumPy prediction.}\label{fig:comp_spo_pred_0.0}
    \end{subfigure}
    \begin{subfigure}{0.49\linewidth}
      \begin{adjustbox}{width=\linewidth}
        \input{figures/comparisons/burgers_rollout_spo_diff_0.0.pgf}
      \end{adjustbox}
      \caption{SciPy \& NumPy difference with target.}\label{fig:comp_spo_diff_0.0}
    \end{subfigure}
    \begin{subfigure}{0.49\linewidth}
      \begin{adjustbox}{width=\linewidth}
        \input{figures/comparisons/burgers_rollout_tdo_pred_0.0.pgf}
      \end{adjustbox}
      \caption{torchdiffeq \& PyTorch prediction.}\label{fig:comp_tdo_pred_0.0}
    \end{subfigure}
    \begin{subfigure}{0.49\linewidth}
      \begin{adjustbox}{width=\linewidth}
        \input{figures/comparisons/burgers_rollout_tdo_diff_0.0.pgf}
      \end{adjustbox}
      \caption{torchdiffeq \& PyTorch difference with target.}\label{fig:comp_tdo_diff_0.0}
    \end{subfigure}
  \end{adjustwidth}
  \caption{Comparison of our model (SpectralSVR) and numerical methods for the Forced Burgers equation with viscosity \(\nu=0.0\).}\label{fig:comparison_burgers_0.0}
\end{figure}

\begin{figure}[H]
  \centering
  \begin{adjustwidth}{-0.05\linewidth}{-0.05\linewidth}
    \begin{subfigure}{0.49\linewidth}
      \begin{adjustbox}{width=\linewidth}
        \input{figures/comparisons/burgers_rollout_model_pred_0.01.pgf}
      \end{adjustbox}
      \caption{SpectralSVR prediction.}\label{fig:comp_model_pred_0.01}
    \end{subfigure}
    \begin{subfigure}{0.49\linewidth}
      \begin{adjustbox}{width=\linewidth}
        \input{figures/comparisons/burgers_rollout_model_diff_0.01.pgf}
      \end{adjustbox}
      \caption{SpectralSVR difference with target.}\label{fig:comp_model_diff_0.01}
    \end{subfigure}
    \begin{subfigure}{0.49\linewidth}
      \begin{adjustbox}{width=\linewidth}
        \input{figures/comparisons/burgers_rollout_spo_pred_0.01.pgf}
      \end{adjustbox}
      \caption{SciPy \& NumPy prediction.}\label{fig:comp_spo_pred_0.01}
    \end{subfigure}
    \begin{subfigure}{0.49\linewidth}
      \begin{adjustbox}{width=\linewidth}
        \input{figures/comparisons/burgers_rollout_spo_diff_0.01.pgf}
      \end{adjustbox}
      \caption{SciPy \& NumPy difference with target.}\label{fig:comp_spo_diff_0.01}
    \end{subfigure}
    \begin{subfigure}{0.49\linewidth}
      \begin{adjustbox}{width=\linewidth}
        \input{figures/comparisons/burgers_rollout_tdo_pred_0.01.pgf}
      \end{adjustbox}
      \caption{torchdiffeq \& PyTorch prediction.}\label{fig:comp_tdo_pred_0.01}
    \end{subfigure}
    \begin{subfigure}{0.49\linewidth}
      \begin{adjustbox}{width=\linewidth}
        \input{figures/comparisons/burgers_rollout_tdo_diff_0.01.pgf}
      \end{adjustbox}
      \caption{torchdiffeq \& PyTorch difference with target.}\label{fig:comp_tdo_diff_0.01}
    \end{subfigure}
  \end{adjustwidth}
  \caption{Comparison of our model (SpectralSVR) and numerical methods for the Forced Burgers equation with viscosity \(\nu=0.01\).}\label{fig:comparison_burgers_0.01}
\end{figure}
\begin{table}
  \caption{Metrics for predicting the exact solution for SpectralSVR (Ours) method}\label{table:comparison_exact_metrics_model}
  \centering
  \begin{tabular}{lcccc}
    \toprule
    \(\nu \) & MSE  & RMSE & MAE  & sMAPE \\
    \midrule
    0.0      & 0.01 & 0.07 & 0.06 & 1.99  \\
    0.01     & 0.00 & 0.07 & 0.06 & 1.61  \\
    0.1      & 0.00 & 0.02 & 0.02 & 1.47  \\
    \bottomrule
  \end{tabular}
\end{table}
\begin{table}
  \caption{Metrics for predicting the exact solution for the lsoda method}\label{table:comparison_exact_metrics_lsoda}
  \centering
  \begin{tabular}{lcccc}
    \toprule
    \(\nu \) & MSE  & RMSE & MAE  & sMAPE \\
    \midrule
    0.0      & 0.00 & 0.00 & 0.00 & 0.00  \\
    0.01     & 0.00 & 0.00 & 0.00 & 0.09  \\
    0.1      & 0.00 & 0.03 & 0.02 & 1.56  \\
    \bottomrule
  \end{tabular}
\end{table}
\begin{table}
  \caption{Metrics for predicting the exact solution for the IABM method}\label{table:comparison_exact_metrics_iabm}
  \centering
  \begin{tabular}{lcccc}
    \toprule
    \(\nu \) & MSE  & RMSE & MAE  & sMAPE \\
    \midrule
    0.0      & 0.00 & 0.00 & 0.00 & 0.00  \\
    0.01     & nan  & nan  & nan  & nan   \\
    0.1      & nan  & nan  & nan  & nan   \\
    \bottomrule
  \end{tabular}
\end{table}
% \section{PROGRAM SATU}
% \section{PROGRAM DUA}


% \chapter{GAMBAR-GAMBAR}



\bibliographystyle{elsarticle-harv}
\bibliography{references.bib}
\end{document}